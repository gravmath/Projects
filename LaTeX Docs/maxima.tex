\documentclass[10pt]{article}
\usepackage[absolute]{textpos}
\usepackage{graphicx}
\setlength{\parindent}{0pt}
\usepackage{color}
\usepackage{amssymb}
\textblockorigin{0.5in}{0.5in}
\definecolor{MintGreen}      {rgb}{0.2,0.85,0.5}
\definecolor{LightGreen}     {rgb}{0.9,1,0.9}
\definecolor{LightYellow}    {rgb}{1,1,0.6}
\definecolor{test}           {rgb}{0.8,0.95,0.95}


\usepackage[absolute]{textpos}
\TPshowboxestrue
\begin{document}
\null
%%%%%%%%%%%%%%%%%%%%%%%%%%%%%%%%%%%%%%%%%%%%%%%%%%%%%%%%%%%%%%%
%          Header - Maxima
%%%%%%%%%%%%%%%%%%%%%%%%%%%%%%%%%%%%%%%%%%%%%%%%%%%%%%%%%%%%%%%
\textblockcolor{MintGreen}
\begin{textblock*}{7.5in}(0mm,0mm)
\begin{tabular*}{7.5in}{c @{\extracolsep{\fill}} c }
       \tiny ~ & ~\\
       \multicolumn{2}{c}{\normalsize \bf Maxima} \\
       \multicolumn{2}{r}{\scriptsize \emph{C. Schiff - 11/22/10}} \\
\end{tabular*}
\end{textblock*}

%%%%%%%%%%%%%%%%%%%%%%%%%%%%%%%%%%%%%%%%%%%%%%%%%%%%%%%%%%%%%%%
%          File Loading
%%%%%%%%%%%%%%%%%%%%%%%%%%%%%%%%%%%%%%%%%%%%%%%%%%%%%%%%%%%%%%%
\scriptsize
\textblockcolor{LightYellow}
\begin{textblock*}{105mm}(0mm,12.54mm)
\begin{tabular*}{105mm}{c @{\extracolsep{\fill}} c }
       \tiny ~ & ~\\
       \multicolumn{2}{c}{\scriptsize \bf File Loading} \\
       \tiny ~ & ~\\	   
\end{tabular*}
To load a file use either full path version, such as \verb#load("c:/work2/qxxx.mac")#, or
the brief version, such as \verb#load(qxxx.mac)#.  For the brief version to work the 
file \verb#qxxx.mac# needs to be placed in one of the folders Maxima searches by default, 
or else, Maxima must be directed to search the desired directory.  To determine
which directories Maxima searches by default, a query at the prompt, of \verb#file_search_maxima#. 
To add to the default path, put a line like:
\begin{verbatim}
  file_search_maxima : append(["c:/work2/###.{mac,mc}"], file_search_maxima )$
\end{verbatim}
and likewise with \verb#maxima# replaced by \verb#lisp#
in the personal startup file \verb#maxima-init.mac#.

The file \verb#maxima-init.mac# is located somewhere like (type \verb#maxima_userdir# to see):\\
\verb$c:\users\cschiff\maxima\maxima-init.mac$.  

\end{textblock*}


%%%%%%%%%%%%%%%%%%%%%%%%%%%%%%%%%%%%%%%%%%%%%%%%%%%%%%%%%%%%%%%
%          Misc. Commands
%%%%%%%%%%%%%%%%%%%%%%%%%%%%%%%%%%%%%%%%%%%%%%%%%%%%%%%%%%%%%%%
\scriptsize
\textblockcolor{LightGreen}
\begin{textblock*}{85.68mm}(104.85mm,12.54mm)
\begin{tabular*}{85.68mm}{l @{\extracolsep{\fill}} l}
                           & \\
\multicolumn{2}{c}{\bf Misc. Commands} \\
                           & \\
\verb#ctrl-q#              & quits Maxima \\
\verb#ctrl-r#              & Evaluates all cells\\
\verb#<command>$#          & runs command and supresses \verb#%o<n># output\\
\verb#ratsimp#             & perform rational simplification (once)\\
\verb#fullratsimp#         & perform rational simplification\\
                           & (stops when no change occurs)\\
\verb#rat#                 & puts an expression into CRE\\
                           & (Canonical Rational Expression)\\				   
\verb#/*<text>*/#          & Insert a comment in file or cell\\
\verb#apropos("foo")#      & returns list of core Maxima names\\
                           & which have \verb#foo# within them\\
\verb#describe("e",<tag>)# & prints to screen a numbered list of all items\\					  
                           & which contain \verb@"e"@ as part of their name.  The\\
						   & \verb%<tag>% can be \verb#exact# or \verb#inexact# which defaults\\
                           & to \verb#exact# if \verb%<tag>% is ommitted\\
\verb#kill#                & \verb#kill(a,b)# will eliminate objects \verb#a# and \verb#b#\\
\verb#ev#                  & \verb#ev ( expr, options )# (or \verb# expr, options# for \\
                           & the interactive version) evaluates the expression \\
						   & \verb#expr# in the environment specified by the \\
						   & arguments \verb#arg_1, ..., arg_n#.  The arguments\\
						   & are switches, assignments, equations, and \\
						   & functions. \\
\verb#evflag#			   & A symbol \verb#x# has the evflag property, the \\
                           & expressions \verb#ev(expre, x)# are equivalent to\\
						   & \verb#ev(expr, x = true)#.\\
\verb#evflag# options      & \verb#algebraic, cauchysum, demoivre, dotscrules,#\\
                           & \verb#%emode, %enumer, exponentialize, exptisolate#,\\ 
						   & \verb#factorflag, float, halfangles, infeval#,\\ 
						   & \verb#isolate_wrt_times, keepfloat, letrat#,\\
						   & \verb#listarith, logabs, logarc, logexpand#,\\
                           & \verb#lognegint, lognumer, m1pbranch#,\\
						   & \verb#numer_pbranch, prgrammode, radexpand#, \\
						   & \verb#ratalgdenom, ratfac, ratmx, ratsimpexpons#,\\
						   & \verb#simp, simpsum, sumexpand, and trigexpand.#\\
\verb#evfun#               & \\
\verb#properties#		   & gives the properties of any expression\\
\verb#declare#             & \\
\verb#fundef#              & \\
\verb#values#              & \\
\verb#map#                 & sgsgds \\
\verb#fullmap#             & \\
\verb#apply#               & \\
\verb#subst#               & \\
\verb#ratsubst#            & \\
\verb#part#                & \\
\verb#substpart#           & \\
\verb#coeff#               & \\
\verb#ratcoef#             & \\

\end{tabular*}
\end{textblock*}

%%%%%%%%%%%%%%%%%%%%%%%%%%%%%%%%%%%%%%%%%%%%%%%%%%%%%%%%%%%%%%%
%          Directory Commands
%%%%%%%%%%%%%%%%%%%%%%%%%%%%%%%%%%%%%%%%%%%%%%%%%%%%%%%%%%%%%%%
\scriptsize
\textblockcolor{test}
\begin{textblock*}{100mm}(0mm,127.7mm)
\begin{tabular*}{100mm}{l @{\extracolsep{\fill}} l}
                & \\
\multicolumn{2}{c}{\bf Directory Commands} \\
                & \\
opendir:        & opendir(DIR,\$dir\_path) opens the directory specified by\\
                &\$dir\_path, assigns a directory handle DIR is possible\\
                &(return of $true$) or returns $false$\\
                & \\
readdir:        & @files = readdir(DIR); puts a list of files in the\\
                & directory handle specified by DIR.  \emph{Note:} that\\
				& @files includes the directories `$.$' and `$..$'\\
                & \\
closedir:       & closedir(DIR) closes the directory specified by the\\
                & directory handle DIR\\
				&\\
system:         & system(``dir'') run a system command with a \$string = ``dir'' or\\
                & system(@command) where @command is a list of strings\\
                & \\
\end{tabular*}
\end{textblock*}

\newpage
\null

%%%%%%%%%%%%%%%%%%%%%%%%%%%%%%%%%%%%%%%%%%%%%%%%%%%%%%%%%%%%%%%
%          Header - FreeFlyer
%%%%%%%%%%%%%%%%%%%%%%%%%%%%%%%%%%%%%%%%%%%%%%%%%%%%%%%%%%%%%%%
\textblockcolor{MintGreen}
\begin{textblock*}{7.5in}(0mm,0mm)
\begin{tabular*}{7.5in}{c @{\extracolsep{\fill}} c }
       \tiny ~ & ~\\
       \multicolumn{2}{c}{\normalsize \bf FreeFlyer} \\
       \multicolumn{2}{r}{\scriptsize \emph{C. Schiff - 1/28/08}} \\
\end{tabular*}
\vspace{0.85mm}
\end{textblock*}

%%%%%%%%%%%%%%%%%%%%%%%%%%%%%%%%%%%%%%%%%%%%%%%%%%%%%%%%%%%%%%%
%          Ephemeris Object
%%%%%%%%%%%%%%%%%%%%%%%%%%%%%%%%%%%%%%%%%%%%%%%%%%%%%%%%%%%%%%%
\TPMargin{1mm}
\textblockcolor{LightYellow}
\begin{textblock*}{101.65mm}(0mm,10mm)
\normalsize \textbf{Ephemeris Object}\\
\scriptsize\\
\textbf{Object Representation}\\
\begin{tabular*}{97.15 mm}{|lll|}\hline
  \textbf{\emph{Parameter}} & \textbf{\emph{Values}} & \textbf{\emph{Explanation}}\\
                            &                        &                            \\
  Name                      & ``Ephemeris''          & String denoting the\\
                            &                        & object name\\
  CentralBody               & ``Earth''              & Central body\\
                            &                        & as origin\\
  StepSize                  & 0 \emph{or} $\# > 0$   & 0 variable step\\
                            &                        & $\# > 0$ fixed step\\
  Branch                    &  ????                  & ???? 		      \\
  WritePosVel               & Record position and    & `0' off   	      \\
                            & velocity               & `1' on\\ 
  WriteAcceleration         & Record accelerations   & `0' off	      \\
                            & (derived from what)    & `1' on\\ 
  WriteAttitude             & Record attitude        & `0' off				      \\
                            & (what parametrization) & `1' on\\ 
  WriteAngularVelocity      & Record attitude rates  & `0' off				      \\
                            & (what parametrization) & `1' on\\ 
  WriteAngularAcceleration  & Record `torque'        & `0' off			      \\
                            & (what parametrization) & `1' on\\ 
\hline
\end{tabular*}\\
\\
\textbf{Create}\\
Create Ephemeris ephem\_name;\\
\\
\textbf{Putting a SC to an Ephem}\\
Put SC to Ephem;\\
Put SC to Ephem as Global;\\
\\
\textbf{Ephem to Disk}\\
Put Ephem to FFephem  ``\textless path \textgreater \textbackslash \textless filename \textgreater'';\\
Put Ephem to STKephem ``\textless path \textgreater \textbackslash \textless filename \textgreater'';\\
Put Ephem to PCephem  ``\textless path \textgreater \textbackslash \textless filename \textgreater'';\\
Put Ephem to PCephem  ``\textless path \textgreater \textbackslash \textless filename \textgreater'' with StepSize = 60 and CS as J2000 or TOD;\\
Put Ephem to ephem    ``\textless path \textgreater \textbackslash \textless filename \textgreater'' with StepSize = 60 and CS as J2000 or TOD;\\
\\
where:\\
~~~FFephem - flat ASCII file in native FF format\\
~~~STKephem - flat ASCII file in native STK format\\
~~~PCephem - Code 500 binary file (little endian)\\
~~~ephem   - Code 500 binary file (big endian)\\
\end{textblock*}

%%%%%%%%%%%%%%%%%%%%%%%%%%%%%%%%%%%%%%%%%%%%%%%%%%%%%%%%%%%%%%%
%          Dynamic Ephemeris Naming
%%%%%%%%%%%%%%%%%%%%%%%%%%%%%%%%%%%%%%%%%%%%%%%%%%%%%%%%%%%%%%%
\textblockcolor{LightGreen}
\begin{textblock*}{89mm}(101.5mm,10mm)
\normalsize \textbf{Dynamic Ephemeris Naming}\\
\scriptsize\\
\textbullet Create String eph\_name\\
\textbullet Create UserInterface my\_gui\\
\begin{tabular*}{60 mm}{llll}
  & \textperiodcentered my\_gui.NumberOfInputs      &     &= 1;\\
  & \textperiodcentered my\_gui.ObjectNameToDisplay &     &= ``'';\\
  & \textperiodcentered my\_gui.ContinueButtonLabel &     &= ``Continue'';\\  
  & \textperiodcentered my\_gui.ExplicitEdit        &(0)  &= 0;\\
  & \textperiodcentered my\_gui.ParameterLabel      &(0)  &= ``Input Ephem Name'';\\
  & \textperiodcentered my\_gui.DefaultValue        &(0)  &= ``'';\\
  & \textperiodcentered my\_gui.MinimumRange        &(0)  &= 999;\\
  & \textperiodcentered my\_gui.MaximumRange        &(0)  &= -999;\\
  & \textperiodcentered my\_gui.Units               &(0)  &= ``'';\\
  & \textperiodcentered my\_gui.ParameterName       &(0)  &= ``eph\_name.Value'';\\
  & \textperiodcentered my\_gui.EntryType           &(0)  &= ``Enter Value'';\\
  & \textperiodcentered my\_gui.FormatString        &(0)  &= ``'';\\
\end{tabular*}\\
\textbullet Create Ephemeris  ephem;\\
\textbullet Create Spacecraft SC using ephem;\\
\textbullet Show my\_gui; \emph{//gets string via UserInterface into eph\_name}\\
\textbullet Get ephem from FFephem eph\_name;\\
\end{textblock*}

\newpage
\null

%%%%%%%%%%%%%%%%%%%%%%%%%%%%%%%%%%%%%%%%%%%%%%%%%%%%%%%%%%%%%%%
%          Header - Maxima
%%%%%%%%%%%%%%%%%%%%%%%%%%%%%%%%%%%%%%%%%%%%%%%%%%%%%%%%%%%%%%%
\textblockcolor{MintGreen}
\begin{textblock*}{7.5in}(0mm,0mm)
\begin{tabular*}{7.5in}{c @{\extracolsep{\fill}} c }
       \tiny ~ & ~\\
       \multicolumn{2}{c}{\normalsize \bf Maxima} \\
       \multicolumn{2}{r}{\scriptsize \emph{C. Schiff - 2/4/08}} \\
\end{tabular*}
\vspace{0.85mm}
\end{textblock*}




















\end{document}