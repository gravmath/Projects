\documentclass[12pt]{article}
\usepackage{pstricks,pst-3dplot}
\begin{document}

\begin{pspicture}(-2,-2)(9,5)
  \uput[0](-2,4.5){Time Derivatives in non-inertial (i.e. rotating) frames}
  \uput[0](4,2){Basic Rule:}
  \uput[0](3.5,1.5){$\vec{\dot{\rho}}_{inertial}=\vec{\dot{\rho}}_{body}+\vec{\omega}\times\vec{\rho}$ }
  \uput[0](-2,-1){\small{Inertial Frame}}
  \uput[0](3,0){\textcolor{red}{\small{Rotating Body Frame}}}
  \uput[0](2.5,2.5){\textcolor{red}{\small{Body CM}}}
  \psset{xMin=0,yMin=0,zMin=0}%
  \pstThreeDCoor[xMax=1,yMax=1,zMax=1,linewidth=2pt,nameX=$\noexpand\hat{I}$,%
    nameY=$\noexpand\hat{J}$,nameZ=$\noexpand\hat{K}$,linecolor=blue]%
%
  \pstThreeDLine[arrows=->](0,0,0)(1,4,5)
  \pstThreeDLine[arrows=->](0,0,0)(1,3.5,2)
  \pstThreeDLine[arrows=->,linecolor=red](1,3.5,2)(1,4,5)
%
%  label vectors
%
  \pstThreeDNode(.5,2,2.5){RM}
  \uput[180](RM){$\vec{\mbox{R}}$}
  \pstThreeDNode(.5,1.75,1){rM}
  \uput[90](rM){$\vec{\mbox{r}}$}
  \pstThreeDNode(1,3.75,3.5){rhoM}
  \uput[0](rhoM){\textcolor{red}{$\vec{\rho}$}}
%
% generate CM symbol
%
  \pstThreeDNode(1,4,5){cM}
  \pscircle[linewidth=.5pt](cM){.15}
  \pswedge[fillstyle=solid,fillcolor=black](cM){.15}{90}{180}
  \pswedge[fillstyle=solid,fillcolor=black](cM){.15}{270}{360}
%
%  second method for axes
%
  \pstThreeDPut[origin=lb](1,3.5,2){%
    \begin{pspicture}(-2,-2)(2,2)
       \pstThreeDCoor[xMax=1,yMax=1,zMax=1,linecolor=red,linewidth=1pt,%
          Alpha=115,Beta=70,nameX=$\noexpand\hat{i}$,%
          nameY=$\noexpand\hat{j}$,nameZ=$\noexpand\hat{k}$]%
    \end{pspicture}%<---- important
   }%
\end{pspicture}

\end{document}
