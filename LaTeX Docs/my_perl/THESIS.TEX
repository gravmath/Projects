\documentclass[12pt]{article}
\usepackage{latexsym}
\usepackage{epic,eepic,graphicx,url,amsmath,amssymb,latexsym,alltt}
\begin{document}
\title{Blah}
\author{Blah}
\maketitle

\section{Spacetime Slicing}

\clearpage
\vspace{5mm}
\hrule
\begin{alltt}
  constant_t_surface.dat
  key St1
\end{alltt}
\hrule
\vspace{5mm}

Define a one-form $ {\tilde \Omega} $ corresponding to surfaces of constant time $t$ by
taking the covariant derivative of the time coordinate
\begin{equation}\label{St1_1}
{ \Omega } _ { a } = {\nabla}_{a} t \doteq \left( \begin{array}{cccc  } 
 1 & 0 & 0 & 0             \end{array} \right)
\end{equation}
subject to the condition that $ { \Omega } _ { a } $ be timelike
\begin{equation}\label{St1_2} \nonumber
{ \Omega } ^ { a } { \Omega } _ { a } = { g } ^ { 00 } \equiv \frac{-1}{ \alpha ^2}
\end{equation}
Introduce the normalized one-form
\begin{equation}\label{St1_3} \nonumber
{ \omega } ^ { a } { \omega } _ { a } = -1
\end{equation}
via the definition
\begin{equation}\label{St1_4} \nonumber
{ \omega } _ { a } = \alpha { \Omega } _ { a }
\end{equation}

A normalized unit-vector $ {n ^ a} $ orthogonal to a constant-$t$ slice can be
constructed by
\begin{equation}\label{St1_4} \nonumber
{n ^ a} = - {g}^{a b} { \omega } _ { b }
\end{equation}
\section{Hamiltonian Formulation for a Single Point Particle}

\clearpage
\vspace{5mm}
\hrule
\begin{alltt}
  ham_part_action.dat
  key Pa1
\end{alltt}
\hrule
\vspace{5mm}

In this section, I present a Hamiltonian variational principle
for geodesic motion. Since our ultimate aim is use
this action as a model for the motion of a point-particle, we
will hereafter refer to any action describing geodesic motion
as the `point-particle' action and we will denote it
as $ I_{pp} $.

The proposed variational principle is
given by
\begin{equation}\label{Pa1_1}
I_{pp} = \int d \lambda m \left( {\dot z}^{\mu} {u}_{\mu} - \Lambda \mathcal{H} \right) \quad ,
\end{equation}
where $ {z ^ \mu} \left( \lambda \right) $ gives the coordinates of the particle
parametrized by the path parameter $ \lambda $ and
$ {\dot z}^{\mu} \equiv \frac{ d {z^ \mu } }{ d \lambda } $.
The 4-velocity is given by $ {u}_{\mu} $ and the Hamiltonian $ \mathcal{ H } $
takes the form
\begin{equation}\label{Pa1_2}
\mathcal{H} = \frac{1}{2} \left( {g}^{\mu \nu} {u}_{\mu} {u}_{\nu} + 1 \right)
\end{equation}
The functions to be varied in Eq. (\ref{Pa1_1}) are $ \Lambda $, $ {z ^ \mu} $,
and $ {u}_{\mu} $.

As can be seen in Eq. (\ref{Pa1_1)}) $ I_{pp} $ is invariant under changes
to the path parameter $ \lambda $ with $ \Lambda $, which acts as a Lagrange multiplier
that enforces 4-velocity normalization, adjusting under each
re-parametrization.


\subsection{Variation With Respect To $ \Lambda $}

\clearpage
\vspace{5mm}
\hrule
\begin{alltt}
  var_pp_L.dat
  key Pav1
\end{alltt}
\hrule
\vspace{5mm}

To begin, we consider the variation of $ I_{pp} $ with respect to the
Lagrange multiplier $ \Lambda $
\begin{equation}\label{Pav1_1}
{ \delta I_{pp} } |_{\delta \Lambda } = \int d \lambda \mathcal{H} \delta \Lambda
\end{equation}
Since $ \delta \Lambda $ is arbitrary this leads immediately to the equation
\begin{equation}\label{Pav1_2}
\mathcal{ H } = {g}^{\mu \nu} {u}_{\mu} {u}_{\nu} + 1 = 0
\end{equation}
which garuentees the normalization of the 4-velocity.

\subsection{Variation With Respect To $ u_\alpha $}

\clearpage
\vspace{5mm}
\hrule
\begin{alltt}
  var_pp_u.dat
  key Pav3
\end{alltt}
\hrule
\vspace{5mm}

Next we consider the variation of $ I_{pp} $ with respect to $ {u}_{\alpha} $
\begin{equation}\label{Pav3_1}
{ \delta I_{pp} } |_{\delta {u}_{\alpha} } = \int d \lambda \left( {\dot z}^{\alpha}
- \Lambda {g}^{\alpha \mu} {u}_{\mu} \right) \delta {u}_{\alpha}
\end{equation}
Setting the variation equal to zero yields
\begin{equation}\label{Pav3_2}
{\dot z}^{\alpha} = \Lambda {g}^{\alpha \mu} {u}_{\mu}
\end{equation}
\subsection{Variation With Respect To $ z ^ \mu $}

\clearpage
\vspace{5mm}
\hrule
\begin{alltt}
  var_pp_z.dat
  key Pav2
\end{alltt}
\hrule
\vspace{5mm}

Finally we consider the variation of $ I_{pp} $ with respect to
the particle position $ {z ^ \mu} $
\begin{equation}\label{Pav2_1}
{ \delta I_{pp} } |_{\delta {z ^ \alpha} } = \int d \lambda \left( {u}_{\alpha} \delta {\dot z}^{\alpha}
- \frac{\Lambda}{2}
{ \delta {g}^{\mu \nu} } |_{\delta {z ^ \alpha} } {u}_{\mu} {u}_{\nu} \right)
\end{equation}
Since the metric $ {g}^{\mu \nu} $ depends only on the particle position
$ {z ^ \mu} $ the corresponding variational derivative is equivalent to the
partial derivative $ \frac{\partial {g}^{\mu \nu}}{\partial {z ^ \alpha}} $.

The next step is to integrate the first term by parts which yields
\begin{equation}\label{Pav2_2}
{ \delta I_{pp} } |_{\delta {z ^ \alpha} } = - \int d \lambda \left( \frac{ d }{d \lambda} {u}_{\alpha}
+ \frac{\Lambda}{2}
\frac{\partial {g}^{\mu \nu}}{\partial {z ^ \alpha}} {u}_{\mu} {u}_{\nu} \right) \delta {z ^ \alpha}
\end{equation}
Since the variation $ \delta {z ^ \alpha} $ is arbitrary its coefficient must
be zero, which results in
\begin{equation}\label{Pav2_3}
\frac{ d }{d \lambda} {u}_{\alpha} + \frac{\Lambda}{2} \frac{\partial {g}^{\mu \nu}}{\partial {z ^ \alpha}} {u}_{\mu} {u}_{\nu} = 0
\end{equation}
\subsection{Obtaining the Geodesic Equations}

\clearpage
\vspace{5mm}
\hrule
\begin{alltt}
  ham_geodesic_eqs.dat
  key Pav3_2
\end{alltt}
\hrule
\vspace{5mm}

In order for Eq. (\ref{Pav3_2}) to be consistent with the
interpretation of $ {u}_{\alpha} $ as a 4-velocity, we must
make the identification
\begin{equation}\label{Pa2_1}
\frac{1}{ \Lambda } \frac{ d }{d \lambda} = \frac{ d }{d \tau}
\end{equation}
Multiplying Eq. (\ref{Pav3_2}) by $ {g}_{\mu \nu} {\dot z}^{\nu} $ yields
\begin{equation}\label{Pa2_2}
{g}_{\mu \nu} {\dot z}^{\mu} {\dot z}^{\nu} = \Lambda ^2 {g}^{\mu \nu} {u}_{\mu} {u}_{\nu}
\end{equation}
Using Eq. (\ref{Pav1_2}) , Eq. (\ref{Pa2_2}) becomes
\begin{equation}\label{Pa2_3}
\Lambda ^2 = - {g}_{\mu \nu} {\dot z}^{\mu} {\dot z}^{\nu}
\end{equation}
or equivalently
\begin{equation}\label{Pa2_4}
\Lambda = \sqrt{ - {g}_{\mu \nu} {\dot z}^{\mu} {\dot z}^{\nu} }
\end{equation}
where the positive root is picked so that $ \frac{ d }{d \lambda} $ and
$ \frac{ d }{d \tau} $ are both future-pointing.
Combining Eq. (\ref{Pa2_1}) with Eq. (\ref{Pav2_3}) yields
\begin{equation}\label{Pa2_5}
\frac{ d {u}_{\alpha}}{d \tau} + \frac{1}{2} \frac{\partial {g}^{\mu \nu}}{\partial {z ^ \alpha}} {u}_{\mu} {u}_{\nu} = 0
\end{equation}
which is the usual form of the geodesic equation for the
4-velocity.
\section{ADM-Like Hamiltonian Formulation for a Single Point Particle}

\clearpage
\vspace{5mm}
\hrule
\begin{alltt}
  ham_part_action_3p1.dat
  key Pa3
\end{alltt}
\hrule
\vspace{5mm}

While the form of the point-particle action $ I_{pp} $ in Eq. (\ref{Pa1_1}) is
convenient for deriving the particle geodesics, it is not in a form compatible
with the ADM field action.

To convert, the action we use the splitting relations in Eq. (\ref{XXXX)}) In
addition, we take the path parameter $ \lambda $ to be coordinate time
so that $ d \lambda = d {z ^ 0} $. With these identifications, $ I_{pp} $
becomes
\begin{equation}\label{Pa3_1}
I_{pp} = \int d \lambda \left( {u}_{0} + {\dot z}^{i} {u}_{i} - \Lambda \mathcal{H} \right)
\end{equation}
where the Hamiltonian now takes the form
\begin{eqnarray}\label{Pa3_2}
\mathcal{H} & = & \frac{1}{2} \left[ - \frac{ {u}_{0} ^2 }{ \alpha ^2 }
+ \frac{ 2 {u}_{0} {\beta}^{i} {u}_{i} }{ \alpha ^2 }
+ \left( {\gamma}^{i j} - \frac{ {\beta}^{i} {\beta}^{j} }
{ \alpha ^2} \right)
{u}_{i} {u}_{j} + 1 \right] \nonumber \\
& = & \frac{1}{2} \left[ {\gamma}^{i j} {u}_{i} {u}_{j}
-\frac{ \left( {\beta}^{i} {u}_{i} - {u}_{0} \right) ^2}{ \alpha ^2 }
+ 1 \right]
\end{eqnarray}
As before, the individual variations must be taken with respect to
$ {u}_{0} $, $ {u}_{i} $, $ {z ^ i} $, and $ \Lambda $.


\subsection{Variation With Respect to $ \Lambda $}

\clearpage
\vspace{5mm}
\hrule
\begin{alltt}
  var_pp3p1_L.dat
  key Pav4
\end{alltt}
\hrule
\vspace{5mm}

Taking the variation of Eq. (\ref{Pa3_1}) with respect to $ \Lambda $ yields
\begin{equation}\label{Pav4_1}
{ \delta I_{pp} } |_{\delta \Lambda } = \int d \lambda \mathcal{H} \delta \Lambda
\end{equation}
which, when set equal to zero yields
\begin{equation}\label{Pav4_2}
\frac{1}{2} \left[ {\gamma}^{i j} {u}_{i} {u}_{j}
-\frac{ \left( {\beta}^{i} {u}_{i} - {u}_{0} \right) ^2}{ \alpha ^2 }
+ 1 \right] = 0
\end{equation}
Solving Eq. (\ref{Pav4_2}) for $ {u}_{0} $ yields the 3+1 equivalent of
the normalization condition Eq. (\ref{XXXX})
\begin{equation}\label{Pav4_3}
{u}_{0} = - {\beta}^{i} {u}_{i} + \alpha
\sqrt{ 1 + {\gamma}^{i j} {u}_{i} {u}_{j} }
\end{equation}
where the sign on the square-root is chosen so that $ {u}_{0} $ is
future-pointing.

The radical term is analogous to the Lorentz contraction familar from
special relativity. It will often be written as
\begin{equation}\label{Pav4_4}
\sqrt{ 1 + \| {u}_{} \| ^ 2 }
\end{equation}
where the 3-norm of $ {u}_{i} $ with respect to the three-metric
$ {\gamma}_{i j} $ is defined by $ \| {u}_{} \| ^2 = {\gamma}^{i j}
{u}_{i} {u}_{j} $.
\subsection{Variation With Respect to $ u_0 $}

\clearpage
\vspace{5mm}
\hrule
\begin{alltt}
  var_pp3p1_u0.dat
  key Pav5
\end{alltt}
\hrule
\vspace{5mm}

In this section, we take the variation of $ I_{pp} $ with respect to
$ {u}_{0} $ to get
\begin{eqnarray}\label{Pav5_1}
{ \delta I_{pp} } |_{\delta {u}_{0} } & = & \int d \lambda \left( \delta {u}_{0}
- \Lambda { \delta \mathcal{H} } |_{\delta {u}_{0} } \right) \nonumber \\
& = & \int d \lambda \delta {u}_{0} \left( 1 -
\frac{ {\beta}^{i} {u}_{i} - {u}_{0} }
{ \alpha ^2 } \right)
\end{eqnarray}
Setting this variation to zero yields the equation
\begin{equation}\label{Pav5_2}
{\beta}^{i} {u}_{i} - {u}_{0} = \alpha ^2
\end{equation}
\subsection{Variation With Respect to $ u_i $}

\clearpage
\vspace{5mm}
\hrule
\begin{alltt}
  var_pp3p1_ui.dat
  key Pav6
\end{alltt}
\hrule
\vspace{5mm}

In this section, we take the variation of $ I_{pp} $ with respect to
$ {u}_{i} $ to get
\begin{eqnarray}\label{Pav6_1}
{ \delta I_{pp} } |_{\delta {u}_{i} } & = & \int d \lambda \left( \delta {u}_{i} {\dot z}^{i}
- \Lambda { \delta \mathcal{H} } |_{\delta {u}_{i} } \right) \nonumber \\
& = & \int d \lambda \left[ \delta {u}_{i} {\dot z}^{i}
- \frac{\Lambda}{2} \left(
2 {\gamma}^{i j} \delta {u}_{i} {u}_{j}
- 2
\frac{ {\beta}^{k} {u}_{k} - {u}_{0} }
{ \alpha ^2 } {\beta}^{i} \delta {u}_{i} \right) \right] \nonumber \\
& = & \int d \lambda \delta {u}_{i} \left[ {\dot z}^{i} +
\frac{ {\beta}^{k} {u}_{k} - {u}_{0} }{ \alpha ^2 } {\beta}^{i}
- {\gamma}^{i j} {u}_{j} \right]
\end{eqnarray}
Setting the variation equal to zero and using Eq. (\ref{Pav5_2}) yields
\begin{equation}\label{Pav6_2}
{\dot z}^{i} = {\beta}^{i} - {\gamma}^{i j} {u}_{j}
\end{equation}

\subsection{Variation With Respect to $ z^i $}

\clearpage
\vspace{5mm}
\hrule
\begin{alltt}
  var_pp3p1_zi.dat
  key Pav7
\end{alltt}
\hrule
\vspace{5mm}

In this section, we take the variation of $ I_{pp} $ with
respect to $ {z ^ i} $ to get
\begin{equation}\label{Pav7_1}
{ \delta I_{pp} } |_{\delta {z ^ i} } = \int d \lambda \left( {u}_{i} {\dot z}^{i} - \Lambda { \delta \mathcal{H} } |_{\delta {z ^ i} } \right) \nonumber \\
\end{equation}
Integrating the first term by parts and expanding the functional
variation of $ \mathcal{H} $ yields
\begin{equation}\label{Pav7_2}
{ \delta I_{pp} } |_{\delta {z ^ i} } = - \int d \lambda \left( \frac{ d }{d t} {u}_{i}
+ \frac{\Lambda}{2}
\left[ \partial_{i} {\gamma}^{k l} {u}_{i} {u}_{j}
+ \frac{ \left( {\beta}^{k} {u}_{k} - {u}_{0} \right) ^2 }{ \alpha ^3 } { \alpha }_{ , k }
- \frac{ \left( {\beta}^{k} {u}_{k} - {u}_{0} \right) }{ \alpha ^2 }
\partial_{i} {\beta}^{k} {u}_{k} \right] \right)
\end{equation}
\section{ADM Variations}

\subsection{Variation With Respect to $ \alpha $}

%var_lapse_vacuum.dat
%key Avl1

In this section we start with the ADM action and calculate the variation
with respect to the lapse $ \alpha $.
The require variation is
\begin{equation}\label{Avl1_1}
{ \delta I_{ADM} } |_{\delta \alpha } = \int d^3x dt \left( - R^0 \right) \delta \alpha
\end{equation}
Setting the variation equal to zero and susbtituting in
the definition of $ R^0 $ found in Eq. (\ref{XXXX})
results in
\begin{equation}\label{Avl1_2}
\boxed{
\sqrt{\gamma} \left( \frac{ Tr( \pi ^2 ) }{ \gamma }
- \frac{ Tr( \pi ) ^2 }{ 2 \gamma } - R \right) = 0
}
\end{equation}
\subsubsection{Converting from $ \pi ^{ij} $ to $ K_{ij} $ }

%K2Pi_var_lapse.dat
%key AcK1

In this section we convert the the ADM Hamiltonian constraint
equation from the set $ \left[ {\gamma}_{i j} , {\pi}^{i j} \right] $ to
the set $ \left[ {\gamma}_{i j} , {K}_{i j} \right] $.

Using the defining relations Eq. (\ref{XXXX)}) Eq. (\ref{XXXX)}) and Eq. (\ref{XXXX)})
Eq. (\ref{Avl1_2}) becomes

\begin{eqnarray}\label{AcK1_1}
\sqrt{\gamma} \left[ \frac{ \gamma \left( Tr( K ) ^2 + Tr( K ^2 ) \right) }{ \gamma }
- \frac{ 4 \gamma Tr( K ) ^ 2}{2 \gamma } - R \right]
& = & \sqrt{\gamma} \left[ Tr( K ) ^2 + Tr( K ^2 ) - 2 Tr( K ) ^2 - R \right] \nonumber \\
& = & \sqrt{\gamma} \left[ Tr( K ^2 ) - Tr( K ) ^2 - R \right]
\end{eqnarray}
Dividing out $ \sqrt{\gamma} $ we arrive at the usual `ADM' Hamiltonian
constraint equation used in numerical relativity \cite{YYYY}
\begin{equation}\label{AcK1_2}
\boxed{
Tr( K ^2 ) - Tr( K ) ^2 - R = 0
}
\end{equation}

\clearpage
\vspace{5mm}
\hrule
\begin{alltt}
  var_hydro_lapse.dat
  key Ahv1
\end{alltt}
\hrule
\vspace{5mm}

The starting action is
\begin{equation}\label{Ahv1_1}
I = \frac{c0}{16 \pi} \int d^3x dt \left[ \partial_{t} {\gamma}_{i j} {\pi}^{i j} - \alpha R^0 - \beta_{k} R^{k} \right]
+ \int d^3a dt { \tilde \rho }_{0} ( 1 + e ) \left[ {u}_{0} + {\dot z}^{i} {u}_{i} - \Lambda \mathcal{H} \right)
\end{equation}
\clearpage
\vspace{5mm}
\hrule
\begin{alltt}
  on_tr_pi_1.dat
  AK1
  9/21/02
\end{alltt}
\hrule
\vspace{5mm}

Here we look at relating the trace of various combinations of $ {\pi}^{i j} $
to the corresponding terms involving $ {K}^{i j} $. To begin start with the
definition Eq. (\ref{XXXX}) $ {\pi}^{i j} = \sqrt{\gamma} \left( Tr( K ) {\gamma}^{i j} - {K}^{i j} \right) $.
Taking the trace of Eq. (\ref{XXXX}) yields
\begin{equation}\label{AK1_1}
Tr( \pi ) = 2 \sqrt{\gamma} Tr( K ) {\epsilon }^{ c0 }
\end{equation}
Substituting Eq. (\ref{XXXX}) into the term $ {\pi}^{i j} {\pi}_{j k} $ and taking
the trace yields
\begin{equation}\label{AK1_2}
Tr( \pi ^2 ) = \gamma \left[ Tr( K ^2 ) + Tr( K ) ^2 \right]
\end{equation}
\clearpage
\vspace{5mm}
\hrule
\begin{alltt}
  on_tr_pi_2.dat
  AK2
  9/21/02
\end{alltt}
\hrule
\vspace{5mm}

With the previous relations in hand, we can convert the second and
third terms in the right-hand-side of the evolution equation for $ {\pi}^{i j} $.

Starting with the second term and substituting in relations
Eq. (\ref{XXXX}) and Eq. (\ref{XXXX}) obtains the formula

\begin{eqnarray}\label{AK2_1}
{\pi}^{i j} Tr( \pi ) - 2 {\pi^i}_{k} {\pi}^{k j}
& = &
\sqrt{\gamma} \left[ Tr( K ) {\gamma}^{i j} - {K}^{i j} \right] 2 \sqrt{\gamma} Tr( K ) \nonumber \\
&   &
- 2 \gamma \left[ Tr( K ) {\gamma ^i }_{ k } - {K^i}_{k} \right]
\left[ Tr( K ) {\gamma}^{k j} - {K}^{k j} \right] \nonumber \\
& = &
2 \gamma \left[ Tr( K ) {K}^{i j} - {K^i}_{k} {K}^{k j} \right]
\end{eqnarray}

Likewise, the third term becomes
\begin{eqnarray}\label{AK2_2}
Tr( \pi ^2 ) - \frac{1}{2} Tr( \pi ) ^2 & = & \gamma \left[ Tr( K ^2 ) + Tr( K ) ^2 \right]
- \frac{1}{2} \left[ 2 \sqrt{\gamma} Tr( K ) \right] ^2 \nonumber \\
& = & \gamma \left[ Tr( K ^2 ) - Tr( K ) ^2 \right]
\end{eqnarray}
\clearpage
\vspace{5mm}
\hrule
\begin{alltt}
  on_tr_pi_3.dat
  AK3
  9/21/02
\end{alltt}
\hrule
\vspace{5mm}

In this note, we convert the terms involving spatial derivatives
of $ {\pi}^{i j} $ to the corresponding terms involving $ {K}^{i j} $.
Again employ Eq. (\ref{XXXX}) to relate the conjugate momentum to the
extrinsic curvature. Some simplification results by noting that
the definition of the 3-dimensional covariant derivative is compatible
with the 3-metric, resulting in $ D^{i} {\gamma}_{k l} = 0 $
and related expressions.

\begin{eqnarray}\label{AK3_1}
\frac{1}{ \sqrt{\gamma} } \left[ \pounds_{{\vec \beta}} {\pi}^{i j} - {\beta}^{i} D_{k} {\pi}^{j k}
- {\beta}^{j} D_{k} {\pi}^{i k} \right]
& = &
\frac{1}{ \sqrt{\gamma} } \pounds_{{\vec \beta}} \sqrt{\gamma} \left( Tr( K ) {\gamma}^{i j} - {K}^{i j} \right)
{\gamma}^{i j} \nonumber \\
&   &
+ \pounds_{{\vec \beta}} Tr( K ) + Tr( K ) \pounds_{{\vec \beta}} {\gamma}^{i j} \nonumber \\
&   &
- \pounds_{{\vec \beta}} {K}^{i j}
- {\gamma}^{i j} {\beta}^{i} D_{k} Tr( K ) \nonumber \\
&   &
- {\beta}^{i} D_{k} {K}^{i j}
\end{eqnarray}
\clearpage
\vspace{5mm}
\hrule
\begin{alltt}
  on_tr_dt_pi.dat
  key TP1
  10/6/02
\end{alltt}
\hrule
\vspace{5mm}

The aim of this note is to calculate the term $ {\gamma}_{i j} \partial_{t} {\pi}^{i j} $.
substituting in the definition Eq. (\ref{XXXX}) and expanding yields

\begin{eqnarray}\label{TP1_1}
{\gamma}_{i j} \partial_{t} {\pi}^{i j}
& = & \partial_{t} Tr( \pi ) - {\pi}^{i j} \partial_{t} {\gamma}_{i j} \nonumber \\
& = & \partial_{t} Tr( \pi ) - {\pi}^{i j} \left[ -2 \alpha {K}_{i j}
+ D_{i} {\beta}_{j} + D_{j} {\beta}_{i} \right] \nonumber \\
& = & \partial_{t} Tr( \pi ) + 2 \alpha {\pi}^{i j}
\left( \frac{ Tr( \pi ) }{ 2 \sqrt{\gamma} } {\gamma}_{i j}
- \frac{ {\pi}^{i j} }{ \sqrt{\gamma} } \right)
- 2 {\pi}^{i j} D_{i} {\beta}_{j} \nonumber \\
& = & \partial_{t} Tr( \pi ) + \frac{ \alpha }{ \sqrt{\gamma} }
\left[ Tr( \pi ^2 ) - \frac{1}{2} Tr( \pi ^2 ) \right]
- 2 {\pi}^{i j} D_{i} {\beta}_{j}
\end{eqnarray}




\clearpage
\vspace{5mm}
\hrule
\begin{alltt}
  on_tr_dt_3k.dat
  key TK1
  10/6/02
\end{alltt}
\hrule
\vspace{5mm}

The aim of this note is to calculate the term $ {\gamma}_{i j} \partial_{t} {K}^{i j} $.
Substituting in the definition Eq. (\ref{XXXX}) and simplifying yields
\begin{eqnarray}\label{TK1_1}
{\gamma}_{i j} \partial_{t} {K}^{i j}
& = &
\partial_{t} Tr( K ) - {K}^{i j} \partial_{t} {\gamma}_{i j} \nonumber \\
& = &
\partial_{t} Tr( K ) - {K}^{i j} \left( -2 \alpha {K}_{i j}
+ D_{i} {\beta}_{j} + D_{j} {\beta}_{i} \right) \nonumber \\
& = &
\partial_{t} Tr( K ) + 2 \alpha Tr( K ^2 ) - 2 {K}^{i j} D_{i} {\beta}_{j}
\end{eqnarray}


\clearpage
\vspace{5mm}
\hrule
\begin{alltt}
  on_dx_pi_1.dat
  DxP1
  10/6/02
\end{alltt}
\hrule
\vspace{5mm}

Since the terms in question involve the spatial derivatives of $ {\pi}^{i j} $
I'll derive the conversion to the corresponding terms $ {K}^{i j} $ here
and will refer to them afterwards.

First consider the term involving the Lie derivative of $ {\pi}^{i j} $
\begin{eqnarray}\label{DxP1_1}
\pounds_{{\vec \beta}} {\pi}^{i }
& = &
{\beta}^{k} D_{k} {\pi}^{i j} - {\pi}^{k j} D_{k} {\beta}^{i}
- {\pi}^{i k} D_{k} {\beta}^{j} + {\pi}^{i j} D_{k} {\beta}^{k} \nonumber \\
& = &
{\beta}^{k} D_{k} \left[ \sqrt{\gamma} \left( Tr( K ) {\gamma}^{i j} - {K}^{i j} \right) \right]
- \left[ \sqrt{\gamma} \left( Tr( K ) {\gamma}^{k j} - {K}^{k j} \right) \right] D_{k} {\beta}^{i} \nonumber \\
&   &
\left[ \sqrt{\gamma} \left( Tr( K ) {\gamma}^{i k} - {K}^{i k} \right) \right] D_{k} {\beta}^{j}
+ \left[ \sqrt{\gamma} \left( Tr( K ) {\gamma}^{i j} - {K}^{i j} \right) \right] D_{k} {\beta}^{k} \nonumber \\
& = &
\sqrt{\gamma} {\gamma}^{i j} \left( {\beta}^{k} D_{k} Tr( K ) + Tr( K ) D_{k} {\beta}^{k} \right)
- \sqrt{\gamma} \pounds_{{\vec \beta}} {K}^{i j} \nonumber \\
&   &
- \sqrt{\gamma} {K}^{i j} D_{k} {\beta}^{k}
- \sqrt{\gamma} Tr( K ) \left( D^{i} {\beta}^{j} + D^{j} {\beta}^{i} \right)
\end{eqnarray}

The terms involving the other spatial derivatives become
\begin{eqnarray}\label{DxP1_2}
{\beta}^{i} D_{k} {\pi}^{j k} + {\beta}^{j} D_{k} {\pi}^{i k}
& = &
{\beta}^{i} D_{k} \left[ \sqrt{\gamma} \left( Tr( K ) {\gamma}^{j k} - {K}^{j k} \right) \right] \nonumber \\
&   &
+ {\beta}^{j} D_{k} \left[ \sqrt{\gamma} \left( Tr( K ) {\gamma}^{i k} - {K}^{i k} \right) \right] \nonumber \\
& = &
\sqrt{\gamma} \left( {\beta}^{i} D^{j} Tr( K ) + {\beta}^{j} D^{j} Tr( K ) \right. \nonumber \\
&   &
- \left. {\beta}^{i} D_{k} {K}^{j k} - {\beta}^{j} D_{k} {K}^{i k} \right)
\end{eqnarray}
\clearpage
\vspace{5mm}
\hrule
\begin{alltt}
  on_TrL_P.dat
  key TrLP
  10/6/02
\end{alltt}
\hrule
\vspace{5mm}

In this note, we calculate the trace of the term $ \pounds_{{\vec \beta}} {K}^{i j} $.
Taking the trace
\begin{eqnarray}\label{TrLP_1}
{\gamma}_{i j} \pounds_{{\vec \beta}} {\pi}^{i j}
& = &
{\gamma}_{i j} \left( {\beta}^{k} D_{k} {\pi}^{i j} - {\pi}^{i k} D_{k} {\beta}^{j}
- {\pi}^{k j} D_{k} {\beta}^{i} + {\pi}^{i j} D_{k} {\vec \beta} \right) \nonumber \\
& = &
{\beta}^{k} D_{k} Tr( \pi ) - 2 {\pi}^{i k} D_{k} {\beta}_{i}
+ Tr( \pi ) D_{k} {\beta}^{k}
\end{eqnarray}

\clearpage
\vspace{5mm}
\hrule
\begin{alltt}
  on_TrL_K.dat
  key TrLK
  10/6/02
\end{alltt}
\hrule
\vspace{5mm}

The aim of this note is to calculate the trace of $ \pounds_{{\vec \beta}} {K}^{i j} $.
Using the standard definitions the relation becomes
\begin{eqnarray}\label{TrLK_1}
{\gamma}_{i j} \pounds_{{\vec \beta}} {K}^{i j}
& = &
{\gamma}_{i j} \left( {\beta}^{k} D_{k} {K}^{i j}
- {K}^{i k} D_{k} {\beta}^{j}
- {K}^{k j} D_{k} {\beta}^{i} \right) \nonumber \\
& = &
{\beta}^{k} D_{k} Tr( K ) - 2 {K}^{i k} D_{k} {\beta}_{i}
\end{eqnarray}
\clearpage
\vspace{5mm}
\hrule
\begin{alltt}
  on_3p1_IF.dat
  key 3P1IF
  8/29/02 & 9/15/02
\end{alltt}
\hrule
\vspace{5mm}

In this note, we want todecompose the ideal fluid stress-energy tensor
according to the 3+1 slicing.
Start with the ideal fluid stress-energy tensor given by:
\begin{equation}\label{3P1IF_1}
{ T } _ { ab } = \rho h {u}_{a} {u}_{b} + P {g}_{a b}
\end{equation}
where $ \rho $ is the baryon density, $ h = \left( 1 + e + P / \rho \right) $ is the
specific enthalpy, and $ P $ is the pressure.
Following Yorke [\cite{YYYY)}] we define the following projections
of $ { T } _ { ab } $:
\begin{eqnarray}\label{3P1IF_2}
{\tilde \rho} & = & { T } _ { {\hat n} } \nonumber \\
& = & {n ^ a} { a } _ { b } {n ^ b} \nonumber \\
& = & \rho h \left( {n ^ a} {u}_{a} \right) ^2 - P \nonumber \\
& = & \rho ( 1 + e ) \left( 1 + || u || ^2 \right) + P || u || ^2 \quad ,
\end{eqnarray}
\begin{eqnarray}\label{3P1IF_3}
{ j } _ { a } & = & - \perp T_{a {\hat n} } \nonumber \\
& = & - {\perp _ a}^{ b } { T } _ { bc } {n ^ c} \nonumber \\
& = & \rho h {\perp _ a}^{ b } {u}_{b} {n ^ c} {u}_{c} \quad ,
\end{eqnarray}
and
\begin{eqnarray}\label{3P1IF_4}
{ S } _ { ab } & = & \perp { T } _ { ab } \nonumber \\
& = & {\perp _ a}^{ c } { T } _ { cd } {\perp ^ d}_{ b } \nonumber \\
& = & \rho h {\perp _ a}^{ c } {u}_{c} {\perp _ b}^{ d } {u}_{d} + P {\gamma}_{a b} {\epsilon }^{ c0 }
\end{eqnarray}

It is trivial to show that, given the above projections, the following
identity holds
\begin{eqnarray}\label{3P1IF_5}
{ T } _ { ab } & = & {\delta ^ a}_{c} { T } _ { cd } {\delta _ d}^{b} \nonumber \\
& = & \left( {\perp ^ a}_{ c } - {n _ a} {n ^ c} \right)
{ T } _ { cd }
\left( {\perp _ d}^{ b } - {n ^ d} {n _ b} \right) \nonumber \\
& = & \perp { T } _ { ab } + { j } _ { a } {n _ b} + { j } _ { b } {n _ a}
{ S } _ { ab }
\end{eqnarray}
Note that since $ { S } _ { ab } $ is a spatial tensor, its indices can
be raised or lowered using either $ {\perp } _ { a b } $ or $ {g}_{a b} $. We will
exploit this freedom as needed to increase the clarity. Also note
that $ {n ^ a} {u}_{a} = \sqrt{ 1 + || u || ^2 } $.
The source terms that obtain in the ADM equations are cited by Yorke to be
\begin{equation}\label{3P1IF_6}
Tr( S ) = { S } _ { ab } {g}^{a b} = \rho h || u || ^2 + 3 P
\end{equation}
and
\begin{eqnarray}\label{3P1IF_7}
{ M } ^ { ab } & = & -8 \pi { S } ^ { ab } + 4 \pi {\perp } ^ { a b } \left[ Tr(S) - {\tilde \rho} \right] \nonumber \\
& = & -8 \pi \rho h { u } ^ { a } { u } ^ { b }
+ 4 \pi {\perp ^ a}_{ b } \left[ 3 P + 2 P || u || ^2 - \rho ( 1 + e ) \right]
\end{eqnarray}
\clearpage
\vspace{5mm}
\hrule
\begin{alltt}
  on_var_rho_3g.dat
  key VR3G
  10/21/02
\end{alltt}
\hrule
\vspace{5mm}

The desire her is to compute $ \frac{\partial \rho}{\partial {\gamma}_{i j}} $.
The density $ \rho $ has
a dependence on $ {\gamma}_{i j} $ throught two terms. The first is through
$ \sqrt{\gamma} $ and the second is through $ || \epsilon || ^2 $.

The variation of $ \sqrt{\gamma} $ easily obtained from the formulae Eq. (\ref{XXXX})
found in the appendix Eq. (\ref{????)}) Instead, focus on the terms involved
in $ || \epsilon || ^ 2$
\begin{eqnarray}\label{VR3G_1}
\frac{\partial }{\partial {\gamma}_{i j}} || \epsilon || ^2 & = & \frac{\partial }{\partial {\gamma}_{i j}} {\epsilon }^{ k } {\epsilon }^{ l } {\gamma}_{k l} \nonumber \\
& = & \frac{\partial {\epsilon }^{ k }}{\partial {\gamma}_{i }} {\epsilon }^{ l } {\gamma}_{k l}
+ {\epsilon }^{ k } \frac{\partial {\epsilon }^{ l }}{\partial {\gamma}_{i j}} {\gamma}_{k l}
+ {\epsilon }^{ i } {\epsilon }^{ j } \nonumber \\
& = & 2 \frac{\partial {\epsilon }^{ k }}{\partial {\gamma}_{i j}} {\epsilon }^{ l } {\gamma}_{k l}
\end{eqnarray}
Now expand the term $ \frac{\partial {\epsilon }^{ k }}{\partial {\gamma}_{i j}} $
\begin{eqnarray}\label{VR3G_2}
\frac{\partial {\epsilon }^{ k }}{\partial {\gamma}_{i j}} & = & \frac{\partial }{\partial {\gamma}_{i j}} \left( {\dot z}^{k} + {\beta}^{k} \right) \nonumber \\
& = & \frac{\partial }{\partial {\gamma}_{i j}} {\beta}^{k} \nonumber \\
& = & \frac{\partial }{\partial {\gamma}_{i j}} {\gamma}^{k m} {\beta}_{m} \nonumber \\
& = & - {\beta}^{(i} {\gamma}^{j) k}
\end{eqnarray}
Substituting back into Eq. (\ref{VR3G_1}) yields
\begin{equation}\label{VR3G_3}
\frac{\partial }{\partial {\gamma}_{i j}} || \epsilon || ^2 = {\epsilon }^{ i } {\epsilon }^{ j }
- 2 {\epsilon }^{ (i } {\beta}^{j)}
\end{equation}
With this result in hand, we can move onto the desired computation by
employing the conservation equation Eq. (\ref{XXXX})
\begin{equation}\label{VR3G_4}
\frac{\partial }{\partial {\gamma}_{i j}} \left[ \frac{ \rho \sqrt{\gamma} |J| }{ \sqrt{1 - || \epsilon || ^2 / \alpha ^2 } } \right]
= \frac{\partial }{\partial {\gamma}_{i j}} { \tilde \rho }_{0} = 0
\end{equation}
yields
\begin{equation}\label{VR3G_5}
\frac{\partial \rho}{\partial {\gamma}_{i j}} = - \frac{\rho}{2} \left[ {\gamma}^{i j}
+ {u}^{i} {u}^{j}
- \frac{ 2 {u}^{(i} {\beta}^{j)} }{ \Lambda } \right]
\end{equation}

\clearpage
\vspace{5mm}
\hrule
\begin{alltt}
  on_var_Ham_3g.dat
  key VH3G
  10/21/02 and 10/22/02
\end{alltt}
\hrule
\vspace{5mm}

In this computation, we explore the variation of the
Hamiltonian with respect to the three-metric. Since
the Hamiltonian depends only on the three-metric and
not on its derivatives, the variation ammounts to
taking a partial derivative.
\begin{eqnarray}\label{VH3G_1}
\frac{\partial }{\partial {\gamma}_{i j}} \mathcal{H} & = & \frac{\partial }{\partial {\gamma}_{i j}} \frac{1}{2}
\left[ {\gamma}^{k l} {u}_{k} {u}_{l} -
\frac{ \left( {\beta}^{k} {u}_{k} - {u}_{0} \right) }
{ \alpha ^ 2 }
+ 1
\right] \nonumber \\
& = &
- \frac{1}{2} {u}^{i} {u}^{j}
- \frac{ \left( {\beta}^{k} {u}_{k} - {u}_{0} \right) }
{ \alpha ^2 }
\frac{\partial }{\partial {\gamma}_{i j}} {\gamma}^{k l} {\beta}_{l} {u}_{k} \nonumber \\
& = &
- \frac{1}{2} {u}^{i} {u}^{j}
+ \frac{ \left( {\beta}^{k} {u}_{k} - {u}_{0} \right) }
{ \alpha ^2 }
{\beta}^{(i} {u}^{j)}
\end{eqnarray}
Using the standard relation $ \left( {\beta}^{k} {u}_{k} - {u}_{0} \right)
= \frac{ \alpha ^2 }{ \Lambda } $ the above relation becomes
\begin{equation}\label{VH3G_2}
\frac{\partial }{\partial {\gamma}_{i j}} \mathcal{H} = \frac{ {u}^{(i} {\beta}^{j)} }{ \Lambda }
- \frac{1}{2} {u}^{i} {u}^{j}
\end{equation}



\clearpage
\vspace{5mm}
\hrule
\begin{alltt}
  slicing_intro.dat
  key SL
  old write-up
\end{alltt}
\hrule
\vspace{5mm}

The initial-value formulation of Arnowitt, Deser, and Misner (ADM) depends on
being able to break spacetime up into a foliation of constant-$t$ spacelike
hypersurfaces $\Sigma_t$. In this section, the definition of this foliation
is presented.
\clearpage
\vspace{5mm}
\hrule
\begin{alltt}
  var_ADM_3g.dat
  key VA3G
  based on the notes 'A quality check on the 3-metric variation' 10/26/02
\end{alltt}
\hrule
\vspace{5mm}

Start with the ADM action with no matter
\begin{equation}\label{VA3G_1}
I_{ADM} = \frac{1}{16 \pi} \int d^3x dt \left[ {\pi}^{i j} \partial_{t} {\gamma}_{i j} - \alpha R^0 - \beta_{k} R^{k} \right]
\end{equation}

Take the variation

\begin{eqnarray}\label{VA3G_2}
{ \delta I_{ADM} } |_{\delta {\gamma}_{i j} } & = & \frac{1}{16 \pi} \int d^3x dt \left[
{\pi}^{i j} \partial_{t} \delta {\gamma}_{i j}
- \frac{\delta \alpha R^0}{\delta {\gamma}_{i j}} \delta {\gamma}_{i j}
- \frac{\delta \beta_{k} R^{k}}{\delta {\gamma}_{i j}} \delta {\gamma}_{i j} \right]
\nonumber \\
& = & \frac{-1}{16 \pi} \int d^3x dt \left[
\partial_{t} {\pi}^{i j}
- \frac{\delta \alpha R^0}{\delta {\gamma}_{i j}}
- \frac{\delta \beta_{k} R^{k}}{\delta {\gamma}_{i j}} \right] \delta {\gamma}_{i j}
\end{eqnarray}
where the last line results from an integration of the first term by parts
and the throwing away of any boundary terms.

Substituting in the relations determined earlier yields
\begin{eqnarray}\label{VA3G_3}
\partial_{t} {\pi}^{i j} & = & - \sqrt{\gamma} { A } ^ { ij }
- \sqrt{\gamma} B {\gamma}^{i j}
+ \pounds_{{\pi}^{i j}} {\vec \beta}
- 2 {\beta}^{(i} { {\pi}^{j) k} } _ { |k } \nonumber \\
& = & - \alpha \sqrt{\gamma} \left( R^{i j} - \frac{1}{2} R \right)
- \frac{ \alpha }{ \sqrt{\gamma} } \left( 2 {\pi^i}_{m} {\pi}^{m j}
- {\pi}^{i j} Tr( \pi ) \right) \nonumber \\
&   & + \sqrt{\gamma} \left( D^{i} D^{j} \alpha
- {\gamma}^{i j} D^{k} D_{k} \alpha \right) \nonumber \\
&   & - \frac{ \alpha }{ 2 \sqrt{\gamma} } {\gamma}^{i j}
\left( \frac{Tr( \pi ^2 )}{2} - Tr( \pi ) ^2 \right) \nonumber \\
&   & + \pounds_{{\vec \beta}} {\pi}^{i j}
- 2 {\beta}^{(i} { {\pi}^{j) k} } _ { |k }
\end{eqnarray}


\end{document}