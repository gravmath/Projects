% ****** Start of file fat_particle_2.tex ******
%
%   This file is the first draft of the paper detailing our approach
%   to the full-backreacting two fat particle system
% 
%
%
%   updates:	 12/17/99 first version of the paper - just the 
%                skeleton - CS
%                

%**************************************************************
%
%  Preliminary matters
%
%**************************************************************
% \documentstyle[preprint,eqsecnum,aps]{revtex}
\documentstyle[eqsecnum,aps,epsfig]{revtex}
\bibliographystyle{prsty}
\newcounter{bean}
\def\.{{\quad .}}
\def\_.{{\quad .}}
\def\_,{{\quad ,}}

%**************************************************************
%
%  Begin the document
%
%**************************************************************
\begin{document}
\draft


%\preprint{GR-QC/????}

%**************************************************************
%
%  Front matter
%
%**************************************************************
\title{A Variational Principle for Two Fat Particles}
%
\author{C. Schiff}
\address{
            Department of Physics, University of Maryland,
            College Park MD 20742-4111 USA\\
           \rm
         e-mail: \tt  cmschiff@erols.com\\
					  Revision 1.0
}
\date{12/17/99}
\maketitle


%**************************************************************
%
%  Abstract
%
%**************************************************************
\begin{abstract}
\end{abstract}

\section{Recasted Geodesics}



\section{Discrete ADM}

\begin{equation}
  I =   \int \, d^4 \, x \frac{R(x)\sqrt{-g(x)}}{16 \pi} 
     +  \sum _{A} m_A \int \left[ {U_A}_{\mu} d {z_A}^{\mu} 
	 - \Lambda _{A} {\mathcal H}_A d \lambda _A \right]
\end{equation}

where
\[
{\mathcal H}_A = \frac{1}{2} \left( {g_A}^{\mu\nu} {U_A}_{\mu} {U_A}_{\nu} + 1 \right)
\]

\begin{equation}
I = I_{HE} + \sum _{A} m_A \int \, d^4 \, x 
             \left\{
			   \int 
			 \right\}
\end{equation}


%\begin{equation}
%\end{equation}

\end{document}
