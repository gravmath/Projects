%
%
%	This is a file to make C W Misner's letter to
%	   Kim New
%
%   re Gowdy analytic test case
%
%	It is designed to be processed by LaTeX and to be emailed 31 Aug'97 
%
%
\documentstyle{letter}
\signature{Charlie
}

\def\half{\mbox{$\frac{1}{2}$}}     % small built-up `one-half'
\def\ref#1{}

%\addtolength{\textheight}{1.0in}
%\addtolength{\topmargin}{-0.5in}

\begin{document}

\begin{letter}{%
Dr.\ Kimberly New\\
Drexel U
}


\opening{Dear Kim}

    I hope you find that setting the lapse makes Gowdy run, but since
I revved up Maple to check the initial conditions, I've gone on to
look at the analytic solutions and will suggest some test cases here.
    The principal change is that I've now put an amplitude factor $A$
on the waves and have run some numerical examples so see what the
solution actually looks like.  
    In testing the ADM code on Gowdy models, you should first run the
case $A=0$ to see that Kasner can be reproduced in this form, before
facing the problems of real Gowdy with waves.

    An interesting case is the lowest mode $n=1$ with a modest
amplitude $A = 0.2$.  
    I'll include with this note some plots of metric values. 
    The the wave energy is initially ($t=1$)  so low that the
solution behaves essentially as Kasner for quite a while.  
    But in this Kasner the 3-direction is contracting while the
volume expands, so the gravitational waves propagating in the
3-direction are getting blue shifted and eventually (near $t=12$ in
the example) the wave energy density makes the Kasner empty space
metric no longer a valid approximation.  
    The collapse of the 3-direction then reverses, joining the
expansion of the other directions, and a more conventional expanding
universe ensues which asymptotically is probably the Kasner case
which is Minkowski spacetime.
    Keith has nearly completed a homogeneous analytic solution that
models this behavior.

    Thus even this analytically solvable case has a lot of physics in
it, and it looks rather hard to reproduce numerically.
    But I think it is not harder than the binary system we aim to do,
so we had better find a way to get it right, as perhaps Joan has
already done when using 1D code some time back.
    The physics is the propagation of gravitational waves in a
dynamic background, plus the back reaction of the waves on the
background which here reverses the Kasner collapse in one direction.
    The numerical difficulty is that when described in terms of the
metric components $g_{ij}$ many of the effects look small but have
big consequences.
    The wave oscillations are only a few percent in the metric
components, yet after a dozen cycles they are sufficient to cause the
collapsing background to bounce.
    During the bounce different parts of the space are showing only a
few parts in a thousand differences in their behavior, yet without
these the bounce might not occur.
    But such things will happen in the binary system also.
    Radiation reaction that may take many cycles to have a major
effect should eventually lead to coalescence.
    This is asking a lot from a second order scheme---let's hope it
can be made to work.

     The plots from the $A = 0.2$, $n=1$ case show that a measure of
wave tidal effects $g_{11}/g_{22}$ has an amplitude of about 10\%
initially ($t=1$) descreasing to about 2\% at $t=20$.
    This occurs at the antinode $z=0$; there is no change in this
quantity at the node $z=1/4$.
    The expansion in the $xy$-plane is controlled by the factor of
$t$ in each $g_{11}$ and $g_{22}$, thus these directions have lengths
expanding like $\sqrt{t}$.
    In the $z$-direction lengths initially contract with a factor
$t^{-1/4}$ and then near $t=12$ this contraction stops and an
expansion in this direction begins as the $t^2$ terms in $\lambda$
from equation~(5\ref{eq-L}) below begin to dominate.
    From the plot of $g_{33}$ in the critical bounce range near
$t=12$ one sees that this behavior is essentially the same at both
the nodes and the antinodes of the waves.
    Superimposed on the collapse and bounce behavior of $g_{33}$
there is a small 0.6\% oscillation which has opposite phase at the
wave nodes and antinodes, but is otherwise similar.

    When there are no gravitational waves in our Gowdy test cases
($A=0$) the solution becomes a Kasner case in nonstandard
coordinates:
    \begin{equation}\label{eq-Kasner}
        ds^2 = t^{-1/2} (- dt^2 + dz^2) + t\,(dx^2 + dy^2)
        \quad .
    \end{equation}
    In general, for polarized analytic Gowdy solutions, we take the
metric to be
        \begin{equation}\label{eq-gdyt}
    ds^2 = t^{-1/2} e^{\lambda/2}(- dt^2 + dz^2)
           +t\, ( e^P\,dx^2 + e^{-P}\,dy^2 )
    \end{equation}
    with 
    \begin{equation} P(t,z) \equiv A\, 
    {\rm BesselJ}(0, 2 \pi n t)\cos(2 \pi n z)
    \quad .
    \end{equation}
    For brevity we shall write this as
    \begin{equation}\label{eq-P}
    P \equiv A\,J_0 \cos(2\pi n z)
    \end{equation}
    with the understanding that $J_0$ always means ${\rm BesselJ}(0,
2 \pi n t)$ and that $J_1$ likewise means ${\rm BesselJ}(1, 2 \pi n
t)$.  
    It is here, in constrast to my previous message, that the
amplitude parameter $A$ is introduced.
    From this definition it follows from the constraint equations
that
    \begin{equation}\label{eq-L}
    \lambda(t,z) \equiv  - 2 \pi n t A^2 J_0 J_1 \cos^2 (2 \pi n z)
    + 2 \pi^2 n^2 t^2 A^2 
    \left( {J_0}^{2} + {J_1}^2 \right)
    \quad .
    \end{equation}
    This has an asymptotic form for large $t$ of
    \begin{equation}\label{eq-Linf}
    \lambda(t,z) \approx  + 2 A^2 n t 
                          + (A^2/\pi) \cos(4\pi n t) \cos^2(2\pi n z)
    \quad .
    \end{equation}
    Then for initial conditions you need the lapse
    \begin{equation}\label{eq-N}
    N \equiv \alpha \equiv t^{-1/4}\exp{(\lambda/4)}
    \end{equation}
    and the metric components
    \begin{eqnarray}
    g_{11} & = & t \exp{(+P)}\\
    g_{22} & = & t \exp{(-P)}\\
    g_{33} & = & - g_{00} = N^2
    \end{eqnarray}
    as well as the second fundamental form components
    \begin{eqnarray}
    K_{11} & = & N^{-1}\exp(+P)  (-\half + \pi n t A J_1 
                    \cos(2 \pi n z) )\\
    K_{22} & = & N^{-1}\exp(-P) (-\half - \pi n t A J_1 
                    \cos(2 \pi n z) )\\
    K_{33} & = & N\,\left(\frac{1}{4 t} -
               t (\pi n A)^2 
                      \left[ {J_1}^2\cos^2(2 \pi n z)
                           + {J_0}^2\sin^2(2 \pi n z) \right] \right)
                      \quad .
    \end{eqnarray}

    In addition to the gzipped postscipt plots, I'm also attaching a
Maple LaTeX listing of my calculations in case you want any more
details, and a Maple session file in case you use Maple and want to
produce plots of some other variables or other cases.

    \closing{%
Cheers,
    }

\end{letter}

\end{document}


