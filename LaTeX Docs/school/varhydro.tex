% ****** Start of file VarHydro.tex ******
%
%   This file is the first draft of the note to Conrad about
%   a continuum variational principle modified from my
%   Sakharov paper in constrained Hamiltonian form
%
%   first draft:     v0.1  28.11.00 first draft - CWM
%   updates:         v0.2  30.11.00 completed T_{\mu\nu} - CS
%                    v0.3  03.01.01 completed z^{\mu}    - CS
%                    v0.4  05.01.02 added references     - CS
%
%**************************************************************
%
%  Preliminary matters
%
%**************************************************************
%\documentstyle[preprint,eqsecnum,aps]{revtex}
%\documentstyle[eqsecnum,aps,epsfig]{revtex}
%\bibliographystyle{prsty}
\bibliographystyle{alpha}
\documentstyle{article}
\newcounter{bean}
\def\half{\mbox{$\frac{1}{2}$}}
\def\.{{\quad .}}
\def\_.{{\quad .}}
\def\_,{{\quad ,}}
\def\be{\begin{equation}}
\def\ee{\end{equation}}
\def\bea{\begin{eqnarray}}
\def\eea{\end{eqnarray}}
\def\po{{\hat \rho}_{0}}


%**************************************************************
%
%  Begin the document
%
%**************************************************************
\begin{document}
%\draft

%\twocolumn[\hsize\textwidth\columnwidth\hsize\csname
%@twocolumnfalse\endcsname
%\preprint{GR-QC/????}

%**************************************************************
%
%  Front matter
%
%**************************************************************
\title{Variational Principle for Relativistic Hydrodynamics}
%
\author{C. Schiff and C. W. Misner\\
        Department of Physics, University of Maryland,\\
         College Park MD 20742-4111 USA\\
         and\\
         Max Planck --- Albert Einstein Institute,\\
         D-14476 Golm, Germany\\
           \rm
         e-mail: \tt  cmschiff@erols.com\\
                 \tt  misner@physics.umd.edu\\
                      Revision 0.4
}
\date{03 January, 2001}
\maketitle


%**************************************************************
%
%  Abstract
%
%**************************************************************
%\widetext
%\begin{abstract}
%\end{abstract}
%\pacs{0425.-g,04.25.Dm,04.30.-w,04.30.Db,04.30.Nk}

%\narrowtext
%\vskip2pc]



\section{Introduction}\label{sec:intro_level1}

This note proposes a variational principle for relativistic
hydrodynamics (for an ideal fluid) which is a variant of that
given in \cite{A:CWM96}.  The difference is that the present
variational principle is based on one for geodesics in the form
\be
     I_p = \int \left( \frac{d z^{\mu}}{d \lambda} u_{\mu} -
                  \Lambda {\mathcal H} \right) d \lambda
\ee
where
\be
 {\mathcal H} = \frac{1}{2} \left[ g^{\mu\nu}(z) u_{\mu} u_{\nu}
                                    + 1 \right]
\ee
while the previous work \cite{A:CWM96} started from
\be
   I_p =  -m \int\!\! \sqrt{ -g_{\mu\nu}(z) \dot{z}^\mu \dot{z}^\nu}
\,d\lambda
\ee
in which $\dot{z}^\mu = dz^\mu/d\lambda$.
Here $\lambda$ is an arbitrary path parameter which is
unrestricted by these simple variational principles.


%******************************************************************
%******************************************************************
%    Lagrangian Hydrodynamics
%******************************************************************
%******************************************************************
\section{Lagrangian Hydrodynamics}\label{sec:1_part_level1}

The aim of the variational principle is to determine a coordinate
transformation
\be
   x^\mu = z^\mu(a^{\bar{\nu}})
\ee
where the $x^\mu$ coordinate system is arbitrarily chosen as one
in which the metric components $g^{\mu\nu}(x)$ can be provided.
The $a^{\bar{\mu}}$ coordinates are comoving with the fluid. That
is, $a^{\bar{k}}$ for $k=1,2,3$ are labels that identify
particular fluid particles.
We expect them to be chosen so that $z^k(0,a^k) = a^k$.
The path parameter $a^{\bar{0}} \equiv \lambda$ along the world
lines $a^{\bar{k}} = \mbox{const}$ is left arbitrary in this
variation principle.
In anticipated applications it would be chosen to be coordinate
time $x^0$, but it could also be set to proper time along the
fluid world lines.

The properties of the ideal fluid are given apriori by a
fundamental thermodynamical function $e(\rho,s)$ specifying the
(specific) internal energy per particle (or per unit rest mass,
or per mole of baryons) as a function of the rest mass density
$\rho$ and the specific entropy $s$.
From this and the relationship
\be\label{eq:first_law}
     de = T\,ds - P\,dv = T\,ds + (P/\rho^2)d\rho
\ee
one obtains the temperature and pressure.
For an ideal fluid the specific entropy $s$ will be constant
along each fluid world line so that $s(a^{\bar{\mu}},a^{\bar{k}})
= s(0,a^{\bar{k}}) = s_0(a^{\bar{k}})$ is independent of
$a^{\bar{0}}$.
We therefore omit any explicit mention of $s$ in the variational
principle.

The rest mass density $\rho$ can be specified arbitrarily at some
initial time; thereafter it is fixed by a conservation law.
This will cause its distribution to vary as the fluid flow
description $z^\mu(a^{\bar{\mu}})$ and/or the metric $g^{\mu\nu}$
are varied.
(Unlike the earlier paper\cite{A:CWM96}, we here do not use
Lagrange multipliers to enforce these constraints on $\rho$, but
follow Schiff \cite{A:CS97} in assuming that $\rho$ is always
defined by a particular form of this conservation law.)
The conservation law is the usual $(\rho u^\mu)_{;\mu} = 0$
written as
\be\label{eq:cons'n-x}
      \frac{\partial}{\partial x^\mu }
\left(\sqrt{-g}\rho \frac{dx^\mu}{d\tau}\right) = 0
\ee
but in the $a^{\bar{\mu}}$ coordinate system.
Here $d/d\tau$ is an abbreviation for the proper time
derivative along the fluid world line:
\be
   \frac{d}{d\tau}
  = \frac{1}{ \sqrt{-g_{\alpha\beta}\dot{x}^\alpha\dot{x}^\beta} }
      \left( \frac{\partial }{\partial a^{\bar{0}}}  \right)
\ee
where
\be
    \dot{x^\mu} \equiv \frac{dx^\mu}{d\lambda} \equiv
   \left( \frac{\partial z^\mu}{\partial a^{\bar{0}}}  \right)
\ee
and the partial derivatives are taken at constant $a^{\bar{k}}$.
In $a^{\bar{\mu}}$ coordinates where $\dot{a}^{\bar{0}} = 1$ and
$\dot{a}^{\bar{k}} = 0$
equation (\ref{eq:cons'n-x}) reads
\be
      \frac{\partial}{\partial a^{\bar{0}} }
\left(\sqrt{ -\bar{g} }\rho
   \frac{1}{ \sqrt{-{g}_{\bar{0}\bar{0}} } }
\right) = 0
\ee
Thus $\rho$ at an arbitrary point $a^{\bar{\mu}}$ is defined by
\be\label{eq:cons'n-def}
\sqrt{ -\bar{g}(a^{\bar{\mu}}) }\rho(a^{\bar{\mu}})
   \frac{1}{ \sqrt{-{g}_{\bar{0}\bar{0}}(a^{\bar{\mu}}) } }
 =
\sqrt{ -\bar{g}(0,a^{\bar{k}}) }\rho(0,a^{\bar{k}})
   \frac{1}{ \sqrt{-{g}_{\bar{0}\bar{0}}(0,a^{\bar{k}}) } }
\ee






%******************************************************************
%******************************************************************
%    Variational Principle
%******************************************************************
%******************************************************************
\section{Variational Principle}

We now have all the tools needed to state our variational
principle.
The action to be varied is the integral of a Lagrangian $L$ along
the world lines, $I = \int L\,d\lambda$.
The Lagrangian in turn is a sum over the fluid particles, $L =
\int {\mathcal L}\,d^3 a$, giving the total action as a four
dimensional integral $I = \int {\mathcal L}\,d^4 a$.
The functions $z^\mu(a^{\bar{\nu}})$ are the central quantities
to be constrained by the variational principle, but some
auxiliary variables $\Lambda$ and $u_\mu$ are introduced.
Thus the quantities to be varied as functions of the four
independent variables $a^{\bar{\mu}}$ are
\be
   z^\mu(a^{\bar{\nu}}) \quad , \quad u_\mu(a^{\bar{\nu}})
      \quad \mbox{and} \quad \Lambda(a^{\bar{\nu}}) \quad .
\ee
The full action is
\be\label{eq:action}
  I = \int \!d^4a\,
\rho(0,a^{\bar{k}})
   \frac{\sqrt{ -\bar{g}(0,a^{\bar{k}}) }}{
\sqrt{-{g}_{\bar{0}\bar{0}}(0,a^{\bar{k}}) } }\,
 (1 + e) \left( \frac{dz^\mu}{d\lambda} u_\mu - \Lambda {\mathcal H}
  \right)
\ee
with
\be
 {\mathcal H} = \frac{1}{2} \left[ g^{\mu\nu}(z) u_{\mu} u_{\nu}
                                    + 1 \right]
     \quad .
\ee

The first consequence of this variation principle is obtained by
varying the Lagrange multiplier $\Lambda$ to find ${\mathcal H} =
0$ or
\be\label{eq:unit-u}
     g^{\mu\nu}(z) u_{\mu} u_{\nu} = -1
\ee
so that $u_{\mu}$ is a timelike unit vector.
The next consequence comes by varying $u_{\mu} (a)$ while holding
$\Lambda (a)$ and $z^\mu (a)$ fixed.
This variation does not cause any change in the internal energy
$e$ which, through its dependence on $\rho$, changes only if
$z^\mu (a)$ (or $g^{\mu\nu}(z(a))$) changes.
The result is to identify $u_{\mu}$ as the fluid 4-velocity:
\be\label{eq:zdot}
   \dot{z}^\mu \equiv \frac{dz^\mu}{d\lambda}
     = \Lambda g^{\mu\nu}(z) u_{\nu}
     \quad .
\ee
This identification is only complete after this equation is
solved for $u_\mu$ and substituted into the
constraint~(\ref{eq:unit-u}) to find
\be\label{eq:Lambda}
     \Lambda\,d\lambda = d\tau \equiv
     \sqrt{ -g_{\mu\nu}(z)\,dz^{\mu}\,dz^{\nu} }
     \quad .
\ee


The variation of $z^\mu(a)$ is more difficult, and we precede it
by a computation of the stress-energy tensor of the fluid.
This we calculate by the definition $-2 \delta I = \int \sqrt{-g}
\,T_{\mu\nu}\delta g^{\mu\nu}\,d^4x$, performing a variation of
$g^{\mu\nu}(x)$ in which none of $z^\mu, u_\mu, \Lambda$ is
being varied.
Note also that since $g^{\mu\nu}(x)$ is assumed to be nonzero
only at interior points of the integration, the functions at
$a^{\bar{0}} = 0$ are not being varied.
Since $g^{\mu\nu}$ only appears in $\mathcal H$ and in $\rho$, we
can in the first step write
\be
  \delta I = \int \!d^4a\,
\rho(0,a^{\bar{k}})
   \frac{\sqrt{ -\bar{g}(0,a^{\bar{k}}) }}{
\sqrt{-{g}_{\bar{0}\bar{0}}(0,a^{\bar{k}}) } }\,\left[
 \left( \frac{\partial e}{\partial \rho}\,\delta\rho \right)
      ( \dot{z}^\mu u_\mu - \Lambda {\mathcal H} )
   - (1+e)\Lambda\delta {\mathcal H}
     \right]  \quad .
\ee
Next one introduces several simplifications.
The variations can be made more explicit:  $\delta {\mathcal H} =
\half u_\mu u_\nu \delta g^{\mu\nu}$ and $\delta \rho = (\partial
\rho / \partial g^{\mu\nu}) \delta g^{\mu\nu}$.
But also in other factors, now that the variation has been
carried out, the other equations of motion can be used such as
${\mathcal H} = 0$ and equation~(\ref{eq:zdot}), to give
\be
  -2\delta I = \int \!d^4a\,
\rho(0,a^{\bar{k}})
   \frac{\sqrt{ -\bar{g}(0,a^{\bar{k}}) }}{
\sqrt{-{g}_{\bar{0}\bar{0}}(0,a^{\bar{k}}) } }\,\Lambda\left[-2
  \frac{\partial e}{\partial \rho}\,
  \frac{\partial\rho}{ \partial g^{\mu\nu} }
      ( u^\mu u_\mu  )
   +(1+e)u_\mu u_\nu
     \right]\delta g^{\mu\nu}  \quad .
\ee
We now convert the integration from $a^\mu$ coordinates to $z^\mu$
coordinates.
By using equations~(\ref{eq:cons'n-def}) and (\ref{eq:unit-u})
the preceding equation becomes
\be
  -2\delta I = \int \!d^4a\,
\rho(a)
   \frac{\sqrt{ -\bar{g}(a) }}{
\sqrt{-{g}_{\bar{0}\bar{0}}(a) } }\,\Lambda\left[
  2 \frac{\partial e}{\partial \rho}\,
  \frac{\partial\rho}{ \partial g^{\mu\nu} }
   +(1+e)u_\mu u_\nu
     \right]\delta g^{\mu\nu}  \quad .
\ee
in which the coordinate transformation $z(a)$  will give $d^4a\,\sqrt{ -\bar{g}(a) } \rho(a) =$ \\
$ d^4x\,\sqrt{-g(x)} \rho(x)$.
Also one can recognize from $a^{\bar{\mu}} = (1,0,0,0)$ that
$\sqrt{-g_{\bar{0}\bar{0}}(a) } =
\sqrt{-g_{\bar{\mu}\bar{\nu}} \dot{a}^{\bar{\mu}} \dot{a}^{\bar{\nu}} }
= \sqrt{-g_{\mu\nu} \dot{z}^\mu \dot{z}^\nu } = d\tau/d\lambda =
\Lambda$ using the invariance of the inner product and
equation~(\ref{eq:Lambda}).
So, using also $\partial e/\partial \rho = P/\rho^2$,
this variation gives
\be
  -2\delta I = \int \!d^4x\,
\rho
   \sqrt{ -g }\,\left[
  2 \frac{P}{\rho^2}\,
  \frac{\partial\rho}{ \partial g^{\mu\nu} }
   +(1+e)u_\mu u_\nu
     \right]\delta g^{\mu\nu}  \quad .
\ee
or
\be
  T_{\mu\nu} = \rho (1+e)u_\mu u_\nu
                 +2 \frac{P}{\rho}\,
                \frac{\partial\rho}{ \partial g^{\mu\nu} }
                         \quad .
\ee
There remains only to calculate $\partial\rho / \partial
g^{\mu\nu}$.

The definition (\ref{eq:cons'n-def}) of $\rho$ let us convert
the left hand side to $x$-coordinates using the fixed
transformation $x = z(a)$ in which the metric does not enter.
Here $\sqrt{ -\bar{g}(a) } = \sqrt{ -g(x) } J$ where
$J$ is the jacobean determinant
    \be\label{eq:J}
        J =  \frac{\partial (z^0, z^1, z^2, z^3)}{\partial
            (a^0, a^1, a^2, a^2)} \equiv
            \frac{\partial(z^\mu)}{\partial(a^{\bar{\nu}})}
    \ee
handles one factor, and of course $\rho(a) = \rho(x)$.
For the third factor, as noted above, $\sqrt{-g_{\bar{0}\bar{0}}(a) }
= \sqrt{-g_{\mu\nu} \dot{z}^\mu \dot{z}^\nu }$.
Thus we have from equation~(\ref{eq:cons'n-def}) that
\be\label{eq:density_cons}
\sqrt{ -g(x) } J \rho(x)
   \frac{1}{ \sqrt{-g_{\mu\nu} \dot{z}^\mu \dot{z}^\nu } }
 =
    [ \mbox{data at } a^{\bar{0}} = 0 ] \quad .
\ee
Now, as we vary $g^{\mu\nu}(x)$ at an interior point of the action
integration, the right hand side remains constant, as does $J$
since $z(a)$ is not being varied.
The variation of the left hand side (done for its logarithm) then
gives
\be
   \frac{\delta\rho}{\rho} + \frac{1}{2}\frac{\delta g}{g}
  - \frac{\delta \sqrt{-g_{\mu\nu} \dot{z}^\mu \dot{z}^\nu }
     }{\sqrt{-g_{\alpha\beta} \dot{z}^\alpha \dot{z}^\beta }}
   = 0  \quad .
\ee
In this we use $\delta g = -g\,g_{\mu\nu}\delta g^{\mu\nu}$ and
$\delta \sqrt{-g_{\mu\nu} \dot{z}^\mu \dot{z}^\nu } = - \half
\dot{z}^\mu \dot{z}^\nu \delta g^{\mu\nu}/ \sqrt{-g_{\alpha\beta}
\dot{z}^\alpha \dot{z}^\beta } = - \half u^\mu \dot{z}^\nu
\delta g^{\mu\nu}$ or
\be
    \frac{\delta \sqrt{-g_{\mu\nu} \dot{z}^\mu \dot{z}^\nu }
     }{\sqrt{-g_{\alpha\beta} \dot{z}^\alpha \dot{z}^\beta } }
    = -\half u^\mu u^\nu \delta g_{\mu\nu}
    = + \half u_\mu u_\nu \delta g^{\mu\nu}
\ee
to find
\be
   2 \frac{\delta \rho}{\rho}
       = (g_{\mu\nu}+ u_\mu u_\nu) \delta g^{\mu\nu}
      \quad .
\ee
Thus we finally find that the desired form of the stress-energy
tensor does arise from this variational principle:
\be
  T_{\mu\nu} = \rho (1+e)u_\mu u_\nu
                 + P (g_{\mu\nu}+ u_\mu u_\nu)
                         \quad .
\ee

Our final task to demonstrate that the variation of
the trajectory function $z^{\mu}(a^{\bar \nu})$ leads to
the relativistic form of Euler's equations.
Taking this variation yields

\bea\label{eq:tot_var_z}
 \delta I & = & \int \!d^4a\, \po
            \left\{
                    \frac{\partial e}{\partial \rho}
                    \frac{\partial \rho}{\partial z^\alpha} \delta z^\alpha
                    {\dot z^\mu} u_\mu
                    \frac{\partial e}{\partial \rho}
                    \frac{\partial \rho}{\partial {z^\alpha}_{,\lambda}} \delta {z^\alpha}_{,\lambda}
                    {\dot z^\mu} u_\mu
                    + \right . \nonumber \\
           &  &
            \left.
                    +
                      (1 + e)u_\mu \delta{\dot z}^\mu
                    - (1 + e)\frac{\Lambda}{2}
                      \frac{\partial g^{\mu\nu}}{\partial z^\alpha}u_\mu u_\nu \delta z^\alpha
            \right\} \,
\eea
where the parameter $\po = \rho(a) \frac{ \sqrt{-\bar{g}(a)} }{ \sqrt{-{g}_{\bar{0}\bar{0}}(a)} }$
has been defined for convenience.  In equation (\ref{eq:tot_var_z}), ${z^\alpha}_{,\lambda}
= \frac{\partial z^\alpha}{\partial a^\lambda}$ and
${\dot z}^{\alpha} = \frac{\partial z^\alpha}{\partial a^0}$.
Owing to the conservation of baryon number density
(\ref{eq:density_cons}) can be used to identify
\be\label{eq:po_def}
  \po = \sqrt{-g(z)}J\rho(z)\frac{ 1 } { \sqrt{-g_{\mu\nu}{\dot z^\mu}{\dot z^\nu}} }
\ee
a relation that proves useful below.



Handling each term in (\ref{eq:tot_var_z}) separately make for easier management
of the algebra.
Using equations (\ref{eq:first_law}), (\ref{eq:zdot}), (\ref{eq:Lambda}), and (\ref{eq:po_def})
the first term can be written as
\be\label{eq:term1}
 term_1 = \int \!d^4a\, \po \frac{\partial e}{\partial \rho}
                   \frac{\partial \rho}{\partial z^\alpha} \delta z^\alpha
                    {\dot z^\mu} u_\mu
=
 \int \!d^4a\, \frac{P}{\rho} \sqrt{-g} J u^{\mu} u_{\mu}
               \frac{\partial \rho}{\partial z^\alpha} \delta z^\alpha \.
\ee
The normalization of the 4-velocity can be used to eliminate the $u^\mu u_\mu$
at the expense of introducing a minus sign and the jacobean determinant
is used to put the integration over the variables $z^\alpha$.
The first term now becomes
\be\label{eq:term1_1}
term_1 = \int \!d^4z\, \frac{P}{\rho}\sqrt{-g} \frac{\partial \rho}{\partial z^\alpha} \delta z^\alpha \.
\ee
Now equation (\ref{eq:density_cons}) can be used to determine $\frac{\partial \rho}{\partial z^\alpha}$
as
\be
 \frac{\partial \rho}{\partial z^\alpha} = \frac{-\rho}{2}
                                           \left(
                                             g_{\mu\nu,\alpha} u^\mu u^\nu  + g^{\mu\nu} g_{\mu\nu,\alpha}
                                           \right) \_,
\ee
which when combined with (\ref{eq:term1_1})  gives
\be\label{eq:term1_2}
  term_1 = \int \!d^4z\, \frac{P}{2} \sqrt{-g}
                                           \left(
                                             g_{\mu\nu,\alpha} u^\mu u^\nu  + g^{\mu\nu} g_{\mu\nu,\alpha}
                                           \right) \delta z^\alpha\.
\ee

We now move onto the second term
\be\label{term2}
term_2 =  \int \!d^4a\, \frac{P}{\rho} \sqrt{-g} J u^{\mu} u_{\mu}
               \frac{\partial \rho}{\partial {z^\alpha}_{,\lambda}} \delta{z^\alpha}_{,\lambda} \.
\ee
Again the same process used for $term_1$ can be used here to obtain
\be\label{eq:term2_1}
  term_2 = \int \!d^4a\, \frac{-P}{\rho}\sqrt{-g} J
  \frac{\partial \rho}{\partial {z^\alpha}_{,\lambda}} \delta{z^\alpha}_{,\lambda} \.
\ee
Equation (\ref{eq:density_cons}) is used to obtain $\frac{\partial \rho}{\partial {z^\alpha}_{,\lambda}}$
given by
\be\label{eq:diff_rho_vel}
\frac{\partial \rho}{\partial {z^\alpha}_{,\lambda}} = \frac{-\po}{\sqrt{-g}J}
                                                       \left(
                                                         g_{\alpha\nu} u^\nu {\delta^\lambda}_0
                                                         + \frac{\Lambda}{J} {J^\lambda}_{\alpha}
                                                       \right) \_,
\ee
where the first term arises from the variation of $\Lambda = \sqrt{-g_{\mu\nu} {\dot z}^\nu {\dot z}^\nu}$
The second term arises from the variation of the jacobean determinant.
The jacobean determinant is given by
\be\label{eq:J_def}
  J = \frac{1}{4!} [\alpha \beta \gamma \delta][\lambda \tau \mu \nu] {z^\alpha}_{,\lambda}
      {z^\beta}_{,\tau} {z^\gamma}_{,\mu} {z^\delta}_{,\nu}
\ee
where the permutation symbol $[\alpha \beta \gamma \delta]$ is equal to $+1$ if the sequence
$\alpha \beta \gamma \delta$ is a even permutation of $1 2 3 4$, is equal to $-1$ if the
sequence $\alpha \beta \gamma \delta$ is an odd permutation of $1 2 3 4$, and is zero if
any two of the indices are equal to each other.
Variation of (\ref{eq:J_def}) yields
\be\label{eq:J_minor}
  \delta J = \left( \frac{1}{3!}[\alpha \beta \gamma \delta][\lambda \tau \mu \nu]
             {z^\beta}_{,\tau} {z^\gamma}_{,\mu} {z^\delta}_{,\nu} \right)
             {\delta z^\alpha}_{,\lambda}
           \equiv {J^\lambda}_{\alpha} {\delta z^\alpha}_{,\lambda} \.
\ee
which serves to define the $\alpha,\lambda$-minor of the jacobean determinant,
which appears in (\ref{eq:diff_rho_vel}).
From this definition, it is immediately seen that
\be\label{eq:diff_J_minor}
  \frac{\partial}{\partial a^\lambda} {J^\lambda}_{\alpha} = 0 \.
\ee
After performing the appropriate integration-by-parts on (\ref{eq:term2_1}),
we arrive at two terms
\be\label{eq:term2_2}
  term_2 = -\int \!d^4a\, \po \frac{\partial}{\partial a^0}
                              \left(
                                \frac{P}{\rho} g_{\alpha\nu} u^\nu
                              \right) \delta z^\alpha
           - \int \!d^4a\, \frac{\partial}{\partial a^\lambda}
                           \left(
                            \sqrt{-g} P {J^\lambda}_{\alpha}
                           \right) \.
\ee
Using (\ref{eq:diff_J_minor}), (\ref{eq:term2_2}) becomes
\be\label{eq:term2_3}
  term_2 = -\int \!d^4a\, \po \frac{\partial}{\partial a^0}
                              \left(
                                \frac{P}{\rho} g_{\alpha\nu} u^\nu
                              \right) \delta z^\alpha
           - \int \!d^4a\, \frac{\partial}{\partial z^\sigma}
                           \left(
                            \sqrt{-g} P
                           \right)
                           \frac{\partial z^\sigma}{\partial a^\lambda} {J^\lambda}_{\alpha} \.
\ee
From the definition of the jacobean determinant (\ref{eq:J_def}) and the properties
of the minor, one arrives at the relation $\frac{\partial z^\sigma}{\partial a^\lambda} {J^\lambda}_{\alpha}
= {\delta^\sigma}_\lambda J$, and thus (\ref{eq:term2_3}) becomes
\be\label{eq:term2_4}
  term_2 = -\int \!d^4a\, \po \frac{\partial}{\partial a^0}
                              \left(
                                \frac{P}{\rho} g_{\alpha\nu} u^\nu
                              \right) \delta z^\alpha
           - \int \!d^4z\, \frac{\partial}{\partial z^\sigma}
                           \left(
                            \sqrt{-g} P
                           \right)\.
\ee
Expanding the first term of $term_2$ and using the relations
$\frac{1}{\Lambda}\frac{\partial z^\beta}{\partial a^0} = u^\beta$
and
$\frac{1}{\Lambda}\frac{\partial u_\beta}{\partial a^0} =
\frac{\partial u_\beta}{\partial z^\sigma}u^\sigma$
we arrive at
\bea\label{eq:term2_4}
  term_2 & = & \int \!d^4a\, \po \frac{P}{\rho^2}\frac{\partial \rho}{\partial a^0} u_\alpha \delta z^\alpha
                      \nonumber \\
           &  &
           - \int \!d^4z\, \sqrt{-g}\left(
                                        \frac{\partial P}{\partial z^\alpha}
                                      + \frac{P}{2} g^{\mu\nu} g_{\mu\nu,\alpha}
                    \right . \nonumber \\
           &  &
                    \left.
                                      + \frac{\partial P}{\partial z^\beta} u^\beta u_\alpha
                                      + P \frac{\partial u_\alpha}{\partial z^\beta}u^\beta
                                    \right) \delta z^\alpha
\eea

We next move onto the third term in (\ref{eq:tot_var_z}).
Manipulations of the kind used above allows us to arrive at
\be\label{eq:term3}
  term_3 = - \int \!d^4a\, \po \frac{P}{\rho^2}\frac{\partial \rho}{\partial a^0} u_\alpha \delta z^\alpha
           - \int \!d^4z\, \sqrt{-g} \rho (1 + e) \frac{\partial u_\alpha}{\partial z^\beta} u^\beta \delta z^\alpha
\ee

Finally, we arrive at the fourth term in (\ref{eq:tot_var_z}), which when expanded as
above yields
\be\label{eq:term4}
  term_4 = - \int \!d^4z\, \sqrt{-g} (1 + e)\frac{\rho}{2} \frac{\partial g^{\mu\nu}}{\partial z^\alpha} u_\mu u_\nu
             \delta z^\alpha \.
\ee
Using the relation $\frac{\partial g^{\mu\nu}}{\partial z^\alpha}
= -g^{\mu\beta} \frac{\partial g_{\beta\sigma}}{\partial z^\alpha}g^{\sigma\nu}$
(\ref{eq:term4}) can be put into the more convenient form
\be\label{eq:term4_1}
  term_4 = - \int \!d^4z\, \sqrt{-g} (1 + e)\frac{\rho}{2} \frac{\partial g_{\beta\sigma}}{\partial z^\alpha}
             u^\beta u_\sigma \delta z^\alpha \.
\ee

Combining equations (\ref{eq:term1_2}), (\ref{eq:term2_4}), (\ref{eq:term3}), and (\ref{eq:term4_1})
we arrive at
\bea
 \delta I & = &  \int \!d^4z\, \sqrt{-g} \delta z^\alpha
            \left\{
              -\rho\left(1 + e + \frac{P}{\rho}\right) \frac{\partial u_\alpha}{\partial z^\beta} u^\beta
                    \right . \nonumber \\
           &  &
                    \left.
              +\rho\left(1 + e + \frac{P}{\rho}\right) \frac{1}{2}\frac{\partial g_{\mu\nu}}{\partial z^\alpha}
                 u^\mu u^\nu
              - \frac{\partial P}{\partial z^\beta}g_{\alpha\mu} u^\mu u^\beta
              - \frac{\partial P}{\partial z^\alpha}
            \right\} \.
\eea
Requiring that the variation equal zero means that the expression within the braces must
be zero.
Organizing this expression and using $\Gamma_{\mu\nu\alpha}u^\mu u^\nu = \frac{1}{2}
\frac{\partial g_{\mu\nu}}{\partial z^\alpha} u^\mu u^\nu$ yields
\be
 \rho \left( 1 + e + \frac{P}{\rho} \right)u_{\alpha ; \beta} u^\beta
      = -\left( {\delta_\alpha}^\beta + u_{\alpha} u^\beta \right) \frac{\partial P}{\partial z^\beta}
      \equiv -{\perp_\alpha}^\beta   \frac{\partial P}{\partial z^\beta}
\ee
which is precisely relation (22.13) in MTW (when identifying $\rho_{MTW} = \rho(1+e)$).

\begin{thebibliography}{2}

\bibitem{A:CWM96}
  C. W. Misner, \emph{Variational Principle for Lagrangian Hydrodynamics},
  Proceedings of the Second International Sakharov Conference of Physics, 1996,
  I. Dremin and A. Semikhatov, eds., World Scientific Pubs.

\bibitem{A:CS97}
  C. Schiff, \emph{Variational Smooth Particle Hydrodynamics II. Relativistic Fluids},
  unpublished

\end{thebibliography}

\end{document}
