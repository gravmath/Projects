%Simple document on the usefulness of index and matrix notations
\documentclass[12pt]{article}
%load packages
\usepackage{latexsym}
\usepackage{epic,eepic,graphicx,url}
%spacer commands
\newcommand{\eqc}{\ensuremath{\: ,} }
\newcommand{\eqp}{\ensuremath{\: .} }
\newcounter{bean}
\newcommand{\surp}[1]{\ensuremath{\left( {#1} \right)} }
\newcommand{\surb}[1]{\ensuremath{\left[ {#1} \right]} }
\newcommand{\surc}[1]{\ensuremath{\left\{ {#1} \right\}} }
\newcommand{\sura}[1]{\ensuremath{\left\langle #1 \right\rangle}}
\newcommand{\sure}[1]{({#1})}
\newcommand{\push} {\ensuremath{\:}}
\newcommand{\nudge}{\ensuremath{\,}}
\newcommand{\back}{\ensuremath{\!\!}}
\newcommand{\iback}{\ensuremath{\!\!\!\!}}
%equation commands
\newcommand{\be} { \begin{equation} }
\newcommand{\ee} { \end{equation}   }
\newcommand{\bes} { \[ }
\newcommand{\ees} { \]  }
\newcommand{\bea}{ \begin{eqnarray} }
\newcommand{\eea}{ \end{eqnarray}   }
\newcommand{\beas}{ \begin{eqnarray*} }
\newcommand{\eeas}{ \end{eqnarray*}   }
%index commands
\newcommand{\idx}[2] {\ensuremath{ {#1}{#2}   }}
\newcommand{\up} [1] {\ensuremath{{}^{#1}     } }
\newcommand{\dn} [1] {\ensuremath{{}_{#1}     } }
%array commands
\newcommand{\cv}  [1]{\ensuremath{\vec {#1}} }
\newcommand{\rv}  [1]{\ensuremath{{\vec {#1}}^{\,T}} }
\newcommand{\cf}  [1]{\ensuremath{\tilde {#1}} }
\newcommand{\rf}  [1]{\ensuremath{{\tilde {#1}}{\,^T}} }
\newcommand{\cs}  [1]{\ensuremath{\bar {#1}} }
\newcommand{\rs}  [1]{\ensuremath{{\bar {#1}}{\,^T}} }
\newcommand{\cuv} [1]{\ensuremath{\hat {#1}} }
\newcommand{\ruv} [1]{\ensuremath{{\hat {#1}}{\,^T}} }
\newcommand{\op}  [1]{\ensuremath{\mathbf{{#1}}}}
\newcommand{\dt}  [1]{\ensuremath{\dot {{#1}}} }
\newcommand{\ddt} [1]{\ensuremath{\ddot {{#1}}} }
\newcommand{\trps}[1]{\ensuremath{{#1}^{\,T} } }
\newcommand{\bv}  [1]{\ensuremath{\cs{e}_{#1}} }
\newcommand{\bd}  [1]{\ensuremath{\cf{\gw}^{#1}} }
%\newcommand{\rank}[2]{\ensuremath{\surp{\begin{array}{c}{#1 \\ #2}\end{array}}} }
\newcommand{\rank}[2]{\ensuremath{\surp{#1, #2}}}
%inner products
\newcommand{\ipp}[2]{\ensuremath{ (  {#1} ,     {#2} ) } }
\newcommand{\ipb}[2]{\ensuremath{ \langle {#1} ,     {#2} \rangle } }
\newcommand{\ipD}[2]{\ensuremath{ \langle {#1} |     {#2} \rangle } }
\newcommand{\ipd}[2]{\ensuremath{         {#1} \cdot {#2}         } }
%derivative and integral commands
\newcommand{\dby}   [1]{\ensuremath{ \frac{d}{d #1}} }
\newcommand{\dxby}  [2]{\ensuremath{ \frac{d #1}{d #2}} }
\newcommand{\pdby}  [1]{\ensuremath{ \frac{\partial}{\partial #1}} }
\newcommand{\pdxby} [2]{\ensuremath{ \frac{\partial #1}{\partial #2}} }
\newcommand{\myint} [1]{\ensuremath{ \int #1 \push} }
%variations
\newcommand{\var}[1]{\ensuremath{\delta {#1}}}
\newcommand{\vwrt}[2]{\ensuremath{\idx{\delta {#1}\vert}{\dn{\delta {#2}}}} }
\newcommand{\IBP}{\ensuremath{\stackrel{\mbox{\tiny{IBP}}}{=}}}
%common abreviations
\newcommand{\wrt}{with respect to }
\newcommand{\bdyterms}{boundary-terms }
\newcommand{\ibp}{integration-by-parts }
\newcommand{\dfunc}{\gd\back-function }
\newcommand{\dfuncs}{\gd\back-functions }
%special characters
\newcommand{\ga}{\ensuremath{\alpha} }
\newcommand{\gba}{\ensuremath{\bar \alpha} }
\newcommand{\gha}{\ensuremath{\hat \alpha} }
\newcommand{\gta}{\ensuremath{\tilde \alpha} }
\newcommand{\gb}{\ensuremath{\beta} }
\newcommand{\gbb}{\ensuremath{\bar \beta} }
\newcommand{\ghb}{\ensuremath{\hat \beta} }
\newcommand{\gtb}{\ensuremath{\tilde \beta} }
\newcommand{\gd}{\ensuremath{\delta} }
\newcommand{\gbd}{\ensuremath{\bar \delta} }
\newcommand{\ghd}{\ensuremath{\hat \delta} }
\newcommand{\gtd}{\ensuremath{\tilde \delta} }
\newcommand{\gD}{\ensuremath{\Delta} }
\newcommand{\gbD}{\ensuremath{\bar \Delta} }
\newcommand{\ghD}{\ensuremath{\hat \Delta} }
\newcommand{\gtD}{\ensuremath{\tilde \Delta} }
\newcommand{\get}{\ensuremath{\eta} }
\newcommand{\gbet}{\ensuremath{\bar \eta} }
\newcommand{\ghet}{\ensuremath{\hat \eta} }
\newcommand{\gtet}{\ensuremath{\tilde \eta} }
\newcommand{\gf}{\ensuremath{\phi} }
\newcommand{\gbf}{\ensuremath{\bar \phi} }
\newcommand{\ghf}{\ensuremath{\hat \phi} }
\newcommand{\gtf}{\ensuremath{\tilde \phi} }
\newcommand{\gF}{\ensuremath{\Phi} }
\newcommand{\gbF}{\ensuremath{\bar \Phi} }
\newcommand{\ghF}{\ensuremath{\hat \Phi} }
\newcommand{\gtF}{\ensuremath{\tilde \Phi} }
\newcommand{\gG}{\ensuremath{\Gamma} }
\newcommand{\gbG}{\ensuremath{\bar \Gamma} }
\newcommand{\ghG}{\ensuremath{\hat \Gamma} }
\newcommand{\gtG}{\ensuremath{\tilde \Gamma} }
\newcommand{\ggm}{\ensuremath{\gamma} }
\newcommand{\gbgm}{\ensuremath{\bar \gamma} }
\newcommand{\ghgm}{\ensuremath{\hat \gamma} }
\newcommand{\gtgm}{\ensuremath{\tilde \gamma} }
\newcommand{\gl}{\ensuremath{\lambda} }
\newcommand{\gbl}{\ensuremath{\bar \lambda} }
\newcommand{\ghl}{\ensuremath{\hat \lambda} }
\newcommand{\gtl}{\ensuremath{\tilde \lambda} }
\newcommand{\gL}{\ensuremath{\Lambda} }
\newcommand{\gbL}{\ensuremath{\bar \Lambda} }
\newcommand{\ghL}{\ensuremath{\hat \Lambda} }
\newcommand{\gtL}{\ensuremath{\tilde \Lambda} }
\newcommand{\gm}{\ensuremath{\mu} }
\newcommand{\gbm}{\ensuremath{\bar \mu} }
\newcommand{\ghm}{\ensuremath{\hat \mu} }
\newcommand{\gtm}{\ensuremath{\tilde \mu} }
\newcommand{\gn}{\ensuremath{\nu} }
\newcommand{\gbn}{\ensuremath{\bar \nu} }
\newcommand{\ghn}{\ensuremath{\hat \nu} }
\newcommand{\gtn}{\ensuremath{\tilde \nu} }
\newcommand{\gp}{\ensuremath{\pi} }
\newcommand{\gbp}{\ensuremath{\bar \pi} }
\newcommand{\ghp}{\ensuremath{\hat \pi} }
\newcommand{\gtp}{\ensuremath{\tilde \pi} }
\newcommand{\gq}{\ensuremath{\theta} }
\newcommand{\gbq}{\ensuremath{\bar \theta} }
\newcommand{\ghq}{\ensuremath{\hat \theta} }
\newcommand{\gtq}{\ensuremath{\tilde \theta} }
\newcommand{\gr}{\ensuremath{\rho} }
\newcommand{\gbr}{\ensuremath{\bar \rho} }
\newcommand{\ghr}{\ensuremath{\hat \rho} }
\newcommand{\gtr}{\ensuremath{\tilde \rho} }
\newcommand{\gs}{\ensuremath{\sigma} }
\newcommand{\gbs}{\ensuremath{\bar \sigma} }
\newcommand{\ghs}{\ensuremath{\hat \sigma} }
\newcommand{\gts}{\ensuremath{\tilde \sigma} }
\newcommand{\gt}{\ensuremath{\tau} }
\newcommand{\gbt}{\ensuremath{\bar \tau} }
\newcommand{\ght}{\ensuremath{\hat \tau} }
\newcommand{\gtt}{\ensuremath{\tilde \tau} }
\newcommand{\gw}{\ensuremath{\omega} }
\newcommand{\gbw}{\ensuremath{\bar \omega} }
\newcommand{\ghw}{\ensuremath{\hat \omega} }
\newcommand{\gtw}{\ensuremath{\tilde \omega} }
\newcommand{\gW}{\ensuremath{\Omega} }
\newcommand{\gbW}{\ensuremath{\bar \Omega} }
\newcommand{\ghW}{\ensuremath{\hat \Omega} }
\newcommand{\gtW}{\ensuremath{\tilde \Omega} }
\newcommand{\gy}{\ensuremath{\psi} }
\newcommand{\gby}{\ensuremath{\bar \psi} }
\newcommand{\ghy}{\ensuremath{\hat \psi} }
\newcommand{\gty}{\ensuremath{\tilde \psi} }
\newcommand{\gY}{\ensuremath{\Psi} }
\newcommand{\gbY}{\ensuremath{\bar \Psi} }
\newcommand{\ghY}{\ensuremath{\hat \Psi} }
\newcommand{\gtY}{\ensuremath{\tilde \Psi} }
%Script letters
\newcommand{\mA}{\ensuremath{\mathcal A} }
\newcommand{\mB}{\ensuremath{\mathcal B} }
\newcommand{\mC}{\ensuremath{\mathcal C} }
\newcommand{\mD}{\ensuremath{\mathcal D} }
\newcommand{\mE}{\ensuremath{\mathcal E} }
\newcommand{\mF}{\ensuremath{\mathcal F} }
\newcommand{\mG}{\ensuremath{\mathcal G} }
\newcommand{\mH}{\ensuremath{\mathcal H} }
\newcommand{\mI}{\ensuremath{\mathcal I} }
\newcommand{\mJ}{\ensuremath{\mathcal J} }
\newcommand{\mK}{\ensuremath{\mathcal K} }
\newcommand{\mL}{\ensuremath{\mathcal L} }
\newcommand{\mM}{\ensuremath{\mathcal M} }
\newcommand{\mN}{\ensuremath{\mathcal N} }
\newcommand{\mO}{\ensuremath{\mathcal O} }
\newcommand{\mP}{\ensuremath{\mathcal P} }
\newcommand{\mQ}{\ensuremath{\mathcal Q} }
\newcommand{\mR}{\ensuremath{\mathcal R} }
\newcommand{\mS}{\ensuremath{\mathcal S} }
\newcommand{\mT}{\ensuremath{\mathcal T} }
\newcommand{\mU}{\ensuremath{\mathcal U} }
\newcommand{\mV}{\ensuremath{\mathcal V} }
\newcommand{\mW}{\ensuremath{\mathcal W} }
\newcommand{\mX}{\ensuremath{\mathcal X} }
\newcommand{\mY}{\ensuremath{\mathcal Y} }
\newcommand{\mZ}{\ensuremath{\mathcal Z} }
%references and citations
\newcommand{\refq}[1]{\sure{\ref{#1}}}
\newcommand{\refp}[1]{\ref{#1}}
\newcommand{\refs}[1]{\cite{#1}}
%Thermo, Fluid, and GR objects
%%%particle kinematics
\newcommand{\ptraj}  [1]{\ensuremath{\idx{z}{\up{#1}}}}
\newcommand{\pdtraj} [1]{\ensuremath{\idx{\dt{z}}{\up{#1}}}}
\newcommand{\pvel}   [1]{\ensuremath{\idx{u}{\dn{#1}}}}
\newcommand{\ppos}   [1]{\ensuremath{\idx{z}{\up{#1}}}}
\newcommand{\pdpos}  [1]{\ensuremath{\idx{\dt{z}}{\up{#1}}}}
\newcommand{\ptvel}  [1]{\ensuremath{\idx{u}{\dn{#1}}}}
\newcommand{\pdtvel} [1]{\ensuremath{\idx{\dt{u}}{\dn{#1}}}}
\newcommand{\ptraja} [2]{\ensuremath{\idx{z}{\dn{#1}\up{#2}}}}
\newcommand{\pdtraja}[2]{\ensuremath{\idx{\dt{z}}{\dn{#1}\up{#2}}}}
\newcommand{\pvela}  [2]{\ensuremath{\idx{u}{\dn{#1#2}}}}
\newcommand{\sm}     [2]{\ensuremath{W\!\surp{\cv{#1} - \cv{#2}}}}
%%%thermodynamic parameters
\newcommand{\gtro}{\ensuremath{\idx{\gtr}{\dn{0}}} }
\newcommand{\ierg}{\ensuremath{e\surp{\gr\surp{x}}} }
\newcommand{\erg} {\ensuremath{e} }
%%%metric functions
\newcommand{\bV}[2] {\idx{\cv{#1}}{\dn{#2}} \nudge }
\newcommand{\bF}[2] {\idx{\cf{#1}}{\up{#2}} \nudge }
\newcommand{\Jac}[2]{\idx{\gL}{\up{#1}\dn{#2}} \nudge }
\newcommand{\jac}[2]{\idx{\gL}{\dn{#1}\up{#2}} \nudge }
\newcommand{\Kd}[2] {\idx{\gd \nudge}{\up{#1}\dn{#2}} \nudge }
\newcommand{\kd}[2] {\idx{\gd}{\dn{#1}\up{#2}} \nudge }
\newcommand{\Cnx}[2]{\idx{\gG}{\up{#1}\dn{#2}} \nudge }
\newcommand{\met} [2]{\ensuremath{\idx{g}{\dn{#1 #2}}}}
\newcommand{\imet}[2]{\ensuremath{\idx{g}{\up{#1 #2}}}}
\newcommand{\metd}   {\ensuremath{ g } }
\newcommand{\smet} [1]{\ensuremath{\sura{\idx{g}{\dn{#1}}}} }
\newcommand{\simet}[1]{\ensuremath{\sura{\idx{g}{\up{#1}}}} }
\newcommand{\smetd}   {\ensuremath{\sura{g}}}
%functional arguments
\newcommand{\contarg}{\ensuremath{\surp{\idx{a}{\up{0}};\cv{a}}} }
\newcommand{\harg}   {\ensuremath{\surp{a^0,\cv{r}_1}} }
\newcommand{\parg}   {\ensuremath{\surp{t,\cv{z}}} }
\newcommand{\farg}   {\ensuremath{\surp{t,\cv{x}}} }
%variations
\newcommand{\vIz}[1] {\ensuremath{\vwrt{I}{\ptraj{#1}}} }
%useful relations
\newcommand{\normeq}[2]{\ensuremath{ \imet{#1}{#2} \pvel{#1}
\pvel{#2} + 1} }

%bibtex stuff
\pagenumbering{arabic}
\bibliographystyle{plain}
%backwards compatibility
\def\a{{\vec a}}
\def\x{{\vec x}}
\def\z{{\vec z}}
\def\P{{\Phi}}
\def\l{{\ell}}
\def\Pl{{\Phi_{\ell}}}
\def\D{{\Delta}}
\def\.{{\quad .}}
%\def\_,{{\quad ,}}
\def\half{\mbox{$\frac{1}{2}$}}
%\def\_.{{\quad .}}
\def\po{{\hat \rho}_{0}}

\begin{document}
\title{On the Usefulness of Index and Matrix Notation}
\maketitle


The purpose of this short note is to give a feel as to when index notation
is useful in comparison to matrix notation.  The basis of the comparison
will be the homework problem 3.30 in the book `A first course in general
relativity' by B. F. Schutz.  The advantages of matrix notation are that
the novice is familiar with it from linear algebra class and that is well
suited (for two indices at least) for those COTSs that have built in matrix
manipulations.  The drawbacks of course are that hand work is often obstructed
by the large object size and that there is no obvious generalization
for more than two indices.  In this regime, index notation is superior but it
can be daunting for the beginner.  We explore both here an show some points of
each.

To begin Schutz asks the student to consider the following two vector fields and the
following scalar field given by
\bea
  \cv{U} & = & \surp{1+t^2, t^2, \sqrt{2} t, 0} \nonumber \\
  \cv{D} & = & \surp{x, 5 t x, \sqrt{2} t, 0}   \nonumber \\
  \gr    & = & x^2 - y^2 + t^2 \eqp
\eea
The context is special relativity, in which the metric tensor is given by
\be
  \get_{\gm\gn} = \surp{ \begin{array}{cccc}
                        -1&0&0&0 \\ 0&1&0&0 \\ 0&0&1&0 \\ 0&0&0&1 \end{array} } \eqp
\ee
Here we find our first example of matrix notation.  In order to determine if the
vector fields \cv{U} and \cv{D} are four-velocities, the norm of each must be
taken with respect to \idx{\get}{\dn{\gm\gn}}.  In matrix notation, the norm
of \cv{U} is given by
\be
  \vert \cv{U} \vert ^2 = \surp{1+t^2, t^2, \sqrt{2} t, 0}
                          \surp{\begin{array}{cccc} -1&0&0&0 \\ 0&1&0&0 \\ 0&0&1&0 \\ 0&0&0&1
                                \end{array} }
                          \surp{\begin{array}{cccc} 1+t^2\\t^2\\ \sqrt{2} t\\0 \end{array}}
                        = -1 \eqp
\ee
Likewise $\vert \cv{D} \vert^2 = -x^2 + 25 t^2 x^ + 2t^2$.  In both cases, it is easier,
to use index notation due to the diagonal nature of the metric.
In another computation, the student is asked to compute \idx{U}{\up{\ga}\dn{,\gb}}.  In
matric notation this is rendered as
\be
  \idx{U}{\up{\ga}\dn{,\gb}} = \surp{
                                   \begin{array}{cccc}
                                   2t&0&0&0 \\ 2t&0&0&0 \\ \sqrt{2}&0&0&0 \\ 0&0&0&0 \end{array}}
\ee
with the \ga index as the row and the \gb index as the column.
The inner product \idx{U}{\dn{\ga}} \idx{U}{\up{\ga}\dn{,\gb}} is then computed as
\be
\idx{U}{\dn{\ga}} \idx{U}{\up{\ga}\dn{,\gb}} =
  \surp{-1-t^2,t^2,\sqrt{2} t, 0 }
  \surp{ \begin{array}{cccc}
         2t&0&0&0 \\ 2t&0&0&0 \\ \sqrt{2}&0&0&0 \\ 0&0&0&0 \end{array}}
 = \surp{0,0,0,0} \eqp
\ee
Again this result is much more easily obtained using index notation in which the
fact that \idx{U}{\dn{\ga}} \idx{U}{\up{\ga}} = -1, automatically yields
\idx{U}{\dn{\ga}} \idx{U}{\up{\ga}\dn{,\gb}} = 0.

So when is matrix notation useful?  Its main use comes when the two-dimensional form is
complicated.  For example, the student is asked to compute
$\surp{\idx{U}{\up{\ga}} \idx{D}{\up{\gb}}}_{,\gb}$.  The intermediate result
of \idx{U}{\up{\ga}}\idx{D}{\up{\gb}} is most easily computed in matrix notation as
\bea
  U^{\ga} D^{\gb} & = & \surp{ \begin{array}{c} 1+t^2\\ t^2\\ \sqrt{2} t\\ 0 \end{array}}
                        \surp{x,5tx, \sqrt{2}t, 0} \nonumber \\
                  & = & \surp{ \begin{array}{cccc}
                               x \surp{1+t^2} & 5 t x \surp{1+t^2} & \sqrt{2} t \surp{1+t^2} & 0 \\
                               x t^2 & 5 t^3 x & \sqrt{2} t^3 & 0 \\
                               \sqrt{2} x t & \sqrt{50} t^2 x& 2 t^2 & 0 \\
                               0 & 0 & 0 & 0
                               \end{array} } \eqp
\eea
The above format is the most efficient for presenting the components.  Furthermore, the
divergence of the tensor is easily computed in matrix notation as
\bea
  \surp{ U^{\ga} D^{\gb}}_{,\gb} & = & \surb{U^\ga D^\gb }
                                       \overleftarrow{\surp{
                                         \begin{array}{c} \partial_t \\
                                                          \partial_x \\
                                                          \partial_y \\
                                                          \partial_z \end{array}
                                              }}
                                   = \surp{\partial_t, \partial_x, \partial_y, \partial_z}
                                     \trps{\surb{U^\ga D^\gb }} \nonumber \\
       & = &
\surp{\partial_t, \partial_x, \partial_y, \partial_z}
\surp{ \begin{array}{cccc}
                               x \surp{1+t^2} & x t^2 & \sqrt{2} x t & 0 \\
                               5 t x \surp{1+t^2} & 5 t^3 x & \sqrt{50} t^2 x & 0 \\
                               \sqrt{2} t \surp{1+t^2}& \sqrt{2} t^3& 2 t^2 & 0 \\
                               0 & 0 & 0 & 0
                               \end{array}} \nonumber \\
       & = & \surp{ \begin{array}{c}
                       2tx + 5t\surp{1+t^2} \\
                       2tx + 5 t^3 \\
                       \sqrt{2} x + \sqrt{50} t^2 \\
                       0
                    \end{array} }\eqp
\eea
Again, no more efficient way of computing these components exists.  However, once this
result is to contracted with \idx{U}{\dn{\ga}} the situation changes.  Carrying
through the matrix notation one finds after several lines that
\be
  U_{\ga} \surp{U^{\ga} D^{\gb}}_{,\gb} = - 5t \eqp
\ee
The result comes more simply in index notation where one first computes
$\idx{D}{\up{\gb}\dn{,\gb}} = 5t$ and then one exploits $U_\ga U^{\ga}_{\gb} = 0$
to determine that
\be
  U_{\ga} \surp{U^{\ga} D^{\gb}}_{,\gb} = U_{\ga} U^{\ga} \idx{D}{\up{\gb}\dn{,\gb}}
                                        = -1 5 t  = -5t \eqp
\ee
The general lesson here, is that for calculations with a great deal of symmetry
(i.e. many terms are zero) then index notation is superior.  It also allows
for direct transcription when writing computer code.  However, when
performing hand computations for components with no obvious cancellations, matrix
notation is best.

As a concrete example of this last statement, consider problem 4.25 in Schutz.  In this
problem the student is to show that the electromagnetic field strength tensor
\idx{F}{\up{\gm\gn}} when acted upon by the rotation portion of the Lorentz group results in
electric and magnetic fields that look as if they where directly transformed under the
rotation.  To do this by hand in terms on components obscures the basic result.
Instead, the result is more straightforwardly obtained using matrix notation.
First \idx{F}{\up{\gm\gn}} is represented as a matrix
\be
  \op{F} \doteq \surp{ \begin{array}{cccc} 0    & E_x  &  E_y &  E_z \\
                                           -E_x & 0    &  B_z & -B_y \\
                                           -E_y & -B_z &  0   &  B_x  \\
                                           -E_z &  B_y & -B_x &  0 \end{array}} \eqp
\ee
Then the transformation of coordinates is also represented as a matrix as
\be
  \op{\mR} \doteq \surp{ \begin{array}{cccc} 1    &   0        &  0        & 0 \\
                                            0    &  \cos{\gq} & \sin{\gq} & 0 \\
                                            0    & -\sin{\gq} & \cos{\gq} & 0 \\
                                            0    &   0        &  0        & 1
                        \end{array}} \eqp
\ee
Expansion of the relation $F^{\gba\gbb} = \Jac{\gba}{\gm} F^{\gm\gn} \jac{\gn}{\gbb}$
yields
\be
  \op{F'} \doteq  \surp{ \begin{array}{cccc}  0 &
                                              E_x c\gq + E_y s\gq &
                                             -E_x s\gq + E_y c\gq &
                                              E_z \\
                                             -E_x c\gq - E_y s\gq &
                                              0 &
                                              B_z &
                                              B_x s\gq - B_y c\gq   \\
                                              E_x s\gq - E_y c\gq &
                                             -B_z &
                                              0   &
                                              B_x c\gq + B_y s\gq \\
                                             -E_z &
                                             -B_x s\gq - B_y c\gq &
                                             -B_x c\gq - B_y s\gq &
                                              0
                        \end{array}}
\ee
where $c\gq \equiv \cos{\gq}$ and $s\gq \equiv \sin{\gq}$ for convenience.
The components in the new systems are then read off as
\bea
  E_x' & = &  E_x \cos{\gq} + E_y \sin{\gq} \nonumber \\
  E_y' & = & -E_x \sin{\gq} + E_y \cos{\gq} \nonumber \\
  E_z' & = &  E_z \nonumber \\
  B_x' & = &  B_x \cos{\gq} + B_y \sin{\gq} \nonumber \\
  B_y' & = & -B_x \sin{\gq} + B_y \cos{\gq} \nonumber \\
  B_z' & = &  B_z
\eea
which is just what one would expect from rotating the electric and magnetic fields
separately.  This result would not have been nearly as clean if it had been
done in index notation.
\end{document}
