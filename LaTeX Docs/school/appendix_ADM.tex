%appendix ADM

\section{Introduction}

Computer solutions of the dynamics of particles and fields in General Relativity
require that general covariance be abandoned in favor of a scheme
in which all of the physical quantities are expressed in terms of a common coordinate
time.
In a very real way then, all computer programs slice spacetime into spacelike slices, each one
labelled by a unique value of coordinate time (page 486 of \refs{MTW}.)
Data for the 3-geometry and its rate of change along with the initial configurations
and rates of change for the relevant physical fields, all expressed on one slice, form the
initial value data.
The integration of the Einstein equations then proceeds by evolving these data
from one slice to another, subject to a range of choices for the how
the slices are connected using the lapse and shift.

In this chapter, the steps required to formulate the Einstein field equations for modelling
via computer are presented.
Section \refp{S:3p1split} deals with how spacetime is foliated or sliced by surfaces of
constant coordinate time and how the metric tensor is subsequently decomposed.
Section \refp{S:spatial_geo} studies the spatial geometry induced on each slice.
The spacetime Riemann tensor is related to the Riemann tensor associated with the 3-geometry
of the slice and a measure of how the curvature changes from one slice to another,
called the extrinsic curvature.
Both of these sections are heavily influenced by the review article of York \refs{york78} and
the style and notation reflect this.
With these results in hand, the next three sections deal with some of the various schemes
used to solve the Einstein equations.  Section \refp{S:ADM_gp} presents a derivation of the
Arnowitt-Deser-Misner (ADM) field equations in their original form with the
spatial metric and associated conjugate momentum as the dynamic variables.  The approach is
based on Appendix E of Wald's book \refs{wald84}.
Section \refp{S:ADM_gk} presents the variant ADM field equations in which the 3-metric
and the extrinsic curvature are chosen as the dynamic variables.
This form seems to be almost always favored over the original ADM equations and the near-universally accepted choice for either numerical work
forming eitherwriting the as with the in The next three


%%%%%%%%%%%%%%%%%%%%%%%%%%%%%%%%%%%%%%%%%%%%%%%%%%%%%%%%%%%%%%%%%%%%%%%%%%%%%%%
\section{Spacetime Foliation}\label{S:3p1split}

The initial-value formulation of Arnowitt, Deser, and Misner (ADM) depends on
being able to break spacetime up into a foliation of constant-$t$ spacelike
hypersurfaces $\Sigma_t$.  In this section, the definition of this foliation
is presented.


\subsection{Surfaces of Constant Time}\label{SS:const_t}

Define a one-form $\Omega_a$ corresponding to surfaces of constant time $t$ by
taking the covariant derivative of the time coordinate
\begin{equation}\label{eq:cov_time}
   \Omega_{a} = \nabla_{a} t
              \doteq \left( \begin{array}{cccc} 1&0&0&0 \end{array} \right) \eqc
\end{equation}
subject to the condition that $\Omega_a$ be timelike
\[
   \Omega^{a} \Omega_{a} = g^{00} \equiv \frac{-1}{N^2} \eqp
\]
Introduce the normalized one-form
\begin{equation}\label{eq:norm_form}
   \omega^{a} \omega_{a} = -1
\end{equation}
via the definition
\[
   \omega_{a} = N \Omega_{a} \eqp
\]

A normalized unit-vector ${n^a}$ orthogonal to a constant-$t$ slice can be
constructed by
\begin{equation}\label{eq:unit_norm}
   n^{a} = - g^{a b} \omega_{b} \eqc
\end{equation}
with the fundamental inner-products
\[
   \langle \tilde n , \bar n \rangle =
   \langle -\tilde \omega , -\bar \omega \rangle =
   \omega_{a} \omega^{a} = -1
\]
and
\[
   \langle \tilde \omega ,  \bar n \rangle =
   \langle \tilde \omega , -\bar \omega \rangle =
   \omega_{a} \omega^{a} = 1 \eqp
\]

The former relation demonstrates that $n^a$ is a timelike vector and the
latter guarantees that $n^a$ is future-pointing.

From equations (\ref{eq:cov_time}) and (\ref{eq:unit_norm}), the unit
one-form $n_a$ is immediately seen to have components
\[
   n_a \doteq \left(-N, 0, 0, 0 \right) \eqp
\]
Using the form of the projection tensor
\begin{eqnarray}\label{eq:proj_tensor}
   h_{ab} & = & g_{ab} + n_{a}n_{b} \nonumber \\
   {h^{a}}_{b} & = &  g^{ac} h_{cb} \equiv {P^{a}}_{b} =
   {\delta^{a}}_{b} + n^{a} n_{b}
\end{eqnarray}
several important relations involving the metric can be worked out.  First of
all, it is clear that the definition of the projection tensor
(\ref{eq:proj_tensor}) ensures that $n^a$ is orthogonal to the projected
surface since
\[
   n^a {h_a}^b = n^a \left( {\delta_a}^b + n_a n^b \right) = n^b + n^a n_a n^b = 0
\]
In what follows, it will often be convenient to refer to arrays of components
in either column or row form.  Arrays with a tilde will be taken to be arrays
of covariant components and arrays with a bar will be taken as arrays of contravariant
components.  Arrays will be represented, by default, as column arrays unless
they possess a $T$ superscript (standing for transpose) in which case they are
interpreted as row arrays.
The spatial portion of the metric follows along the same line.  Thus
$g_{ij} \equiv h_{ij} = {\tilde{\tilde h}}$ and when needed
${\tilde {\tilde h}}^{-1} = \left( h_{ij} \right)^{-1} = h^{ij} =
{\bar {\bar h}}$.


\subsection{Determining the Metric Components}\label{SS:g_comps}

Assume that the metric and the inverse metric are given by
\[
   g_{ab} \doteq
   \left(
     \begin{array}{c c}
       g_{00}     & {\tilde N}^{T} \\
       {\tilde N} & {\tilde {\tilde h}}
     \end{array}
   \right)
\]
and
\[
   g^{ab} \doteq
   \left(
     \begin{array}{c c}
       -N^{-2}        & {\bar \delta}^{T} \\
       {\bar \delta}  & {\bar {\bar \sigma}}
     \end{array}
   \right)
\]
respectively.

In this notation, the unit-normal to the constant-$t$ slice is given by
\begin{eqnarray*}
   n^a & = &  g^{ab} n_{b} \\
       & \doteq &
   \left(
     \begin{array}{c c}
        -N^{-2}       & {\bar \delta}^{T} \\
       {\bar \delta}  & {\bar {\bar \sigma}}
     \end{array}
   \right)
%
   \left(
     \begin{array}{c}
       -N \\
       0
     \end{array}
   \right) \\
%
   & = &
   \left(
     \begin{array}{c}
       N^{-1} \\
       -N \bar \delta
     \end{array}
   \right) \eqp
\end{eqnarray*}
Now $h_{ab}$ can be constructed via two routes.  First ${h^{a}}_{b}$ can be
determined
\begin{eqnarray*}
   {h^{a}}_{b} & = &  {\delta^{a}}_{b} + n^{a} n_{b} \\
             & = & \left(
                      \begin{array}{cc}
                         1 & 0 \\
                         0 & 1
                      \end{array}
                   \right) +
                   \left(
                      \begin{array}{c}
                          N^{-1} \\ -N \bar \delta
                      \end{array}
                   \right)
                   \left(
                      \begin{array}{cc}
                         -N & 0
                      \end{array}
                   \right) \\
             & = & \left(
                      \begin{array}{cc}
                         1 & 0 \\
                         0 & 1
                      \end{array}
                   \right) +
                   \left(
                      \begin{array}{cc}
                         -1             & 0 \\
                         N^2 \bar \delta & 0
                      \end{array}
                   \right) \\
             & = & \left(
                      \begin{array}{cc}
                         0               & 0 \\
                         N^2 \bar \delta & 1
                      \end{array}
                   \right)
\end{eqnarray*}
and its index lowered to yield
\begin{eqnarray*}
   h_{ab} & = &  g_{a c} {h^{c}}_{b}  \\
             & = & \left(
                      \begin{array}{cc}
                         g_{00}   & {\tilde N}^T \\
                         \tilde N & {\tilde {\tilde h}}
                      \end{array}
                   \right)
                   \left(
                      \begin{array}{cc}
                         0               & 0 \\
                         N^2 \bar \delta & 1
                      \end{array}
                   \right) \\
             & = & \left(
                      \begin{array}{cc}
                         N^2 {\tilde N}^T \bar \delta & {\tilde N}^T \\
                         N^2 {\tilde {\tilde h}} \bar \delta & {\tilde {\tilde h}}
                      \end{array}
                   \right)
\end{eqnarray*}
Finally $h_{ab}$ can be constructed directly by
\begin{eqnarray*}
   h_{ab} & = &  g_{ab} + n_{a}n_{b}  \\
             & = & \left(
                      \begin{array}{cc}
                         g_{00}   & {\tilde N}^T \\
                         \tilde N & {\tilde {\tilde h}}
                      \end{array}
                   \right) +
                   \left(
                      \begin{array}{c}
                         -N \\
                          0
                      \end{array}
                   \right)
                   \left(
                      \begin{array}{c c}
                         -N & 0
                      \end{array}
                   \right)\\
             & = & \left(
                      \begin{array}{cc}
                         g_{00}   & {\tilde N}^T \\
                         \tilde N & {\tilde {\tilde h}}
                      \end{array}
                   \right) +
                   \left(
                      \begin{array}{cc}
                         N^2 & 0 \\
                         0   & 0
                      \end{array}
                   \right) \\
             & = & \left(
                      \begin{array}{cc}
                         g_{00} + N^2  & {\tilde N}^T \\
                         \tilde N & {\tilde {\tilde h}}
                      \end{array}
                   \right) \eqp
\end{eqnarray*}
Equating the two forms for $h_{a b}$ give the relations
\begin{eqnarray*}
   \bar \delta & = & \frac{1}{N^2} {\tilde {\tilde h}}^{-1} {\tilde N} \\
               & = & \frac{h^{ij} N_j}{N^2} \\
               & \equiv & \frac{N^i}{N^2}
\end{eqnarray*}
and
\begin{eqnarray*}
   g_{00} & = &  -N^2 + {\tilde N}^T {\tilde {\tilde h}}^{-1} \tilde N \\
          & = &  -N^2 + N^{i} N_{i} \eqp
\end{eqnarray*}
The relation for ${\bar {\bar \sigma}}$ can be determined from the requirement
that ${\tilde {\tilde g}} {\bar {\bar g}} = 1$ and yields
\[
   {\bar {\bar \sigma}} = {\tilde {\tilde h}}^{-1}
                          - \frac{ \bar N {\bar N}^T }{N^2} \eqp
\]
Combining these relations arrives at the ADM form of the metric
\begin{eqnarray}\label{eq:g_ADM}
   g_{ab} & = & \left( -N^2 + N^i N_i \right) dt^2 \nonumber \\
          &   &  + 2 N_i dt dx^i + h_{ij} dx^i dx^j \nonumber \\
          & = & -N^2 dt^2 \nonumber \\
          &   & + h_{ij}\left(dx^i + N^i dt\right) \left(dx^j + N^j dt\right)
\end{eqnarray}

%%%%%%%%%%%%%%%%%%%%%%%%%%%%%%%%%%%%%%%%%%%%%%%%%%%%%%%%%%%%%%%%%%%%%%%%%%%%%%%
\section{Studying the Spatial Geometry}\label{S:spatial_geo}

In this section, the physical attributes of the constant-$t$ spacelike
hypersurfaces $\Sigma_t$ is presented.  First the foliation developed in the
preceding section will be interpreted.  The extrinsic curvature is then
defined and will be shown to be the rate-of-change of the 3-metric along
the vector field,$n^a$, normal to the set \{ $\Sigma_t$ \} of spacelike hypersurfaces.
Finally, the 3-dimensional Riemann tensor will be presented in terms of a
covariant derivative defined within the hypersurface.

%\begin{figure}
%\begin{center}
%\epsfig{file=shift_lapse.eps, width = 3.0 in}
%\end{center}
%\caption{test}
%\end{figure}


\subsection{Physical Interpretation of the Foliation}

Physical meaning can be attached to ${\tilde {\tilde h}}$ (the spatial portion
of $h_{ab}$ ) by noting that from the first expression for the ADM metric
(\ref{eq:g_ADM}) $h_{ij}$ acts as a metric on the spatial constant-$t$ slice.
The spatial vector $N^i$, which is called the shift vector, can be
physically interpreted in terms of the movement of time.  Defining a vector
corresponding to positive time movement as
\begin{eqnarray*}
  t^a & \doteq & \left( \begin{array}{cccc} 1 & 0 & 0 & 0 \end{array} \right) \\
      &   =    &  N \left( \begin{array}{cc}
                             \frac{1}{N} & \frac{-\bar N}{N}
                           \end{array}
                    \right) +
                    \left( \begin{array}{cc}
                             0 & \bar N
                           \end{array}
                    \right) \\
      &   =    & N n^a + N^a
\end{eqnarray*}
gives an immediate relation between the time vector and the normal vector, in
terms of the shift.
Geometrically, the time vector `stretches' between points with the same spatial
coordinates on the two slices.
A particle at rest on the lower surface will have a worldline whose tangent
is $t^a$.
By considering two adjacent spatial hypersurfaces defined
as times $t$ and $t+dt$ respectively, allows a fairly simple physical
interpretation.
The separation, in proper time, between the two slices, given by the projection
of $\bar t$ along the normal $\bar n$, is
\[
  \langle \tilde \omega , \bar t \rangle = N
\]
This result motivates the name of `lapse' for the function $N$.
The normal $n^a$, which is perpendicular to the lower surface, connects a
point on the lower surface, with coordinates $x^i(lower)$, to a point on
the upper surface, with coordinates $x^i(upper)$.
The two coordinates are related by
\[
   x^i(upper) = x^i(lower) - N^i dt \eqp
\]

It is important to realize that the hypersurface normal field does not satisfy
the geodesic equation even though the timelike vector field $t^a$ does.  The
amount that the hypersurface normal field fails the geodesic equation is given
by its four-acceleration
\[
   a^{b} = n^a {n^b}_{;a} \eqp
\]
In addition, since $n^a$ is normalized, it is always orthogonal to its four-
acceleration
\[
   n^b a_b = n^b n^c \nabla_c n_b= n^c \left( n^b \nabla_c n_b \right) = 0 \eqp
\]


\subsection{The Many Faces of the Extrinsic Curvature $K_{ab}$}\label{S:K_faces}

As discussed above, once the spacetime foliation is determined the metric can
be split into the shift, the lapse, and the 3-metric.  The central idea for
performing this foliation was to define a one-form that is a differential
$\Omega_a = dt$ of the coordinate time $t$.  This definition corresponds to
surfaces of constant coordinate time and implies
\[
  \begin{array}{ccc}
    d \Omega = 0 & \leftrightarrow & \nabla_{[a} \Omega_{b]} = 0
  \end{array}
\]
which is a consequence of the Poincar\'{e} lemma.  Defining the normalized
one-form as in (\ref{eq:norm_form}) yields the corresponding relation
\[
  \begin{array}{ccc}
    \omega \wedge d \omega = 0 & \leftrightarrow & \omega_{[a}
    \nabla_{b} \omega_{c]} = 0
  \end{array}
\]
which is an instance of the Froebinius theorem.  The one-form dual to the
normal vector to a constant-$t$ surface, allows the Froebinius relation
to be rewritten as
\begin{equation}\label{eq:norm_Froeb}
  n_{[a} \nabla_{b} n_{c]} = 0
\end{equation}
which is the crucial relationship that reveals the many faces of the
extrinsic curvature.  By contracting (\ref{eq:norm_Froeb}) with $n^c$ the
relationship
\begin{equation}\label{eq:norm_Fr_con}
 \begin{array}{ccccc}
  n^c n_{[a} \nabla_{b} n_{c]} & = & 0 & = & n_{[a} a_{b]} + \nabla_{[a} n_{b]}
 \end{array}
\end{equation}
is obtained.  With this in hand, the general definition of the extrinsic
curvature
\[
   K_{ab} = -\nabla_{a} n_{b} - n_{a} a_{b}
\]
is seen to be entirely symmetric in its two indices.  This follows from
\[
   K_{ab} = -\nabla_{(a} n_{b)} - n_{(a} a_{b)} -\nabla_{[a} n_{b]} - n_{[a} a_{b]}
\]
and (\ref{eq:norm_Fr_con}).  Thus the extrinsic curvature can also be written as
\[
   K_{ab}  =  -\nabla_{(a} n_{b)} - n_{(a}a_{b)} \eqp
\]
Define $\perp$ as shorthand for applying the projection tensor to all free
indices.  By applying $\perp$ to $K_{ab}$ and repeatedly using the identities
$n^a \nabla_b n_a = 0$ and $n^a {h_a}^b = 0$gives
\begin{eqnarray*}
   \perp K_{ab} & = & {h_a}^c {h_b}^d K_{cd} \\
%                & = & -\nabla_{(a} n_{b)} - n_{a} n^c \nabla_c n_{b} \\
%                &   & -n_b n^d \nabla_d n_a \\
                & = & -\nabla_{(a} n_{b)} - n_{(a} a_{b)} \\
                & = & K_{ab}
\end{eqnarray*}
which demonstrates that $K_{ab}$ is entirely a spatial tensor.

Finally, the extrinsic curvature can be related to the Lie derivative of the
three metric along the normal vector field giving
\[
   {\mathcal L}_{\mathbf{n}} h_{ab} = 2 \left[ \nabla_{(a} n_{b)} + n_{(a}a_{b)} \right]
\]
which when compared with (\ref{eq:norm_Froeb}) yields the desired connection
\begin{equation}\label{eq:Lie_K}
   K_{ab} = - \frac{1}{2} {\mathcal L}_{\mathbf{n}} h_{ab}
\end{equation}
between the extrinsic curvature and the 3-metric.


\subsection{The 3-Dimensional Riemann Tensor}
To complete the study of the spatial hypersurface, the 3-dimensional Riemann
tensor must be defined in terms of the spatial covariant derivative $D_a$ which
takes the form
\[
   D_a {T^{b_1 \ldots b_k}}_{c_1 \ldots c_m} = {h^{b_1}}_{s_1}
                                               \ldots {h_{c_m}}^{t_m} {h_a}^{d}
                                               \nabla_d
                                               {T^{s_1 \ldots s_k}}_{t_1 \ldots t_m}
\]
on the spatial slice.  That this is a good definition of a covariant derivative
is seen from its application to the 3-metric which yields
\[
   D_a h_{bc} = {h_a}^{r} {h_b}^s {h_c}^t \nabla_r
                \left( g_{st} + n_s n_t \right) = 0
\]
since $g_{ab}$ is covariantly constant and since ${h_a}^b n_b = 0$.  The
3-dimensional Riemann tensor is then defined by
\begin{equation}\label{eq:3_d_Riemann}
   {}^{(3)}{R^c}_{dab} V^d = \left( D_{a}D_{b} - D_{b}D_{a} \right) V^c
\end{equation}
which is exactly the same form as in 4-dimensional spacetime.  To achieve the
3+1 decomposition, the spatial Riemann must be related to the spacetime Riemann.
This process starts with the term
\begin{eqnarray*}
D_b V^c & = & {h_b}^a {h^c}_e \nabla _a V^e \\
        & = & {h_b}^a \left( {\delta ^c}_e + n^c n_e  \right) \nabla _a V^e \\
        & = & {h_b}^a {\delta ^c}_e \nabla _a V^e + {h_b}^a n^c n_e \nabla _a V^e \\
        & = & {h_b}^a \nabla _a V^c - {h_b}^a n^c V^e \nabla _a n_e \\
        & = & {h_b}^a \nabla _a V^c - {h_b}^a n^c V^e
              \left( -K_{ae} - n_a a_e \right) \\
        & = & {h_b}^a \nabla _a V^c + K_{be} n^c V^e \eqp
\end{eqnarray*}
The next step involves a repeated application of the 3-dimensional covariant
derivative which when expanded yields
\begin{eqnarray*}
  D_a \left( D_b \, V^c \right) & = &   D_a \left({h_b}^d \nabla _d V^c \right)
                                   + D_a \left(K_{bd} V^d n^c \right) \\
                             & = &   {h_b}^d {h_a}^e {h_d}^f {h^c}_g
                                     \nabla _e \nabla _f V^g
                                   + D_a \left(K_{bd} V^d n^c \right) \\
                             & = &   {h_b}^f {h_a}^e {h^c}_g
                                     \nabla _e \nabla _f V^g
                                   + D_a \left(K_{bd} V^d n^c \right) \,.
\end{eqnarray*}
The last term can be simplified
\begin{eqnarray*}
  D_a \left( K_{bd} V^d n^c \right) & = & {h_a}^p {h_b}^q {h^c}_r
                                          \nabla _p
                                          \left( K_{qd} V^d n^r \right) \\
                                    & = & {h_a}^p {h_b}^q {h^c}_r
                                          K_{qd} V^d \nabla _p n^r \\
                                    & = & K_{bd} {h_a}^p {h^c}_r V^d
                                          \left(-{K_p}^r - n_p a^r \right) \\
                                    & = & -K_{bd}{K_a}^c V^d \, ,
\end{eqnarray*}
where ${h^c}_r n^r = 0$ was used to eliminate two terms.
To complete the evaluation of the 3-dimensional Riemann, the above expressions
must be anti-symmetrized over $a$ and $b$ giving
\begin{eqnarray*}
 \left( D_{a}D_{b} - D_{b}D_{a} \right) V^c & = & {h_b}^f {h_a}^e {h^c}_g
                                                 \left(   \nabla _e \nabla _f
                                                        - \nabla _f \nabla _e
                                                 \right)  V^g \\
                                            &    &
                                                 +
                                                 \left(   K_{ad} {K_b}^c
                                                        - K_{bd} {K_a}^c
                                                 \right) V^d \eqp
\end{eqnarray*}
Substituting in the definitions of the 3- and 4-dimensional Riemann tensors
gives
\begin{eqnarray*}
 {}^{(3)} {R^c}_{dab} V^d & = & {h_b}^f {h_a}^e {h^c}_g \, {}^{(4)} {R^g}_{mef} V^m \\
                                            &    &
                                                 +
                                                 \left(   K_{ad} {K_b}^c
                                                        - K_{bd} {K_a}^c
                                                 \right) V^d \eqp
\end{eqnarray*}

\begin{eqnarray}\label{eq:gauss}
 {}^{(3)} {R^c}_{dab} & = & {h_b}^f {h_a}^e {h^c}_g {h_d}^m
                            \, {}^{(4)} {R^g}_{mef} \nonumber \\
                      &   &
                            +
                            \left(   K_{ad} {K_b}^c
                                   - K_{bd} {K_a}^c
                            \right) \nonumber \\
                       & = & \perp {R^c}_{dab} + K_{ad}{K_b}^c - K_{bd}{K_a}^{c} \, .
\end{eqnarray}
between the spatial and spacetime Riemann tensors.

%%%%%%%%%%%%%%%%%%%%%%%%%%%%%%%%%%%%%%%%%%%%%%%%%%%%%%%%%%%%%%%%%%%%%%%%%%%%%%%
\section{The ADM Field Equations}\label{S:ADM_gp}

Finally we arrive at the field equations for the ADM system.  The equations are
obtain by first recasting the Hilbert-Einstein action in terms of the shift, the
lapse, the 3-metric, and the extrinsic curvature.  Once done, a conjugate
momentum to the 3-metric is defined and the action is put into full Hamiltonian
form.

\subsection{Expressing the Hilbert Action in 3+1 Language}\label{SS:Hilbert_A}

Before transforming the Hilbert action the final relation between $K_{ab}$ and
the time derivative of the spatial geometry must be derived.  A straightforward
manipulation of (\ref{eq:Lie_K}) gives
\begin{eqnarray}\label{eq:Lie_K_2}
   K_{ab} & = & -\frac{1}{2} {\mathcal L}_{\mathbf{n}} h_{ab} \nonumber \\
          & = & -\frac{1}{2} \left[
                                \left(\nabla_c h_{ab}\right)n^c
                                + h_{ac} \nabla_b n^c + h_{cb} \nabla_a n^c
                               \right] \nonumber \\
          & = & -\frac{1}{2} \left[
                                \left(
                                  \nabla_c h_{ab}
                                \right)n^c \right. \nonumber \\
          &   &                   \left. + \frac{1}{N}h_{ac}
                                  \left\{ \nabla_b \left(N n^c \right)
                                          - n^c \nabla_b N \right\} \right. \nonumber \\
          &   &                   \left. + \frac{1}{N}h_{cb}
                                  \left\{ \nabla_a \left(N n^c \right)
                                          - n^c \nabla_a N \right\}
                             \right] \nonumber \\
          & = & -\frac{1}{2N} {\mathcal L}_{N\mathbf{n}} h_{ab} \eqp
\end{eqnarray}
From the definition of the normal vector in terms of the time vector and the shift
the above relation becomes
\begin{eqnarray}
K_{ab}    & = & -\frac{1}{2 N} {\mathcal L}_{\mathbf{t} - \mathbf{N}} h_{ab} \nonumber \\
          & = & -\frac{1}{2 N} \left[ {\dot h}_{ab}
               - {\mathcal L}_{\mathbf{N}} h_{ab} \right] \eqp
\end{eqnarray}

The inherent flexibility of the Lie derivative can be exploited to simplify the
calculation of the Lie derivative of the metric along the shift vector by
using the spatial covariant derivative (and in the process getting rid of all
derivatives of the 3-metric except for the $\mathbf{\dot h}$ term).
Doing so gives
\[
   {\mathcal L}_{\mathbf{N}} h_{ab} = h_{ab|c} N^c + h_{cb} {N^{c}}_{|a}
                                                  + h_{ac} {N^{c}}_{|b}
\]
where the vertical bar stands for the spatial covariant derivative.  Combining
this result with (\ref{eq:Lie_K_2}) gives
\[
   K_{ab} = -\frac{1}{2 N} \left( {\dot h}_{ab} - D_{a} N_{b} - D_{b} N_{a} \right) \eqp
\]
With the above results in hand, it is relatively simple to recast the Hilbert
action.  From the definition of the Einstein tensor the Ricci scalar can be
isolated by contracting the Einstein tensor relation with unit-normals to give
\begin{equation}\label{eq:Ricci}
   R = 2 \left( G_{ab}n^an^b - R_{ab} n^a n^b \right) \eqp
\end{equation}
The term involving the Einstein tensor in (\ref{eq:Ricci}) can be related to
contractions on the Riemann tensor through an equivalent route.  Starting from
\begin{eqnarray*}
   \frac{1}{2} R_{abcd} h^{ac} h^{bd} & = & R + R_{ab} n^a n^b \nonumber \\
                                      & = & G_{ab} n^a n^b
\end{eqnarray*}
and using (\ref{eq:gauss}) gives
\begin{eqnarray*}
   R_{abcd} h^{ac} h^{bd} & = & R_{abcd}{h^a}_m h^{mq} {h_q}^c {h^b}_s h^{st} {h_t}^d \nonumber \\
                          & = & \perp R_{msqt} h^{mq} h^{st} \nonumber \\
                          & = & \perp \left( {}^{(3)}R_{msqt} + K_{st}K_{qm} \right. \nonumber \\
                          &   & \left. - K_{qt}K_{sm} \right) h^{mq} h^{st} \nonumber \\
                          & = & {}^{(3)}R + {Tr(K)}^2 - K^{ms}K_{ms} \nonumber \\
                          & = & {}^{(3)}R + {Tr(K)}^2 - Tr({\mathbf K}^2)
\end{eqnarray*}
which is seen to be the first term in equation (\ref{eq:Ricci}).
The second term on the right-hand side of
(\ref{eq:Ricci}) is
\begin{eqnarray*}
   R_{ab} n^a n^b & = & {R^c}_{acb} n^a n^b \\
                  & = & n^b {R^c}_{acb}n^a \\
                  & = & n^b \left( \nabla_c \nabla_b - \nabla_b \nabla_c \right) n^c \\
                  & = & \nabla_c(n^b \nabla_b n^c)-(\nabla_c n^b)(\nabla_b n^c) \\
                  &   & - \nabla_b(n^b \nabla_c n^c)+(\nabla_b n^b)(\nabla_c n^c) \\
                  & = & K^2 - K_{ab} K^{ab} -
                          \nabla_c \left( n^b \nabla_b n^c \right) \\
                  &   & + \nabla_b \left( n^b \nabla_c n^c \right)  \\
                  & = & Tr(K)^2 - Tr({\mathbf K}^2) -
                          \nabla_c \left( n^b \nabla_b n^c \right) \\
                  &   & + \nabla_b \left( n^b \nabla_c n^c \right)  \eqp
\end{eqnarray*}
Since the last two terms are divergences they will be ignored.  Finally using
$\sqrt{-g} = N \sqrt{h}$ gives the ADM Lagrange density
\begin{equation}\label{eq:Hilbert_L}
   {\mathcal L} = N \sqrt{h} \left[ {}^{(3)}R + Tr({\mathbf K^2}) - Tr(K)^2 \right] \eqp
\end{equation}


\subsection{The ADM Action in Canonical Coordinates}\label{SS:Can_ADM}

The starting point for deriving the ADM field equations is the Hilbert Lagrangian
density (\ref{eq:Hilbert_L}) expressed in $3+1$ language in terms of the
3-dimensional Ricci scalar and the extrinsic curvature of the spatial
hypersurface (with $K= {K^a}_{a}$).  The first step is to define the conjugate
momenta by
\[
   \pi^{ab} = \frac{\partial {\mathcal L} } {\partial {\dot h}_{ab}}
            = \frac{\partial {\mathcal L} } {\partial K_{ab}}
              \frac{\partial K_{ab}}{\partial {\dot h}_{ab} }
\]
and to use the relationship between the extrinsic curvature and the
time-derivative of the 3-metric (\ref{eq:Lie_K_2}) to arrive at
\begin{equation}\label{eq:pi_def}
   \pi^{ab} = -\sqrt{h} \left[ K^{ab} - h^{ab} K \right] \eqp
\end{equation}
The next step is to define the Hamiltonian density as
\[
   {\mathcal H} = \pi^{ab} {\dot h}_{ab} - {\mathcal L} \eqp
\]
Solving (\ref{eq:Lie_K_2}) for
\[
   {\dot h}_{ab} = - 2 N K_{ab} + 2 N_{(a|b)}
\]
and use this in conjunction with (\ref{eq:pi_def}) contracted with $h_{ab}$
\begin{eqnarray*}
   \pi^{ab} h_{ab} & = & -\sqrt{h} \left[ K - K {h^b}_{b} \right] \\
                   & = & -\sqrt{h} \left[ K - 3 K \right] \\
                   & = & 2\sqrt{h} K
\end{eqnarray*}
allows (\ref{eq:pi_def}) to be inverted, giving $K^{ab}$ in terms of $\pi^{ab}$
as
\[
   K^{ab} = -\frac{\pi^{ab}}{\sqrt{h}} + \frac{\pi}{2 \sqrt{h}} h^{ab} \eqp
\]
Using these relations,, the Hamiltonian density becomes
\begin{eqnarray}\label{eq:H_den_1}
   {\mathcal H} & = & \sqrt{h} N \left[ \frac{\pi^{ab} \pi_{ab}}{h} -
                                   \frac{ \pi^2}{2 h} - {}^{(3)}R
                            \right] \nonumber \\
               &   & + 2 N_a {\pi^{ab}}_{|b} - 2 \left( N_a \pi^{ab} \right)_{|b} \eqp
\end{eqnarray}
Since the last term in (\ref{eq:H_den_1}) is a total divergence, it can be
ignored and the Lagrangian density becomes
\begin{eqnarray}
   {\mathcal L} = {\dot h}_{ab} \pi^{ab} - N \left(
                 \underbrace{ \sqrt{h}
                 \left[ \frac{\pi^{ab} \pi_{ab}}{h} -
                        \frac{ \pi^2}{2 h} - {}^{(3)}R
                 \right]}_{R^0} \right) \nonumber \\
                 - N_a \left( \underbrace{ -2  {\pi^{ab}}_{|b}}_{R^i} \right)
\end{eqnarray}
which is the standard ADM Lagrangian density.

The ADM action is then
\begin{equation}\label{eq:ADM_action}
   I = \frac{1}{16 \pi} \int dt d^3x \left[ {\dot h}_{ab} \pi^{ab}
                                            - N R^0 - N_i R^i \right]
\end{equation}
where the change in index notation was done to emphasize the spatial nature of
$N_i$ and $R^i$.


\subsection{The Variations}\label{SS:Variations}

In this section we present the variations of the action that lead to the
ADM field equations.  The variations are with respect to the shift, the lapse,
the 3-metric and its conjugate momentum.


\subsubsection{Varying the Shift and Lapse}

In canonical form, the ADM action (\ref{eq:ADM_action}) contains two non-
dynamical entities, the shift and the lapse, which act effectively as Lagrange
multipliers.  Variation of the ADM action with respect to these parameters
yields the initial value equations
\[
   R^0 = -\sqrt{h} \left[ {}^{(3)}R + h^{-1} \left( \frac{1}{2} {\pi}^2 -
                                                Tr \left( \mathbf{\pi}^2 \right)
                                             \right)
                   \right] = 0
\]
and
\[
   R^i = -2 {\pi^{ij}}_{|j} = 0 \eqp
\]


\subsubsection{Varying the Conjugate Momenta}

The next variation to consider is that of the conjugate momenta.  This variation
is relatively straightforward and is carried out in two steps.  The first step
involves taking the variation of $N R^0$ with respect to $\pi^{ij}$ which results
in
\[
\frac{ \delta N R^0}{\delta \pi^{ij}} = \frac{2 N}{\sqrt{h}}
                                      \left(
                                        \pi_{ij} - h_{ij} \frac{\pi}{2}
                                      \right) \eqp
\]
the second step involves taking the variation of $N_i R^i$ with respect to $\pi^{ij}$
, noting in the course of the variation, that from its definition $\pi^{ij}$ must
be symmetric.  Thus
\[
   \frac{\delta N_i R^i}{\delta \pi^{ij}} = D_{(i} N_{j)}
\]
and the variation results in the equation for the time-derivative of the 3-metric
as
\[
   {\dot h}_{ij} = \frac{2 N}{\sqrt{h}} \left( \pi_{ij} - h_{ij} \frac{\pi}{2}
                                        \right) + D_{(i} N_{j)} \eqp
\]
\subsubsection{Varying the 3-Metric}

Finally, the variation of the ADM action with respect to the 3-metric $h_{ij}$
must be performed.  Due to is complexity, this variation is best performed
on each term in the action and the results combined at the end.  Starting with
$R^0$, the variation is

\begin{eqnarray*}
   \frac{\delta N R^0}{\delta h_{ij}} & = &
   \underbrace{\frac{\delta}{\delta h_{ij}} \left( N \sqrt{h} {}^{(3)}R \right)}
              _{A1}
  +\underbrace{\frac{\delta}{\delta h_{ij}} \left( \frac{N \pi^2}{2 \sqrt{h}} \right)}
              _{A2} \\
  & &
  -\underbrace{\frac{\delta}{\delta h_{ij}} \left( \frac{N Tr \left(\pi^2 \right)}
                                                   {\sqrt{h}} \right)}
              _{A3} \eqp
\end{eqnarray*}
The term labeled $A1$ yields two basic terms

\[
   A1 = N \sqrt{h} \left[ \frac{1}{2} {}^{(3)}R h^{ij} - {}^{(3)}R^{ij} \right]
        + N \sqrt{h} h^{k \ell} \frac{\delta R_{k \ell}}{\delta h_{ij}} \eqp
\]
the first term arise directly from the variations of the metric factors $\sqrt{h}$
and the $h^{ab}$ in the contraction to form the Ricci scalar (as it does in
the conventional Hilbert action).
The second term
\[
  N \sqrt{h} h^{ab} \delta R_{ab} = N \sqrt{h} h^{ab}
                                   \left(
                                     \delta {\Gamma^c}_{ab;c} -
                                     \delta {\Gamma^c}_{ac;b}
                                   \right)
\]
normally vanishes under a covariant integration by parts when $N=constant$.  This can
be seen by writing out the first term (with $N=1$)
\[
  \sqrt{h} h^{ab} \delta {\Gamma^c}_{ab;c} = \left(
                                               \sqrt{h} h^{ab} \delta {\Gamma^c}_{ab}
                                             \right)_{;c} -
                                             \left(
                                               \sqrt{h} h^{ab}
                                             \right)_{;c} \delta {\Gamma^c}_{ab}
\]
as representative.
Since both $\sqrt{h}$ and $h^{ij}$ are covariantly constant, the second term is
identically zero.  The first term is a total divergence and can be dropped.
However, the presence of the lapse function yields a `boundary' term of the form
\begin{equation}\label{eq:piece_of_A}
   \sqrt{h} h^{ab} \left(   N_{;b} \delta {\Gamma ^c}_{ac}
                          - N_{;c} \delta {\Gamma ^c}_{ab} \right) .
\end{equation}
Using the standard expression for the connection coefficients
${\Gamma^a}_{bc}$ their variations can be expressed as
\[
   \delta {\Gamma^c}_{ab} = \delta \left[ \frac{1}{2} h^{cs}
                            \left( h_{sa,b} + h_{sb,a} - h_{ab,s} \right) \right]
\]
and \
\[
   \delta {\Gamma^c}_{ac} = \delta \left[ \frac{1}{2} h^{c s} h_{cs,a} \right]
\]
Now to simplify the calculations, Riemann normal coodinates are
employed to assume $h_{cs,a} = 0$.
The first part of
(\ref{eq:piece_of_A}) becomes
\begin{eqnarray}
\sqrt{h} h^{a b} N_{;b} \delta {\Gamma^c}_{ac}
& = & \sqrt{h} h^{a b} N_{;b} \frac{1}{2}
              \left[  -h^{c q} \delta h_{q p} h^{p s} h_{c s,a} \right. \nonumber \\
&   &         \left.  + h^{c s} \delta h_{c s,a} \right]
\nonumber \\
& = &    \frac{1}{2} \sqrt{h} h^{a b} N_{;b} h^{c s} \delta h_{c s, a}
\nonumber \\
& = &   -\frac{1}{2} \left( \sqrt{h} N^{;a} h^{c s} \right)_{,a} \delta h_{c s}
\nonumber \\
& = &   -\frac{1}{2} {N^{;a}}_{,a} \sqrt{h} h^{c s} \delta h_{c s}
\nonumber \\
&   &   - \frac{1}{2} N^{;a} \left( \sqrt{h} h^{c s} \right)_{,a}
\delta h_{c s}
\nonumber \\
& = &   -\frac{1}{2} {N^{;a}}_{;a} \sqrt{h} h^{c s} \delta h_{c s}
\end{eqnarray}
where in the last line the `comma-to-semicolon' rule was applied.
Continuing in this fashion and making the common notational switch
from $Q_{;b}$ to $Q_{|b}$ to emphasize the three-dimensional
nature of the computation, the $A1$-term becomes
\begin{eqnarray*}
   A1 & = & N \sqrt{h} \left[ \frac{1}{2} {}^{(3)}R h^{ij} - {}^{(3)}R^{ij} \right] \\
      &   &  + \sqrt{h} \left( N^{|i|j} - h^{ij} {N^{|k}}_{|k} \right) \eqp
\end{eqnarray*}
The $A2$ and $A3$ terms become (taking into account the `hidden' $h_{ij}$'s in
the lowered indices on the $\pi^{ij}$'s in the various contractions
\[
   A2 = N \left( \pi^{ij} \frac{\pi}{\sqrt{h}} - \frac{\pi^2}{4 \sqrt{h}} h^{ij}
          \right)
\]
and
\[
   A3 = \left(
               \frac{-2 N}{\sqrt{h}} {\pi^i}_{k} \pi^{kj} +
               \frac{N}{2 \sqrt{h}}  Tr\left( \pi^2 \right) h^{ij}
        \right) \eqp
\]
The final piece is the variation of $R^i$ which from a naive look seems to be
zero since neither $\pi^{ij}$ of $N_{i}$ depend on $h_{ij}$.  However the
covariant derivative operator does and the term gives a nontrivial result.
The calculation is simplified by noting that, from (\ref{eq:pi_def}), $\pi^{ij}$
is a tensor density.
This gives the following simple relation
\[
   \delta \left( {\pi^{ij}}_{|j} \right)= \pi^{jk} \delta {\Gamma^i}_{jk}
\]
which can be immediately used to get
\[
    \frac{\delta N_k R^k}{\delta h_{ij}}
= - \left( \pi^{ik} N^j \right)_{|k}
  - \left( \pi^{jk} N^i \right)_{|k}
  + \left( \pi^{ij} N^k \right)_{|k} \eqp
\]
Combining all of these results gives the equation for the time-derivative of
the conjugate momenta
\begin{eqnarray*}
{\dot \pi}^{ij} & = &
  - N \sqrt{h} \left( {}^{(3)}R^{ij} - \frac{1}{2} h^{ij} {}^{(3)}R \right) \\
  & &
  + \frac{N}{2\sqrt{h}} h^{ij} \left(
                                      Tr\left( \pi^2\right) - \frac{1}{2} \pi^2
                               \right) \\
  & &
  - 2 \frac{N}{\sqrt{h}} \left(
                                \pi^{ik}{\pi_k}^j - \frac{\pi}{2} \pi^{ij}
                         \right) \\
  & &
  + \sqrt{h} \left( N^{|ij} - h^{ij} {N^{|i}}_{|i} \right)
  + \left( \pi^{ij} N^k \right)_{|k} \\
  & &
  - \left( \pi^{ik} N^j \right)_{|k}
  - \left( \pi^{jk} N^i \right)_{|k} \eqp
\end{eqnarray*}

%%%%%%%%%%%%%%%%%%%%%%%%%%%%%%%%%%%%%%%%%%%%%%%%%%%%%%%%%%%%%%%%%%%%%%%%%%%%%%%
\section{The ADM `\ggm-K` Variant}\label{S:ADM_gk}

%%%%%%%%%%%%%%%%%%%%%%%%%%%%%%%%%%%%%%%%%%%%%%%%%%%%%%%%%%%%%%%%%%%%%%%%%%%%%%%
\section{The BSSN Field Equations}\label{S:BSSN}

%%%%%%%%%%%%%%%%%%%%%%%%%%%%%%%%%%%%%%%%%%%%%%%%%%%%%%%%%%%%%%%%%%%%%%%%%%%%%%%
\section{Gauge Choices for the Shift and Lapse}\label{S:shift_lapse}

%%%%%%%%%%%%%%%%%%%%%%%%%%%%%%%%%%%%%%%%%%%%%%%%%%%%%%%%%%%%%%%%%%%%%%%%%%%%%%%
\section{Boundary Conditions}\label{S:BCs}
