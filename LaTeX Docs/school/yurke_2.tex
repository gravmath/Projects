% ****** Start of file yurke2.tex ******
%
%   This file is the first draft of the paper detailing our approach
%   to the study of gravitational radiation using the simple
%   conservative damped oscillator model of yurke 
% 
%
%
%   updates:	 9/13/99 major reworking to include new
% 			     explanation of the radial and along-track
%				 forces - CS
%
%           	 10/8/99 inclusion of the balance incoming and 
%                outgoing radiation solutions - CS
%
%  

%**************************************************************
%
%  Preliminary matters
%
%**************************************************************
% \documentstyle[preprint,eqsecnum,aps]{revtex}
\documentstyle[eqsecnum,aps,epsfig]{revtex}
\bibliographystyle{prsty}
\newcounter{bean}
\def\.{{\quad .}}
\def\_.{{\quad .}}
\def\_,{{\quad ,}}

%**************************************************************
%
%  Begin the document
%
%**************************************************************
\begin{document}
\draft



\twocolumn[\hsize\textwidth\columnwidth\hsize\csname
@twocolumnfalse\endcsname
\preprint{GR-QC/????}

%**************************************************************
%
%  Front matter
%
%**************************************************************
\title{Determining Radiation of Adiabatically Damped Motion 
       Via a Simple Model}
%
\author{C. Schiff and C. W. Misner}
\address{
            Department of Physics, University of Maryland,
            College Park MD 20742-4111 USA\\
           \rm
         e-mail: \tt  cmschiff@erols.com\\
                      misner@umail.umd.edu\\
					  Revision 2.0
}
\date{\today}
\maketitle


%**************************************************************
%
%  Abstract
%
%**************************************************************
\widetext
\begin{abstract}
We present a method for determining approximately the radiative 
loss due the coupling of a mechanical system with a field in the 
case where the motion of the mechanical system can be 
characterized as circular with an adiabatically changing radius. 
The mechanical system, which is coupled to a pair of infinite 
strings, is driven in circular motion by a motor.
The motion of the strings obeys the linear wave equation subject 
to outgoing-wave boundary conditions.  
The motor, which continuously provides energy to replace the loss 
radiated away via the strings, can be used to approximately 
determine the relationships between the radius, the angular 
frequency, and the radiated energy.  
For a given circular radius, the oscillator is driven at its 
natural (undamped) frequency when the radial component of the 
force due to the motor is zero.  
The along-track component of the force then uniquely determines 
the the amount of damping present and as a consequence the amount 
of energy being radiated.
This method, which grew out of our analysis of the coupled 
harmonic oscillator-string model studied by Yurke (1984), is 
applicable to damped circular motion in an arbitrary central 
potential.  
We believe that the model offers valuable insight into possible 
numerical approaches for calculating the gravitational radiation 
produced by a binary neutron star system during the the stages of 
inspiral prior to the final plunge and coalescence.
\end{abstract}
\pacs{0425.-g,04.25.Dm,04.30.-w,04.30.Db,04.30.Nk}

\narrowtext
\vskip2pc]


\section{Introduction}\label{sec:intro_level1}

The inspiral of two compact objects (\emph{e.g} neutron stars), 
within the framework of general relativity, is expected to be a 
prominent source of weak gravitational waves (see section 36.6 
of \cite{mtw}).  
Due to the lack of symmetry and the inherent nonlinearities, 
analytical solutions have been mostly confined to the relatively 
weak-field post-Newtonian regime.  
Full numerical simulations of the final stages where the binary 
approaches the last stable circular orbit and begins to coalesce, 
have been undertaken by the Grand Challenge.  
However, these simulations are not stable for more than a few 
orbits in the weak- to moderate- field regimes \cite{whelan}.
Thus there is a `gap' regime in the study of a radiating, 
inspiralling compact-object binaries.

We present a method, based on a simple model from classical 
physics, that offers insights into how this `gap' may be filled 
by a numerical scheme which approximately determines the energy 
loss due to the gravitational radiation produced during most of 
an inspiral.
In addition, our method is compatible with existing codes for solving
either Einstein's equations or simpler scalar theories of gravity.
Central to our method is the fact that motion in this regime is 
essentially circular \cite{peters63,peters64}.
In addition, the energy loss per orbit is small compared to the 
binding energy of the system.  
These observations lead naturally to a frame that rotates with the 
binary's orbital angular velocity in which the compact objects 
are quasi-stationary.  
This approach stills places a heavy burden on numerical codes, 
since equations of the motion for both the compact objects and 
the gravitational field must be solved.  
An ideal approach is one in which the compact objects can be 
considered truly stationary without losing the gravitational 
radiation produced because they are not.
Our solution is to allow a fictitious driving force to act on the 
binary, instantaneously supplying just enough energy to counteract 
the loss due to gravity.
Analysis of the components of the force in the rotating frame 
yields the desired relationship between the radius, the angular 
frequency, and the energy loss.  

To motivate this approach, we discuss an analogous model from 
classical mechanics before we sketch out some of the aspects of 
the numerical approach used for relativity.  
The classical mechanics model benefits from both is calculation 
simplicity and the fact that one's intuition is easier to apply.

\section{The Simple Radiative Model}

We begin our presentation of the simple classical mechanics model 
by summarizing the main ingredients our model must have to be 
analogous with the binary inspiral.
The important qualitative features of the inspiraling binary 
system are:

\begin{list}
   {$\bullet$}{\usecounter{bean}
    \setlength{\rightmargin}{\leftmargin}}
	\item Conservation of angular momentum 
 	\item Nearly circular motion throughout much of the evolution
	\item Coupling of the mechanical motion to a field medium that can radiate 
	      energy away
	\item Nonlinear behavior near the source
	\item Linear behavior at infinity
\end{list}

Yurke has presented a model of a conservative system consisting 
of a one-dimensional mechanical oscillator coupled to a single string, 
semi-infinite in extent.
This system locally shows dissipation in the motion of the 
oscillator through the imposition of outgoing wave 
boundary conditions \cite{yurke84}.
We modify this model by allowing the mechanical body, of mass $m$, 
to move in a general (\emph{i.e.} nonlinear) central potential in 
a plane perpendicular to pair of semi--infinite strings (see 
Figure 1).  
\begin{figure}
\begin{center}
\epsfig{file=yurke_dev.eps, width = 4.0 in}
\end{center}
\caption{test}
\end{figure}
The central potential guarantees the conservation of angular 
momentum and the strings provide coupling to a medium which can 
radiate energy from the mechanical system.  
The central body's dynamical state is described by its location 
${\vec R}(t) = \left[ X(t), Y(t) \right]$ in the x-y plane as a 
function of time.  
The dynamical state of the strings is analogously described by 
the transverse amplitudes ${\vec r}(z,t) = \left[ x(z,t), y(z,t) 
\right]$.  
Consistency is guaranteed at $z = 0$ by
\[
  {\mathcal L}_{n} h_{ab}
\]
\begin{eqnarray}\label{eq:consist_cond}
	x(0,t) & = & X(t) \nonumber \\
   y(0,t) & = & Y(t)  
\end{eqnarray}

The strings are assumed to obey the linear wave equation
\begin{equation}\label{eq:wave_eq}
	\mu \frac{\partial ^2 {\vec r}}{\partial t^2} 
       - T  \frac{\partial ^2 {\vec r}}{\partial z^2} 
	= 0
\end{equation}
where $\mu$ is the string's mass per unit length and $T$ is the 
tension.  
In addition to the above modifications, we allow a wider range of 
boundary conditions in our model than used by Yurke.  
Our primary focus will be on the radiation losses, which are 
modeled with outgoing wave boundary conditions 

\[
	{\vec r}(z,t) = {\vec r} \left( t - \frac{|z|}{c} \right)
\]

(where $c = \sqrt{ \frac{T}{\mu}}$ is wave speed along the string) 
on both transverse amplitudes.
However, we will impose a mixture of incoming and outgoing 
wave boundary conditions on each component of the string's 
amplitude.  
The corresponding wave form is then
\begin{equation}\label{eq:mixed_bcs}
  {\vec r}(z,t) = \alpha {\vec r}\left( t - \frac{|z|}{c} \right) 
                + \left(1 - \alpha \right){\vec r} \left( t + \frac{|z|}{c} \right)
\end{equation}
where the real parameter $0 \le \alpha \le 1$ controls the 
mixture.  Outgoing wave boundary conditions are then recovered by 
setting $\alpha = 1$ 

 
\subsection{Deriving the Equations of Motion}

As discussed above, the mechanical body moves within a central 
potential $V(R) \equiv V(|\vec R|)$.
The force is then derived from the gradient as

\[
	\vec F = - \nabla V( R )
\]

\noindent Using the chain rule and the definition of the radius 
$R = \sqrt{X^2 + Y^2}$ the central force can be written as 
\begin{equation}\label{eq:F_pot}
  \vec F = -\frac{1}{R} \frac{d V(R)}{d R} {\vec R}(t)
\end{equation}
In order to calculate the motion of the body, the forces provided 
by the strings must also be taken into account.  
Assuming that the strings are held at a constant tension $T$, 
the force they exert on the oscillator is

\begin{equation}\label{eq:F_tension}
   {\vec F}_{tension} = T \left[ \left.\frac{d}{ds} \right|_{right} + 
                          \left.\frac{d}{ds}  \right|_{left} \right]_{z=0}
\end{equation}

The tension along the right string $\left. T \frac{d}{ds} 
\right|_{right} $ can be expressed in a coordinate basis as 

\[
	\left. T \frac{d}{ds} \right|_{right}  = T \left[ 
			\frac{dx}{ds}\frac{\partial}{\partial x} +
		        \frac{dy}{ds}\frac{\partial}{\partial y} +
		        \frac{dz}{ds}\frac{\partial}{\partial z}
			\right]
\]

In addition, we will approximate that the strings make a small 
deflection from the horizontal allowing us to express the 
right-ward tension in its final form

\[
	\left. T \frac{d}{ds} \right|_{right}  = T \left[ 
			\frac{dx}{dz}\frac{\partial}{\partial x} +
		        \frac{dy}{dz}\frac{\partial}{\partial y} +
		        \frac{\partial}{\partial z}
			\right]
\]

A similar calculation for the left string gives the same form 
except for a sign change on the last term.  
Combining these with (\ref{eq:F_tension}) gives

\begin{eqnarray}\label{eq:mixed_motion}
 	m {\ddot {\vec R} }(t) & = & -\frac{1}{R} \frac{d V(R)}
	                            {d R} {\vec R}(t) \nonumber \\
			& & +  T 
			\left[ 
			\left. \frac{\partial \vec r}{\partial z} \right|_{z = 0^+} 
		     -  \left. \frac{\partial \vec r}{\partial z} \right|_{z = 0^-}
			\right]
\end{eqnarray}

\noindent for the equation of motion for the mechanical body.  
This form is awkward due to the appearance of the partial 
derivatives of the string amplitudes on the right-hand side.
These can be eliminated by using the consistency condition
(\ref{eq:consist_cond}) with the mixed boundary conditions
(\ref{eq:mixed_bcs}) 
\begin{eqnarray*}
	\frac{\partial {\vec r}}{\partial z} & = &   \frac{-\alpha \, signum(z)}{c} \, 
	                                       \frac{\partial {\vec r}}{\partial t}
										   \left(t - \frac{|z|}{c} \right) \\
										 &   & +
										   \frac{(1-\alpha) \, signum(z)}{c} \,
	                                       \frac{\partial {\vec r}}{\partial t}
										   \left(t + \frac{|z|}{c} \right)
\end{eqnarray*}
to yield
\[
	\left. \frac{\partial {\vec r}}{\partial z} \right|_{z=0^{\pm}} 
	= \pm \left( 1 - 2 \alpha \right)\frac{{\dot {\vec R}}(t)}{c} 
\]

Substituting into (\ref{eq:mixed_motion}) gives the equation of 
motion for the mechanical oscillator 

\begin{equation}\label{eq:osc_motion}
    {\ddot {\vec R}}(t) - 2 \Gamma \left(1 - 2 \alpha \right) {\dot {\vec R}}(t)  
   +\frac{1}{R} \frac{d V(R)}{d R} {\vec R}(t) = 0
\end{equation}
where the damping parameter $\Gamma = \frac{T}{cm}$ is defined in 
terms specified physical parameters of both the mechanical body 
and the radiative system.  Defining the impedance of a string
as $Z = \frac{T}{c}$ \cite{pain} shows that the damping is
characterized by the ratio of the impedance of the string to 
the mass of the mechanical body.
Equations (\ref{eq:consist_cond}-\ref{eq:mixed_bcs}) together with 
(\ref{eq:osc_motion}) provide the complete equations of motion for 
the combined string-oscillator system.  

These equations can be further condensed to 
\begin{eqnarray}\label{eq:equ_motion}
    \left[ \mu + m \delta (z) \right] \frac{\partial ^2}{\partial t^2} {\vec r}
& = &  T \frac{\partial ^2}{\partial z^2}{\vec r} \nonumber \\
&   & \left(-\frac{1}{R} \frac{d V(R)}{d R} \right) 
      \, {\vec r} \, \delta (z)
\end{eqnarray}
by using the Dirac $\delta$-function.  
This form emphasizes the coupled nature of the mechanical 
body-string system and leads directly to a variational formulation.  
Using it, we note that the equations can be 
derived from a variation of the action
\begin{eqnarray}
  I & = & \int dt L(t) \nonumber \\ 
    & = & \int dt dz 
      \left[
	    \frac{1}{2}
	    \left(\mu + m \delta(z)\right)
        \left( {x_{,t}}^2 + {y_{,t}}^2 \right) \right. \nonumber \\
	&   &
		\left.
	   -\frac{T}{2} \left( {x_{,z}}^2 + {y_{,z}}^2 \right)
       - V(R) \delta(z)
  	  \right]
\end{eqnarray}
with respect to $x(z,t)$ and $y(z,t)$.
The corresponding energy functional
\begin{eqnarray}
  E & = & \int dz 
      \left[
	    \frac{1}{2}
	    \left(\mu + m \delta(z)\right)
        \left( {x_{,t}}^2 + {y_{,t}}^2 \right) \right. \nonumber \\
	&   &
		\left.
	   +\frac{T}{2} \left( {x_{,z}}^2 + {y_{,z}}^2 \right)
       + V(R) \delta(z)
  	  \right]
\end{eqnarray}
is formally infinite, when integrated over the entire range of $z$.
However, the energy in the interval $-L \leq z \leq L$ is 
well-defined.  
Using this expression and the assuming the equations of motion 
are satisfied, the change in the energy in this interval is given 
in terms of the Poynting-like flux by
\[
\frac{dE}{dt} = T \left[x_{,t} x_{,z} + y_{,t} y_{,z}\right]{\left.{}^{} \right|}_{z=-L}^{z=L}	
\]
Applying the mixed boundary conditions yields
\begin{eqnarray}\label{eq:Poynt_1}
\frac{dE}{dt} & = &  \frac{2T}{c}\left[ -\alpha
                                               \left( {x_{,t}}^2 + {y_{,t}}^2 \right)
                                                     (t - |z|/c)
								  \right. \nonumber \\
			  &   & 			+ 
			                      \left. 
								   (1 - \alpha) \left( {x_{,t}}^2 + {y_{,t}}^2 \right)
								                     (t + |z|/c)
							\right]_{z=L}
\end{eqnarray}
where $x$ and $y$ are solutions of the form of 
(\ref{eq:mixed_bcs}).


%%%%%%%%%%%%%%%%%%%%%%%%%%%%%%%%%%%%%%%%%%%%%%%%%%%%%%%%%%%%%%%%%%
%            Stationary Circular Motion
%%%%%%%%%%%%%%%%%%%%%%%%%%%%%%%%%%%%%%%%%%%%%%%%%%%%%%%%%%%%%%%%%%
\section{Solving the Equations of Motion for Steady Circular Motion}

In this section, we discuss solutions to the equations of motion 
(\ref{eq:mixed_motion}) that allow stationary circular motion.
Stationary circular motion is characterized by the kinematical 
condition
\begin{equation}\label{eq:R_circ}
{\vec R}(t) = \rho cos(\omega t) {\hat \imath} + \rho sin(\omega t) {\hat \jmath}
\end{equation}
where ${\hat \imath}$ and ${\hat \jmath}$ are unit vectors lying 
along the x- and y-axes respectively and $\rho$ is the constant
radius of the circle.
Our approach will be to start with (\ref{eq:mixed_motion}) and 
impose different types of wave boundary conditions.
If the resulting vector equation of motion for the mechanical body 
is consistent with (\ref{eq:R_circ}) then the stationary circular 
motion will result.  
We will show that these types of solutions can be obtained in 
two distinct ways.
In the first way, the outgoing radiation can be exactly balanced 
by incoming radiation from infinity.
In the second way, the energy carried away by the outgoing 
radiation is replaced by a motor.
We discuss both of these physical systems in turn below.   

%%%%%%%%%%%%%%%%%%%%%%%%%%%%%%%%%%%%%%%%%%%%%%%%%%%%%%%%%%%%%%%%%%
%            Mixed Outgoing and Incoming Wave Boundary Conditions
%%%%%%%%%%%%%%%%%%%%%%%%%%%%%%%%%%%%%%%%%%%%%%%%%%%%%%%%%%%%%%%%%%
\subsection{Mixed Outgoing and Incoming Wave Boundary Conditions}

Equation (\ref{eq:osc_motion}) is the starting point for our
analysis of the motion with mixed wave boundary conditions.
Combining this equation with the kinematical condition
(\ref{eq:R_circ}) we arrive at the matrix equation
\[
            \underbrace{
                  \left[
			       \begin{array}{cc}
				   {\omega _{\rho} }^2 - \omega ^2 & - 2 \Gamma (1 - 2 \alpha) \omega \\
                   - 2 \Gamma (1 - 2 \alpha) \omega & {\omega _{\rho} }^2 - \omega ^2 				    
    		       \end{array}
			      \right]
				 }_{M}
				  \left(
				   \begin{array}{c}
				    \rho \cos ( \omega t) \\
					\rho \sin ( \omega t)
				   \end{array}
				  \right)
				  = 
				  0
\] 
where ${\omega _{\rho}}^2$ is the constant defined by
\[
  {\omega _{\rho}}^2 = \left(\frac{1}{R} \frac{dV}{dR}\right)_{R=\rho}
\]
This equation has a nontrivial solution only when $\det(M) = 0$.  
Carrying out the determinant yields a bi-quadratic equation for the
rotation frequency $\omega$.
Solving the equation gives
\[
  \omega = \pm i \, \Gamma (1-2\alpha)  
           \pm \sqrt{ {\omega _{\rho}}^2 - \Gamma ^ 2 (1 - 2 \alpha)^2}
\]
for the rotation frequency.
Real solutions for $\omega$ are obtained by setting $\alpha = 1/2$
giving an exact balance between incoming and outgoing radiation.
The resulting rotation frequency is then given by
\[
  \omega^2 = \left( \frac{1}{R} \frac{dV}{dR} \right)_{R = \rho}
\]
and the corresponding equation of motion is given by
\begin{equation}\label{eq:osc_motion_bal}
 {\ddot {\vec R}}(t) 
  + \left( \frac{1}{R} \frac{\partial V(R)}{\partial R} \right)_{R=\rho} {\vec R}(t) = 0
\end{equation}

At this point, it is worth asking if (\ref{eq:mixed_bcs}) offer the most general
solutions consistent with the boundary conditions.  
It is easy to see, in fact, that (\ref{eq:consist_cond}) and (\ref{eq:wave_eq})
are satisfied by the inclusion of a set of traveling wave 
solutions of the form
\[
  {\vec h}(z,t) = {\vec h}\left(t - \frac{z}{c}\right) 
                - {\vec h}\left(t + \frac{z}{c}\right)
\] 
However, the inclusion of these terms adds an inhomogeneous term to 
the equation of motion, transforming it to
\begin{equation}\label{eq:osc_motion_homo}
 {\ddot {\vec R}}(t) 
  + \left( \frac{1}{R} \frac{\partial V(R)}{\partial R} \right)_{R=\rho} {\vec R}(t) = 
   - 4 \Gamma {\vec h}'(t)
\end{equation}
where the prime denotes differentiation with respect to the argument 
$t \pm z/c$ followed by an evaluation at $z=0$.
Since the inhomogeneous term is incompatible with the kinematical 
condition (\ref{eq:R_circ}) the traveling waves must be rejected. 
 
%%%%%%%%%%%%%%%%%%%%%%%%%%%%%%%%%%%%%%%%%%%%%%%%%%%%%%%%%%%%%%%%%%
%            Motorized Circular Motion
%%%%%%%%%%%%%%%%%%%%%%%%%%%%%%%%%%%%%%%%%%%%%%%%%%%%%%%%%%%%%%%%%%
\subsection{Motorized Circular Motion}

Finally, we allow the inclusion of an external force, 
${\vec F}_{driving}$, acting solely on the mechanical body 
provided by a motor.
Pure outgoing wave boundary conditions will be imposed 
(\emph{i.e.} $\alpha =1$).
Equation (\ref{eq:osc_motion}) can then be expressed as 
\begin{equation}\label{eq:forced_osc_motion}
    {\ddot {\vec R}}(t) + 2 \Gamma {\dot {\vec R}}(t) + 
\left(\frac{1}{m r} \frac{\partial V(r)}{\partial r} \right) {\vec R}(t) = 
\frac{{\vec F}_{driving}}{m}
\end{equation}
Inserting the kinematical condition (\ref{eq:R_circ}) leads directly 
to the vector equation 
\begin{equation}\label{eq:F_vector}
\frac{{\vec F}_{driving}}{m} =
                  \left[
			       \begin{array}{c}
				    \left(
					  {\omega _{\rho}}^2 - \omega^2
					\right) \rho \cos(\omega t) 
					  - 2 \Gamma \rho \omega \sin( \omega t)\\
				    \left(
					  {\omega _{\rho}}^2 - \omega^2
					\right) \rho \cos(\omega t) 
					  - 2 \Gamma \rho \omega \sin( \omega t)\\
    		       \end{array}
			      \right]
\end{equation}
where ${\omega _{\rho}^2}$ is as defined above.
However, the inertial components of the driving force in 
(\ref{eq:F_vector}) do not lend themselves to the simple physical 
interpretation that obtains in a coordinate system that rotates 
with the mechanical oscillator.  
In this coordinate system, one axis, the radial axis, always lies 
along the line joining the origin to the oscillator. 
The other axis, the along-track axis, lies in the plane of the 
motion at a right-angle to the radial axis.  
The radial component of the driving force $F_r$ is obtained from 
\begin{eqnarray}\label{eq:F_r}
	F_r & = &  
		\frac{ {\vec F}_{driving} \cdot {\vec r} }{|{\vec r}|} \nonumber \\
            & = & m \rho \left(
			               {\omega _{\rho}^2} - \omega^2
                         \right)  
\end{eqnarray}

and along-track component of the driving force $F_a$ is obtained 
from

\begin{eqnarray}\label{eq:F_a}
	F_a & = &  
		\frac{ {\vec F}_{driving} \cdot {\vec v} }{|{\vec v}|} \nonumber \\
            & = & 2 m \Gamma \omega \rho
\end{eqnarray}
Applying the outgoing boundary conditions yields
\begin{equation}\label{eq:Poynt_1_forced}
\frac{dE}{dt} = \frac{-2T}{c}\left[{x'}^2 + {y'}^2 \right]_{z=L}
\end{equation}
where $x$ and $y$ are solutions of the form of 
(\ref{eq:mixed_bcs}) with $\alpha = 1$ and the prime denotes 
differentiation with respect to the argument $t - \frac{|z|}{c}$.

Taking the limit $L \rightarrow 0$ in (\ref{eq:Poynt_1_forced})
gives 
\begin{equation}\label{eq:Poynt_2}
 \frac{dE}{dt} = \frac{-2T}{c}\rho^2 \omega^2 = -2 \Gamma m \rho^2 \omega^2
\end{equation}
as the energy loss per unit time of the mechanical body.
Equations (\ref{eq:F_r}) and (\ref{eq:F_a}) are the fundamental 
equations from which the natural frequency and the energy 
radiated for a given radius may be determined.  
In particular, the system is driven at its natural frequency, 
defined to be the frequency of a circular orbit in the absence of 
damping, for a given radius when $F_r = 0$.  
Measuring the remaining component of force $F_a$ immediately 
determines the damping parameter $\Gamma$, which represents the 
coupling between the mechanical and radiative systems.  
These calculations will be first presented in the context of the 
linear oscillator, where an exact analytic solution is readily 
obtained.
They will then be extended to the case of a nonlinear oscillator 
where comparisons are made between the theoretical approximations 
and the results from numerical experiments.  
Finally, we examine a Schwarzschild-like potential that possesses
a critical value of angular momentum below which the orbit inspirals to the
origin.  
This potential differs markedly from the previous potentials in 
that the instantaneous energy loss grows as the radius decreases 
rather than diminishes.

\subsubsection{The Linear Oscillator}

For the case of the linear oscillator the potential is given by 
\[
  V(r) = \frac{1}{2}k r^2
\]
and the natural frequency, $\omega_0$, is independent of radius.  
Equation (\ref{eq:F_r}) determines a unique driving frequency
\[
  \omega = \omega _0 = \sqrt{ \frac{k}{m}}
\]
 for 
all sized circular orbits.  
Before proceeding, however, it is instructive to look at equation 
(\ref{eq:forced_osc_motion}) with the driving force set to zero.  
The solution for initially circular motion is given as
\[
{\vec R}(t) = e^{-\Gamma t} 
                  \left[
			       \begin{array}{c}
			         \rho _0 \cos( \omega _D t) \\
					 \rho _0 \sin( \omega _D t) \\
    		       \end{array}
			      \right]
\]
where $\omega_D = \sqrt{\omega_0^2 - \Gamma^2}$.
Note that the absence of the driving force prevents stationary 
circular motion from occuring; $\omega_0$ would have to an 
imaginary component that just balanced the damping term given 
by $\Gamma$.
Since $\omega_0$ represents a real rotation frequency, stationary
motion without driving is unphysical.  
The energy of the mechanical system is given by
\begin{eqnarray*}
  E(t) & = & \frac{m}{2}{\dot {\vec R}}(t) + 
             \frac{1}{2} k {\vec R}(t) \\
       & = & k {\rho _0}^2 e^{-2 \Gamma t}
\end{eqnarray*}
and the radiative loss can be directly calculated to yield
\begin{eqnarray*}
  \frac{d E}{d t} & = & -2 \Gamma k {\rho _0}^2 e^{-2 \Gamma t} \\
                  & = & -2 \Gamma k r(t)^2
\end{eqnarray*}
The result of (\ref{eq:Poynt_2}) also gives this answer with the 
understanding that the stationary radius $\rho$ referred to in
(\ref{eq:Poynt_2}) is chosen to be the same value as the dynamic
$r(t)$ at the desired time.  
A tighter connection can be made as follows.
The energy of the mechanical system can be written a solely a 
function of the radius $r$ by using the kinematical condition 
(\ref{eq:R_circ}).  
Combining this relation with the instantaneous energy loss 
$\frac{dE}{dt}$ given by (\ref{eq:Poynt_2}) provides a differential 
equation 
\begin{equation}\label{eq:dr_dt}
 \frac{dr}{dt} = \frac{ \frac{dE}{dt} }{ \frac{dE}{dr}}
\end{equation}
for $r = r(t)$.  
In the case of the linear oscillator, the energy $E$ as a 
function of $r$ is 
\[
  E(r) = \frac{1}{2} m r^2 \omega^2 + \frac{1}{2}k r^2
\]
which yields 
\[
  \frac{dE}{dr} = 2 k r
\]
for the derivative.
The corresponding differential equation for the radius
\[
  \frac{dr}{dt} = -\Gamma r
\] 
has the familiar solution 
\[
  r(t) = \rho _0 e^{-\Gamma t}
\] 
which we derived by more elementary means above.

All of this analysis can be cast in an intuitive picture as 
follows.  
Imagine we run the motor driving the mechanical body and that 
someone else has attached the strings without disclosing the tension 
or linear mass density.  
Watching the motion (\emph{i.e.} kinematics) and monitoring the 
work done by our motor in both the radial and along-track 
directions allows us to deduce the impedance of the strings.  
By sampling different radii we can determine the amount energy 
radiated and by solving the equation for $r(t)$ we can 
reconstruct the entire dynamics.

\subsubsection{The Nonlinear Oscillator}

Very little changes in the transition from the linear to the nonlinear regime.
The potential we examined is given by 
\[
  V(r) = \frac{1}{2}k r^2 + \frac{\lambda}{4} r^4
\]
with $\lambda \ne 0$.
The natural frequency for circular orbits, $\omega_0$, obtained
from (\ref{eq:F_r}), is no longer independent of radius but is 
given by
\[ 
{\omega_{0}}^2 = \left( \frac{k}{m} + \frac{\lambda}{m} \rho^2 \right)
\]

Figure 2a shows the comparison between $\frac{d E}{d t}(\rho)$ obtained
from the quasi-stationary method compared to what was obtained by direct 
numerical integration.  Figure 2b shows the comparison between 
$\frac{d E}{d t}(t)$ reconstructed from the quasi-stationary method
with the results obtained from the numerical integration.

\subsubsection{The Schwarzschild-like Potential}

The final potential we examined
\[
	V(|\vec r|) = \frac{1}{2} - \frac{M}{r} - \frac{M \lambda}{r^3}
\]
was constructed to, as closely as possible, mimic the dynamics
arising in general relativity.  Analyzing the potential using the
standard techniques of classical mechanics, one obtains the 
effective potential  
\[
  V_{eff} = 1 - \frac{2 M}{r} + \frac{ {p_{\phi}}^2}{r^2} 
              - \frac{2 M \lambda}{r^3}
\]
where $M$ and $\lambda$ are freely specifiable parameters of the 
potential and $p_{\phi}$ is the conserved angular momentum.
This effective potential is formally the same as that arising 
from the Schwarzschild metric with the identification
$\lambda = p_{\phi}^2$. 
The extrema of the effective potential occur at
\[
  r_{\pm} = \frac{ p_{\phi}^2 \pm \sqrt{ p_{\phi}^4 - 
                                         12 M^2 \lambda
									   } 
				 }{2M}
\]
Real solutions occur only for orbits with angular momentum greater 
than the critical value
\[
  p_{\phi_{crit}} = \left( 12 M^2 \lambda \right)^{\frac{1}{4}}
\]
Figure 3 shows a set effective potentials with $p_{\phi}$ greater than, 
equal to, and less than $p_{\phi_{crit}}$.

Using (\ref{eq:F_r}) gives a natural frequency
\[
  \omega _0 = \sqrt{\frac{M}{\rho^5}\left(\rho^2 + 3 \lambda \right)  }
\]

 
\section{Application to General Relativity}

We conclude by sketching how our method would apply to a numerical simulation
in general relativity.  Central to the discussion above was the concept of using
the driving force as a method for determining the relationship between the 
radius, angular frequency, and the radiated energy.  The presence of the 
radiation, mediated by the strings, required a non-zero driving force to keep 
the oscillator in stationary circular motion at is natural frequency and 
radius.  Analogously, the presence of gravitational radiation would require a
driving force to halt the inspiral of two compact objects 
(\emph{e.g} neutron stars).  This driving force would signal its presence in 
general relativity by the failure of the center-of-masses for the compact 
objects to follow a geodesic.

The equation for geodesic motion
\[
   \nabla_{\mathbf{U}} \mathbf{U} = U^{b}_{;a} U^a = 0 \_,
\]
would then become
\[
   \nabla_{\mathbf{U}} \mathbf{U} = U^{b}_{;a} U^a = A^b \_,
\]
where $A^b$ is the identified with the four-acceleration.  In the rotating frame 
the two compact objects will appear fixed, yielding a stationary matter 
distribution 
\[
   \frac{\partial}{\partial t} T^{\mu \nu} = 0
\]
In this frame, the radial and along-track directions are 
unambiguously defined.  The method detailed above can then be applied;
%
\begin{list}
   {\arabic{bean})}{\usecounter{bean}
    \setlength{\rightmargin}{\leftmargin}}
	\item Pick a radius 
 	\item Vary the frequency until $F_r = 0$
	\item Calculate the work done by the driving force over one orbit 
	\item Estimated instantaneous energy loss equal to the work divided by 
	      the period
\end{list}
%
The frequency can be varied in a numerically efficient manner by the 
implementation of a Newton method for finding the root for $F_r = F_r(\omega) 
= 0$.  The energy loss per change in angular frequency $\frac{dE}{d \omega}$ can
be obtained with little extra effort by running through the above algorithm for
a given frequency $\omega$ and a nearby frequency $\omega + \delta \omega$.  The
derivative can then be approximated by a finite difference.



\bibliography{gr,oscil}
\end{document}

