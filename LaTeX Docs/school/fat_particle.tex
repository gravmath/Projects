% ****** Start of file fat_particle.tex ******
%
%   This file is the first draft of the paper detailing our approach
%   to the full-backreacting two fat particle system
% 
%
%
%   updates:	 10/29/99 first version of the paper - just the 
%                skeleton - CS
%                
%                11/6/99 major reworking - finished 11/14/99
%  

%**************************************************************
%
%  Preliminary matters
%
%**************************************************************
% \documentstyle[preprint,eqsecnum,aps]{revtex}
\documentstyle[eqsecnum,aps,epsfig]{revtex}
\bibliographystyle{prsty}
\newcounter{bean}
\def\.{{\quad .}}
\def\_.{{\quad .}}
\def\_,{{\quad ,}}

%**************************************************************
%
%  Begin the document
%
%**************************************************************
\begin{document}
\draft


\preprint{GR-QC/????}

%**************************************************************
%
%  Front matter
%
%**************************************************************
\title{Two Fat Particles}
%
\author{C. Schiff}
\address{
            Department of Physics, University of Maryland,
            College Park MD 20742-4111 USA\\
           \rm
         e-mail: \tt  cmschiff@erols.com\\
					  Revision 2.0
}
\date{\today}
\maketitle


%**************************************************************
%
%  Abstract
%
%**************************************************************
\begin{abstract}
\end{abstract}


\section{Discrete ADM}

What follows is my specifications for our two-fat particle
code.  
To ensure that we are all on the same page I've attempted to
make these `specs' self-contained.  
To this end, I'll start with baryon conservation as the 
symmetry that we will impose and I'll derive a form for the 
density.
Next I'll derive a spatially discrete form of Einstein equations
in their covariant formulation and show that the requirement
of consistency between the density definition derived from 
baryon conservation and the one derived from the variational
formulation uniquely determines how the grid-defined metric
functions are smoothed to the particle position.
Finally, I'll derive the discrete equations in ADM $3+1$ formulation.  

\subsection{Baryon Conservation}

The covariant form of baryon conservation is given by 
\[
  \left( \rho U^{\alpha} \right)_{;\alpha} = 0
\]
where $U^{\alpha}$ is the normalized four velocity 
(i.e. $U^{\alpha} U_{\alpha} = -1$) and $\rho$ is the 
density.
This equation can also take on the form
\[
  \frac{1}{\sqrt{-g}} \left( \rho \sqrt{-g} U^{\alpha} \right)_{,\alpha} = 0 \, .
\]
By integrating the above equation over the spatial variables we
obtain the conservation law
\[
  \int \, d^3 x \rho(x) \sqrt{-g(x)} {U^{0}}(x) = const = \sum _A m_A
\]
where the constant $const$ is taken to be the sum of the rest masses.

As we will eventually move to $3+1$, we can regard  
the coordinate velocity as computationally fundamental.  
With this in mind we recast the zeroth component of the four-velocity
\[
  U^0 = \frac{1}{\sqrt{ -g_{\mu \nu} 
                        \frac{d z^{\mu}}{d \lambda}
                        \frac{d z^{\nu}}{d \lambda}				  
    	              } }
		\frac{d z^0}{d \lambda}
\]
as
\[
  U^0 = \frac{1}{ \frac{d z^0}{d \lambda} \sqrt{ -g_{\mu \nu} 
                        \frac{d z^{\mu}}{d z^0}
                        \frac{d z^{\nu}}{d z^0}				  
    	              } }
		\frac{d z^0}{d \lambda} 
	  \equiv \frac{1}{ \sqrt{ -g_{\mu \nu} V^{\mu} V^{\nu} } } 
\]
where the four-dimensional coordinate velocity is defined as
\[
  V^{\mu} {\dot = } \left(1, \frac{d z^i}{d z^0} \right) \equiv \left(1, V^i \right) \, .  
\]
Substituting this form into the conservation integral yields
\[
  \int \, d^3 x \rho(x)
                \left( 
                \sqrt{
                      \frac{ g } {g_{\mu \nu} V^{\mu} V^{\nu}}  
                     }
				\right)(x)  = \sum _A m_A
\]
which can be satisfied, identically, by the density definition
\[
  \rho(x) = \left(
            \sqrt{
                   \frac{g_{\mu \nu} V^{\mu} V^{\nu}}{g}
                 }
		    \right)(x)
            \sum _A m_A \delta^3(\vec x - \vec z_A) \, .
\]
Discretization of the spatial coordinates is accomplished through the  
following prescription.  
Firstly the integral becomes a sum over a grid points 
\[
  \int \, d^3 x \rightarrow \sum _{\ell} k^3 
\]
denoted by 
a multi-index $\ell$ and where $k$ is the linear spatial extent
of a single cell.
The delta function is now approximated by a smoothing kernel
\[
  \delta^3(\vec x - \vec z_A ) = \frac{W(\vec x_{\ell} - \vec z_A)}
                                       {\sum _m k^3 W(\vec x_m - \vec z_A)}
\]
where $W(\vec x_{\ell} - \vec z_A)$ is non-zero only within a 
finite region.
Finally, we will regard the fluid to be comprised by 
discrete fluid elements.
The location of these elements will be given by the location
of their center-of-mass.
To prevent gravitational collapse, their mass will be
smoothed out onto the grid and we will assume that there is no overlap
between two smoothing kernels.
The coordinate velocity at a grid point is then set equal to
the center-of-mass coordinate velocity of the particle whose 
smoothing kernel envelopes the grid point.  
Those grid points which are enveloped by no kernel have a zero 
coordinate velocity.  
These conditions are concisely summarized by 
\[
  V^{\mu}( \vec x_{\ell},x^0) =    \sum _A {V_A}^{\mu}(x^0) 
                                            \Theta \left[ W(\vec x_{\ell} - \vec z_A) 
											       \right]
\]
where 
\[
  \Theta(x) = \left\{
                     \begin{array}{cc}
					   1  & x > 0 \\
					   0  & otherwise
					 \end{array}
			   \right. \, .
\]
The density then becomes
\[
  \rho(\vec x_{\ell},x^0) = \left(
                              \sqrt{
                                    \frac{g_{\mu \nu} V^{\mu} V^{\nu}}{g}
                                    }
                              \right)(\vec x_{\ell},x^0)
            \sum _A m_A \frac{W(\vec x_{\ell} - \vec z_A)}
			                 {\sum _n k^3 W(\vec x_n - \vec z_A)} \, .
\]
which when substituted into the discrete form of the 
conservation integral is
\[
  \sum _{\ell} k^3 \rho( \vec x_{\ell},x^0)
                   \left( 
                   \sqrt{
				      \frac{g}{ g_{\mu \nu} V^{\mu} V^{\nu} } 
					    }
				   \right)(\vec x_{\ell},x^0)
				   =
				   \sum _A m_A \, .
\]


\subsection{Covariant Discretization}

The starting point is the combined gravitational and matter action 
\[
  I = I_{HE} + I_{matter}
\]
where the gravitational action is given by the usual
Hilbert-Einstein action
\[
  I_{HE} = \frac{1}{16 \pi} \int \, d^4 x \, R(x) \sqrt{ -g(x) } 
\]
and the matter action by
\[
  I_{matter} = - m_1 \int \, d \lambda_1 
                     \sqrt{ -g_{\mu \nu}(z_1(\lambda_1)) 
					        \frac{d {z_1}^{\mu}(\lambda_1)}{d \lambda_1} 
					        \frac{d {z_1}^{\nu}(\lambda_1)}{d \lambda_1} }
               - m_2 \int \, d \lambda_2 
                     \sqrt{ -g_{\mu \nu}(z_2(\lambda_2)) 
					        \frac{d {z_2}^{\mu}(\lambda_2)}{d \lambda_2} 
					        \frac{d {z_2}^{\nu}(\lambda_2)}{d \lambda_2} }\, .				
\]

Again using the arbitrariness of the path parameter in the
matter action, $\lambda_A$ can be eliminated in favor of
${z_A}^0$ to give
\[
  I_{matter} = - m_1 \int \, d {z_1}^0 
                     \sqrt{ -g_{\mu \nu}({z_1}^i({z_1}^0);{z_1}^0) 
					        {V_1}^{\mu}({z_1}^0) {V_1}^{\nu}({z_1}^0) } 
               - m_2 \int \, d {z_2}^0 
                     \sqrt{ -g_{\mu \nu}({z_2}^i({z_2}^0);{z_2}^0) 
					        {V_2}^{\mu}({z_2}^0) {V_2}^{\nu}({z_2}^0) }\, .				
\]

The combined action can be written in terms of the field variables $x^{\mu}$ 
\begin{eqnarray*}
  I & = & \int \, d^4 x \left\{ 
                         \frac{R(x) \sqrt{-g(x)} }{16 \pi}
						 -
						 \left( m_1 \int \, d {z_1}^0
						        \sqrt{ -g_{\mu \nu}({z_1}^i({z_1}^0);{z_1}^0) 
								        {V_1}^{\mu}({z_1}^0) 
										{V_1}^{\nu}({z_1}^0) }
						 \right)
						 \delta^4(x - z_1) \right .\\
	&   & 
    					 - \left.
						 \left( m_2 \int \, d {z_2}^0
						        \sqrt{ -g_{\mu \nu}({z_2}^i({z_2}^0);{z_2}^0) 
								        {V_2}^{\mu}({z_2}^0) 
										{V_2}^{\nu}({z_2}^0) }
						 \right)
						 \delta^4(x - z_2)
                    \right\} \, .
\end{eqnarray*}
The integration over $d {z_A}^0$ can be eliminated by using $\delta(x^0 - {z_A}^0)$
to yield
\begin{eqnarray*}
  I  & = &   \int \, d^4 x \left\{ 
                         \frac{R(x) \sqrt{-g(x)} }{16 \pi}
						 -
						 \left( m_1 
						        \sqrt{ -g_{\mu \nu}({z_1}^i(x^0);x^0) 
								        {V_1}^{\mu}(x^0) 
										{V_1}^{\nu}(x^0) }
						 \right)
						 \delta^3({\vec x} - {\vec z_1}) \right.\\
	 &    &
    					 - \left.
						 \left( m_2 
						        \sqrt{ -g_{\mu \nu}({z_2}^i(x^0);x^0) 
								        {V_2}^{\mu}(x^0) 
										{V_2}^{\nu}(x^0) }
						 \right)
						 \delta^3({\vec x} - {\vec z_2})
                    \right\} \.
\end{eqnarray*}
The above action is now ready for the discretization program discussed above.
However, a freedom exists in the above continuum expression that is absent
in the discrete form; namely where the metric functions are to be evaluated.
With the spatial $\delta$-functions present 
\[
  g_{\mu\nu}(\vec x,x^0)\delta(\vec x - \vec z_A) = g_{\mu\nu}(\vec z_A(x^0);x^0) 
                                                    \delta(\vec x - \vec z_A)  
\]
but when the spatial $\delta$-functions are transitioned to smoothing
kernels the above equation no longer holds.
Since after discretization the metric is only known at grid points, we
choose the left-hand form before smoothing the spatial $\delta$-functions.
Introducing the $K$-term 
\[
  K({V_A}^{\mu},g_{\mu\nu}(\vec x_{\ell},x^0)) 
               \equiv \sqrt {-g_{\mu \nu}(\vec x_{\ell},x^0) 
			                 {V_A}^{\mu}(x^0) {V_A}^{\nu}(x^0)
							}
\]
and the $S$-term
\[
  S_{(\ell,A)} = \frac{ W({\vec x}_{\ell} - {\vec z_A})}
                      {\sum _n k^3 W({\vec x}_{n} - {\vec z_A}) }
\]
for brevity, the discrete action becomes
\[
  I  =  \int \, d x^0 \sum _{\ell}  k^3 \left\{ 
                         \frac{R({\vec x}_{\ell},x^0) \sqrt{-g({\vec x}_{\ell},x^0)} }{16 \pi}
						 - m_1   K({V_1}^{\mu},g_{\mu\nu}(\vec x_{\ell},x^0)) 
						         S_{(\ell,1)} 
						 - m_2   K({V_2}^{\mu},g_{\mu\nu}(\vec x_{\ell},x^0))
						         S_{(\ell,2)} 
                    \right\} \, .
\]

Finally, varying the discrete action with respect to the metric functions defined 
at each grid point gives 
\begin{eqnarray*}
  \delta I \vert_{\delta g_{\alpha \beta}(\vec x_{\ell},x^0)} & = &  
        \int \, d x^0 \sum _{\ell}  k^3 \left\{ 
                         \frac{ -G^{\alpha\beta}({\vec x}_{\ell},x^0) 
						        \sqrt{-g({\vec x}_{\ell},x^0)} 
							  }
							  { 16 \pi } \right. \\
					&   & \left.
					  +  \frac{ m_1 {V_1}^{\alpha}(x^0) {V_1}^{\beta}(x^0) S_{(\ell,1)} }
					          { 2 K({V_1}^{\mu},g_{\mu\nu}(\vec x_{\ell},x^0))  }
					  +  \frac{ m_2 {V_2}^{\alpha}(x^0) {V_2}^{\beta}(x^0) S_{(\ell,2)} }
					          { 2 K({V_2}^{\mu},g_{\mu\nu}(\vec x_{\ell},x^0))  }
                    \right\} \delta g_{\alpha \beta}(\vec x_{\ell},x^0) \, .  
\end{eqnarray*}
Setting the variation to zero yields a discrete form of the Einstein equations
\[
  G^{\alpha\beta}({\vec x}_{\ell},x^0) = 8 \pi 
                                         \left\{
					     \frac{ m_1 {V_1}^{\alpha}(x^0) {V_1}^{\beta}(x^0) }
					          { 2 K({V_1}^{\mu},g_{\mu\nu}(\vec x_{\ell},x^0))  
							    \sqrt{-g({\vec x}_{\ell},x^0)}
							  }
							  S_{(\ell,1)} 
					  +  \frac{ m_2 {V_2}^{\alpha}(x^0) {V_2}^{\beta}(x^0) }
					          { 2 K({V_2}^{\mu},g_{\mu\nu}(\vec x_{\ell},x^0))  
							    \sqrt{-g({\vec x}_{\ell},x^0)}							  
							  }
							  S_{(\ell,2)} 
                                         \right\} \, .
\]
This can be put into a form in which the density can be read off as follows.
Use the relationship between the coordinate velocities defined at a particle 
and at a grid point to rewrite the above as
\[
  G^{\alpha\beta}({\vec x}_{\ell},x^0) = 8 \pi 
                                         \left(
										 \frac{K(V^{\mu}(\vec x_{\ell},x^0),
										             g_{\mu\nu}(\vec x_{\ell},x^0))
											  }
                 					          {                                           										 
                							    \sqrt{-g({\vec x}_{\ell},x^0)} 
											  }
										 \sum _A m_A S_{(\ell,A)}								 
										 \right)
										 \frac{
                                                {V}^{\alpha}(\vec x_{\ell},x^0) 
                                                {V}^{\beta} (\vec x_{\ell},x^0)
											  }
											  {
											    \left(
                                                  K(V^{\mu}(\vec x_{\ell},x^0),
												    g_{\mu\nu}(\vec x_{\ell},x^0))
												\right)^2
											  } \, .
\]
Now since $U^0 = 1/K$ and $V^{\mu} U^0 = U^{\mu}$, the above discrete 
form of Einstein equations becomes
\[
  G^{\alpha\beta}({\vec x}_{\ell},x^0) = 8 \pi 
                                         \left(
										 \frac{K(V^{\mu}(\vec x_{\ell},x^0),
										             g_{\mu\nu}(\vec x_{\ell},x^0))
											  }
                 					          {                                           										 
                							    \sqrt{-g({\vec x}_{\ell},x^0)} 
											  }
										 \sum _A m_A S_{(\ell,A)}								 
										 \right)
										 U^{\alpha} U^{\beta} \, .
\]
The right-hand side is formally identical to the stress-energy tensor of 
dust and thus the quantity in the parentheses can be identified as the density
\[
  \rho(\vec x_{\ell},x^0) = \left(
							       \frac{K(V^{\mu}(\vec x_{\ell},x^0),
							        g_{\mu\nu}(\vec x_{\ell},x^0))
								  }
                 			      {                                           										 
                				    \sqrt{-g({\vec x}_{\ell},x^0)} 
								  }
							 \sum _A m_A S_{(\ell,A)}								 
							\right) \, .
\]
Replacing the `K'- and `S'-terms by their definitions gives
\[
  \rho(\vec x_{\ell},x^0) = \left( 
                              \sqrt{
                                 \frac{g_{\mu\nu} V^{\mu} V^{\nu}}
								      {g}
								   }
						    \right)(\vec x_{\ell},x^0)
							\sum _A m_A 
							\frac{ W({\vec x}_{\ell} - {\vec z_A})}
                                 {\sum _n k^3 W({\vec x}_{n} - {\vec z_A}) } 
\]
which is identical to the expression obtained by applying baryon 
conservation.  
\\
\\
\\
\\
\noindent
\textbf{Thus the prescription of defining all metric functions directly on the 
computational grid in the discrete action is consistent with the 
discrete form of baryon conservation.}

\newpage 


%%%%%%%%%%%%%%%%%%%%%%%%%%%%%%%%%%%%%%%%%%%%%%%%%%%%%%%%%%%%%%%%%%%%%%%%%%%%%%%
%                       3+1 ADM
%%%%%%%%%%%%%%%%%%%%%%%%%%%%%%%%%%%%%%%%%%%%%%%%%%%%%%%%%%%%%%%%%%%%%%%%%%%%%%% 
\subsection{Discretizing the Action in 3+1 - Conventional Hamiltonian}

I begin again with the combined gravitational and matter actions
\[
  I = I_{HE} + I_{matter}
\]
where the gravitational action is given by the usual
Hilbert-Einstein action 
\[
  I_{HE} = \frac{1}{16 \pi} \int \, d^4 x R(x) \sqrt{ -g(x) } 
\]
and the matter action by
\[
  I_{matter} = - m_1 \int \, d \lambda_1 
                     \sqrt{ -g_{\mu \nu}(z_1(\lambda_1)) 
					        \frac{d {z_1}^{\mu}(\lambda_1)}{d \lambda_1} 
					        \frac{d {z_1}^{\nu}(\lambda_1)}{d \lambda_1} }
               - m_2 \int \, d \lambda_2 
                     \sqrt{ -g_{\mu \nu}(z_2(\lambda_2)) 
					        \frac{d {z_2}^{\mu}(\lambda_2)}{d \lambda_2} 
					        \frac{d {z_2}^{\nu}(\lambda_2)}{d \lambda_2} }\, .				
\]
Using the standard ADM prescription for particles this can be re-written
as
\begin{eqnarray*}
  I_{matter} & = & - m_1 \int \, d {z_1}^0 L_{matter}(\vec z_1({z_1}^0),\tilde p_1({z_1}^0);g_{\mu\nu}(\vec z_1({z_1}^0);{z_1}^0)) \\
             &   &  - m_2 \int \, d {z_2}^0 L_{matter}(\vec z_2({z_2}^0),\tilde p_2({z_2}^0);g_{\mu\nu}(\vec z_2({z_2}^0);{z_1}^0)) 
\end{eqnarray*}
where the individual matter Lagrangians take the form  
\begin{eqnarray*}
  L_{matter}(\vec z_A({z_A}^0),\tilde p_A({z_A}^0);g_{\mu\nu}(\vec z_A({z_A}^0);{z_A}^0))
    & = &    \left\{
			         {p_A}_k({z_A}^0) \frac{d {z_A}^k}{d {z_A}^0}({z_A}^0) \right. \\
	&   &    \left.
					 - H \left( \vec z_A({z_A}^0),
					            \tilde p_A({z_A}^0);
								g_{\mu\nu}(\vec z_A({z_A}^0);{z_A}^0)
						 \right) \frac{}{}
			 \right\}
\end{eqnarray*}
where
\begin{eqnarray*}
  H(\vec z_A({z_A}^0),
    \tilde p_A({z_A}^0);
	g_{\mu\nu}(\vec z_A({z_A}^0);{z_A}^0))  & = &  
	                                 - {p_A}_k({z_A}^0) {N_A}^k(\vec z_A({z_A}^0)) \\ 
    &   &                             + N(\vec z_A({z_A}^0)) \sqrt{
			            					    m_A^2 + h^{ij}(\vec z_A({z_A}^0);{z_A}^0) 
						            			{p_A}_i({z_A}^0) {p_A}_j({z_A}^0)
								               } 
\end{eqnarray*}
and the particle index $A$ takes on values 1 or 2.
For brevity each particles contravariant components of position 
will be denoted by ${\vec z}_A \equiv {z_A}^k$ and covariant components of
momenta by ${\tilde p}_A \equiv {p_A}_k$.  It is to be understood that both
the position and the conjugate momenta are parameterized by ${z_A}^0$.

The total action may then be written as a function of 
the coordinates $x^{\mu}$ through the $\delta$-functions 
to yield 
\begin{eqnarray*}
  I & = & \int \, d^4 x \left\{
  	  	   	  	  		  \frac{R(x) \sqrt{-g(x)}}{16 \pi} 
						  -m_1 \int d{z_1}^0 
						  L_{matter}(\vec z_1({z_1}^0),
						             \tilde p_1({z_1}^0);
									 g_{\mu\nu}(\vec z_1({z_1}^0);{z_1}^0))
						  \delta^4(x - z_1) \right. \\
	&    &                \left.		
						  -m_2 \int d{z_2}^0 
						  L_{matter}(\vec z_2({z_2}^0),
						             \tilde p_2({z_2}^0);
									 g_{\mu\nu}(\vec z_2({z_2}^0);{z_2}^0))
						  \delta^4(x - z_2)
  	  	   	  	  	\right\} \, .
\end{eqnarray*}
The integrations over $d {z_A}^0$ can again be eliminated using $\delta(x^0 - {z_A}^0)$
\begin{eqnarray*}
  I & = & \int \, d^4 x \left\{
  	  	   	  	  		  \frac{R(x) \sqrt{-g(x)}}{16 \pi} 
						  -m_1  
						  L_{matter}(\vec z_1(x^0),
						             \tilde p_1(x^0);
									 g_{\mu\nu}(\vec z_1(x^0);x^0))
						  \delta^3(\vec x - \vec z_1(x^0)) \right. \\
	&    &                \left.
						  -m_2  
						  L_{matter}(\vec z_2(x^0),
						             \tilde p_2(x^0);
									 g_{\mu\nu}(\vec z_2(x^0);x^0))
						  \delta^3(\vec x - \vec z_2(x^0))
  	  	   	  	  	\right\} \, .
\end{eqnarray*}
Now perform the ADM decomposition (and renaming $x^0 = t$)
\begin{eqnarray*}
  I & = &  \int \, dt d^3 x \left\{
                             \frac{1}{16 \pi}
                             \left( {\dot h}_{ij}(\vec x, t) \pi^{ij}(\vec x, t) - 
							        N(\vec x, t) R^0(\vec x, t) - N_i(\vec x, t) R^i(\vec x, t) 
							 \right) \right. \\
    &   &        						
	             -m_1 
     			  L_{matter}(\vec z_1(t),
				             \tilde p_1(t);
							 g_{\mu\nu}(\vec z_1(t);t))
		    				  \delta^3(\vec x - \vec z_1(t)) \\
    &   &         \left.
				 -m_2
     			  L_{matter}(\vec z_2(t),
				             \tilde p_2(t);
							 g_{\mu\nu}(\vec z_2(t);t))
		    				  \delta^3(\vec x - \vec z_1(t))
  	  	   	  	  	\right\} 
\end{eqnarray*}
where 
\[
  R^0 = -\sqrt{h}\left[ {}^3R + 
         h^{-1} \left( \frac{1}{2}Tr(\pi)^2 - Tr(\pi^2)\right) \right]
\]
and
\[
  R^i = 2 {\pi^{ij}}_{|j} \, .
\]
Now discretize the spatial portion of the fields by allowing 
$\int \, d^3 x \rightarrow \sum_{\ell} k^3$ where $\ell$ is a multi-index uniquely 
denoting each discrete grid point.
This process compels the definition
\[
  \delta^3(\vec x - {\vec z}_A) \rightarrow 
                                \frac{W({\vec x}_{\ell} - {\vec z}_A)}
								     {\sum _n k^3 W(\vec x_n - \vec z_A) } \, .
\]
Again choose the action to contain only the metric function defined at grid points
by using the freedom of the $\delta$-function before discretization.  
This gives
\begin{eqnarray*}
  I & = &  \int \, dt \sum _{\ell} k^3 \left\{
                             \frac{1}{16 \pi}
                             \left(  {\dot h}_{ij}(\vec x_{\ell}, t) 
							         \pi^{ij}(\vec x_{\ell}, t) 
									- 
							         N(\vec x_{\ell}, t) 
									 R^0(\vec x_{\ell}, t) 
									- 
									 N_i(\vec x_{\ell}, t) 
									 R^i(\vec x_{\ell}, t) 
							 \right) \right. \\
    &   &        \left.							
	             -m_1 
     			  L_{matter}(\vec z_1(t),
				             \tilde p_1(t);
							 g_{\mu\nu}(\vec x_{\ell};t))
			      \frac{W({\vec x}_{\ell} - {\vec z}_1)}
    				   {\sum _{n} k^3 W({\vec x}_n - {\vec z}_1) }
				 -m_2
     			  L_{matter}(\vec z_2(t),
				             \tilde p_2(t);
							 g_{\mu\nu}(\vec x_{\ell};t))
			      \frac{W({\vec x}_{\ell} - {\vec z}_2)}
					   {\sum _{n} k^3 W({\vec x}_{n} - {\vec z}_2) }
                  \frac{}{}						   
  	  	   	  	  	\right\} \, .
\end{eqnarray*}
Introducing the (slightly new) notation 
\[
   \frac{W({\vec x}_{\ell} - {\vec z}_A)}
        {\sum _n k^3 W(\vec x_n - \vec z_A) }
	    \equiv S({\vec x}_{\ell} - {\vec z}_A)
\]
allows for a slightly denser notation and the final form of the action is  
\begin{eqnarray*}
  I & = &  \int \, dt \sum _{\ell} k^3 \left\{
                             \frac{1}{16 \pi}
                             \left(  {\dot h}_{ij}(\vec x_{\ell},t) 
							         \pi^{ij}(\vec x_{\ell},t) 
									- 
							         N(\vec x_{\ell},t) 
									 R^0(\vec x_{\ell},t) 
									- 
									 N_i(\vec x_{\ell},t) 
									 R^i(\vec x_{\ell},t) 
							 \right) \right. \\
    &   &        \left.							
	             -m_1 
     			  L_{matter}(\vec z_1(t),
				             \tilde p_1(t);
							 g_{\mu\nu}(\vec x_{\ell};t))
				  S({\vec x}_{\ell} - {\vec z}_1)
				 -m_2 
     			  L_{matter}(\vec z_2(t),
				             \tilde p_2(t);
							 g_{\mu\nu}(\vec x_{\ell};t))
				  S({\vec x}_{\ell} - {\vec z}_2)
                  \frac{}{}						   
  	  	   	  	  	\right\} \, .
\end{eqnarray*}

\subsubsection{Varying the conjugate momenta $\tilde p_A(t)$}
Taking the variation of the above integral with respect to 
the conjugate momenta $\tilde p_A(t)$ yields
\begin{eqnarray*}
  \delta I \vert_{\delta {p_A}_i(t)} & = & - \int \, dt \sum _{\ell} k^3 \sum _{A} m_A
                                        \left(
										  \frac{ d {z_A}^i(t) }{d t}
										 -\frac{ \partial H(\vec z_A(t),
										                    \tilde p_A(t);
                                                            g_{\mu\nu}(\vec x_{\ell};t)) }
											   { \partial {p_A}_i(t)}
										\right)											   
										   S({\vec x}_{\ell} - {\vec z}_A(t))			
                                           \delta {p_A}_i(t) \, . 
\end{eqnarray*}
When this variation is set to zero the resulting particle equation becomes 
\[
   \frac{ d {z_A}^i(t) }{d t} = \sum _{\ell} k^3 
                                  \frac{ \partial H(\vec z_A(t),
										            \tilde p_A(t);
                                                    g_{\mu\nu}(\vec x_{\ell};t)) 
									   }
									   { \partial {p_A}_i(t)}
  									   S({\vec x}_{\ell} - {\vec z}_A(t))													  
\]
which is equivalent to 
\[
   \frac{ d {z_A}^i(t) }{d t} = \frac{
                                          \sum _{\ell}
										  \left( 
										  -N^i(\vec x_{\ell},t) 
										  +
										  \frac{
										        N(\vec x_{\ell},t) 
												h^{ij}(\vec x_{\ell},t)
												{p_A}_j(t)
										       }
											   {\mu(\tilde p_A(t),
											        h^{ij}(\vec x_{\ell},t)
											   } 
										  \right)                                    
										  W(\vec x_{\ell} - \vec z_A(t))									 
									 }
									 {
    									 \sum _{n} W(\vec x_{n} - \vec z_A(t))
									 }
\]
where I have used the $\mu$-parameter 
\[
  \mu(\tilde p_A(t),h^{ij}(\vec x_{\ell},t)) = \sqrt{
			            					             m_A^2 
													   + h^{ij}(\vec x_{\ell},t) 
            						            		{p_A}_i(t) 
            						            		{p_A}_j(t)
													   }
\]
defined by CWM in \emph{Geodesic Integrator Formulae}.

\subsubsection{Varying the particle trajectories $\vec z_A(t)$}

Taking the variation of the action integral with respect to 
the particle trajectory $\vec z_A(t)$ yields
\begin{eqnarray*}
  \delta I \vert_{\delta {z_A}^i(t)} & = & - \int \, dt \sum _{\ell} k^3 \sum _{A} m_A
                                      \left[
                                        \left(
										  {p_A}_i(t)
										  \frac{ d }{d t}
										  \delta{z_A}^i(t)
										 -\frac{ \partial H(\vec z_A(t),
										                    \tilde p_A(t);
                                                            g_{\mu\nu}(\vec x_{\ell};t)) }
											   { \partial {z_A}^i(t)}
										\right)											   
										S({\vec x}_{\ell} - {\vec z}_A(t)) \right. \\
								&  & \left.
                                      + \left(
									      {p_A}_i(t)
										  \frac{ d {z_A}^i(t) }{d t}
										  - H(\vec z_A(t),
    						                  \tilde p_A(t);
                                              g_{\mu\nu}(\vec x_{\ell};t))
									    \right)						
										\frac{ \partial
										       S({\vec x}_{\ell} - {\vec z}_A(t))
											 }
											 { \partial {z_A}^i(t)}
                                     \right] \, . 				   			
\end{eqnarray*}
The only dependence on $\vec z_A$ in $H$ was through the metric functions.
However, this dependence is now absent because of our prescription.
Taking this into account and performing an integration-by-parts (IBP)
on the first term yields 
\begin{eqnarray*}
  \delta I \vert_{\delta {z_A}^i(t)} & = & - \int \, dt \sum _{\ell} k^3 \sum _{A} m_A
                                      \left[
     	  							   -\frac{ d }{d t}									  
                                       \left(
										  {p_A}_i(t)
   										  S({\vec x}_{\ell} - {\vec z}_A(t)) 
									   \right)
									  + 
  								      {p_A}_i(t)
									  \frac{ d {z_A}^i(t) }{d t}
   									  \frac{ \partial
										      S({\vec x}_{\ell} - {\vec z}_A(t))
										   }
										   { \partial {z_A}^i(t)} \right. \\
				                        &  & \left.
										  - H(\vec z_A(t),
    						                  \tilde p_A(t);
                                              g_{\mu\nu}(\vec x_{\ell};t))
									  \frac{ \partial
										     S({\vec x}_{\ell} - {\vec z}_A(t))
										   }
										   { \partial {z_A}^i(t)}
                                     \right] \, . 				   			
\end{eqnarray*}
When the derivative $d/dt$ in the first term hits the smoothing it cancels the second
term identically.  
Switching the derivative from $\partial/ \partial {z_A}^i(t)$ 
to $\partial/\partial {x_{\ell}}^i$ and performing a summation by parts 
on the last term gives
\begin{eqnarray*}
  \delta I \vert_{\delta {z_A}^i(t)} & = & - \int \, dt \sum _{\ell} k^3 \sum _{A} m_A
                                      \left[
     	  							   -\frac{ d {p_A}_i(t) }{d t}									  
									  - 
                                        \frac{ \partial H(\vec z_A(t),
										                  \tilde p_A(t);
                                                          g_{\mu\nu}(\vec x_{\ell};t)) 
									         }
									         { \partial {x_{\ell}}^i}
									  \right]
										     S({\vec x}_{\ell} - {\vec z}_A(t)) \, .
\end{eqnarray*}
Setting the variation to zero finally gives the particle position updates
\[
  \frac{ d {p_A}_i(t) }{d t}  =  -\sum _{\ell} k^3 \sum _{A} m_A
                                    \frac{ \partial H(\vec z_A(t),
	         							              \tilde p_A(t);
                                                      g_{\mu\nu}(\vec x_{\ell};t)) 
									     }
									     { \partial {x_{\ell}}^i}
									     S({\vec x}_{\ell} - {\vec z}_A(t)) 
\]
which can also be written as
\begin{eqnarray*}
  \frac{ d {p_A}_i(t) }{d t}  & = & - \sum _{A} m_A \sum _{\ell} \\
                              &   &  \frac{ \left(
									        -{p_A}_k(t) 
											{N_A}^k_{,i}(\vec x_{\ell},t)
											+ {N_A}_{,i}(\vec x_{\ell},t)
											\mu(\tilde p_A(t),h^{ij}(\vec x_{\ell},t))
											-\frac{{N_A}(\vec x_{\ell},t)}
											      {\mu(\tilde p_A(t),h^{ij}(\vec x_{\ell},t))}
											 {h^{jk}}_{,i}(\vec x_{\ell},t) 
											 {p_A}_j(t)
											 {p_A}_k(t)
									       \right) 
										   W({\vec x}_{\ell} - {\vec z}_A(t))
									     }
									     { \sum _n W({\vec x}_n - {\vec z}_A(t)) }\, .
\end{eqnarray*}


\end{document}

