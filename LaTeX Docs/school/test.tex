\documentclass[twocolumn]{article}
%\usepackage{makeidx,babel}


\begin{document}

%%
%%Top matter
%%
\title{A Brief Review of the Literature on Noether's Theorem}
\author{Conrad Schiff \thanks{Research support by my family} \\
        Applied Modeling Group\\
	  Bowie MD 20716\\
	  \texttt{cmschiff@erols.com}}
\date{\today}
\maketitle

%%
%%Abstract
%%
\begin{abstract}
We review various recent sources on Noether's theorem and various 
related results such as the Rund--Trautman identity.  We then show how 
these results can be more cleanly obtained through a method which sharpens 
the distinction between invariance and extremezation of the action.
\end{abstract}

\section{Introduction}\label{S:intro}

\section{A Brief Review of Current Sources}\label{S:review}

\subsection{Hill's Seminal Article}\label{SS:H}

\subsection{Jose and Saletan}\label{SS:JS}

\subsection{American Journal of Physics}\label{SS:AMJ}

\section{A Clean Derivation}

\[
	\left( \prod_{j=1}^{n} {\hat x}_{j} \right) H_{c}
         = \frac{1}{2} {\hat k}_{ij} \det \hat{ \mathbf{K} }(i|i)
\]
which is described in \cite{J&S98}.

\begin{thebibliography}{99}
\bibitem{J&S98}
  Jorge V. Jose and Eugene J. Saletan 
  \emph{Classical Dynamics: A Contemporary Approach},
  Cambridge University Press, 1998
\end{thebibliography}
\end{document}