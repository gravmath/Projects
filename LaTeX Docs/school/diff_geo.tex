\documentclass[twocolumn]{article}
\usepackage{amsmath,latexsym}
%\usepackage[dvipdfm]{graphicx}
%\usepackage[dvipdfm]{color}
%\usepackage[dvipdfm]{hyperref}
\pagenumbering{arabic}
\bibliographystyle{plain}
\def\.{{\quad .}}  
\def\_,{{\quad ,}}
\newenvironment{defit}[1]
  {\noindent{$\bullet$}\underline{\textbf{#1}}\\}
  {\\} 
\newcounter{bean}

\begin{document}

%**************************************************************
%
%  Front matter
%
%**************************************************************
\title{Notes on the Foundation of Differential Geometry}
\author{Conrad Schiff \thanks{Physics Graduate Student} \\
        University of Maryland\\
	  \texttt{cmschiff@erols.com}}
\date{\today}
\maketitle


%**************************************************************
%
%  Abstract
%
%**************************************************************
%\widetext
\begin{abstract}
In this notes I present a synthesis of my work on manifold theory.  
They are a compilation of a simple reorganization and union of ideas 
from a variety of sources combined with my own reflections on the 
various subjects.  As a result of both my personnel interests and my 
own style another reader most assuredly will find them overly pedantic 
in some areas and terse in others.  To this I make no attempt at 
rectification.

Update history: initial work August 13, 1999
\end{abstract}

%\narrowtext
%\vskip2pc]
%%%%%%%%%%%%%%%%%%%%%%%%%%%%%%%%%%%%%%%%%%%%%%%%%%%%%%%%%%%
%%Elementary Topology 
%%%%%%%%%%%%%%%%%%%%%%%%%%%%%%%%%%%%%%%%%%%%%%%%%%%%%%%%%%%
\section{Elementary Topology}\label{S:elem_top}

The notion of topology gives us a sense of how to talk about 
continuity and how to say when one point is in a neighborhood of 
another.  A \textbf{topological space} $\mathcal{T}$ $= (M, T)$ is 
defined to be a set $M$ with a distinguished collection of open 
subsets $T$ that satisfy 
\cite{bryant,frankel,lipschutz,love_rund,munkres,schutz80,wald}.

\begin{list}
   {$\bullet$}{\usecounter{bean}
    \setlength{\rightmargin}{\leftmargin}}
	\item Both $M$ and the empty set $\emptyset$ are open
 	\item The intersection of any finite collection of open sets               
          is open
	\item The arbitrary union of open sets is open
\end{list}


These notations can easily exist without a prescription for 
determining the distance between points (\emph{i.e.} without a 
metric).  However, for everything that is discussed below a metric 
will be assumed.  Our topological space $M$ will hereafter be referred 
to as a metric space.
\\\\
\begin{defit}{Metric}
Given a set $X$, a metric on $X$ is a function $d: X \times X 
\rightarrow \mathbf{R}$ such that (i) $d( \mathbf{x}, \mathbf{y} ) =  
d( \mathbf{y}, \mathbf{x} )$ (ii) $d( \mathbf{x}, \mathbf{y} ) \geq 0$ 
with the equality satisfied only when $ x= y$, and (iii) 
$d( \mathbf{x}, \mathbf{z} ) \leq d( \mathbf{x}, \mathbf{y} ) + 
d( \mathbf{y}, \mathbf{z} )$.  
Notice that we have used the metric 
$d(\mathbf{x},\mathbf{y})$ to define neighborhoods and thereby open 
sets and thus we say that $d(\mathbf{x},\mathbf{y})$ has induced a 
topology on $\mathbf{R}^{n}$ \cite{schutz80}.
\end{defit}

\begin{defit}{Neighborhood}
 An open set $A$ is a neighborhood of $p$ if $p 
\in A$ and $A$ is open \cite{love_rund}.  If the $M$ is a metric space 
with the metric $d$ then given $x_0 \in M$ and given $\epsilon > 0$, 
then the set $U(x_0;\epsilon) = \{ x | d(x,x_0) < \epsilon \}$ is 
called the $\epsilon$-neighborhood of $x_0$ \cite{munkres}.
\end{defit}  

\begin{defit}{Discrete Topology}
A set of points is discrete if each point has a neighborhood
which contains no other points of the set \cite{schutz80}.  The 
discrete topology can be made by taking $\mathcal{T} = (M,\emptyset)$ 
\cite{wald}.
\end{defit}

\begin{defit}{Closed Ball}
${\bar B}_{a}(\epsilon) = \{ x \in \mathbf{R}^n | \, 
   \left\| {\bar x} - {\bar a} \right\| \leq \epsilon \}$ 
\cite{frankel}
\end{defit}   

\begin{defit}{Open Ball}
${B}_{a}(\epsilon) = \{ x \in \mathbf{R}^n | \,
   \left\| {\bar x} - {\bar a} \right\| < \epsilon \}$ \cite{frankel}
\end{defit}

\begin{defit}{Interior}
$A^{\circ}$ of a subset $A \subset M$ is the union 
of all open sets contained in A and from the item 3 above $A^{\circ}$ 
is open \cite{love_rund}.
\end{defit}

\begin{defit}{Closure}
${\bar A}$ of $A$ is the intersection of all closed 
sets that contain $A$ and clearly ${\bar A}$ is closed 
\cite{love_rund}.
\end{defit}

\begin{defit}{Open set}
A set is open if it coincides with an interior 
\cite{love_rund}.
\end{defit}

\begin{defit}{Closed set}
A set is closed if it coincides with a closure 
\cite{love_rund}.
\end{defit}

\begin{defit}{Boundary}
A boundary of $A$ is the set $\partial A = {\bar A} 
- {A}^{\circ}$\cite{love_rund}.
\end{defit}

\begin{defit}{Connected space}
A topological space is connected if the only 
sets that are both open and closed are $\emptyset$ and $M$ 
\cite{love_rund}.
\end{defit}

\begin{defit}{Dense Subset}
A subset $A$ of $M$ is said to be dense in $M$ if 
$\bar A = M$ \cite{love_rund}.
\end{defit}

\begin{defit}{Separable Space}
A topological space $M$ is separable if it contains 
a countable dense subset \cite{love_rund}.
\end{defit}

\begin{defit}{Hausdorff space}
A topological space $M$ is a Hausdorff space if, 
for every pair of distinct points $p \in M$ and $q \in M$, there exist 
open subsets $A$, $B$ such that $p \in A$ and $q \in B$ and 
$A \cap B = \emptyset$ 
\cite{love_rund}.  Any two points of $\mathbf{R}^{n}$ have 
neighborhoods which do not intersect \cite{schutz80}.
\end{defit}

\begin{defit}{Cover}
A cover $\mathcal{A}$ is a class of subsets of $M$ if 
$M$ coincides with the union of all $A \in \mathcal{A}$ 
\cite{love_rund}.
\end{defit}

\begin{defit}{Open cover}
A cover made up of open sets \cite{love_rund}.
\end{defit}

\begin{defit}{Subcover}
A subclass of a cover that is itself a cover 
\cite{love_rund}.
\end{defit}

\begin{defit}{Compact}
A topological space $M$ is said to be compact if every open
cover has a finite subcover \cite{love_rund}.
\end{defit}

\begin{defit}{Heine-Borel Theorem}
Every closed and bounded subset of $\mathbf{R}^{n}$ is compact 
\cite{love_rund}.
\end{defit}

\begin{defit}{Refinement}
A cover $\mathcal{B}$ of $M$ is a refinement of a cover 
$\mathcal{A}$ of $M$ if for every set $B \in \mathcal{B}$ there exists 
at least one set $A \in mathcal{A}$ such that $B \subset A$ 
\cite{love_rund}.
\end{defit}

\begin{defit}{Locally finite cover}
The cover $\mathcal{A}$ is said to be locally 
finite if for each $p \in M$ there exists an open set $W_p$ such that 
the set $\{ A \in \mathcal{A} | A \cap W_{p} \neq \emptyset \}$ is 
finite.  That is the neighborhood $W_p$ of $p$ has a nonempty 
intersection with only a finite number of sets of the cover  
\cite{love_rund}.
\end{defit}

\begin{defit}{Paracompact}
A Hausdorff topological space $M$ is paracompact 
if every open cover of $M$ has a locally finite refinement 
\cite{love_rund}. 
Paracompactness is a property which manifolds are required to satisfy 
in order to prevent them from being ``too large''.  Indeed, it is not 
easy to construct examples of topological spaces which satisfy the 
requirements for a manifold but are not paracompact.  For a manifold, 
paracompactness implies that the manifold admits a Riemannian metric 
and it is second countable meaning that manifold can be covered 
locally by a finite number of charts \cite{wald}.  
\end{defit}

\begin{defit}{Partition of unity}
The most important consequence of paracompactness 
is that provides for the existence of a partition of unity.  A 
partition of unity subordinate to the finite open cover 
$\{O_{\alpha}\}$ is a collection of smooth functions 
$\{ f_{\alpha} \}$ such that (i) the support of $f_{\alpha}$
is contained in $\{O_{\alpha}\}$ (ii) $0 \leq f_{\alpha} \leq 1$, 
and (iii) $\sum_{\alpha} f_{\alpha} = 1$ \cite{wald}.
\end{defit}

%%%%%%%%%%%%%%%%%%%%%%%%%%%%%%%%%%%%%%%%%%%%%%%%%%%%%%%%%%%
%%Manifold basics
%%%%%%%%%%%%%%%%%%%%%%%%%%%%%%%%%%%%%%%%%%%%%%%%%%%%%%%%%%%
\section{Manifold Basics}\label{S:mani_bas}

A differentiable manifold may be roughly described as a topological 
space of a certain kind that can be covered by appropriate coordinate 
neighborhoods 
\cite{love_rund}.

A set $M$ is defined to be a manifold if each point of $M$ has an open 
neighborhood which has a continuous $1-1$ map onto an open set of 
$\mathbf{R}^n$ for some $n$.  It is important that the definition 
involves only open sets and not the whole of $M$ and $\mathbf{R}^n$, 
because we do not want to restrict the global topology of $M$.  All we 
want from this elementary geometric picture is that we are able to 
ensure that the local topology of our space is the same as that of 
$\mathbf{R}^n$ \cite{schutz}.  

\subsection{Functions and Curves}\label{SS:F_and_C}

By definition, the $1-1$ map from a point $P \in M$ to $\mathbf{R}^n$
associates with the point $P$ an $n$-tuple $\left(x^1(P), \ldots, 
x^n(P) 
\right)$.  These numbers $x^1(P), \ldots, x^n(P)$ are called the 
coordinates of $P$ under this map.  Symbolically this is denoted
\[
  \phi:M \rightarrow \mathbf{R}^n \, \phi^i:P \rightarrow x^i
\]  

Given the sets $U \subset \mathbf{R}^{m}$ and $V \subset 
\mathbf{R}^{n}$
and a map between them $\phi : U \rightarrow V$


\subsection{Two Theorems}\label{SS:two_theorems}

\noindent\underline{Inverse Function Theorem} \\
The inverse function theorem gives conditions under which a 
differentiable function from $\mathbf{R}^{n}$ to $\mathbf{R}^{n}$ has 
a differentiable inverse \cite{munkres}.   
\\\\
\noindent\underline{Implicit Function Theorem} \\
Let $A$ be open in $\mathbf{R}^{k+n}$; let $f:A \rightarrow 
\mathbf{R}^{n}$ be of class $C^{r}$.  Write $f$ in the form 
$f(\bar x, \bar y)$, for $\bar x \in \mathbf{R}^{k}$ and $\bar y \in 
\mathbf{R}^{n}$.  Suppose that $(\bar a, \bar b)$ is a part of $A$ 
such that $f(\bar a, \bar b) = 0$ and 
$det \left[ \frac{\partial f}{\partial \bar y}(\bar a, \bar b)\right]
\neq 0$.  Then there is a neighborhood $B$ of $\bar a$ in 
$\mathbf{R}^{k}$ and a unique continuous function $g:B \rightarrow 
\mathbf{R}^{n}$ such that $g(\bar a) = \bar b$ and $f\left(\bar x, 
g(\bar x)\right) = 0$ for all $x \in B$.  The function $g$ is in fact 
of class $C^{r}$ \cite{munkres}.

\subsection{Sub-Manifolds}
Suppose there are $r$ function of $n+r$ variables, $F^{j}(x^1, \cdot, 
x^{n+r}) j = 1, \cdots, r$.  If $F^j = c^j$, where $c^j$ are 
constants, and for each ${\bar x}_0$ which satisfies this condition 
the Jacobian matrix $\left( \frac{\partial F^j}{\partial x^i} \right) 
j = 1, \cdots, r; i = 1, \cdots, n+r$ has rank $r$, then $F^j = c^j$ 
define an $n$-dimensional submanifold of $\mathbf{R}^{n+r}$ 
\cite{frankel}.

\subsection{Basis Objects and Transformations}\label{SS:basis_obj}

In this section, I outline how the simple objects defined on a 
manifold change under a redefinition of the underlying coordinates.  
The notation I use for basis vectors
\begin{equation}
   {\bar b}_i = \frac{\partial}{\partial x^i}
\end{equation}
and basis forms 
\begin{equation}
   {\bar \beta}^i = d x^i
\end{equation}
is derived from \cite{gock_schuck}.

From a straightforward application of the chain-rule, the basis 
vectors transform as 
\begin{equation}
   \frac{\partial}{\partial x^{j'}} = 
          \frac{\partial x^i}{\partial x^{j'}}                           
          \frac{\partial}{\partial x^i}
\end{equation}
while the basis forms transform in the dual fashion
\begin{equation}
   d x^{j'} = \frac{\partial x^{j'}}{\partial x^{i}} d x^i \.
\end{equation}

From their definition, the basis vectors must transform as
\begin{eqnarray}
   {\bar b}_{j'} = {\Lambda^i}_{j'} {\bar b}_{i} \\
   {\bar \beta}^{j'} = {\Lambda^{j'}}_{i} {\bar \beta}^{i}
\end{eqnarray}
where 



\subsection{Pullbacks and Pushforwards}\label{SS:pull_push}

\subsection{Vector Fields}\label{SS:vec_fields}

\subsection{Integral Curves}\label{SS:int_curves}

\subsection{Connections and the Metric}\label{SS:conx_g}

\subsection{Fiber Bundles}\label{SS:fib_bun}


%%%%%%%%%%%%%%%%%%%%%%%%%%%%%%%%%%%%%%%%%%%%%%%%%%%%%%%%%%%
%%Defining Derivatives
%%%%%%%%%%%%%%%%%%%%%%%%%%%%%%%%%%%%%%%%%%%%%%%%%%%%%%%%%%%
\section{Defining Derivatives}\label{S:def_deriv}

The five conditions required of a derivative operator are \cite{wald}:

\begin{list}
   {$\bullet$}{\usecounter{bean}
    \setlength{\rightmargin}{\leftmargin}}
	\item linearity
 	\item Leibnitz rule
	\item commutes with contraction
	\item consistent with directional derivatives
	\item torsion free $\nabla_a \nabla_b f = \nabla_b \nabla_a f$
\end{list}


\subsection{Partial derivative}\label{SS:par_dervi}
The partial derivative of a function of $r$ variables $f(\bar x)$
with respect to $x^i$ will be denoted by $f_{,i} \equiv 
\frac{\partial f}{\partial x^i}$

\subsection{Directional derivative}\label{SS:dir_deriv}

Let $A \subset \mathbf{R}^{m}$; let $f: A \rightarrow \mathbf{R}^{n}$.
In addition, let $A$ contain a neighborhodd of $\bar a$.  Given
$\bar u \neq 0 \in \mathbf{R}^{m}$ then the directional derivative is 
defined as
\[
  f'(\bar a; \bar u) = \lim_{t \rightarrow 0} 
  \frac{f(\bar a + t \bar u) - f(\bar a)}{t}
\]
This definition lacks in that it does not follow that 
differentiability implies continuity and that composites of 
differentiable functions are differentiable \cite{munkres}. 

\subsection{Lie derivative}\label{SS:Lie_deriv}

\subsection{Exterior derivative}\label{SS:ext_deriv}

\subsubsection{Hodge Star Operator}\label{SSS:hodge}

The Hodge operator (or duality transformation) takes $p$-forms in an
$n$-dimensional to $(n-p)$-forms via
\begin{eqnarray*}
  {}^{\ast} \left( dx^{i_1} \wedge dx^{i_2} \wedge \cdots 
                            \wedge dx^{i_p} \right)  = 
					   \frac{1}{(n-p)!} 
					   \epsilon_{i_1 i_2 \ldots i_p i_{p+1} i_n} 
\\ 
  dx^{i_{p+1}} \wedge dx^{i_{p+2}} \wedge \cdots \wedge dx^{i_n} \.
\end{eqnarray*}

Repeated application of the Hodge star operator gives \cite{ryder}:
\[
  {}^{\ast} {}^{\ast} \omega_p = (-1)^{p(n-p)} \omega_p \. 
\]

\subsubsection{Co-derivative}\label{SSS:co_deriv}

The co-derivative (or adjoint exterior derivative \cite{ryder})
is defined by 
\[
  \delta = (-1)^{np+n+1} {}^{\ast} d {}^{\ast}
\]
where $p$ is the degree of the form $\omega_p$ on which $\delta$ is 
applied and $n$ is the dimension of the space \cite{ryder}.

In addition, the co-derivative satisfies the relation 
$\delta^2 = 0$\cite{ryder}.
 
\subsubsection{Laplacian}\label{SSS:laplace}

The Laplacian is defined as
\[
  \Delta = (d + \delta)^2 = d \delta + \delta d
\]
\cite{ryder}.

\subsubsection{Classical Vector Calculus}

Connection can be made between the exterior derivative and classical 
vector calculus by using the Hodge star operator.  The resulting 
approach is so easy and compact that it obviates the need for look-up 
tables for the components of the gradient, divergence, curl, and 
laplacian in curvilinear coordinates (see \cite{frankl} for such a 
table).

Take for instance the expression for the curl in cylindrical 
coordinates given by
\begin{eqnarray*}
	\nabla \times {\vec A} & = & 
	                        \left(
	                         \frac{1}{r} 
							 \frac{\partial A_{z}}
							      {\partial \theta}
							 -
							 \frac{\partial A_{\theta}}
							      {\partial z}
	                        \right) {\hat e}_{r} \\
							& + & 
							\left(
							 \frac{\partial A_{r}}
							      {\partial z}
						     -
							 \frac{\partial A_{z}}
							      {\partial r}
							\right) {\hat e}_{\theta} \\
							& + & 
							\frac{1}{r}
							\left(
							 \frac{\partial 
							        \left(r A_{\theta}\right)}
							      {\partial r}
							-
							 \frac{\partial A_{r}}
							      {\partial \theta}
							\right) {\hat e}_{z}
\end{eqnarray*}
this formidable expression can be derived easily as follows.

The first step is to recognize that the presence of the unit-vectors 
in the above expression means that we are operating in a 
non-coordinate basis.  The fundamental correspondence in this 
non-coordinate basis is given by
\begin{eqnarray*}
	{\hat e}_{r} = \frac{\partial}{\partial r} 
	               & \Leftrightarrow &
				   dr = {\tilde r} \\
	{\hat e}_{\theta} = \frac{1}{r} \frac{\partial}
	                                     {\partial \theta}
					& \Leftrightarrow &
				   r d \theta = {\tilde \theta} \\
	{\hat e}_{z} = \frac{\partial}{\partial z}
					& \Leftrightarrow &
        		   dz = {\tilde z} \.
\end{eqnarray*}
To calculate the curl, start with the desired vector field
\[
	{\vec A} =  A_{r} {\hat e}_{r} + A_{\theta} {\hat e}_{\theta}
	          + A_{z} {\hat e}_{z}
\]
and to use the fundamental correspondence to express it in terms of 
the unit basis forms as
\[
	{\tilde A} =  A_{r} {\tilde r} + A_{\theta} {\tilde \theta}
	            + A_{z} {\tilde z} \.
\]
The curl is obtained by taking the exterior derivative of the $\tilde 
A$ followed by the application of the Hodge star operator.  The only 
point that isn't mechanical, is that the Hodge star operator must be 
applied to the unit basis forms $\tilde r, \tilde \theta, \tilde z$ 
and not the coordinate forms $dr, d\theta, dz$.  The application of
these two operation yields
\begin{eqnarray*}
	\ast d {\tilde A} & = & \ast \left[
	                           \left( 
							    \frac{\partial A_z}
								     {\partial \theta}
							    -
								r \frac{\partial A_{\theta}}
								       {\partial z}
							   \right) d\theta \wedge dz 
							   \right. \\
							  & &
							 \left(
							  \frac{\partial A_r}
							       {\partial z}
							 -
							  \frac{\partial A_z}
							       {\partial r}
							 \right) dz \wedge dr \\
							 & &
							 \left.
							 \left(
							  \frac{\partial \left( 
							        r A_{\theta}\right)}
									{\partial r}
							 -
							  \frac{\partial A_r}
							       {\partial \theta} 
							 \right) dr \wedge d \theta
							 \right] 
\end{eqnarray*}
which when expressed in terms of the unit basis forms gives
\begin{eqnarray*}
	\ast d {\tilde A} & = & \ast \left[
							   \frac{1}{r}
	                           \left(
							    \frac{\partial A_z}
								     {\partial \theta}
							    -
								r \frac{\partial A_{\theta}}
								       {\partial z}
							   \right) 
							   \tilde \theta \wedge \tilde z 
							   \right. \\
							  & &
							 \left(
							  \frac{\partial A_r}
							       {\partial z}
							 -
							  \frac{\partial A_z}
							       {\partial r}
							 \right) 
							 \tilde z \wedge \tilde r \\
							 & &
							 \left.
							 \frac{1}{r}
							 \left(
							  \frac{\partial \left( 
							        r A_{\theta}\right)}
									{\partial r}
							 -
							  \frac{\partial A_r}
							       {\partial \theta} 
							 \right) 
							 \tilde r \wedge \tilde \theta
							 \right] \. 
\end{eqnarray*}
Carrying out the Hodge star operation finally yields
\begin{eqnarray*}
	\ast d {\tilde A} & = & \ast \left[
							   \frac{1}{r}
	                           \left(
							    \frac{\partial A_z}
								     {\partial \theta}
							    -
								r \frac{\partial A_{\theta}}
								       {\partial z}
							   \right) 
							   \tilde r 
							   \right. \\
							  & &
							 \left(
							  \frac{\partial A_r}
							       {\partial z}
							 -
							  \frac{\partial A_z}
							       {\partial r}
							 \right) 
							 \tilde \theta \\
							 & &
							 \left.
							 \frac{1}{r}
							 \left(
							  \frac{\partial \left( 
							        r A_{\theta}\right)}
									{\partial r}
							 -
							  \frac{\partial A_r}
							       {\partial \theta} 
							 \right) 
							 \tilde z
							 \right] \. 
\end{eqnarray*}
which, through the fundamental correspondence can be immediately
written in vector form
\begin{eqnarray*}
	\nabla \times {\vec A} & = & 
	                        \left(
	                         \frac{1}{r} 
							 \frac{\partial A_{z}}
							      {\partial \theta}
							 -
							 \frac{\partial A_{\theta}}
							      {\partial z}
	                        \right) {\hat e}_{r} \\
							& + & 
							\left(
							 \frac{\partial A_{r}}
							      {\partial z}
						     -
							 \frac{\partial A_{z}}
							      {\partial r}
							\right) {\hat e}_{\theta} \\
							& + & 
							\frac{1}{r}
							\left(
							 \frac{\partial 
							        \left(r A_{\theta}\right)}
							      {\partial r}
							-
							 \frac{\partial A_{r}}
							      {\partial \theta}
							\right) {\hat e}_{z} \.
\end{eqnarray*}
The gradient, divergence, and the laplacian all follow the same
way and will be briefly presented here for completeness.  First start 
from a scalar function $f(r,\theta,z)$.  Its gradient in cylindrical 
coordinates is obtained by taking the exterior derivative
\begin{eqnarray*}
	df & = & 
	     \frac{\partial f}{\partial r} dr +
	     \frac{\partial f}{\partial \theta} d \theta +
		 \frac{\partial f}{\partial z} dz \\
	   & = & 
	     \frac{\partial f}{\partial r} \tilde r +
		 \frac{1}{r}
	     \frac{\partial f}{\partial \theta} \tilde \theta +
    	 \frac{\partial f}{\partial z} \tilde z
\end{eqnarray*}
and again using the fundamental correspondence to get
\[
	\nabla f =
         \frac{\partial f}{\partial r} {\hat e}_r +
		 \frac{1}{r}
	     \frac{\partial f}{\partial \theta} {\hat e}_{\theta} +
    	 \frac{\partial f}{\partial z} {\hat e}_z \.
\] 
The divergence is obtained like the curl except this time the order
of application of the exterior derivative and the Hodge star operator
is reversed.  Thus
\begin{eqnarray*}
	d \left( \ast {\tilde A} \right) 
	                  & = & d \left[
	                      A_r \tilde \theta \wedge \tilde z +
						  A_{\theta} \tilde z \wedge \tilde r +
						  A_z \tilde r \wedge \tilde \theta
						  \right] \\
					  & = &
						  \left[
	                      r A_r d\theta \wedge dz +
						  A_{\theta} dz \wedge dr + 
						  \right.\\
				      &   &
						  \left.
						  r A_z dr \wedge d\theta
                          \right] \\
				      & = &
					      \left[
						    \frac{1}{r}
							\frac{\partial}
							     {\partial r}
								 \left(r A_r \right) 
							+
							\frac{1}{r}
							\frac{\partial A_{\theta}}
							     {\partial \theta}
							+
							\frac{\partial A_z}
							     {\partial z}
						  \right]
						  \tilde r \wedge 
						  \tilde \theta \wedge
						  \tilde z \.
\end{eqnarray*}
Operating a final time with the Hodge star operator yields the desired
result of
\[
	\nabla \cdot \vec A =
					      \left[
						    \frac{1}{r}
							\frac{\partial}
							     {\partial r}
								 \left(r A_r \right) 
							+
							\frac{1}{r}
							\frac{\partial A_{\theta}}
							     {\partial \theta}
							+
							\frac{\partial A_z}
							     {\partial z}
						  \right] \.
\]
Finally the laplacian is obtained by taking the divergence of the 
gradient leading to the identification of
\[
	\nabla^2 = \ast d \ast d
\] 
which is a special case of the laplacian defined above.


\subsection{Covariant derivative}\label{SS:cov_deriv}


%%%%%%%%%%%%%%%%%%%%%%%%%%%%%%%%%%%%%%%%%%%%%%%%%%%%%%%%%%%
%%Defining Integrals
%%%%%%%%%%%%%%%%%%%%%%%%%%%%%%%%%%%%%%%%%%%%%%%%%%%%%%%%%%%
\section{Defining Integrals}\label{S:def_int}

\subsection{Proper Measure}\label{SS:prop_meas}

\subsection{Relative Tensors}\label{SS:rel_ten}

\subsection{Chains, Boundaries and Forms}\label{SS:chns_bndrs}
\noindent{$\bullet$}\underline{\textbf{Chains and Forms}}
Start by considering the meaning of the ordinary line and surface 
integrals
\begin{eqnarray*}
  I_1 & = & \int_{C} \left\{F_x dx + F_y dy + F_z dz \right\} \\ 
      & = & \int_{C} {\vec F} \cdot d{\vec r} \\
  I_2 & = & \int_{S} \left\{G_x dy \wedge dz + G_y dz \wedge dx + 
                       G_z dx \wedge dy \right\} \\
	  & = & \int_{S} {\vec G} \cdot d{\vec S}													    
\end{eqnarray*}
Since $I_1$ and $I_2$ are numbers, then the `thing' over which the 
forms are being integrated over must in some sense be dual to the 
forms.  Define the notion of a chain where a generic chain in $n$ 
dimensions is denoted $C_n$.  Thus a point is a 0-chain $C_0$, a line 
a 1-chain $C_1$, \emph{etc.}.

Now the boundary of an $n$-chain is an $(n-1)$-chain.  The boundary 
operator $\partial$ can be defined which maps $C_n$ into $C_{n-1}$
\[
  C_n \stackrel{\partial}{\rightarrow} C_{n-1}  
\]
or 
\[
  \partial C_n = C_{n-1} \.
\]
\noindent{$\bullet$}\underline{\textbf{Closed Chain}} A closed chain, 
is one which has no boundary and satisfy $\partial Z_n = 0$ 
\cite{ryder}. 
\\\\
\noindent{$\bullet$}\underline{\textbf{Cycle}} A cycle, denoted by 
$Z_n$, is 
a closed chain \cite{ryder}. 
\\\\
\noindent{$\bullet$}\underline{\textbf{Boundary}} A boundary, denoted 
by $B_n$, is a chain which serves as the boundary of a higher 
dimensional chain $C_{n+1}$ and satisfies $B_n = \partial C_{n+1}$ 
\cite{ryder}. 
\\\\
Since the boundaries $B_n$s have no boundaries (i.e. they are closed) 
it can be immediately be deduced that $\partial^2 = 0$.  

\subsection{Stokes Theorem}\label{SS:stokes}


%%%%%%%%%%%%%%%%%%%%%%%%%%%%%%%%%%%%%%%%%%%%%%%%%%%%%%%%%%%
%%Variational Principles
%%%%%%%%%%%%%%%%%%%%%%%%%%%%%%%%%%%%%%%%%%%%%%%%%%%%%%%%%%%
\section{Variational Principles}\label{S:var_prin}

\subsection{Cross Section Variations}\label{SS:xsec_var}

\subsection{Base-Space Variations}\label{SS:base_var}

\subsection{Equations of Motion}\label{SS:eqn_mot}

\subsection{N\"{o}ether's Theorem}\label{SS:noether}

\subsubsection{Particle Mechanics}

\underline{The theorem:}



\subsubsection{Field Theories}

\noindent \underline{The theorem:}\\ \\
Suppose that the action S is invariant under a group of 
transformations on 
$x^{\mu}$ and the field $\phi$, which for infinitesimal 
transformations take on the form
\[
  \Delta x^{\mu}= {X^{\mu}}_{\nu} \delta \omega^{\nu}, 
  \Delta \phi = {\Phi}_{\mu} \delta \omega^{\mu}
\]
characterized by the infinitesimal parameter $\delta \omega^{\mu}$.  
Then the current defined by
\[
  {J^{\mu}}_{\nu} = \frac{\partial \mathcal{L}}
                        {\partial ( \partial_{\mu} \phi)}
                   \Phi_{\nu} - 
				        {T^{\mu}}_{\kappa} {X^{\kappa}}_{\nu}
\]   
where the stress-energy tensor ${T^{\mu}_{\nu}}$ is given by
\[
  {T^{\mu}}_{\nu} = \frac{\partial \mathcal{L}}
                          {\partial ( \partial_{\mu} \phi)}
						  \partial_{\nu} \phi -
						  {\delta^{\mu}}_{\nu} \mathcal{L}
\]
is divergenceless (\emph{i.e.} conserved).  This in turn yields a 
conserved charge 
\[
  Q_{\nu} = \int_{V} {J^{0}}_{\nu} d^3 \,x 
\]   
\cite{ryder}.

%\begin{figure}
%  \begin{center}
%    \includegraphics{eva04l.jpg}
%  \end{center}
%
%  \caption{A photograph of the author.}
%  \label{fig:author}
%\end{figure}


\bibliography{diff_geo,gr}
\end{document}
