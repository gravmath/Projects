% ****** Start of file SG_notes.tex ******
%
%   Some notes on the SGSolvr paper
% 
%
%
%   updates:	 1/1/00 first version of the paper - CS
%                

%**************************************************************
%
%  Preliminary matters
%
%**************************************************************
% \documentstyle[preprint,eqsecnum,aps]{revtex}
\documentstyle[eqsecnum,aps,epsfig]{revtex}
\bibliographystyle{prsty}
\newcounter{bean}
\def\.{{\quad .}}
\def\_.{{\quad .}}
\def\_,{{\quad ,}}

%**************************************************************
%
%  Begin the document
%
%**************************************************************
\begin{document}
\draft


%\preprint{GR-QC/????}

%**************************************************************
%
%  Front matter
%
%**************************************************************
\title{Some notes on SGSolvr2}
%
\author{C. Schiff}
\address{
            Department of Physics, University of Maryland,
            College Park MD 20742-4111 USA\\
           \rm
         e-mail: \tt  cmschiff@erols.com\\
					  Revision 0.0
}
\date{1/11/00}
\maketitle


%**************************************************************
%
%  Abstract
%
%**************************************************************
\begin{abstract}
\end{abstract}

\section{Introduction}

Hello.  
Thought I would way in on the cross-talk about the scalar gravity solver.
My input should be regarded as questions and comments rather than 
definitive corrections.

\section{Scalar action in Rotating Coordinates}

The first thing I did was to obtain the inverse metric for flat Mink. space
in cylindrical coords.
The metric is 
\[
ds^2 = -dt^2 + dx^2 + dy^2 + dz ^2 = -dt^2 + dr^2 + r^2 d {\theta}^2 + dz^2
\] 
and thus the inverse is
\[
\left( \frac{\partial}{\partial s} \right) ^2 = 
   - \left( \frac{\partial}{\partial t} \right) ^2
   + \left( \frac{\partial}{\partial r} \right) ^2
   + \frac{1}{r^2}\left( \frac{\partial}{\partial \theta} \right) ^2
   + \left( \frac{\partial}{\partial z} \right) ^2          
\]
I next calculated the Jacobian matrix of the transformation from inertial
cylindrical to rotating-cylindrical coords.
\[
{\Lambda _{\nu}}^{\tilde \mu} = \left( \begin{array}{cccc}
                                     1 & 0 & - \omega & 0 \\
						             0 & 1 &    0     & 0 \\
								     0 & 0 &    1     & 0 \\
								     0 & 0 &    0     & 1 
							        \end{array} \right)
\]
which resulted in an inverse metric 
\[
{\eta}^{\tilde \mu \tilde \nu}= \left( \begin{array}{cccc}
                                     -1     & 0 & \omega            & 0 \\
						             0      & 1 &    0              & 0 \\
								     \omega & 0 &  1/r^2 - \omega^2 & 0 \\
								     0      & 0 &    0              & 1 
							        \end{array} \right)
\]
As expected the determinant $\sqrt{-\eta}$ is r but to be sure I calculated 
the determinant of ${\eta}^{\tilde \mu \tilde \nu}$ by hand and in MathCAD and Maple.
Everybody was in agreement.

The bi-linear form for the action now became
\begin{eqnarray}
{\eta}^{\tilde \mu \tilde \nu} 
\Phi _{,\tilde \mu} \Phi _{,\tilde \nu} & = & 
     -\left( \Phi _{,\tilde t } \right)^2
	 +\left( \Phi _{,\tilde r } \right)^2
	 +\frac{1}{r^2} \left( \Phi _{,\tilde \theta } \right)^2
	 +\left( \Phi _{,\tilde z } \right)^2
	 +2 \omega \Phi _{,\tilde t} \Phi _{\tilde \theta}
	 -\omega^2 \left( \Phi _{,\tilde \theta } \right)^2 
\end{eqnarray}
Since
\[
  \nabla \Phi =   \frac{\partial \Phi}{\partial \tilde r} 
                      {\hat e}_{\tilde r} 
                + \frac{1}{r} \frac{\partial \Phi}{\partial \tilde \theta}
				      {\hat e}_{\tilde \theta}
			    + \frac{\partial \Phi}{\partial \tilde z} 
				      {\hat e}_{\tilde z}
\]
the bilinear form for the action becomes
\[
{\eta}^{\tilde \mu \tilde \nu} 
\Phi _{,\tilde \mu} \Phi _{,\tilde \nu} = 
     -\left( \Phi _{,\tilde t } \right)^2
	 +\left( \nabla \Phi \right)^2
	 +2 \omega \Phi _{,\tilde t} \Phi _{,\tilde \theta}
	 -\omega^2 \left( \Phi _{,\tilde \theta } \right)^2 
\]
Carrying out the variation is straightforward since each term must be integrated-by-parts
giving
\[
- \Phi _{,\tilde t \tilde t } 
+ \nabla^2 \Phi 
+ 2 \omega \Phi _{,\tilde \theta \tilde t} 
- \omega^2 \Phi _{,\tilde \theta \tilde \theta} + 4 \pi S = 0 
\]  
where 
\[
  S = \frac{\delta L_{matter}}{\delta g^{\mu\nu}}\frac{\delta g^{\mu\nu}}{\delta \Phi}
\]
Adding the damping term $-2D {\Phi}_{,\tilde t}$ yields
\[
- \Phi _{,\tilde t \tilde t } 
+ \nabla^2 \Phi 
+ 2 \omega \Phi _{,\tilde \theta \tilde t} 
- \omega^2 \Phi _{,\tilde \theta \tilde \theta}
- 2 D \Phi_{,\tilde t} 
+ 4 \pi S = 0 
\]  
Defining $\Pi$ as
\[
  \Pi = \Phi_{,t} = \Phi_{,\tilde t} - \omega \Phi_{, \tilde \theta}
\]
gives
\[
  \Pi_{,\tilde t} = \Phi_{,\tilde t \tilde t} - \omega \Phi_{,\tilde \theta \tilde t} \.
\]
Upon elimination of $\Phi_{,\tilde t}$ using
\[
  \Phi_{,\tilde t} = \Pi + \omega \Phi_{, \tilde \theta}
\]
I arrive at
\[
  \Pi_{\tilde t}  = 
  \nabla^2 \Phi
 +\omega \Pi_{,\tilde \theta}
 -2 D \Pi
 -2 D \omega \Phi_{,\tilde \theta}
 + 4 \pi S
\]
 
Now if I interpret $\partial _{\tilde \theta} = -{\tilde y} \partial _{\tilde x}
+ {\tilde x} \partial _{\tilde y}$ and assume I'm in rotating cartesian coordinates 
(such that $\sqrt{-\eta} = 1$) then I get
exactly what Keith has in his equation (1.20).

\section{Inverse Metric}

Starting at the scalar field metric
\[
  ds^2 = - e^{2 \Phi} dt^2 + e^{-2 \Phi} \left( dx^2 + dy^2 + dz^2 \right) 
\]
and making the standard transformation I also end up with Keith's
equation (1.25)
\[
  ds^2 = \left(-e^{2 \Phi} + r^2 \omega^2 e^{-2 \Phi} \right) dt^2 
        + e^{-2 \Phi} \left(dx^2 + dy^2 + dz^2 \right) 
		+ 2 \omega r^2 e^{-2 \Phi} d {\tilde \theta} dt
\]
or in matrix form
\[
g_{\mu\nu} = \left(
             \begin{array}{cccc}
               -e^{2 \Phi} + (x^2 + y^2) \omega^2 e^{-2 \Phi} & 
			   -y \omega e^{-2 \Phi} &
			    x \omega e^{-2 \Phi} &
				0 \\
			   -y \omega e^{-2 \Phi} &
			    e^{-2 \Phi} &
			    0  &
			    0 \\
     		    x \omega e^{-2 \Phi} &
				0 &
				e^{-2 \Phi} &
				0 \\
				0 &
				0 &
				0 & 
				e^{-2 \Phi}
			 \end{array}
			 \right)
\]
I then used this expression to calculate $g^{\mu\nu}$ and the variational derivative
$\frac{\delta g^{\mu\nu}}{\delta \Phi}$ in MathCAD and Maple.
I didn't get anything as clean as Charlie did, for example
\[
g^{00} = 1/\left(-e^{2 \Phi} + z^2 \omega^2 e^{-2 \Phi} \right)
\]

\end{document}
