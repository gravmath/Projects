%% 
%% Delete all preceding lines before processing this file by LaTeX
%>>*
%% 
%% File: geothy.tex version 1.2
%% Last revised Sat 97Mar29  at 09:50 by Misner
%% 
% 1.0  Sun 96Nov24 23:00 Misner   first draft
% 1.1  Mon 96Nov25 10:59 Misner   typos corrected
% 1.2  Sat 97Mar29 09:50 Misner   another typo corrected
%%
%% 
\documentstyle[11pt]{article}
\def\half{\mbox{$\frac{1}{2}$}}     % small built-up `one-half'

\begin{document}
%% @nokeywords@

\makeatletter
\@addtoreset{equation}{section}
\makeatother
\renewcommand{\theequation}{\thesection.\arabic{equation}}

\def\x{{\bf x}}
\def\z{{\bf z}}

\newcommand{\avg}[1]{\langle #1 \rangle}

%*<<

%
\title{Geodesic Integrator Formulae
      }
%
\author{   \sc
            Charles W. Misner\\
           \em
            Department of Physics, University of Maryland%,
           \\ \em
            College Park MD 20742-4111 USA\\
           \rm
         e-mail: \tt misner@umail.umd.edu
        }
\date{29 Mar 1997 \qquad   ver. 1.2}
%
\maketitle

%\begin{abstract}
%\end{abstract}


\section*{}
    To the BNS collaboration,\\[1.2ex]
    Here is an analytic version of the equations we (mostly KNew and
KWatt immediately) are writing a Fortran program to integrate.
    From the snippet of Alliance code we looked at in Phila Friday
(11/22) it appears that the names for the metric components should be
    \begin{equation}\label{eq-ds2}
    ds^2 = - \alpha^2 dt^2 + g_{ij} (dx^i + \beta^i\,dt)(dx^j +
\beta^j\,dt) \quad .
    \end{equation}
    Although I noticed the $\alpha$, I didn't pay attention to
whether it was $g_{ij}$ or $\gamma_{ij}$ or something else.
    We should perhaps try to get a look at Empire code also, as from
Jim York's paper I suspect they may use the names I've used before.
    As Kim and Keith look at Alliance code they should make a table
of comparable names and their mathematical, ADM code, and (when
convenient) Empire code equivalences.
    With such a table we could probably do some bits of rewriting to
Empire compatibility (should that become desirable) with some editor
macros, and I could keep the text descriptions in parallel to the
code.

    Among the quantities we need to call from Einstein solver code
are not just the metric ($\alpha$, $\beta^k$, and $g_{ij}$ or
$g^{ij}$), but also space (perhaps time) derivatives: $\alpha_{,k}$,
${\beta^\ell}_{,k}$, and ${g^{ij}}_{,k}$.
    Some of these derivatives may have names in the existing code that
we should notice if we see them.
    For instance, the quantity $-{g^{ij}}_{,k} = g^{im}
{\Gamma^j}_{mk} + g^{jm} {\Gamma^i}_{mk}$ occurs in the geodesic
equations but may be more easily available in the $\Gamma$ forms than
as derivatives of the $g^{ij}$.


\section{Geodesic Equations}
    Our current geodesic equations come in Hamiltonian form from
    \begin{equation}\label{eq-H}
    H = - \beta^k p_k + \alpha \sqrt{m^2 + g^{ij} p_i p_j}
    \quad .
    \end{equation}
    We should probably define $H = m\gamma$ and check for
conservation of $\gamma$ in the cases of time independent metrics.
    We will also need an abbreviation $\mu \equiv \sqrt{m^2 + g^{ij}
p_i p_j}$ for this square root which should not be recalculated
unnecessarily.  
    Advise me if these names cause conflicts with established code.

    The position update equations are
    \begin{equation}\label{eq-v}
    dz^k/dt = v^k(p,z,t)
    \end{equation}
    for a generic numerical integration scheme, and we have
    \begin{equation}\label{eq-gv}
    v^k(p,z,t) = -\beta^k + (\alpha/\mu) g^{kj} p_j
    \quad .
    \end{equation}
    Similarly there are momentum update equations
    \begin{equation}\label{eq-p}
    dp_k/dt = f_k(p,z,t)
    \end{equation}
    for a generic numerical integration scheme, and our equations
give
    \begin{equation}\label{eq-gp}
    f_k(p,z,t) = p_i{\beta^i}_{,k} - \mu \alpha_{,k}
    - (\alpha/2\mu) {g^{ij}}_{,k}\, p_i p_j
    \quad .
    \end{equation}

    The $z$-dependent functions in these equations vary from one
metric to another, and will eventually be calls to the ADM or Empire
code.  In the following section I give values for several sample
cases that we will be testing.

\section{Newtonian metric}

    A metric that gives the correct Newtonian limit is
    \begin{equation}\label{eq-Newton}
    ds^2 = - (1-M/r)^2 dt^2 + \delta_{ij}\,dx^i\,dx^j
    \end{equation}
    so the specification for the metric quantities is
    \begin{eqnarray}\label{eq-gNewton}
    \alpha  & = & 1 - M/r\\
    \beta^k & = & 0\\
    g^{ij} & = & \delta^{ij}
    \end{eqnarray}
    where $r^2 = z^k z^k$.
    The only nonzero metric gradient is
    \begin{equation}\label{eq-dgNewton}
    \alpha_{,k} = z^k (M/r^3)
    \quad .
    \end{equation}

\section{Schwarzschild escape metric}

    In an unfamiliar time coordinate the Schwarzschild metric can be
written
    \begin{equation}\label{eq-Schwe}
    ds^2 = - dt^2 + \delta_{ij} (dx^i - x^i \sqrt{2 M/r^3}\,dt)(dx^j -
x^j \sqrt{2 M/r^3}\,dt)
    \end{equation}
    so the specification for the metric quantities is
    \begin{eqnarray}\label{eq-gSchwe}
    \alpha  & = & 1\\
    \beta^k & = & - z^k \sqrt{2 M/r^3}\\
    g^{ij} & = & \delta^{ij}
    \end{eqnarray}
    where $r^2 = z^k z^k$.
    The metric gradients are
    \begin{eqnarray}\label{eq-dgSchwe}
    \alpha_{,k}  & = & 0\\
    {\beta^i}_{,k} & = & \sqrt{\frac{2 M}{r^3}} 
        \left( \frac{3}{2}\frac{z^i z^k}{r^2} - \delta^{ik} \right)\\
    {g^{ij}}_{,k} & = & 0 
    \quad .
    \end{eqnarray}

\section{Scalar gravity metric}

    In the spherically symmetric static case the metric in a scalar
theory of gravity can be written
    \begin{equation}\label{eq-Sclr}
    ds^2 = (1-M/r)^2 [- dt^2 + \delta_{ij}\,dx^i\,dx^j]
    \end{equation}
    so the specification for the metric quantities is
    \begin{eqnarray}\label{eq-gSclr}
    \alpha  & = & 1 - M/r\\
    \beta^k & = & 0\\
    g^{ij} & = & \delta^{ij} (1 - M/r)^{-2}
    \end{eqnarray}
    where $r^2 = z^k z^k$.
    The metric gradients are
    \begin{eqnarray}\label{eq-dgSclr}
    \alpha_{,k}  & = & z^k (M/r^3)\\
    {\beta^i}_{,k} & = 0\\
    {g^{ij}}_{,k} & = & -(2/\alpha^3)\delta^{ij} \alpha_{,k}
    \quad .
    \end{eqnarray}

\section{Schwarzschild isotropic metric}

    In isotropic spatial coordinates the Schwarzschild metric is
    \begin{equation}\label{eq-Schwiso}
    ds^2 = - \left( \frac{1-M/2r}{1+M/2r} \right)^2 dt^2 
    + \left( 1 + \frac{M}{2r} \right)^4 \delta_{ij}\,dx^i\,dx^j
    \end{equation}
    so the specification for the metric quantities is
    \begin{eqnarray}\label{eq-gSchwiso}
    \alpha  & = & \frac{1-M/2r}{1+M/2r} \\
    \beta^k & = & 0 \\
    g^{ij} & = & \delta^{ij} \left( 1 + \frac{M}{2r} \right)^{-4} 
    \end{eqnarray}
    where $r^2 = z^k z^k$.
    The metric gradients are
    \begin{eqnarray}\label{eq-dgSchwiso}
    \alpha_{,k}  & = & \frac{z^k}{r} \frac{M}{r^2}
            \left( 1 + \frac{M}{2r} \right)^{-2} \\
    {\beta^i}_{,k} & = & 0 \\
    {g^{ij}}_{,k} & = & \delta^{ij} \frac{z^k}{r} \frac{2 M}{r^2}
    \left( 1 + \frac{M}{2r} \right)^{-5} 
    \quad .
    \end{eqnarray}

\section{Schwarzschild standard metric}

    The standard Schwarzschild metric when converted to rectangular
coordinates is
    \begin{equation}\label{eq-Schwstd}
    ds^2 = - \left( 1 -\frac{2 M}{r} \right) dt^2 + \left( \delta_{ij} 
    + \frac{x^i x^j}{r^2} \frac{2M/r}{1-2M/r} \right) \,dx^i\,dx^j
    \end{equation}
    so the specification for the metric quantities is
    \begin{eqnarray}\label{eq-gSchwstd}
    \alpha  & = & \sqrt{1-2M/r}\\
    \beta^k & = & 0 \\
    g^{ij} & = & \delta^{ij} - \frac{2M}{r} \frac{z^i z^j}{r^2} 
    \end{eqnarray}
    where $r^2 = z^k z^k$.
    The metric gradients are
    \begin{eqnarray}\label{eq-dgSchwstd}
    \alpha_{,k}  & = & \frac{z^k}{r} \frac{M}{r^2}
            \left( 1 - \frac{2M}{r} \right)^{-1/2} \\
    {\beta^i}_{,k} & = & 0 \\
    {g^{ij}}_{,k} & = & \frac{2 M}{r^2} 
        \left( 3 \frac{z^i z^j z^k}{r^3}
        -\delta^{ik} \frac{z^j}{r} -\delta^{jk} \frac{z^i}{r} 
        \right) 
    \quad .
    \end{eqnarray}

\section{Vaidya metric}
    There is a spherically symmetric time dependent solution of the
Einstein equations in which an arbitrary flux of radially flowing
radiation is present.  
    This results in an arbitrary time dependence of the Schwarzschild
mass $M(t)$ and could be a useful test case for the behavior of any
numerical scheme when energy is not conserved. 
    Whether an analytic formula for the geodesic's $\gamma(t)$ can be
found to use in this check I don't know. 
    We always have the option of using a high order integrations
(which could not be used with calls to the Einstein code) as a check
on any quadratic scheme.
    I don't have the details on this case yet.

\section{Angular momentum conservation}

    Each of the above metrics is spherically symmetric, so with this
symmetry there comes a conservation law of angular momentum.
    Each metric has been written with space coordinates that are
rectangular in the sense that orthogonal rotations $x^i \mapsto A^{ij}
x^j$ with an orthogonal matrix $A$ leave the metric invariant.
    These transformations leave  $p_k dz^k$ invariant and are therefor
canonical transformations.  
    But they also leave $p_k p_k$, $r^2$, $p_k z^k$ and thus the
Hamiltonian invariant, so their generator is a constant of motion.
    Since the canonical transformation is identical to the classical
case, so is its generator.
    Thus we conclude that the angular momentum defined as
    \begin{equation}\label{eq-L}
    L_k \equiv [kij] z^i p_j
    \end{equation}
is conserved in all these examples.
    Here $[kij]$ is the antisymmetric symbol with values $0,\pm 1$ as
used in MTW.


\end{document}

    
    \begin{equation}\label{eq-tau}
             \frac{dt}{d\tau} 
            = -\frac{1}{m\alpha^2} (p_0 - \beta^k p_k)
            = \frac{1}{m\alpha}\sqrt{m^2 + g^{k\l} p_k p_\l}
            \quad .
    \end{equation}



 
