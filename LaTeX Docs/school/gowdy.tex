%
%
%	This is a file to make C W Misner's letter to
%	   Kim New
%
%   re Gowdy analytic test case
%
%	It is designed to be processed by LaTeX and to be emailed 28 Aug'97 
%
%
\documentstyle{letter}
\signature{Charlie
}

%\addtolength{\textheight}{1.0in}
%\addtolength{\topmargin}{-0.5in}

\begin{document}

\begin{letter}{%
Dr.\ Kimberly New\\
Drexel U
}


\opening{Dear Kim}

     As a try at your problems getting the Gowdy polarized case to
converge, I am sending you my formulas for an analytic solution that
I suppose you are using as the test case.  If your initial condition
or later analytic/exact solution values are different from these,
that could be part of the problem.

    I take the metric to be 
        \begin{equation}\label{eq-gdyt}
    ds^2 = t^{-1/2} e^{\lambda/2}(- dt^2 + dz^2)
           +t\, dw^2
    \end{equation}
with
    \begin{equation}\label{eq-cag}
    dw^2  =  \cosh Q (e^P\,dx^2 + e^{-P}\,dy^2) 
                + 2 \sinh Q \,dx\,dy
    \end{equation}
    but now we set $Q=0$ everywhere.
    Thus the metric components (with $x^0 = t$) are
    $g_{00} = -t^{-1/2} \exp{(\lambda/2)}$, 
    $g_{33} = -g_{00} = t^{-1/2} \exp{(\lambda/2)}$,
    $g_{11} = t \exp{(P)}$, and 
    $g_{22} = t \exp{(-P)}$
    with all others zero.

    For $P$ and $\lambda$\, I choose
    \begin{equation}
{P}(\,{t}, {z}\,) \equiv {\rm BesselJ}(\,0, 2\,{ \pi}\,{
n}\,{t}\,)\,{\rm cos}(\,2\,{ \pi}\,{n}\,{z}\,)
    \end{equation}
    and
    \begin{eqnarray}
\lefteqn{{\lambda}(\,{t}, {z}\,) \equiv  - 2\,{t}\,{ \pi}\,{n}
\,{\rm BesselJ}(\,0, 2\,{ \pi}\,{n}\,{t}\,)\,{\rm BesselJ}(\,1, 2
\,{ \pi}\,{n}\,{t}\,)\,{\rm cos}(\,2\,{ \pi}\,{n}\,{z}\,)^{2}} 
    \nonumber\\
 & & \mbox{} + 2\,{ \pi}^{2}\,{n}^{2}\,{t}^{2}\, \left( \! \,
{\rm BesselJ}(\,0, 2\,{ \pi}\,{n}\,{t}\,)^{2} + {\rm BesselJ}(\,1
, 2\,{ \pi}\,{n}\,{t}\,)^{2}\, \!  \right) 
\quad . \mbox{\hspace{40pt}}
    \end{eqnarray}
    This last was the hard part, finding a solution to the constraint
equations.  I used MapleV to check my intermediate guesses and help
me converge to a solution that it then verified.

    You should be able to just change names (e.g., switch $z$ and
$x$) but don't forget that the lapse condition must set the lapse
squared to the metric component in the propagation direction.

    \closing{%
Good luck,
    }

\end{letter}

\end{document}

