\documentclass[twocolumn]{article}
%\usepackage{makeidx,babel}


\begin{document}

%%
%%Top matter
%%
\title{A Derivation of the Transformation Relations between the 
       Clohessy-Wiltshire and the GCI Frames}
\author{Conrad Schiff \thanks{System Engineer} \\
        AI Solutions Inc.\\
	  \texttt{schiff@ai-solutions.com}}
\date{\today}
\maketitle

%%
%%Abstract
%%
\begin{abstract}
In this brief note I present the derivation of the transformations between the 
Clohessy-Wiltshire and GCI frames.  During the derivation a modest approximation
is made in order to allow a simple construction of the frames.
\end{abstract}

\section{Introduction}\label{S:intro}
The purpose of this note is to layout the necessary transformation relations 
between the Clohessy-Wiltshire (CW) and GCI frames.  We assume that the 
spacecraft's position and velocity are available in GCI and are denoted by 
${\vec r}$ and ${\vec v}$.  The first step involved will be to define the basis 
vectors of the CW frame and to evaluate the first time-derivatives of the basis 
vectors.  From these vectors the tranformation matrix from the CW to GCI frame 
and its first time-derivative will be constructed.  Finally the actual 
transformation relations between the two frames will be presented.


\section{CW Basis Vectors}\label{S:CW_vecs}
The basis vector along the CW X-axis is defined as

\begin{equation}\label{eq:Xhat}
	{\hat X} = \frac{\vec r}{  | \vec r |}.
\end{equation}

The CW Z-axis is defined as 

\begin{equation}\label{eq:Zhat}
	{\hat Z} = \frac{ \vec r \times \vec v} { |  \vec r \times \vec v  |}.
\end{equation}

And the CW Y-axis is defined as

\begin{equation}\label{eq:Yhat}
 	\hat Y = \hat Z \times \hat X.
\end{equation}

Since the CW frame is a moving (rotating) frame the time derivatives of the 
basis vector must be calculated.  Fortunately only the time derivative of 
$\hat Z$ is really complicated and once it and ${\dot {\hat X}}$ are computed 
the derivative of $\hat Y$ follows from equation (\ref{eq:Yhat}).  

Starting with $\hat X$ we obtain
\begin{eqnarray}\label{eq:Xhatdot}
	\dot{	\hat X} & = &  \frac{\vec v}{| \vec r |} - \frac{\vec r}{| \vec r |^2} 
	                     \frac{d}{dt} | \vec r | \nonumber \\
 	    	        & = &  \frac{\vec v}{| \vec r |} - \frac{\vec r}{| \vec r |^2}
 	    	               \frac{d}{dt} \sqrt{ \vec r \cdot \vec r} \nonumber \\
			        & = &  \frac{\vec v}{| \vec r |} - \frac{\vec r}{| \vec r |^2} 
			               \frac{1}{2 \sqrt{ \vec r \cdot \vec r} } \frac{d}{dt} 
			               \left( { \vec r \cdot \vec r} \right) \nonumber \\
			        & = &  \frac{\vec v}{| \vec r |} - \frac{\vec r}{| \vec r |^3} 
			               \left( { \vec r \cdot \vec v} \right).
\end{eqnarray}

For $\hat Z$ we obtain
\begin{eqnarray}\label{eq:Zhatdot}
	\dot{	\hat Z} & = & \frac{ \vec v \times \vec v + \vec r \times \vec a} 
	                         { |  \vec r \times \vec v  |} - 
	                    \frac{ \vec r \times \vec v} 
	                         { |  \vec r \times \vec v  |^2 } 
	                    \frac{d}{dt} |  \vec r \times \vec v  |  \nonumber \\
	              & = & \frac{ \vec v \times \vec v + \vec r \times \vec a} 
	                         { |  \vec r \times \vec v  |} - 
	                    \frac{ \vec r \times \vec v} 
	                         { |  \vec r \times \vec v  |^2 } 
	                    \frac{d}{dt} 
	                    \sqrt{\left( \vec r \times \vec v \right) \cdot 
	                    \left( \vec r \times \vec v \right) } \nonumber \\
	              & = & \frac{ \vec v \times \vec v + \vec r \times \vec a} 
	                         { |  \vec r \times \vec v  |} - 
	                    \frac{ \vec r \times \vec v} 
	                         { 2 |  \vec r \times \vec v  |^3 } 
	                    \frac{d}{dt} 
	                    \left( \vec r \times \vec v \right) \nonumber \\
	              & = & \frac{ \vec v \times \vec v + \vec r \times \vec a} 
	                         { |  \vec r \times \vec v  |} - 
	                    \frac{ \vec r \times \vec v} 
	                         { 2 |  \vec r \times \vec v  |^3 } 
	                    \left( \vec v \times \vec v + \vec r \times \vec a \right)
\end{eqnarray}
where $\vec a$ is the acceleration.
 
The terms in equation (\ref{eq:Zhatdot}) of the form $\vec v \times \vec v$ are 
identically zero.  This leaves 
\begin{equation}\label{eq:Zhatdot_2}
	\dot{	\hat Z} = \frac{ \vec r \times \vec a} 
	                     { |  \vec r \times \vec v  |} - 
	                \frac{ \vec r \times \vec v} 
	                     { 2 |  \vec r \times \vec v  |^3 } 
	                \left( \vec r \times \vec a \right).
\end{equation}
In this form equation (\ref{eq:Zhatdot_2}) is almost impossible to evaluate 
since the accelerations are usually evaluated via a complex set of model 
(\emph{e.g.} geopotential).  Fortunately, these terms are almost always small 
perturbations to the central force motion due to the $\frac{1}{r}$ potential 
due to the central body.  This observation then motivates the assumption that 
$\vec r$ and $\vec a$ are essentially parallel.  Thus equation 
(\ref{eq:Zhatdot_2}) reduces to 
\begin{equation}\label{eq:Zhatdot_3}
	\dot {\hat Z} = 0.
\end{equation}

Finally $\dot {\hat Y}$ is given by
\begin{equation}\label{eq:Yhatdot}
	\dot{\hat Y} = {\hat Z} \times \dot{\hat X}.
\end{equation}

\section{CW to GCI}
The transformation from the CW frame to the GCI frame is done in two steps.  In 
the first the position is converted via
\begin{equation}\label{eq:CW2GCI_pos}
	{\vec r}_{CW} = A \cdot {\vec r},
\end{equation}
where the matrix $A$ is given by
\begin{equation}\label{eq:A}
A = 
\left(
  \begin{array}{clcr}
	 \hat X \cdot \hat \imath  & \hat X \cdot \hat \jmath & \hat X \cdot \hat k  \\
	 \hat Y \cdot \hat \imath  & \hat Y \cdot \hat \jmath & \hat Y \cdot \hat k  \\
	 \hat Z \cdot \hat \imath  & \hat Z \cdot \hat \jmath & \hat Z \cdot \hat k  \\
  \end{array}
\right) \quad .
\end{equation}
The velocity is tranformed by taking the time derivative of equation 
(\ref{eq:CW2GCI_pos})
to give
\begin{equation}\label{eq:CW2GCI_vel}
	{\vec v}_{CW} = A \cdot {\vec v} + {\dot A} \cdot {\vec r},
\end{equation}
where $\dot A$ is given by
\begin{equation}\label{eq:Adot}
{\dot A}
\left(
  \begin{array}{clcr}
	 \dot {\hat X} \cdot \hat \imath  & \dot {\hat X} \cdot \hat \jmath & \dot {\hat X} \cdot \hat k  \\
	 \dot {\hat Y} \cdot \hat \imath  & \dot {\hat Y} \cdot \hat \jmath & \dot {\hat Y} \cdot \hat k  \\
	 \dot {\hat Z} \cdot \hat \imath  & \dot {\hat Z} \cdot \hat \jmath & \dot {\hat Z} \cdot \hat k  \\
  \end{array}
\right)
\end{equation}
and where the time derivatives are given in equations (\ref{eq:Xhatdot}, 
\ref{eq:Zhatdot_3},\ref{eq:Yhatdot}).


\section{GCI to CW}

Finally the transformations from the GCI to CW frame can be done by inverting 
equation (\ref{eq:CW2GCI_pos}) to give
\begin{equation}
	\vec r = A^{T} \cdot {\vec r}_{CW}
\end{equation}
and then using this result when inverting equation (\ref{eq:CW2GCI_vel}) to give
\begin{equation}
	\vec v = A^{T} \cdot \left( {\vec v}_{CW} - {\dot A} \cdot A^{T} 
	               \cdot {\vec r}_{CW} \right).
\end{equation}





                  
\end{document}
