\documentstyle[twocolumn]{article}
\bibliographystyle{plain}
\def\.{{\quad .}}  
\def\_,{{\quad ,}} 
\newcounter{bean}

\begin{document}

%**************************************************************
%
%  Front matter
%
%**************************************************************
\title{Notes on Fluid Dynamics}
\author{Conrad Schiff \thanks{Physics Graduate Student} \\
        University of Maryland\\
	  \texttt{cmschiff@erols.com}}
\date{\today}
\maketitle


%**************************************************************
%
%  Abstract
%
%**************************************************************
%\widetext
\begin{abstract}
In this notes I present a synthesis of my work on fluid dynamics.  They are a 
compilation of a simple reorganization and union of ideas from a variety of 
sources combined with my own reflections on the various subjects.  As a result
of both my personnel interests and my own style another reader most assuredly will
find them overly pedantic in some areas and terse in others.  To this I make
no attempt at rectification.

Update history: initial work August 19, 1999
\end{abstract}

%\narrowtext
%\vskip2pc]
%%%%%%%%%%%%%%%%%%%%%%%%%%%%%%%%%%%%%%%%%%%%%%%%%%%%%%%%%%%
%%What is a Fluid 
%%%%%%%%%%%%%%%%%%%%%%%%%%%%%%%%%%%%%%%%%%%%%%%%%%%%%%%%%%%
\section{What is a Fluid}\label{S:fluid_def}
In simple terms a fluid is a substance which cannot resist 
a shear force or stress without moving as can a solid.  A liquid
has intermolecular forces which hold it together so that is possesses
volume but no definite shape \cite{hughes}.

A mathematical model of a fluid is based on the idea of a continuum.
In the continuum, we don't have to keep track of individual particles.
Rather, it is convenient to ask the question: at some point in space what
are the fluid's velocity, acceleration, and thermodynamic properties.
This description of a fluid is known as an Euler description \cite{hughes}.

The number of basic variables in fluid mechanics is five: three velocity
components and two thermodynamic properties.  Any two of the thermodynamic
properties, such as pressure, temperature, density, enthalpy, entropy, etc.,
suffice to determine the state and hence all the other properties.  Hence we
need five independent equations.  these are usually the three components 
of the equation of motion, a continuity equation, and an energy equation.
Often an equation of state is introduced in order to allow the writing
of the energy equation in terms of three variables (temperature, density,
and pressure) instead of just two.  In this case we have six equations
and six unknowns.  In an incompressible fluid the energy equation is not
needed since density is taken as known and only the pressure along with the
velocity need be found in order to completely describe the fluid flow \cite{hughes}.
\\\\
\noindent $\bullet$ \textbf{Pressure} \\
The pressure in a static fluid is defined as the normal compressive force per
unit area (normal stress) acting on a surface immersed in the fluid.  The
pressure at a point is isotropic in a fluid at rest and is called 
hydrostatic pressure.  Hydrostatic pressure is the pressure used in 
thermodynamics as a fundamental property.  If the pressure varies from
place to place in the fluid a net pressure force would exist on any 
fixed volume of fluid and must be balanced by a body force, such as 
gravity, or else the fluid will move since the pressure force will
generate an acceleration.  In a dynamical situation there may exist not 
only pressure forces but shear forces and stresses as well.  However, the
pressure is still isotropic as defined above but must be measured as the
normal stress on an area which moves along locally with the fluid 
\cite{hughes}.
\\\\
\noindent $\bullet$ \textbf{Viscosity} \\
All fluids have viscosity which causes friction.  If the friction is 
negligible, we say the flow is ideal and can be describe by potential 
flow theory.  Viscosity is a measure of the fluid's resistance to shear 
when a fluid is in motion.  The defining equation for the viscosity 
$\mu$ for a Newtonian fluid is
\[
  \tau = \mu \nabla {\vec u}
\]
where $\tau$ is the shear stress and $\nabla {\vec u}$ the velocity
gradient.  In general, this hold over a large class of fluids, however
nonlinear relationships can also exist.  The two most notable are the 
plastic flows, in which the fluid behaves like a solid until it yields
and begins to flow like a viscous fluid, and the dilatant flows, where
the fluid exhibits little viscosity for small stresses but become much
more like solids for larger stresses.  The flow of air over objects
is considered ideal flow except in a thin layer, called the boundary
layer just next to the surface \cite{hughes}.
\\\\
\noindent $\bullet$ \textbf{Laminar and Turbulent Flow} \\
The terms laminar flow and purely viscous flow are used synonymously to 
mean a fluid which flows in layers. A turbulent flow in one in which the 
velocity components have random fluctuations imposed upon their mean 
values.  For a given fluid the velocity and channel configuration 
determines whether the flow will be laminar or turbulent \cite{hughes}.      
\\\\
\noindent $\bullet$ \textbf{Surface Tension} \\
Surface tension is used to loosely to identify the apparent stress in the
surface layer of a liquid.  This layer behaves like s stretched membrane
and can give rise to a pressure difference across a curved liquid surface.
Actually the surface tension is an energy associated with any fluid-fluid
interface \cite{hughes}.
\\\\
\noindent $\bullet$ \textbf{Compressible and Incompressible Flow} \\
It is customary to divide fluids into two groups - gases and liquids.
Gases are compressible and their density and changes readily with 
temperature and pressure.  Liquids are rather difficult to compress and
for most problems one might consider them incompressible \cite{hughes}.
\\\\  
\noindent $\bullet$ \textbf{Subsonic and Supersonic Flow} \\
In compressible flows there is a great distinction between flow involving
velocities less than (subsonic) or more than (supersonic) the speed of
sound.  Shock waves can only occur in flows which are supersonic 
\cite{hughes}.   
\\\\
\noindent $\bullet$ \textbf{Steady Flow} \\
A steady flow is one in which the physical properties of the fluid, the
velocity and the thermodynamic terms, at a given point in space, do not
change as a function of time \cite{hughes}
\\\\
\noindent $\bullet$ \textbf{Internal Flow} \\
By internal flow we mean pipe and channel flow, where the fluid flows
within a confining structure.  Such flow in the main part of the channel
may be considered as approximately ideal for gases, but even so a 
boundary layer (usually turbulent) develops on the walls.  In both
viscous and turbulent flow the boundary layer thickens downstream so
that it eventually extends all the way across the channel \cite{hughes}.
\\\\ 
\noindent $\bullet$ \textbf{External Flow} \\
External flow is the flow of a fluid over an object such as in 
aerodynamics.  The flow region around an object may be divided into 
three regions.  Far from the body the flow is essentially ideal, with
friction unimportant.  Near the body the fluid develops a shear layer 
(since the velocity must be zero relative to the body at the surface)
where viscosity and turbulence is important.  This frictional layer is
called a boundary layer, and may be laminar or turbulent.  the third 
layer, which is behind the body is known as the wake.  The wake is 
region of high turbulence and low pressure (hence the drag due to 
the wake) and is due to the separation of the boundary layer from 
the body surface \cite{hughes}.    






\bibliography{fluids}
\end{document}
