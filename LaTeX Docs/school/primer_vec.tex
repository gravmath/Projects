\documentclass{article}
%load packages
\usepackage{latexsym}
\usepackage{epic,eepic,graphicx,url}
%spacer commands
\newcommand{\eqc}{\ensuremath{\: ,} }
\newcommand{\eqp}{\ensuremath{\: .} }
\newcounter{bean}
\newcommand{\surp}[1]{\ensuremath{\left( {#1} \right)} }
\newcommand{\surb}[1]{\ensuremath{\left[ {#1} \right]} }
\newcommand{\surc}[1]{\ensuremath{\left\{ {#1} \right\}} }
\newcommand{\sura}[1]{\ensuremath{\left\langle #1 \right\rangle}}
\newcommand{\sure}[1]{({#1})}
\newcommand{\push} {\ensuremath{\:}}
\newcommand{\nudge}{\ensuremath{\,}}
\newcommand{\back}{\ensuremath{\!\!}}
\newcommand{\iback}{\ensuremath{\!\!\!\!}}
%equation commands
\newcommand{\be} { \begin{equation} }
\newcommand{\ee} { \end{equation}   }
\newcommand{\bes} { \[ }
\newcommand{\ees} { \]  }
\newcommand{\bea}{ \begin{eqnarray} }
\newcommand{\eea}{ \end{eqnarray}   }
\newcommand{\beas}{ \begin{eqnarray*} }
\newcommand{\eeas}{ \end{eqnarray*}   }
%index commands
\newcommand{\idx}[2] {\ensuremath{ {#1}{#2}   }}
\newcommand{\up} [1] {\ensuremath{{}^{#1}     } }
\newcommand{\dn} [1] {\ensuremath{{}_{#1}     } }
%array commands
\newcommand{\cv}  [1]{\ensuremath{\vec {#1}} }
\newcommand{\rv}  [1]{\ensuremath{{\vec {#1}}^{\,T}} }
\newcommand{\cf}  [1]{\ensuremath{\tilde {#1}} }
\newcommand{\rf}  [1]{\ensuremath{{\tilde {#1}}{\,^T}} }
\newcommand{\cs}  [1]{\ensuremath{\bar {#1}} }
\newcommand{\rs}  [1]{\ensuremath{{\bar {#1}}{\,^T}} }
\newcommand{\cuv} [1]{\ensuremath{\hat {#1}} }
\newcommand{\ruv} [1]{\ensuremath{{\hat {#1}}{\,^T}} }
\newcommand{\op}  [1]{\ensuremath{\mathbf{{#1}}}}
\newcommand{\dt}  [1]{\ensuremath{\dot {{#1}}} }
\newcommand{\ddt} [1]{\ensuremath{\ddot {{#1}}} }
\newcommand{\trps}[1]{\ensuremath{{#1}^{\,T} } }
\newcommand{\bv}  [1]{\ensuremath{\cs{e}_{#1}} }
\newcommand{\bd}  [1]{\ensuremath{\cf{\gw}^{#1}} }
%\newcommand{\rank}[2]{\ensuremath{\surp{\begin{array}{c}{#1 \\ #2}\end{array}}} }
\newcommand{\rank}[2]{\ensuremath{\surp{#1, #2}}}
%inner products
\newcommand{\ipp}[2]{\ensuremath{ (  {#1} ,     {#2} ) } }
\newcommand{\ipb}[2]{\ensuremath{ \langle {#1} ,     {#2} \rangle } }
\newcommand{\ipD}[2]{\ensuremath{ \langle {#1} |     {#2} \rangle } }
\newcommand{\ipd}[2]{\ensuremath{         {#1} \cdot {#2}         } }
%derivative and integral commands
\newcommand{\dby}   [1]{\ensuremath{ \frac{d}{d #1}} }
\newcommand{\dxby}  [2]{\ensuremath{ \frac{d #1}{d #2}} }
\newcommand{\pdby}  [1]{\ensuremath{ \frac{\partial}{\partial #1}} }
\newcommand{\pdxby} [2]{\ensuremath{ \frac{\partial #1}{\partial #2}} }
\newcommand{\myint} [1]{\ensuremath{ \int #1 \push} }
%variations
\newcommand{\var}[1]{\ensuremath{\delta {#1}}}
\newcommand{\vwrt}[2]{\ensuremath{\idx{\delta {#1}\vert}{\dn{\delta {#2}}}} }
\newcommand{\IBP}{\ensuremath{\stackrel{\mbox{\tiny{IBP}}}{=}}}
%common abreviations
\newcommand{\wrt}{with respect to }
\newcommand{\bdyterms}{boundary-terms }
\newcommand{\ibp}{integration-by-parts }
\newcommand{\dfunc}{\gd\back-function }
\newcommand{\dfuncs}{\gd\back-functions }
%special characters
\newcommand{\ga}{\ensuremath{\alpha} }
\newcommand{\gba}{\ensuremath{\bar \alpha} }
\newcommand{\gha}{\ensuremath{\hat \alpha} }
\newcommand{\gta}{\ensuremath{\tilde \alpha} }
\newcommand{\gb}{\ensuremath{\beta} }
\newcommand{\gbb}{\ensuremath{\bar \beta} }
\newcommand{\ghb}{\ensuremath{\hat \beta} }
\newcommand{\gtb}{\ensuremath{\tilde \beta} }
\newcommand{\gd}{\ensuremath{\delta} }
\newcommand{\gbd}{\ensuremath{\bar \delta} }
\newcommand{\ghd}{\ensuremath{\hat \delta} }
\newcommand{\gtd}{\ensuremath{\tilde \delta} }
\newcommand{\gD}{\ensuremath{\Delta} }
\newcommand{\gbD}{\ensuremath{\bar \Delta} }
\newcommand{\ghD}{\ensuremath{\hat \Delta} }
\newcommand{\gtD}{\ensuremath{\tilde \Delta} }
\newcommand{\get}{\ensuremath{\eta} }
\newcommand{\gbet}{\ensuremath{\bar \eta} }
\newcommand{\ghet}{\ensuremath{\hat \eta} }
\newcommand{\gtet}{\ensuremath{\tilde \eta} }
\newcommand{\gf}{\ensuremath{\phi} }
\newcommand{\gbf}{\ensuremath{\bar \phi} }
\newcommand{\ghf}{\ensuremath{\hat \phi} }
\newcommand{\gtf}{\ensuremath{\tilde \phi} }
\newcommand{\gF}{\ensuremath{\Phi} }
\newcommand{\gbF}{\ensuremath{\bar \Phi} }
\newcommand{\ghF}{\ensuremath{\hat \Phi} }
\newcommand{\gtF}{\ensuremath{\tilde \Phi} }
\newcommand{\gG}{\ensuremath{\Gamma} }
\newcommand{\gbG}{\ensuremath{\bar \Gamma} }
\newcommand{\ghG}{\ensuremath{\hat \Gamma} }
\newcommand{\gtG}{\ensuremath{\tilde \Gamma} }
\newcommand{\ggm}{\ensuremath{\gamma} }
\newcommand{\gbgm}{\ensuremath{\bar \gamma} }
\newcommand{\ghgm}{\ensuremath{\hat \gamma} }
\newcommand{\gtgm}{\ensuremath{\tilde \gamma} }
\newcommand{\gl}{\ensuremath{\lambda} }
\newcommand{\gbl}{\ensuremath{\bar \lambda} }
\newcommand{\ghl}{\ensuremath{\hat \lambda} }
\newcommand{\gtl}{\ensuremath{\tilde \lambda} }
\newcommand{\gL}{\ensuremath{\Lambda} }
\newcommand{\gbL}{\ensuremath{\bar \Lambda} }
\newcommand{\ghL}{\ensuremath{\hat \Lambda} }
\newcommand{\gtL}{\ensuremath{\tilde \Lambda} }
\newcommand{\gm}{\ensuremath{\mu} }
\newcommand{\gbm}{\ensuremath{\bar \mu} }
\newcommand{\ghm}{\ensuremath{\hat \mu} }
\newcommand{\gtm}{\ensuremath{\tilde \mu} }
\newcommand{\gn}{\ensuremath{\nu} }
\newcommand{\gbn}{\ensuremath{\bar \nu} }
\newcommand{\ghn}{\ensuremath{\hat \nu} }
\newcommand{\gtn}{\ensuremath{\tilde \nu} }
\newcommand{\gp}{\ensuremath{\pi} }
\newcommand{\gbp}{\ensuremath{\bar \pi} }
\newcommand{\ghp}{\ensuremath{\hat \pi} }
\newcommand{\gtp}{\ensuremath{\tilde \pi} }
\newcommand{\gq}{\ensuremath{\theta} }
\newcommand{\gbq}{\ensuremath{\bar \theta} }
\newcommand{\ghq}{\ensuremath{\hat \theta} }
\newcommand{\gtq}{\ensuremath{\tilde \theta} }
\newcommand{\gr}{\ensuremath{\rho} }
\newcommand{\gbr}{\ensuremath{\bar \rho} }
\newcommand{\ghr}{\ensuremath{\hat \rho} }
\newcommand{\gtr}{\ensuremath{\tilde \rho} }
\newcommand{\gs}{\ensuremath{\sigma} }
\newcommand{\gbs}{\ensuremath{\bar \sigma} }
\newcommand{\ghs}{\ensuremath{\hat \sigma} }
\newcommand{\gts}{\ensuremath{\tilde \sigma} }
\newcommand{\gt}{\ensuremath{\tau} }
\newcommand{\gbt}{\ensuremath{\bar \tau} }
\newcommand{\ght}{\ensuremath{\hat \tau} }
\newcommand{\gtt}{\ensuremath{\tilde \tau} }
\newcommand{\gw}{\ensuremath{\omega} }
\newcommand{\gbw}{\ensuremath{\bar \omega} }
\newcommand{\ghw}{\ensuremath{\hat \omega} }
\newcommand{\gtw}{\ensuremath{\tilde \omega} }
\newcommand{\gW}{\ensuremath{\Omega} }
\newcommand{\gbW}{\ensuremath{\bar \Omega} }
\newcommand{\ghW}{\ensuremath{\hat \Omega} }
\newcommand{\gtW}{\ensuremath{\tilde \Omega} }
\newcommand{\gy}{\ensuremath{\psi} }
\newcommand{\gby}{\ensuremath{\bar \psi} }
\newcommand{\ghy}{\ensuremath{\hat \psi} }
\newcommand{\gty}{\ensuremath{\tilde \psi} }
\newcommand{\gY}{\ensuremath{\Psi} }
\newcommand{\gbY}{\ensuremath{\bar \Psi} }
\newcommand{\ghY}{\ensuremath{\hat \Psi} }
\newcommand{\gtY}{\ensuremath{\tilde \Psi} }
%Script letters
\newcommand{\mA}{\ensuremath{\mathcal A} }
\newcommand{\mB}{\ensuremath{\mathcal B} }
\newcommand{\mC}{\ensuremath{\mathcal C} }
\newcommand{\mD}{\ensuremath{\mathcal D} }
\newcommand{\mE}{\ensuremath{\mathcal E} }
\newcommand{\mF}{\ensuremath{\mathcal F} }
\newcommand{\mG}{\ensuremath{\mathcal G} }
\newcommand{\mH}{\ensuremath{\mathcal H} }
\newcommand{\mI}{\ensuremath{\mathcal I} }
\newcommand{\mJ}{\ensuremath{\mathcal J} }
\newcommand{\mK}{\ensuremath{\mathcal K} }
\newcommand{\mL}{\ensuremath{\mathcal L} }
\newcommand{\mM}{\ensuremath{\mathcal M} }
\newcommand{\mN}{\ensuremath{\mathcal N} }
\newcommand{\mO}{\ensuremath{\mathcal O} }
\newcommand{\mP}{\ensuremath{\mathcal P} }
\newcommand{\mQ}{\ensuremath{\mathcal Q} }
\newcommand{\mR}{\ensuremath{\mathcal R} }
\newcommand{\mS}{\ensuremath{\mathcal S} }
\newcommand{\mT}{\ensuremath{\mathcal T} }
\newcommand{\mU}{\ensuremath{\mathcal U} }
\newcommand{\mV}{\ensuremath{\mathcal V} }
\newcommand{\mW}{\ensuremath{\mathcal W} }
\newcommand{\mX}{\ensuremath{\mathcal X} }
\newcommand{\mY}{\ensuremath{\mathcal Y} }
\newcommand{\mZ}{\ensuremath{\mathcal Z} }
%references and citations
\newcommand{\refq}[1]{\sure{\ref{#1}}}
\newcommand{\refp}[1]{\ref{#1}}
\newcommand{\refs}[1]{\cite{#1}}
%Thermo, Fluid, and GR objects
%%%particle kinematics
\newcommand{\ptraj}  [1]{\ensuremath{\idx{z}{\up{#1}}}}
\newcommand{\pdtraj} [1]{\ensuremath{\idx{\dt{z}}{\up{#1}}}}
\newcommand{\pvel}   [1]{\ensuremath{\idx{u}{\dn{#1}}}}
\newcommand{\ppos}   [1]{\ensuremath{\idx{z}{\up{#1}}}}
\newcommand{\pdpos}  [1]{\ensuremath{\idx{\dt{z}}{\up{#1}}}}
\newcommand{\ptvel}  [1]{\ensuremath{\idx{u}{\dn{#1}}}}
\newcommand{\pdtvel} [1]{\ensuremath{\idx{\dt{u}}{\dn{#1}}}}
\newcommand{\ptraja} [2]{\ensuremath{\idx{z}{\dn{#1}\up{#2}}}}
\newcommand{\pdtraja}[2]{\ensuremath{\idx{\dt{z}}{\dn{#1}\up{#2}}}}
\newcommand{\pvela}  [2]{\ensuremath{\idx{u}{\dn{#1#2}}}}
\newcommand{\sm}     [2]{\ensuremath{W\!\surp{\cv{#1} - \cv{#2}}}}
%%%thermodynamic parameters
\newcommand{\gtro}{\ensuremath{\idx{\gtr}{\dn{0}}} }
\newcommand{\ierg}{\ensuremath{e\surp{\gr\surp{x}}} }
\newcommand{\erg} {\ensuremath{e} }
%%%metric functions
\newcommand{\bV}[2] {\idx{\cv{#1}}{\dn{#2}} \nudge }
\newcommand{\bF}[2] {\idx{\cf{#1}}{\up{#2}} \nudge }
\newcommand{\Jac}[2]{\idx{\gL}{\up{#1}\dn{#2}} \nudge }
\newcommand{\jac}[2]{\idx{\gL}{\dn{#1}\up{#2}} \nudge }
\newcommand{\Kd}[2] {\idx{\gd \nudge}{\up{#1}\dn{#2}} \nudge }
\newcommand{\kd}[2] {\idx{\gd}{\dn{#1}\up{#2}} \nudge }
\newcommand{\Cnx}[2]{\idx{\gG}{\up{#1}\dn{#2}} \nudge }
\newcommand{\met} [2]{\ensuremath{\idx{g}{\dn{#1 #2}}}}
\newcommand{\imet}[2]{\ensuremath{\idx{g}{\up{#1 #2}}}}
\newcommand{\metd}   {\ensuremath{ g } }
\newcommand{\smet} [1]{\ensuremath{\sura{\idx{g}{\dn{#1}}}} }
\newcommand{\simet}[1]{\ensuremath{\sura{\idx{g}{\up{#1}}}} }
\newcommand{\smetd}   {\ensuremath{\sura{g}}}
%functional arguments
\newcommand{\contarg}{\ensuremath{\surp{\idx{a}{\up{0}};\cv{a}}} }
\newcommand{\harg}   {\ensuremath{\surp{a^0,\cv{r}_1}} }
\newcommand{\parg}   {\ensuremath{\surp{t,\cv{z}}} }
\newcommand{\farg}   {\ensuremath{\surp{t,\cv{x}}} }
%variations
\newcommand{\vIz}[1] {\ensuremath{\vwrt{I}{\ptraj{#1}}} }
%useful relations
\newcommand{\normeq}[2]{\ensuremath{ \imet{#1}{#2} \pvel{#1}
\pvel{#2} + 1} }

%bibtex stuff
\pagenumbering{arabic}
\bibliographystyle{plain}
%backwards compatibility
\def\a{{\vec a}}
\def\x{{\vec x}}
\def\z{{\vec z}}
\def\P{{\Phi}}
\def\l{{\ell}}
\def\Pl{{\Phi_{\ell}}}
\def\D{{\Delta}}
\def\.{{\quad .}}
%\def\_,{{\quad ,}}
\def\half{\mbox{$\frac{1}{2}$}}
%\def\_.{{\quad .}}
\def\po{{\hat \rho}_{0}}

%constraint equations
\newcommand{\scpos}{\ensuremath{\idx{\cv{x}}{(t)}} }
\newcommand{\scvel}{\ensuremath{\idx{\cv{v}}{(t)}} }
\newcommand{\tvec} {\ensuremath{\cuv{\idx{u}{\dn{\gG}}}} }
\newcommand{\tvect}{\ensuremath{\ruv{\idx{u}{\dn{\gG}}}} }
%Lagrange multipliers
\newcommand{\lx} {\idx{\cv{\gl}}{\dn{x}} }
\newcommand{\lxt}{\idx{\cv{\gl}}{\dn{x}\up{T}} }
\newcommand{\lv} {\idx{\cv{\gl}}{\dn{v}} }
\newcommand{\lvt}{\idx{\cv{\gl}}{\dn{v}\up{T}} }
\newcommand{\lm} {\idx{\gl}{\dn{m}} }
\newcommand{\dlx} {\idx{\dt{\cv{\gl}}}{\dn{x}} }
\newcommand{\dlxt}{\idx{\dt{\cv{\gl}}}{\dn{x}\up{T}} }
\newcommand{\dlv} {\idx{\dt{\cv{\gl}}}{\dn{v}} }
\newcommand{\dlvt}{\idx{\dt{\cv{\gl}}}{\dn{v}\up{T}} }
\newcommand{\dlm} {\idx{\dt{\gl}}{\dn{m}}}
%equations
\newcommand{\poseq}{\ensuremath{\dt{\cv{x}} - \cv{v}} }
\newcommand{\veleq}{\ensuremath{\dt{\cv{v}} - \cv{g} -
                                       \frac{\gG}{m}\cuv{\idx{u}{\dn{\gG}}}} }
\newcommand{\meq}{\ensuremath{\dt{m} - \frac{\gG}{c}} }
\newcommand{\veceq}{\ensuremath{\ipd{\tvect}{\tvec} - 1} }
\newcommand{\teq}  {\ensuremath{\gG \surb{\idx{\gG}{\dn{max}} - \gG} - \idx{q}{\up{2}}} }
\newcommand{\gravjac}{\frac{\partial \cv{g}}{\partial \cv{x}}}
\newcommand{\peq}  {\ensuremath{\dlx  + \ipd{\gravjac}{\lv}}}
\newcommand{\peqt} {\ensuremath{\dlxt + \ipd{\lvt}{\gravjac}}}



\begin{document}

\title{A Primer Vector Cookbook \thanks{Work performed under NASA Contract ?????} }
\author{Conrad Schiff \thanks{Chief Scientist} \\
        a.i. solutions, Inc.\\
      \texttt{schiff@ai-solutions.com}}
\date{11/12/01}
\maketitle

%%
%%Abstract
%%
%\begin{abstract}
%
%\end{abstract}

\section{Introduction}


In this document, we detail the derivation and application of the
primer vector equations in obtaining fuel optimal trajectories. We
have undertaken this work for three reasons.  First of all, while
there are numerous works available in the literature, none of them
were developed solely as a tutorial to start the mission designer
working with the system. Secondly, many of them leave the details
of the derivation to Lawden's text, which while a classic is
difficult to read.  Finally, by directly exploiting the
Hamiltonian formulation of the problem, these works ignore the
possible contribution of boundary terms to the formulation.  We
attempt to remedy these points by focusing on a more `tutotial'
approach.  This document is to read in tandem with the
accompanying document \cite{leonardo}.

\section{Deriving the Primer Equations}\label{derive_primer}

There are numerous ways to derive the primer vector equations.  We
start with the Lagrangian approach in which a functional is built
that minimizes the fuel used subject to a set of system
constraints.  The constraints are included in the formulation by
the use of Lagrange multipliers.  Variations of the functional
with respect to the system variables yields the full set of primer
vector equations.  This set is then reduced to a fundamental few
equations which can then be applied to the desired spacecraft
trajectory.

To begin, we consider what our system variables are.  They are the
position and velocity of the spacecraft, \scpos  and \scvel, the
mass of the spacecraft $m$, and the thrust magnitude and
direction, \gG and \tvec. (We avoid using the symbol $T$ to denote
thrust since it will be used exclusively to denote the matrix
transpose.)

Now that we have identified our system variables, we have next to
identify the constraints that they must obey.  The constraints
are:
\begin{itemize}
  \item Spacecraft position and velocity must obey the correct equations of motion
  (position and velocity)
  \item Thrust direction must be specified as a unit vector
  \item Thrust magnitude must be bounded between 0 and
  \idx{\gG}{\dn{max}}
  \item Mass flow rate must be given by the standard relation
\end{itemize}
To each of the items above, we can associate a constraint equation
that mathematical enforces the constraints.  For the spacecraft
position the proper equation is
\be
  \poseq = 0 \eqc
\ee
where \dt{\cv{x}} is the first derivative of \scpos.  Likewise
the proper equation for the spacecraft velocity is \be
  \veleq = 0 \eqc
\ee where \dt{\cv{v}} is the first derivative of \scvel, \cv{g} is
the acceleration due to gravity, and $\frac{\gG}{m} \tvec$ is the
acceleration due to the thrust.

For the thrust direction given by \tvec the constraint equation is
\be
  \veceq = 0 \eqp
\ee
The thrust magnitude presents a challenge.  Unlike the other
constraint equations, the one for the thrust is an inequality
constraint
\[
0 \leq \gG \leq \idx{\gG}{\dn{max}} \eqp
\]
Using the standard method of introducing a slack variable to
eliminate the inequality \cite{in}, the thrust magnitude
constraint equation can now be written as
\be
  \teq = 0 \eqc
\ee
where $q$ is the slack variable ensuring the inequality.



\subsection{Building the Lagrangian}
Once the constraints equations have been identified, the next step
is to build a Lagrangian functional that can be varied to subject
to those constraints.  This is straightforwardly obtained by
forming a sum of each constraint equation multiplied by a Lagrange
multiplier.  The resulting form is
\bea\label{eq:def_L}
  L & = &   \ipd{\lxt}{\surp{\poseq}} + \ipd{\lvt}{\surp{\veleq}} + \lm \surp{\meq} \nonumber \\
    &   & + \gs_1 \surp{\veceq} + \gs_2 \surp{\teq} \eqp
\eea The Lagrange multipliers, \lx, \lv, and \lm are distinguished
from the multipliers \idx{\gs}{\dn{1}} and \idx{\gs}{\dn{2}} since
they are actually adjoined to dynamical constraints (\textit{i.e.}
the differential equations of motion).

\subsection{The Variations}

The next step is to take the variation of the action functional
defined as $S = \int dt L$ (where L is defined in
\sure{\ref{eq:def_L}}) \wrt following variables: \lx, \lv, \lm,
\idx{\gs}{\dn{1}}, \idx{\gs}{\dn{2}}, \scpos, \scvel, \gG, $m$,
$q$, and \tvec (11 in all). Fortunately, the variations of $S$
\wrt the first five variables (\lx, \lv, \lm, \idx{\gs}{\dn{1}},
and \idx{\gs}{\dn{2}}) yield nothing more than the constraint
equations we started with.  Let's illustrate this point.  The
variation of $S$ \wrt to \lx is given by
\be
  \vwrt{S}{\lx} = \int dt \push \ipd{\var{\lxt}}{\surp{\poseq}} \eqp
\ee
Since \var{\lxt} is arbitrary, setting the variation
\vwrt{S}{\lx} = 0 implies that the term it multiplies,
\surp{\poseq}, must be identically zero. Thus we recover the
dynamical constraint equation \poseq that we started with.

Less trivial are the equations that result from the other
variations (\textit{i.e.} \scpos, \scvel, \gG, $m$, $q$, and
\tvec). Of these, the variations of \scpos, \scvel, and $m$ are
most interesting since the enter the functional in terms of their
time derivatives.  This causes their corresponding Lagrange
multipliers, \lx, \lv, and \lm to acquire a time evolution.  To
see this consider the variation of $S$ \wrt \scpos.

\be \vwrt{S}{\scpos} = \int dt \push \surp{
  \ipd{\lxt}{\var{\dt{\cv{x}}}} - \ipd{\lvt}{\gravjac} \cdot \push \var{\scpos} } \ee

An \ibp is used to take the time derivative off of the
\var{\dt{\cv{x}}} and to place it on the term \lxt at the expense
of a minus sign and \bdyterms.

\be \vwrt{S}{\scpos} = \int dt \push
\ipd{\surp{\peqt}}{\var{\scpos}} + \bdyterms  \ee

\[
\idx{\dt{\cv{\gl}}}{\dn{x}\up{T}}
\]
Setting the variation of the





\subsection{Reducing the Equation Set}

\section{The Optimization Algorithm}

\section{References}
\bibliography{primer}

\end{document}
