where the partial derivatives of the density
\[
  \rho = \sqrt{
           \frac{-g_{\mu\nu}(z) {\dot z}^{\mu}(t) {\dot z}^{\nu}(t)}
                {-g(x)}
              }
         m W(\vec x - \vec z)
\]
need to be computed.  Before tackling the entire computation it is
better to start with the partial derivative of the `smoothed'
metric functions $g^{\mu\nu}(z)$ with respect to the metric
functions defined at the field coordinates $g^{\alpha\beta}(x)$.
Since the defining relation between the two is linear we arrive at
the simple relation (assuming the values of the time coordinate in
the two expression are equal)
\bea
  \frac{\partial}{\partial g^{\alpha \beta}(x)} \int d^3x \,
   g^{\mu\nu}(x) W(\vec x - \vec z)
& = &
  \int d^3x \,\frac{\partial g^{\mu\nu}(x)}{\partial g^{\alpha\beta}(x)}
   W(\vec x -\vec z) \nonumber \\
& = &
  \int d^3x \, {\delta^\mu}_{\alpha} {\delta^{\nu}}_{\beta}
W(\vec x - \vec z) = {\delta^\mu}_{\alpha} {\delta^{\nu}}_{\beta}
\eqp  \nonumber \eea
This result will form the basis for all
subsequent results.  For example the derivative of the metric
functions $g_{\mu\nu}(z)$ with respect to the $g^{\alpha\beta}(x)$
is easily related to the above result using the chain rule.  Doing
so yields
\[
  \frac{\partial g_{\mu\nu}(z)}{\partial g^{\alpha\beta}(x)} =
  \frac{\partial g_{\mu\nu}(z)}{\partial g^{\sigma\tau}(z)}
  \frac{\partial g^{\sigma\tau}(z)}{\partial g^{\alpha\beta}(x)}
  \eqp
\]
Using the fundamental relation between the field and `smoothed'
g's derived above, this reduces to
\[
  \frac{\partial g_{\mu\nu}(z)}{\partial g^{\alpha\beta}(x)} =
  - g_{\alpha\mu}(z)g_{\beta\nu}(z) \eqp
\]
We are ready to calculate the derivative of $\rho$. Operating the
derivative on the definition of the density gives
\bea
  \frac{\partial \rho}{\partial g^{\alpha\beta}(x)}
   & = &
     \frac{\partial}{\partial g^{\alpha\beta}(x)}
       \left( m W(\vec x - \vec z)
       \sqrt{ \frac{-g_{\mu\nu}(z){\dot z}^{\mu} {\dot z}^{\nu}}{-g(x)} } \right) \nonumber \\
   & = &
     m W(\vec x - \vec z) \left\{
       \frac{1/2}{ \sqrt{-g(x)} \sqrt{-g_{\mu\nu}(z){\dot z}^{\mu} {\dot z}^{\nu}}}
       \frac{\partial}{\partial g^{\alpha\beta}(x)}
       \left( -g_{\mu\nu}(z){\dot z}^{\mu} {\dot z}^{\nu}\right) \right. \nonumber \\
   &   &
     + \left. \sqrt{-g_{\mu\nu}(z){\dot z}^{\mu} {\dot z}^{\nu}}
       \frac{1}{\left( \sqrt{-g(x)} \right)^2}
       \frac{\partial}{\partial g^{\alpha\beta}} \sqrt{-g(x)}
     \right\} \nonumber \eqp
\eea
The only term not dealt with above is the term
$\frac{\partial \sqrt{-g(x)}}{\partial g^{\alpha\beta}(x)}$ which
can be evaluated using the standard relationship to yield
\[
  \frac{\partial \sqrt{-g(x)}}{\partial g^{\alpha\beta}(x)} =
  \frac{-1}{2} \sqrt{-g(x)}g_{\mu\nu}(x)
  \frac{\partial g^{\mu\nu}(x)}{\partial g^{\alpha\beta}(x)} =
  \frac{-1}{2} \sqrt{-g(x)}g_{\alpha\beta}(x) \eqp
\]
Using all the relations derived above and using the definition of
the density, we arrive at our final expression for the derivative
of $\rho$ with respect to the metric function defined on the field
coordinates.
\[
\frac{\partial \rho}{\partial g^{\alpha\beta}} = \frac{1}{2} \rho
\left\{u_{\alpha}(t) u_{\beta}(t) + g_{\alpha\beta} \right\} \eqp
\]
The presence of the kernel in $\rho$ guarantees that
$u_{\alpha}(t;\vec x) = u_{\alpha}(t)$ within the support of
$W(\vec x - \vec z)$.  Likewise for the pressure, which we assume
is defined via an equation of state $P = P(\rho)$ . This behavior
shows that our fat particle moves as a rigid body. Substituting
this result into the variation of the action yields the following
integral relation
\bea
 \delta I \vert_{\delta g^{\mu\nu}}
 & = &
   \int d^4x \, \delta g^{\alpha\beta}
   \left\{ \frac{1}{16\pi} G_{\mu\nu} \sqrt{-g} -
   \frac{P}{\rho^2}\frac{\rho}{2}
   \left(u_{\alpha}u_{\beta} + g_{\alpha\beta} \right)
   W(\vec x - \vec z) m \Lambda \right. \nonumber \\
 &   &
   \left. -m(1+e) \Lambda \frac{1}{2} u_{\alpha} u_{\beta} W(\vec x -\vec z)
   \right\} \nonumber \eqp
\eea
Assuming that the variations can be arbitrarily specified and
setting the variation equal to zero yields the following field
equations
\[
G_{\alpha\beta} = 8 \pi \left[ P g_{\alpha\beta} + \rho \left( 1 +
e + \frac{P}{\rho} \right) u_{\alpha}u_{\beta} \right] \eqc
\]
which are the `smoothed' equivalents of the corresponding
continuum ones.

At this point we have used up all the freedom we had in picking
the combination of coordinates used in the defining relation for
the density.  To summarize, we have found that we could derive the
expected `smoothed' relations for the continuum equations obtained
from the variations of $\Lambda$, $u_{\mu}$, and $g^{\alpha\beta}$
by using the odd combination
\[
  \rho = m W(\vec x - \vec z) \frac{ \sqrt{-g_{\mu\nu}(z) {\dot
  z}^{\mu} {\dot z}^{\nu}}}{\sqrt{-g(x)}}
\]
in defining the density in conjunction with the SPH smoothing
approximation for rational functions.  The final set of equations,
derived from the variation of the particle path $z^{\mu}$ must now
be calculated.  If it provides a reasonable set of equations, that
is to say that a single fat particle that starts at rest will stay
at rest, we can then feel that the entire theory is on good
footing.
