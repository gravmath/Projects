\section{Introduction}

In our previous work we developed a Lagrangian variational
principle for a self-gravitating fluid based on the action \bes
I_{matter} = I_{grav} + \myint{d^4a} \gtro (1 + \erg) \surb{
\pdtraj{\gm} \pvel{\gm} - \gL \mH} \eqc \ees where the parameter
\gtro is related to the number density \gr by \bes
  \gtro = \gr \contarg \sqrt{
        \frac{ -{\bar g} \contarg }{ -g_{\bar 0 \bar
        0}\contarg}}
\ees and, as the notation implies, the metric functions in \gtro
are evaluated in terms of a co-moving frame that is at rest with
respect to the fluid flow. The internal energy $e = e(\gr)$ gives
rise to the pressure required to keep the fluid from collapsing.
The last term leads to the Euler equations determining how the
fluid flows.  The super-Hamiltonian \bes
  \mH = \frac{1}{2}\surp{ \imet{\gm}{\gn} \pvel{\gm} \pvel{\gn} + 1}
\ees expresses the standard four-velocity normalization and is
identically zero along the proper geodesic motion \ptraj{\gm} of
the fluid. The Lagrange multiplier \gL enforces this constraint.
All told, this variational principle requires the variation of the
metric components \imet{\gm}{\gn}, the fluid-element trajectories
\ptraj{\gn}, the conjugate four-velocities \pvel{\gm}, and the
Lagrange multiplier \gL \back. These variations are performed
subject to the background constraint of the conservation of baryon
number. In our earlier work we wrote this conservation equation as
\[
  \gtro = \gtr J
\]
where \gtr has analogous definition as \gtro  above
\[
\gtr(z) = \gr(z) \sqrt{\frac{-\metd(z)}{-\met{\gm(z)}{\gn}
\ptraj{\gm} \ptraj{\gn}}} \eqc
\]
$\ptraj{\gm} \equiv \frac{d z^\gm}{d a^0}$, and the metric
functions are evaluated along the fluid flow lines. The Jacobean
determinant is given by $J \equiv \det \surp{ \frac{\partial
z^{\mu}}{\partial a^{\nu}}}$.

What we wish to do in this work is to develop a coupled
matter-field system for a single self-gravitating fluid element.
The basic approach will be based on the above action but will be
specialized to a single particle whose extent will be spread out
over a finite coordinate distance using a smoothing kernel.  This
discretization will be affected using the relation $\gtro = m \gd
\surp{\cv{a} - \cv{r}_1}$.  The internal energy will not be set
equal to zero in the action.  At a glance, this may seem like
nonsense since we expect the internal energy to arise from an
interaction amongst several particles.  However, in physics we
often refer to collections of particles like a star, or a lake, or
a glass of water in singular terms.  What we are doing here is to
extend this concept to a single fluid element in General
Relativity.  There will necessarily be an approximation that all
of the constituents of the fluid element, which we are ignoring,
move with the same four-velocity.  Thus the element will resist
shear stresses and remain rigid through its course. Our hope for
this program is that the resulting field equations provide a
computational tool for modelling black holes in 3-d numerical
relativity.  However, before we try to model this difficult
problem we will attempt to see if a single fat particle can
provide a black hole-like solution to the Einstein field
equations.

\section{Deriving the Equations of Motion}

In this section we derive the equations of motion for our single
fluid element, hereafter denoted as a single fat particle. As
discussed in the Introduction, we arrive at a single fat particle
action by making the substitution
\[
 \gtro = m \gd(\cv{a} - \cv{r}_1) \eqp
\]
Inserting this expression into the action leads to
\[
  I = I_{grav} + \int da^0 m (1 + \erg_1) \surp{ \pdtraja{1}{\gm}
  \pvela{1}{\gm} - \gL {\mathcal H}_1 } \eqc
\]
where we use the following notation
\\ \\
\bea
  \pdtraja{1}{\gm} & = &  \pdtraj{\gm}\harg     \push \nonumber \\
  \pvela{1}{\gm}   & = &  \pvel{\gm} \harg      \push \nonumber \\
  \erg_1              & = &  \erg\surp{\gr\harg}      \push \nonumber \\
  {\mathcal H}_1   & = & \frac{1}{2}(\imet{\gm}{\gn}\harg \pvela{1}{\gm} \pvela{1}{\gn} + 1) \eqp   \nonumber
\end{eqnarray}
The `1' index at this point is excess baggage reminding us of our
humble beginnings with one particle but serving no other purpose.
Thus we drop it hereafter and just keep the single particle
concept in the back of our minds.

The introduction of the four-dimensional \dfunc $\gd(x^\gm -
z^\gm) = \gd(x^0 - z^0) \, \gd(\cv{x} - \cv{z})$ allows the
hydrodynamic action to be written in term of the gravitational
field variables. \beas
I & = & \frac{1}{16 \pi} \myint{d^4x} R(x) \sqrt{-g(x)} \\
  &   & + m \myint{da^0} d^4x \push (1 + \erg) \surb{ \ptraj{\gm} \pvel{\gm} -
\gL \mathcal{H} } \gd(x^0 - z^0) \gd(\cv{x} - \cv{z}) \eeas Now
use path invariance and say $da^0 = dz^0$.  This allows us to
perform the $da^0$ integration and eliminate one \dfunc. The
remaining three-dimensional \dfunc over the spatial variables is
then approximated by a smoothing kernel leading to:
\[
I =  \int d^4x \push \surc{ \frac{R(x) \sqrt{-g(x)}}{16 \pi} + m
(1 + \erg) \surb{ \ptraj{\gm} \pvel{\gm} - \gL\mathcal{H} }
\sm{x}{z} }
\]
A delicate point arises here.  The presence of the \dfuncs meant
that we could be sloppy, allowing ourselves the luxury of
switching back and forth between the fluid trajectory variables
\ptraj{\gm} and the field variables $x^\mu$.  Once the
approximation of the spatial \dfuncs with the smoothing kernel has
been made, we are compelled to pick where our functions live.

In general, the choice for each function appearing in the action
is relatively straightforward.  Only the density, tentatively
defined by
\[
  \gr = \frac { \sqrt{-\met{\gm}{\gn} \pdtraj{\gm} \pdtraj{\gn}}}
              { \sqrt{-g}} m \sm{x}{z} \eqc
\]
presents any difficulty.  Here we are confronted with two choices
for where the metric functions are defined.  In great measure, we
are free to pick any convenient definition from the numerous ones
possible. We have this freedom since the prescription for baryon
number conservation can be written as
\[
  \int d^3x \, \rho \frac{ \sqrt{-g}}
                         { \sqrt{-\met{\gm}{\gn} \ptraj{\gm} \ptraj{\nu}}}
                         m \sm{x}{z}
  = m \eqp
\]
As long as the smoothing kernel $W$ is normalized any combination
of $\frac{ \sqrt{-\met{\gm}{\gn} \pdtraj{\gm} \pdtraj{\gn} }}{
\sqrt{-g}}$ can be used in the definition of \gr as long as the
same combination is used when calculating the total baryon number.
The primary complication is the `coupling' term
$\sqrt{-\met{\gm}{\gn} \pdtraj{\gm} \pdtraj{\gn} }$. It is
directly related to $u^0$ and is identified with the Lagrange
multiplier \gL in the continuum case by simultaneously using the
four-velocity normalization and the equation $\pdtraj{\gm} = \gL
\imet{\gm}{\gn} \pvel{\gn}$.  We will see below that in order to
make the `smoothed' analogs of the equations for the four-velocity
normalization and the relation between \pdtraj{\gm} and \pvel{\gm}
look natural we will have to adopt the following convention.  Any
time a metric function is used in combination with a fat particle
hydrodynamic function it should be smoothed according to the
prescription \be\label{eq:inv_sm_met}
  \simet{\gm\gn}\parg = \myint{d^3x} \imet{\gm}{\gn}\farg \sm{x}{z} \eqp
\ee We define the smoothed metric \sura{\met{\gm}{\gn}} by
requiring \be\label{eq:sm_met} \simet{\gm\gn} \smet{\gn\gr} =
\Kd{\gm}{\gr} \eqp \ee The density can be explicitly written as
\be\label{eq:rho_def}
    \gr = m \frac{ \sqrt{-\smet{\gm\gn} \pdtraj{\gm} \pdtraj{\gn}}}
                 { \sqrt{-g}} \sm{x}{z} \eqp
\ee

Also we will be compelled to regard the internal energy, where
this coupling between the field and the hydrodynamic variables is
most entangled, as being a function of both $x^{\mu}$ subject to
the particle trajectories \ptraj{\gm}.  Functionally, we will
denote this dependence, when needed, as $\erg = \ierg$.  Finally
we will take the term $\sqrt{-g} = \sqrt{-g(x)}$.  Again the
reason for this choice is that it will make the field equations
derived by varying $\imet{\ga}{\gb}(x)$ look natural.  However, we
emphasize that once these two choices have been adopted we no
longer have the ability to adjust anything else in the action. The
resulting equations for the fluid motion will then reflect the
viability of these choices.

\subsection{Action Variation With Respect to \gL}
First we look at the variation of the action with respect to \gL
which yields
\[
\vwrt{I}{\gL} = -\int d^4x \push m(1+\erg) \mathcal{H} \sm{x}{z}
\var{\gL} \eqp
\]
Using the definition of the Hamiltonian reduces the above equation
to
\[
\int dt \push d^3x \push \sm{x}{z} (1 + \erg)
\surb{\imet{\gm}{\gn}(x) \pvel{\gm} \pvel{\gn} + 1} = 0 \eqp
\]
Defining an associated smoothing
\[
    \langle f(z) \rangle _\erg = \int d^3x \, \sm{x}{z} (1+\erg) f(x)
\]
where the inclusion of the internal energy term is unavoidable due
to the entanglement of the field and hydrodynamic degrees of
freedom as discussed above.  Ordinarily this would present a
substantial complication, creating a self-coupled variational
principle that would require extensive modification of the simple
equations of motion that resulted from the continuum theory.  In
order to avoid this we employ the standard SPH approximation of
rational functions that states
\[
  \sura{ \frac{A(\vec r)}{B(\vec r)} }= \frac{ \langle
  A(\vec r) \rangle}{\langle B(\vec r) \rangle} + O(h^2) \eqp
\]
Writing the equation resulting from $\delta \Lambda$ in terms of
the auxiliary smoothing we arrive at
\[
  \langle g^{\mu\nu}(t,\vec z) \rangle_e u_{\mu}(t) u_{\nu}(t) +
  \langle 1 \rangle_\erg = 0 \eqp
\]
Dividing by $\langle 1 \rangle_e$ we arrive at
\[
  \frac{\langle g^{\mu\nu}(t,\vec z) \rangle_\erg}{\langle 1
  \rangle_\erg}u_{\mu}(t) u_{\nu}(t) + 1 = 0 \eqp
\]
The first term is simplified using the SPH approximation of
rational functions to yield
\[
\frac{\langle g^{\mu\nu}(t,\vec z) \rangle_\erg}{\langle 1
  \rangle_\erg}  =   \myint{d^3x} \frac{g^{\mu\nu}(t,\vec x)}{1+\erg} (1+\erg)
  W(\vec z - \vec x) = \myint{d^3x} g^{\mu\nu}(t,\vec x) W(\vec z
  - \vec x)
\]
Using this relation simplifies the four-velocity normalization
condition to \be\label{eq:norm_cond} g^{\mu\nu}(t,\vec z)
u_{\mu}(t) u_{\nu}(t) + 1 = 0 \ee where $g^{\mu\nu}(t,\vec z) =
\int d^3x \, g^{\mu\nu}(x) W(\vec x - \vec z)$.

This equation agrees with what one would obtain by taking the
corresponding equation of the continuum theory and directly
smoothing.  This correspondence is used to justify the claim made
above that the metric functions can be smoothed using solely the
smoothing kernel.

\subsection{Action Variation with Respect to \pvel{\gm}}
Next we look at the variation of the action with respect to the
covariant four-velocity $u_{\mu}$ which yields
\[
  \delta I_{\vert \delta u_{\mu}} = \myint{dt} m (1 + \erg) \left\{
  {\dot z}^{\mu} - \Lambda g^{\mu\nu}(t,\vec z) u_{\nu}(t)
  \right\} \delta u_{\mu}(t) \eqp
\]
Again using the SPH approximation as above we arrive at
\be\label{eq:4vel_def} {\dot z}^{\mu} = \Lambda g^{\mu\nu}(t,\vec
z) u_{\nu}(t) \ee which is the `smoothed' analog of the continuum
equation relating the coordinate velocities ${\dot z}^{\mu}$ to
the covariant four-velocities $u_{\mu}$. Combining the two
equations and solving for $\Lambda$ yields \be\label{eq:Lambda}
 \Lambda^2 = -g_{\mu\nu}(t,\vec z) {\dot z}^{\mu}{\dot z}^{\nu}
\ee where $g_{\mu\nu}(t,\vec z)$ is the inverse matrix of
$g^{\mu\nu}(t,\vec z)$ formally defined by
\[
g_{\mu\nu}(t,\vec z) g^{\mu\sigma}(t,\vec z) =
{\delta^{\sigma}}_{\nu} \eqp
\]
Solving the above equation for $\Lambda$ gives us an equation
consistent with our smoothing convention; namely that since
$g_{\mu\nu}$ appears in conjunction with fluid degrees of freedom
it must be the smoothed form of that function.

%%%%%%%%%%%%%%%%%%%%%%%%%%%%%%%%%%%%%%%%%%%%%%%%%%%%%%%%%%%%%%%%%%
\subsection{Action Variation with Respect to \imet{\gm}{\gn}(x)}
%%%%%%%%%%%%%%%%%%%%%%%%%%%%%%%%%%%%%%%%%%%%%%%%%%%%%%%%%%%%%%%%%%

We now turn to the variation of the combined action with respect
to $g^{\mu\nu}(x)$. Taking this variation yields
\bea\label{eq:var_I_g}
 \vwrt{I}{g^{\mu\nu}} & = &
   \frac{1}{16\pi} \myint{d^3x \nudge dt} G_{\mu\nu} \sqrt{-g} \,\,
   \var{\imet{\gm}{\gn}} \nonumber \\
& &
 + \myint{d^3x \nudge dt} m \frac{P}{\gr^2} \vwrt{\gr}{\imet{\gm}{\gn}}
     \surc{ \pdtraj{\gm} u_{\mu} - \gL \mathcal{H} } W(\vec x - \vec z) \nonumber \\
& &
 - \myint{d^3x \nudge dt} m(1 + \erg)\frac{\gL}{2} u_{\mu} u_{\nu} \var{\imet{\gm}{\gn}} \eqp
\eea The first term in \refq{eq:var_I_g} is the usual term that
comes from the variation of the Hilbert action.  The second term
depends on the variation of the density, as defined in
\refq{eq:rho_def}, which is comprised of the two terms
\be\label{eq:var_rho} \vwrt{\gr}{\imet{\gm}{\gn}} = \frac{-m
\sm{x}{z}}{2} \surc{ \frac{1}{\gL\sqrt{-g}} \var{\smet{\gm\gn}}
\pdtraj{\gm}\pdtraj{\gn} + \frac{\gL}{\surp{-g}^{3/2}} \var{g}}
\ee with \gL defined as in \refq{eq:Lambda} and $g(x) =
\det{\surp{\met{\gm}{\gn}}}$. The first term in \refq{eq:var_rho},
\var{\smet{\gm\gn}}, can be related to the variation of the
smoothed inverse metric defined in \refq{eq:inv_sm_met} by taking
the variation of \ref{eq:sm_met} \bes
  \var{\smet{\gs\gt}} = \smet{\gs\ga} \var{\simet{\ga\gb}} \smet{\gb\gt} \eqp
\ees The variation of \simet{\ga\gb} \bes
  \var{\simet{\ga\gb}} = \myint{d^3x} \sm{x}{z} \var{\imet{\ga}{\gb}}
\ees is easily obtained from the equation \refq{eq:inv_sm_met}.
The variation of the determinant of the metric follows from the
standard relation \bes
  \var{\metd} = - \metd \met{\gm}{\gn} \var{\imet{\gm}{\gn}} \eqp
\ees

Substituting these relations in \refq{eq:var_rho} and collecting
terms against the definition of \gr in \refq{eq:rho_def} yields
\bes
  \vwrt{\gr}{\imet{\gm}{\gn}}    =   \frac{\gr}{2} \surp{\met{\gm}{\gn}
                                     \var{\imet{\gm}{\gn}}
                                   + u_{\gm}(t) u_{\gn}(t)
                                     \myint{d^3y} \var{\imet{\gm}{\gn}}(\cv{y},t) \sm{y}{z}} \eqp
\ees Using equation \refq{eq:4vel_def}, in the form \pdtraj{\mu}
\ptvel{\mu} = -\gL, in combination with \mH = 0, renders the
action variation as \beas
 \var{I} & = & \myint{d^3x dt}    \surc{ \frac{\idx{G}{\dn{\gm\gn}}}{16 \gp}
                                - \frac{\gr}{2} (1+e) \ptvel{\mu} \ptvel{\gn}
                                - \frac{P}{2} \met{\gm}{\gn}
                                       } \sqrt{-\metd} \nudge \var{\imet{\gm}{\gn}} \\
         & + & \myint{d^3x dt} P \nudge \ptvel{\gm} \ptvel{\gn} \sqrt{-\metd}
               \myint{d^3y} \sm{y}{z} \var{\imet{\gm}{\gn}(\cv{y},t)} \eqp
\eeas By switching the dummy variables of integration in the
latter integral, $x \rightarrow y$ and $y \rightarrow x$, and
collecting terms under one integral, we arrive at \beas \var{I} &
= & \myint{d^3x dt} \left\{ \frac{\idx{G}{\dn{\gm\gn}}}{16 \gp} -
\frac{\gr}{2}
  (1+e) \ptvel{\gm} \ptvel{\gn} - \frac{P}{2} \met{\gm}{\gn} \right. \\
        &   &  \left. + \frac{\sm{x}{z}}{\sqrt{-\metd}}  \ptvel{\gm} \ptvel{\gn}
        \myint{d^3y} P \nudge \sqrt{-\metd} \right\} \sqrt{-\metd} \nudge \var{\imet{\gm}{\gn}} \eqp
\eeas Setting the variation to zero, we arrive at the `smoothed'
form of the Einstein equations \bea
   \idx{G}{\dn{\gm\gn}} = 8 \gp \surc{   \gr (1+e) \ptvel{\gm} \ptvel{\gn} + P \met{\gm}{\gn}
                                  + \frac{W(\cv{x} - \cv{z})}{\sqrt{g(x)}}
                                    \ptvel{\gm} \ptvel{\gn}
                                    \surb{\myint{d^3y} P(y) \sqrt{-g(y)}} }
\eea with the right-hand side being taken as the definition of the
`smoothed' stress-energy tensor.


\subsection{Action Variation with Respect to \ptraj{\gm}}

The final step is to take the variation of the action \wrt
\ptraj{\gm} to obtain the Euler equations for the particle. To
facilitate this end, the separation between space and time will be
observed.  The action, thus becomes \be\label{eq:smthI} I = m \int
dt \push \pdtraj{\gm} \pvel{\gm} \sura{1+e} - m \int dt \push
\frac{\gL}{2} \sura{\surp{1+e}\surp{\imet{\gm}{\gn} \pvel{\gm}
\pvel{\gn} + 1}} \eqp \ee
%
Now taking the variation of \surp{\ref{eq:smthI}} yields
\[
\vIz{i} = m \int dt \push \surc{ \var{\pdpos{i}} \ptvel{i}
\sura{1+e} + \pdtraj{\gm} \pvel{\gm} \var{\sura{1+e}} -
\frac{\gL}{2} \var{ \sura{ \surp{1+e} \surb{\normeq{\gm}{\gn}}}}}
\eqp
\]
Integrate the first term by parts to move the time derivative off
of \ptraj{\mu}.  Next use the relation $\pdtraj{\gm}\!\pvel{\gm} =
-\gL$ in the second term.  For the third term, the relation
\be\label{eq:norm}
 \sura{f(x)\surb{\normeq{\gm}{\gn}}} =
\sura{f(x)}\sura{\normeq{\gm}{\gn}} = 0 \ee
%
is used to expand the variation.  Follow this by switching the
derivative on the smoothing kernel from
\idx{\partial}{\dn{\cv{z}}} to \idx{\partial}{\dn{\cv{x}}} and
integrate by parts.  Performing these steps yields \bea \vIz{i} &
= & m \myint{dt} \surc{
   \dby{t} \surp{\pvel{i}\sura{1+\erg}} \var{\ptraj{i}}
 - \gL \var{\sura{1+\erg}} } \nonumber \\
        &   & - m \myint{dt}
   \frac{\gL}{2} \myint{d^3x} \pdby{x^i} \surc{ \surp{1+\erg}
     \surp{\imet{\gm}{\gn} \pvel{\gm} \pvel{\gn} + 1}} \sm{x}{z} \var{\ptraj{i}}
\eea
%
Again use \surp{\ref{eq:norm}} when expanding
\pdby{x^i}\surc{\surp{1+\erg}\surb{\normeq{\gm}{\gn}}} to obtain
%
\bea \vIz{i} & = & -m \myint{dt} \surc{
   \dby{t} \surp{ \pvel{i} \sura{1+\erg} } \var{\ptraj{i}}
 - \gL \var{\sura{1+\erg}}} \nonumber \\
             &   & - m \myint{dt}
   \frac{\gL}{2} \myint{d^3x} \surp{1+\erg}
   \pdxby{\imet{\gm}{\gn}}{x^i} \pvel{\gm} \pvel{\gm} \sm{x}{z}
   \var{\ptraj{i}} \eqp
\eea
%
Now expand the \var{\sura{1+\erg}} term to obtain
%
\bea \vIz{i} & = & -m \myint{dt} \dby{t} \surp{\pvel{i}
\sura{1+\erg}} \var{\ptraj{i}} \nonumber \\
             &   & -m \myint{dt} \gL \surp{
             \pdxby{\sura{1+\erg}}{\ptraj{i}}  \var{\ptraj{i}} +
             \pdxby{\sura{1+\erg}}{\pdtraj{i}} \var{\pdtraj{i}} }
             \nonumber \\
             &   & -m \myint{dt} \frac{\gL}{2} \pvel{\gm}
             \pvel{\gn} \sura{ \surp{1+\erg}
             \pdxby{\imet{\gm}{\gn}}{x^i}} \var{\ptraj{i}} \eqc
\eea
%

Integrating the third term by parts and collecting yields
%
\bea \vIz{i} & = & -m \myint{dt} \left\{ \dby{t} \surp{\pvel{i}
\sura{1+\erg} - \gL \pdxby{\sura{1+\erg}}{\pdtraj{i}} } \right.
\nonumber \\
        &   & \left. \gL \pdxby{\sura{1+\erg}}{\ptraj{i}}
        + \frac{\gL}{2} \pvel{\gm} \pvel{\gn} \sura{ \surp{1+\erg}
        \pdxby{\imet{\gm}{\gn}}{x^i}} \right\} \var{\ptraj{i}}
        \eqp
\eea
%
To proceed further, the terms involving
\pdxby{\sura{1+\erg}}{\ptraj{i}} and
\pdxby{\sura{1+\erg}}{\pdtraj{i}} must be expanded. Several
intermediate results are helpful in tackling the
\pdxby{\sura{1+\erg}}{\ptraj{i}} term first:
\[
\pdxby{\gr}{\ptraj{i}} = \gr \surb{ \frac{1}{\sm{x}{z}}
\pdxby{\sm{x}{z}}{\ptraj{i}} + \frac{1}{2}
\pdxby{\simet{\ga\gb}}{\ptraj{i}} \pvel{\ga} \pvel{\gb} }
\]
and
\[
\pdxby{\gr}{x^i} = \gr \surb{ \frac{-1}{\sm{x}{z}}
\pdxby{\sm{x}{z}}{\ptraj{i}} + \frac{1}{2} \met{\gm}{\gn}
\pdxby{\imet{\gm}{\gn}}{x^i}} \eqp
\]
Using these relations gives
%
\bea \pdxby{\sura{1+\erg}}{\ptraj{i}} & = & \myint{d^3x}
\pdxby{\erg}{\gr} \surc{\pdxby{\gr}{\ptraj{i}} +
\pdxby{\gr}{x^i}} \sm{x}{z} \nonumber \\
& = & \sura{\frac{P}{\gr}} \frac{1}{2}
\pdxby{\simet{\ga\gb}}{\ptraj{i}} + \frac{1}{2} \sura{
\frac{P}{\gr} \met{\gm}{\gn} \pdxby{\imet{\gm}{\gn}}{x^i}} \eea
%
Similarly, the term of \pdxby{\sura{1+\erg}}{\pdtraj{i}} is
tackled by using the intermediate relation
\[
\pdxby{\gr}{\pdtraj{i}} = -\frac{\gr \pvel{i}}{\gL}
\]
to obtain
\[
\pdxby{\sura{1+\erg}}{\pdtraj{i}} =
-\frac{\pvel{i}}{\gL}\sura{\frac{P}{\gr}} \eqp
\]
Finally, the term \sura{\surp{1+\erg}\pdxby{\imet{\gm}{\gn}}{x^i}}
can be simplified using the SPH rules used before to yield
\[
\sura{\surp{1+\erg}\pdxby{\imet{\gm}{\gn}}{x^i}} = \sura{1+\erg}
\sura{\pdxby{\imet{\gm}{\gn}}{x^i}} = \sura{1+\erg}
\pdxby{\simet{\gm\gn}}{\ptraj{i}} \eqp
\]
Putting all of these pieces together and setting the variation
equal to zero gives the SPH version of the Euler equations
%
\be\label{eq:SPHEuler}
 \frac{1}{\gL} \dby{t} \surb{ \sura{1+\erg+\frac{P}{\gr}}
\pvel{i} } + \frac{1}{2} \sura{1+\erg+\frac{P}{\gr}}
\pdxby{\simet{\gm\gn}}{\ptraj{i}} \pvel{\gm} \pvel{\gn} +
\frac{1}{2} \sura{\frac{P}{\gr} \met{\gm}{\gn}
\pdxby{\imet{\gm}{\gn}}{x^i}} = 0 \eqp \ee
%
The covariant form of the Euler equations for an ideal fluid
($\idx{T}{\dn{\gm\gn}} = \gr \surp{1+\erg+\frac{P}{\gr}}
\idx{u}{\dn{\gm}} \idx{u}{\dn{\gn}} + P \met{\gm}{\gn}$) take the
form
\[
\gr\surp{1+\erg+\frac{P}{\gr}} \pvel{\ga;\gn} \idx{u}{\up{\gn}} =
-\idx{P}{\dn{,\ga}} - \pvel{\ga} \idx{u}{\up{\gn}}
\idx{P}{\dn{,\gn}}
\]
Making the identifications $\pvel{\ga,\gn} \idx{u}{\up{\gn}} =
\dxby{\pvel{\ga}}{\gt}$ and $\idx{P}{\dn{,\gn}} \idx{u}{\up{\gn}}
= \dxby{P}{\gt}$, yields the equivalent form
\[ \gr \surp{1 + \erg
+ \frac{P}{\gr}} \surp{\dxby{\pvel{i}}{\gt} -
\idx{\gG}{\up{\ga}\dn{\gn i}} \pvel{\ga} u^{\gn} } =
-\idx{P}{\dn{,i}} - \pvel{i} \dxby{P}{\gt} \eqc
\]
which is more suitable for comparison.  Continuing along, the
identities $\idx{\gG}{\up{\gm}\dn{\ga\gb}} \pvel{\gm}
\idx{u}{\up{\gb}} = \frac{1}{2} \idx{u}{\up{\gs}}
\idx{u}{\up{\gb}} \met{\gs}{\gb , \ga} = -\frac{1}{2} \pvel{\gs}
\pvel{\gb} \idx{\imet{\gs}{\gb}}{\dn{,\ga}}$ and $\dby{\gt}\surp{1
+ \erg + \frac{P}{\gr}} = \frac{1}{\gr} \dxby{P}{\gt}$ allow for
further simplification, yielding \be\label{eq:Euler_final_form}
\dby{\gt} \surb{ \surp{1 + \erg + \frac{P}{\gr}} \pvel{i}} +
\frac{1}{2} \surp{1 + \erg + \frac{P}{\gr}}
\pdxby{\imet{\ga}{\gn}}{x^i}\pvel{\ga}\pvel{\gn} + \frac{1}{\gr}
\idx{P}{\dn{,i}} = 0 \ee as our final form. Comparison between
\sure{\ref{eq:Euler_final_form}} and \sure{\ref{eq:SPHEuler}}
shows that the functional forms of the two equations are very
similar. The predominant difference arises in the pressure
gradient term. In \sure{\ref{eq:Euler_final_form}} the spatial
gradient of the pressure is directly computed while in
\sure{\ref{eq:SPHEuler}} the pressure gradient is inferred from
the spatial gradient of the metric. This coupling between the
pressure and the metric is also present in the expression for the
stress-energy tensor.

\section{Analysis of the Fat Particle Equations}

TBS
