%Simple document on the basic relations in tensor analysis
%basis and some formulas in each
\documentclass[12pt]{article}
%load packages
\usepackage{latexsym}
\usepackage{epic,eepic,graphicx,url}
%spacer commands
\newcommand{\eqc}{\ensuremath{\: ,} }
\newcommand{\eqp}{\ensuremath{\: .} }
\newcounter{bean}
\newcommand{\surp}[1]{\ensuremath{\left( {#1} \right)} }
\newcommand{\surb}[1]{\ensuremath{\left[ {#1} \right]} }
\newcommand{\surc}[1]{\ensuremath{\left\{ {#1} \right\}} }
\newcommand{\sura}[1]{\ensuremath{\left\langle #1 \right\rangle}}
\newcommand{\sure}[1]{({#1})}
\newcommand{\push} {\ensuremath{\:}}
\newcommand{\nudge}{\ensuremath{\,}}
\newcommand{\back}{\ensuremath{\!\!}}
\newcommand{\iback}{\ensuremath{\!\!\!\!}}
%equation commands
\newcommand{\be} { \begin{equation} }
\newcommand{\ee} { \end{equation}   }
\newcommand{\bes} { \[ }
\newcommand{\ees} { \]  }
\newcommand{\bea}{ \begin{eqnarray} }
\newcommand{\eea}{ \end{eqnarray}   }
\newcommand{\beas}{ \begin{eqnarray*} }
\newcommand{\eeas}{ \end{eqnarray*}   }
%index commands
\newcommand{\idx}[2] {\ensuremath{ {#1}{#2}   }}
\newcommand{\up} [1] {\ensuremath{{}^{#1}     } }
\newcommand{\dn} [1] {\ensuremath{{}_{#1}     } }
%array commands
\newcommand{\cv}  [1]{\ensuremath{\vec {#1}} }
\newcommand{\rv}  [1]{\ensuremath{{\vec {#1}}^{\,T}} }
\newcommand{\cf}  [1]{\ensuremath{\tilde {#1}} }
\newcommand{\rf}  [1]{\ensuremath{{\tilde {#1}}{\,^T}} }
\newcommand{\cs}  [1]{\ensuremath{\bar {#1}} }
\newcommand{\rs}  [1]{\ensuremath{{\bar {#1}}{\,^T}} }
\newcommand{\cuv} [1]{\ensuremath{\hat {#1}} }
\newcommand{\ruv} [1]{\ensuremath{{\hat {#1}}{\,^T}} }
\newcommand{\op}  [1]{\ensuremath{\mathbf{{#1}}}}
\newcommand{\dt}  [1]{\ensuremath{\dot {{#1}}} }
\newcommand{\ddt} [1]{\ensuremath{\ddot {{#1}}} }
\newcommand{\trps}[1]{\ensuremath{{#1}^{\,T} } }
\newcommand{\bv}  [1]{\ensuremath{\cs{e}_{#1}} }
\newcommand{\bd}  [1]{\ensuremath{\cf{\gw}^{#1}} }
%\newcommand{\rank}[2]{\ensuremath{\surp{\begin{array}{c}{#1 \\ #2}\end{array}}} }
\newcommand{\rank}[2]{\ensuremath{\surp{#1, #2}}}
%inner products
\newcommand{\ipp}[2]{\ensuremath{ (  {#1} ,     {#2} ) } }
\newcommand{\ipb}[2]{\ensuremath{ \langle {#1} ,     {#2} \rangle } }
\newcommand{\ipD}[2]{\ensuremath{ \langle {#1} |     {#2} \rangle } }
\newcommand{\ipd}[2]{\ensuremath{         {#1} \cdot {#2}         } }
%derivative and integral commands
\newcommand{\dby}   [1]{\ensuremath{ \frac{d}{d #1}} }
\newcommand{\dxby}  [2]{\ensuremath{ \frac{d #1}{d #2}} }
\newcommand{\pdby}  [1]{\ensuremath{ \frac{\partial}{\partial #1}} }
\newcommand{\pdxby} [2]{\ensuremath{ \frac{\partial #1}{\partial #2}} }
\newcommand{\myint} [1]{\ensuremath{ \int #1 \push} }
%variations
\newcommand{\var}[1]{\ensuremath{\delta {#1}}}
\newcommand{\vwrt}[2]{\ensuremath{\idx{\delta {#1}\vert}{\dn{\delta {#2}}}} }
\newcommand{\IBP}{\ensuremath{\stackrel{\mbox{\tiny{IBP}}}{=}}}
%common abreviations
\newcommand{\wrt}{with respect to }
\newcommand{\bdyterms}{boundary-terms }
\newcommand{\ibp}{integration-by-parts }
\newcommand{\dfunc}{\gd\back-function }
\newcommand{\dfuncs}{\gd\back-functions }
%special characters
\newcommand{\ga}{\ensuremath{\alpha} }
\newcommand{\gba}{\ensuremath{\bar \alpha} }
\newcommand{\gha}{\ensuremath{\hat \alpha} }
\newcommand{\gta}{\ensuremath{\tilde \alpha} }
\newcommand{\gb}{\ensuremath{\beta} }
\newcommand{\gbb}{\ensuremath{\bar \beta} }
\newcommand{\ghb}{\ensuremath{\hat \beta} }
\newcommand{\gtb}{\ensuremath{\tilde \beta} }
\newcommand{\gd}{\ensuremath{\delta} }
\newcommand{\gbd}{\ensuremath{\bar \delta} }
\newcommand{\ghd}{\ensuremath{\hat \delta} }
\newcommand{\gtd}{\ensuremath{\tilde \delta} }
\newcommand{\gD}{\ensuremath{\Delta} }
\newcommand{\gbD}{\ensuremath{\bar \Delta} }
\newcommand{\ghD}{\ensuremath{\hat \Delta} }
\newcommand{\gtD}{\ensuremath{\tilde \Delta} }
\newcommand{\get}{\ensuremath{\eta} }
\newcommand{\gbet}{\ensuremath{\bar \eta} }
\newcommand{\ghet}{\ensuremath{\hat \eta} }
\newcommand{\gtet}{\ensuremath{\tilde \eta} }
\newcommand{\gf}{\ensuremath{\phi} }
\newcommand{\gbf}{\ensuremath{\bar \phi} }
\newcommand{\ghf}{\ensuremath{\hat \phi} }
\newcommand{\gtf}{\ensuremath{\tilde \phi} }
\newcommand{\gF}{\ensuremath{\Phi} }
\newcommand{\gbF}{\ensuremath{\bar \Phi} }
\newcommand{\ghF}{\ensuremath{\hat \Phi} }
\newcommand{\gtF}{\ensuremath{\tilde \Phi} }
\newcommand{\gG}{\ensuremath{\Gamma} }
\newcommand{\gbG}{\ensuremath{\bar \Gamma} }
\newcommand{\ghG}{\ensuremath{\hat \Gamma} }
\newcommand{\gtG}{\ensuremath{\tilde \Gamma} }
\newcommand{\ggm}{\ensuremath{\gamma} }
\newcommand{\gbgm}{\ensuremath{\bar \gamma} }
\newcommand{\ghgm}{\ensuremath{\hat \gamma} }
\newcommand{\gtgm}{\ensuremath{\tilde \gamma} }
\newcommand{\gl}{\ensuremath{\lambda} }
\newcommand{\gbl}{\ensuremath{\bar \lambda} }
\newcommand{\ghl}{\ensuremath{\hat \lambda} }
\newcommand{\gtl}{\ensuremath{\tilde \lambda} }
\newcommand{\gL}{\ensuremath{\Lambda} }
\newcommand{\gbL}{\ensuremath{\bar \Lambda} }
\newcommand{\ghL}{\ensuremath{\hat \Lambda} }
\newcommand{\gtL}{\ensuremath{\tilde \Lambda} }
\newcommand{\gm}{\ensuremath{\mu} }
\newcommand{\gbm}{\ensuremath{\bar \mu} }
\newcommand{\ghm}{\ensuremath{\hat \mu} }
\newcommand{\gtm}{\ensuremath{\tilde \mu} }
\newcommand{\gn}{\ensuremath{\nu} }
\newcommand{\gbn}{\ensuremath{\bar \nu} }
\newcommand{\ghn}{\ensuremath{\hat \nu} }
\newcommand{\gtn}{\ensuremath{\tilde \nu} }
\newcommand{\gp}{\ensuremath{\pi} }
\newcommand{\gbp}{\ensuremath{\bar \pi} }
\newcommand{\ghp}{\ensuremath{\hat \pi} }
\newcommand{\gtp}{\ensuremath{\tilde \pi} }
\newcommand{\gq}{\ensuremath{\theta} }
\newcommand{\gbq}{\ensuremath{\bar \theta} }
\newcommand{\ghq}{\ensuremath{\hat \theta} }
\newcommand{\gtq}{\ensuremath{\tilde \theta} }
\newcommand{\gr}{\ensuremath{\rho} }
\newcommand{\gbr}{\ensuremath{\bar \rho} }
\newcommand{\ghr}{\ensuremath{\hat \rho} }
\newcommand{\gtr}{\ensuremath{\tilde \rho} }
\newcommand{\gs}{\ensuremath{\sigma} }
\newcommand{\gbs}{\ensuremath{\bar \sigma} }
\newcommand{\ghs}{\ensuremath{\hat \sigma} }
\newcommand{\gts}{\ensuremath{\tilde \sigma} }
\newcommand{\gt}{\ensuremath{\tau} }
\newcommand{\gbt}{\ensuremath{\bar \tau} }
\newcommand{\ght}{\ensuremath{\hat \tau} }
\newcommand{\gtt}{\ensuremath{\tilde \tau} }
\newcommand{\gw}{\ensuremath{\omega} }
\newcommand{\gbw}{\ensuremath{\bar \omega} }
\newcommand{\ghw}{\ensuremath{\hat \omega} }
\newcommand{\gtw}{\ensuremath{\tilde \omega} }
\newcommand{\gW}{\ensuremath{\Omega} }
\newcommand{\gbW}{\ensuremath{\bar \Omega} }
\newcommand{\ghW}{\ensuremath{\hat \Omega} }
\newcommand{\gtW}{\ensuremath{\tilde \Omega} }
\newcommand{\gy}{\ensuremath{\psi} }
\newcommand{\gby}{\ensuremath{\bar \psi} }
\newcommand{\ghy}{\ensuremath{\hat \psi} }
\newcommand{\gty}{\ensuremath{\tilde \psi} }
\newcommand{\gY}{\ensuremath{\Psi} }
\newcommand{\gbY}{\ensuremath{\bar \Psi} }
\newcommand{\ghY}{\ensuremath{\hat \Psi} }
\newcommand{\gtY}{\ensuremath{\tilde \Psi} }
%Script letters
\newcommand{\mA}{\ensuremath{\mathcal A} }
\newcommand{\mB}{\ensuremath{\mathcal B} }
\newcommand{\mC}{\ensuremath{\mathcal C} }
\newcommand{\mD}{\ensuremath{\mathcal D} }
\newcommand{\mE}{\ensuremath{\mathcal E} }
\newcommand{\mF}{\ensuremath{\mathcal F} }
\newcommand{\mG}{\ensuremath{\mathcal G} }
\newcommand{\mH}{\ensuremath{\mathcal H} }
\newcommand{\mI}{\ensuremath{\mathcal I} }
\newcommand{\mJ}{\ensuremath{\mathcal J} }
\newcommand{\mK}{\ensuremath{\mathcal K} }
\newcommand{\mL}{\ensuremath{\mathcal L} }
\newcommand{\mM}{\ensuremath{\mathcal M} }
\newcommand{\mN}{\ensuremath{\mathcal N} }
\newcommand{\mO}{\ensuremath{\mathcal O} }
\newcommand{\mP}{\ensuremath{\mathcal P} }
\newcommand{\mQ}{\ensuremath{\mathcal Q} }
\newcommand{\mR}{\ensuremath{\mathcal R} }
\newcommand{\mS}{\ensuremath{\mathcal S} }
\newcommand{\mT}{\ensuremath{\mathcal T} }
\newcommand{\mU}{\ensuremath{\mathcal U} }
\newcommand{\mV}{\ensuremath{\mathcal V} }
\newcommand{\mW}{\ensuremath{\mathcal W} }
\newcommand{\mX}{\ensuremath{\mathcal X} }
\newcommand{\mY}{\ensuremath{\mathcal Y} }
\newcommand{\mZ}{\ensuremath{\mathcal Z} }
%references and citations
\newcommand{\refq}[1]{\sure{\ref{#1}}}
\newcommand{\refp}[1]{\ref{#1}}
\newcommand{\refs}[1]{\cite{#1}}
%Thermo, Fluid, and GR objects
%%%particle kinematics
\newcommand{\ptraj}  [1]{\ensuremath{\idx{z}{\up{#1}}}}
\newcommand{\pdtraj} [1]{\ensuremath{\idx{\dt{z}}{\up{#1}}}}
\newcommand{\pvel}   [1]{\ensuremath{\idx{u}{\dn{#1}}}}
\newcommand{\ppos}   [1]{\ensuremath{\idx{z}{\up{#1}}}}
\newcommand{\pdpos}  [1]{\ensuremath{\idx{\dt{z}}{\up{#1}}}}
\newcommand{\ptvel}  [1]{\ensuremath{\idx{u}{\dn{#1}}}}
\newcommand{\pdtvel} [1]{\ensuremath{\idx{\dt{u}}{\dn{#1}}}}
\newcommand{\ptraja} [2]{\ensuremath{\idx{z}{\dn{#1}\up{#2}}}}
\newcommand{\pdtraja}[2]{\ensuremath{\idx{\dt{z}}{\dn{#1}\up{#2}}}}
\newcommand{\pvela}  [2]{\ensuremath{\idx{u}{\dn{#1#2}}}}
\newcommand{\sm}     [2]{\ensuremath{W\!\surp{\cv{#1} - \cv{#2}}}}
%%%thermodynamic parameters
\newcommand{\gtro}{\ensuremath{\idx{\gtr}{\dn{0}}} }
\newcommand{\ierg}{\ensuremath{e\surp{\gr\surp{x}}} }
\newcommand{\erg} {\ensuremath{e} }
%%%metric functions
\newcommand{\bV}[2] {\idx{\cv{#1}}{\dn{#2}} \nudge }
\newcommand{\bF}[2] {\idx{\cf{#1}}{\up{#2}} \nudge }
\newcommand{\Jac}[2]{\idx{\gL}{\up{#1}\dn{#2}} \nudge }
\newcommand{\jac}[2]{\idx{\gL}{\dn{#1}\up{#2}} \nudge }
\newcommand{\Kd}[2] {\idx{\gd \nudge}{\up{#1}\dn{#2}} \nudge }
\newcommand{\kd}[2] {\idx{\gd}{\dn{#1}\up{#2}} \nudge }
\newcommand{\Cnx}[2]{\idx{\gG}{\up{#1}\dn{#2}} \nudge }
\newcommand{\met} [2]{\ensuremath{\idx{g}{\dn{#1 #2}}}}
\newcommand{\imet}[2]{\ensuremath{\idx{g}{\up{#1 #2}}}}
\newcommand{\metd}   {\ensuremath{ g } }
\newcommand{\smet} [1]{\ensuremath{\sura{\idx{g}{\dn{#1}}}} }
\newcommand{\simet}[1]{\ensuremath{\sura{\idx{g}{\up{#1}}}} }
\newcommand{\smetd}   {\ensuremath{\sura{g}}}
%functional arguments
\newcommand{\contarg}{\ensuremath{\surp{\idx{a}{\up{0}};\cv{a}}} }
\newcommand{\harg}   {\ensuremath{\surp{a^0,\cv{r}_1}} }
\newcommand{\parg}   {\ensuremath{\surp{t,\cv{z}}} }
\newcommand{\farg}   {\ensuremath{\surp{t,\cv{x}}} }
%variations
\newcommand{\vIz}[1] {\ensuremath{\vwrt{I}{\ptraj{#1}}} }
%useful relations
\newcommand{\normeq}[2]{\ensuremath{ \imet{#1}{#2} \pvel{#1}
\pvel{#2} + 1} }

%bibtex stuff
\pagenumbering{arabic}
\bibliographystyle{plain}
%backwards compatibility
\def\a{{\vec a}}
\def\x{{\vec x}}
\def\z{{\vec z}}
\def\P{{\Phi}}
\def\l{{\ell}}
\def\Pl{{\Phi_{\ell}}}
\def\D{{\Delta}}
\def\.{{\quad .}}
%\def\_,{{\quad ,}}
\def\half{\mbox{$\frac{1}{2}$}}
%\def\_.{{\quad .}}
\def\po{{\hat \rho}_{0}}

\begin{document}
\title{On the Basic Relations of Tensor Analysis}
\maketitle


The purpose of this short note is summarize the basic relations
and formulas of tensor analysis for both coordinate and
non-coordinate bases.

To start, consider the test case study of basis vectors in both cartesian
and cylindrical coordinates.  Of course, by their very definition,
cartesian coordinates suppose the space flat.  So there will be
no chance to obtain curvature.  However, this simple situation will
allow us to see some of the basic relations amongst the connection coefficients
in a simple context where the mathematics are easy.  In addition, the
concept of scaffolding will become apparent.

To begin, consider the simple case of the position of a point
\be
  \cv{r} = r \cos(\gf) \partial_x + r \sin(\gf) \partial_y
\ee
specified in terms of the cartesian basis vectors $\partial_x$ and
$\partial_y$.
The coordinate basis vectors are given by partial derivative of the
above expression with respect to each coordinate.  Thus
$\partial \cv{r} / \partial r$ and $\partial \cv{r} / \partial \gf$
give the r- and \gf-coordinate basis vectors
vector
\bea
 \bv{r}   & = &    \cos(\gf) \partial_x +   \sin(\gf) \partial_y \nonumber \\
 \bv{\gf} & = & -r \sin(\gf) \partial_x + r \cos(\gf) \partial_y \eqp
\eea
These vectors can be normalized so that they have unit length
(assuming for the time begin the standard cartesian form for
determining length - an idea we'll return to later.)
\bea
  \bv{\hat r}   & = &   \cos(\gf) \partial_x + \sin(\gf) \partial_y \nonumber \\
  \bv{\hat \gf} & = & - \sin(\gf) \partial_x + \cos(\gf) \partial_y \eqp
\eea
Now since each basis vector depends on the coordinates themselves (\textit{i.e.} they
change their direction and possibly size from place to place), we must
consider the partial derivatives of each vector.  Since the basis vectors form a vector
space, changes in these vectors must be able to be expressed in terms of the vectors
themselves.  In other words, the basis vectors form a closed space.  Proceeding to
calculate these partial derivatives, we obtain
\bea
  \pdxby{\bv{r}}{r}     = 0                    & \rightarrow &
      \nabla_r     \bv{r}   =  0 \nonumber \\
  \pdxby{\bv{r}}{\gf}   = \frac{1}{r} \bv{\gf} & \rightarrow &
      \nabla_{\gf} \bv{r}   =  \frac{1}{r} \bv{\gf} \nonumber \\
  \pdxby{\bv{\gf}}{r}   = \frac{1}{r} \bv{\gf} & \rightarrow &
      \nabla_r     \bv{\gf} =  \frac{1}{r} \bv{\gf} \nonumber \\
  \pdxby{\bv{\gf}}{\gf} = -r \bv{r}            & \rightarrow &
      \nabla_{\gf} \bv{\gf} =  -r \bv{r}
\eea
for the coordinate basis and
\bea
  \pdxby{\bv{\hat r}}{r}   = 0               & \rightarrow &
     \nabla_{\hat r} \bv{\hat r} \nonumber \\
  \pdxby{\bv{\hat r}}{\gf} = \bv{\hat \gf}   & \rightarrow &
     \nabla_{\hat \gf} \bv{\hat r} = \frac{1}{r} \bv{\hat \gf} \nonumber \\
  \pdxby{\bv{\hat \gf}}{r} = 0               & \rightarrow &
     \nabla_{\hat r} \bv{\hat \gf} = 0 \nonumber \\
  \pdxby{\bv{\hat \gf}}{\gf} = - \bv{\hat r} & \rightarrow &
     \nabla_{\hat \gf} \bv{\hat \gf} = -\frac{1}{r} \bv{\hat r}
\eea
for the non-coordinate basis.  The functions multiplying the basis vectors
are called the connection coefficients and the non-zero ones are summarized as
\bea
  \Cnx{\gf}{r \gf} & = & \frac{1}{r} \nonumber \\
  \Cnx{\gf}{\gf r} & = & \frac{1}{r} \nonumber \\
  \Cnx{r}{\gf \gf} & = & -r
\eea
and
\bea
  \Cnx{\ghf}{\hat r \ghf} & = & \frac{1}{r} \nonumber \\
  \Cnx{\hat r}{\ghf \ghf} & = & -\frac{1}{r} \eqp
\eea
The metric is defined in terms of the basis vectors as
\be
  \met{a}{b} = \ipd{\bv{a}}{\bv{b}}
\ee
and contains all of the information that is contained in the basis vectors.

Now one question no doubt springs to mind.  Why ever use the metric when one
can obtain the connection coefficients in the same way one did above?  The answer
is that one cannot tell, in general, how to find the cartesian vectors which are
connection-free, or even if they exist.  Without them as a scaffold, there is no way
to be sure how to take partial derivatives of the basis vectors, since expressions like
$\partial_r$ or $\partial_{\hat r}$ may have coordinate dependence that is
not explicit in their form.  Thus in general, one must either know something about
them with respect to how they change from place to place or one must impose a metric.
The latter is usually easier and is the path which we now follow.
Taking partial derivatives of the metric and using the standard definition of
the connection
\be
  \gG_{abc} = \bv{a} \cdot \pdxby{\bv{b}}{x^c}
\ee
yields
\bea
  \met{a}{b}_{,c} & = & \ipd{\bv{a,c}}{\bv{b}} + \ipd{\bv{a}}{\bv{b,c}} \nonumber \\
                  & = & \ipd{\Cnx{s}{ac} \bv{s}}{\bv{b}} + \ipd{\bv{a}}{\Cnx{s}{bc} \bv{s}} \nonumber \\
                  & = & \met{s}{b} \Cnx{s}{ac} + \met{a}{s} \Cnx{s}{bc} \nonumber \\
                  & = & \idx{\gG}{\dn{bac}} + \idx{\gG}{\dn{abc}} \eqp
\eea
Taking the usual three-fold combination yields
\bea
  \met{ab}{,c} + \met{ac}{,b} - \met{bc}{,a} & = &
    2 \gG_{\surp{ab}c} + 2 \gG_{\surp{ac}b} - 2 \gG_{\surp{bc}a} \nonumber \\
        & = & \gG_{abc} + \gG_{bac} + \gG_{acb} + \gG_{cab} - \gG_{bca} - \gG_{cba} \nonumber \\
        & = & \gG_{abc} + \gG_{acb} + 2 \gG_{b\surb{ac}} + 2 \gG_{c\surb{ab}} \nonumber \\
        & = & 2 \gG_{abc} - \gG_{abc} + \gG_{acb} + 2 \gG_{b\surb{ac}} + 2 \gG_{c\surb{ab}} \nonumber \\
        & = & 2 \gG_{abc} - 2 \gG_{a\surb{bc}} + 2 \gG_{b\surb{ac}} + 2 \gG_{c\surb{ab}} \eqp
\eea
Now employing the definition of the commutation coefficients
\be
  \bv{a} \bv{b} - \bv{b} \bv{a} = \nabla_{a} \bv{b} - \nabla_{b} \bv{a}
    = \idx{C}{\dn{ab}\up{c}} \bv{c} \nonumber \\ \eqc
\ee
which, ironically, can be defined without the scaffold of cartesian vectors, yields
the relation
\be
   \surp{\Cnx{c}{ba} - \Cnx{c}{ab}} \bv{c} = - 2 \Cnx{c}{\surb{ab}} \bv{c}
     = \idx{C}{\dn{ab}\up{c}} \bv{c} \eqp
\ee
Substituting this into the usual three-fold metric combination gives
\bea
 \gG_{abc} & = & \frac{1}{2} \surp{   \met{ab}{,c}     + \met{ac}{,b}     - \met{bc}{,a}
                                    - \gG_{c\surb{ab}} - \gG_{b\surb{ac}} + \gG_{a\surb{bc}}  } \nonumber \\
           & = & \frac{1}{2} \surp{   \met{ab}{,c}     + \met{ac}{,b}     - \met{bc}{,a}
                                    + C_{abc}          + C_{acb}          - C_{bca} } \eqp
\eea
All the pieces are now in place to define the covariant derivative.  Begin by taking
the partial derivative of an arbitrary vector
\bea
  \pdby{x^b} \surp{\cv{V}} & = & \pdby{x^b} \surp{ V^a \bv{a}} \nonumber \\
                           & = & \idx{V}{\up{a}\dn{,b}} \bv{a} + V^a \pdxby{\bv{a}}{x^b} \nonumber \\
                           & = & \idx{V}{\up{a}\dn{,b}} \bv{a} + V^a \Cnx{c}{ab} \bv{c} \nonumber \\
                           & = & \surp{\idx{V}{\up{a}\dn{,b}} + \Cnx{a}{cb} V^c} \bv{a} \eqp
\eea
Again expressing this vector in terms of the basis vectors
\be
  \nabla_b \cv{V} \equiv \pdby{x^b}\cv{V} \equiv \idx{V}{\up{a}\dn{;b}} \bv{a}
\ee
leads naturally to a shorthand notation for the components
\be
  \idx{V}{\up{a}\dn{;b}} = \idx{V}{\up{a}\dn{,b}} + \Cnx{a}{bc} V^c \eqc
\ee
called the covariant derivative.
Now we can determine what the partial derivatives of the basis forms are by simply
taking the partial derivative of the standard contraction or duality formula
\bea
 \pdby{x^c} \sura{ \bd{a},\bv{b} } & = &  0 \nonumber \\
 \sura{ \pdxby{\bd{a}}{x^c},\bv{b}} + \sura{\bd{a},\pdxby{\bv{b}}{x^c}} & = & 0 \nonumber \\
 \sura{ \idx{\gY}{\up{a}\dn{sc}} \bd{s}, \bv{b}} + \sura{ \bd{a}, \Cnx{s}{bc} \bv{s}} & = & 0 \nonumber \\
 \idx{\gY}{\up{a}\dn{bc}} + \Cnx{a}{bc} & = & 0 \nonumber \\
 \idx{\gY}{\up{a}\dn{bc}} & = & - \Cnx{a}{bc} \eqp
\eea
Once obtained, the same manipulation
\bea
  \pdby{x^c} \surp{ \gs_a \bd{a}} & = & \gs_{a,c} \bd{a} + \gs_a \pdxby{\bd{a}}{x^c} \nonumber \\
                                  & = & \gs_{a,c} \bd{a} - \gs_{a} \Cnx{a}{sc} \bd{s} \nonumber \\
                                  & = & \surp{ \gs_{a,c} - \Cnx{s}{ac} \gs_s } \bd{a} \nonumber \\
                                  & \equiv & \gs_{a;c} \bd{a}
\eea
leads to the form of the covariant derivative of a form.

Finally, taking the anti-symmetrized second partial derivative of an arbitrary vector yields
\bea
  \pdxby{^2 \cv{V}}{x^c \partial x^d} - \pdxby{^2 \cv{V}}{x^d \partial x^c} & = &
    \surp{ \Cnx{a}{bd,c} - \Cnx{a}{bc,d} + \Cnx{a}{sc} \Cnx{s}{bd} - \Cnx{a}{sd} \Cnx{s}{bc}}
    V^c \bv{a} \nonumber \\
    & = & \idx{R}{\up{a}\dn{bcd}} V^b \bv{a}
\eea
while the anti-symmetrized second partial derivative of an arbitrary form yields
\bea
  \pdxby{^2 \cf{\gs}}{x^c \partial x^d} - \pdxby{^2 \cf{\gs}}{x^d \partial x^c} & = &
   \surp{ \Cnx{a}{bc,d} - \Cnx{a}{bd,c} + \Cnx{a}{sd}\Cnx{s}{bc} - \Cnx{a}{sc}\Cnx{s}{bd} } \gs_a \bd{b} \nonumber \\
   & = & - \idx{R}{\up{a}\dn{bcd}} \gs_a \bd{b} \eqp
\eea
Both of these can be written solely in terms of the components as
\be
  \surp{\nabla_c \nabla_d - \nabla_d \nabla_c} V^a = \idx{R}{\up{a}\dn{bcd}} V^b
\ee
and
\be
  \surp{\nabla_c \nabla_d - \nabla_d \nabla_c} \gs_b = -\idx{R}{\up{a}\dn{bcd}} \gs_a \eqp
\ee
\end{document}
