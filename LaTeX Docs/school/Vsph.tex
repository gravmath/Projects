\documentstyle[fleqn]{article}

\def\a{{\vec a}}
\def\x{{\vec x}}
\def\z{{\vec z}}

\begin{document}

\title{Variational Smooth Particle Hydrodynamics}
%
\author{   \sc
             Conrad Schiff {\rm and} Charles W. Misner\\
           \em
            Department of Physics, University of Maryland,
           \\
            College Park MD 20742-4111 USA\\
           \rm
         e-mail: \tt  cmschiff@erols.com\\
                     misner@umail.umd.edu
        }
\date{13 December 1997}
% 
\maketitle

\section{Introduction}

The aim here is to provide a summary of smooth particle hydrodynamics
(SPH) for use in our BNS work. 
    The presentation is most easily followed, we hope, by starting
with classical fluid mechanics and only after covering that rather
thoroughly moving to the analogous formalism in general relativity. 
    Section~\ref{cfm} covers the variational approach of Mittag,
Stephen, and Yourgrau (MSY) to classical fluid mechanics.  
    Section~\ref{sph} reviews the some of the SPH basics and touches
upon some of the existing work in which variational methods play a
role, written here to emphasize the MSY style variational principles
that we use.
    Section~\ref{Newton} introduces self-gravitation of the fluid
while treating the gravitational potential as a field satisfying
partial differential equations on a discrete grid.
    This is intended as preparation for the relativistic case where
gravity cannot be treated as a particle-particle interaction.


%%%%%%%%%%%%%%%%%%%%%%%%%%%%%%%%%%%%%%%%%%%%%%%%%%%%%%%%%%%%%%%%%%%%%%
%%%%%%%%%%%%%%%%%%%%%%%%%%%%%%%%%%%%%%%%%%%%%%%%%%%%%%%%%%%%%%%%%%%%%%
%%%%%%%%%%%%%%%%%%%%%%%%%%%%%%%%%%%%%%%%%%%%%%%%%%%%%%%%%%%%%%%%%%%%%%
    \section{Classical Fluid Mechanics}\label{cfm}
    Start with a fluid and label each fluid element by its postion 
${\vec a}$ at $t=0$.  
    MSY then introduce what we call the trajectory function ${\vec
z}({\vec a},t)$ which gives any fluid element's position at some
later time $t$ by 
    \begin{equation} 
    {\vec x} = {\vec z}({\vec a},t) .
    \end{equation} 
    In discrete approximations later there will only be a finite
number of particle trajectories $\z$ computed, and fields such as the
gravitational potential $\Phi(\x)$ will be computed only on a finite
set of grid locations $\x$.
    The goal of making and combining these two discretization
approximations makes it important that a clear distinction be kept in
mind between the particle trajectories $\z$ and the field points
$\x$, even though the equation $\x = \z(t)$ imbeds a particle
trajectory within the grid where field values will be known.
    The trajectory function has the obvious boundary condition ${\vec
z}({\vec a},0) = {\vec a}$.  
    Assuming the fluid to be ideal, its motion is subject to two
constraints.  
    The first is the conservation of mass which takes the form 
    \begin{equation} 
    \rho({\vec z},t)\,d^3 z = \rho({\vec a},0)\,d^3 a. 
    \end{equation} 
    Introducing the Jacobian $J =
{\partial(z_1,z_2,z_3)}/{\partial(a_1,a_2,a_3)}$ allows the
conservation of mass equation to be given by 
    \begin{equation}\label{eq:rho}
    \rho J({\vec z}(\a,t)) = \rho_0(\a) .  
    \end{equation} 
    Thus $\rho$ depends on the family of trajectories $\z(\a,t)$
under consideration, and is entirely determined by it.
    The second constraint is the conservation of entropy which takes
on the form
    \begin{equation} 
    s(\a,t) = s({\vec a},0) = s_0(\vec a) . 
    \end{equation}

Although MSY incorporate these constraints in a variational principle
by the use of the Lagrange multipliers, we find the equations shorter
and easier to read if one takes $\rho$ to be defined by
Eqn.~(\ref{eq:rho}). 
    The MSY Lagrangian then is :
    \begin{equation}\label{eq:MSY}
    I  =  \int\!\! dt \int\!\! d^3a \left[\frac{\rho_0}{2} \left(
\frac{\partial z_i}{\partial t}\right)^2 - \rho_0(e+\Phi) \right]
    \end{equation} 
    where $e(\rho,s)$ is the specific internal energy and
$\Phi(\z(\a,t),t)$ is the gravitational potential energy per unit
mass at the position of particle $\a$.  
    In this formalism, ${\vec z}(\a,t)$ are the dynamical variables
that are varied to produce the equations of motion.

The variation of the action with respect to the trajectory function
${\vec z}$ requires the various properties of the Jacobian that MSY
employ in their formulation.
    We must remember that since $\rho$ is defined by Eqn.~(\ref{eq:rho})
both $\rho$ and $e(\rho,s)$ will vary when $\z$ is varied.
    Thus we find
    \begin{equation}
    \delta I = \int\!\! dt \int\!\! d^3a \left[ \rho_0
    \left(\frac{\partial z_i}{\partial t}\right)\delta
    \left(\frac{\partial z_i}{\partial t} \right) 
    - \rho_0 \frac{P}{\rho^2} \delta \rho
    - \rho_0 \frac{\partial \Phi}{\partial z_i} \delta {z_i} 
    \right].
    \end{equation}
    Here we have used the thermodynamic relationship $de = T\,ds -
P\,d(1/\rho)$ in the form $(\partial e/\partial \rho)_s = P/\rho^2$
in evaluating how the internal energy $e$ changes when the trajectory
variations cause changes in $\rho$ but not in $s$ at a particular
fluid element $\a$.
    The problematic term here is that containing $\delta \rho$.
    But from $\rho J = \rho_0(\a)$ where the right hand side is
independent of $\z$ one has $J \delta \rho + \rho \delta J = 0$.
    Thus the term $-(\rho_0 {P}/{\rho^2}) \delta \rho$  becomes 
$-(P/\rho) J \delta \rho = + P \delta J$.
    Using $\delta J = ({\partial J}/{\partial z_{i,j}}) \delta
z_{i,j} = J_{ij} \delta z_{i,j}$, where $z_{i,j} = {\partial
z_i}/{\partial a_j}$ and $J_{ij}$ is the (i,j) minor of the Jacobian,
the second ($P$) term can be simplified.  
    Exchanging the order of the variation and the partial
differention in the first and second terms and integrating these terms
by parts gives
    \begin{equation}
    \delta I = -\int\!\! dt \int\!\! d^3a \left[ \rho_0
    \left(\frac{\partial^2 z_i}{\partial t^2}\right) 
    + \frac{\partial}{\partial
    a_j}\left(P J_{ij}\right) 
    + \rho_0 \frac{\partial \Phi}{\partial z_i} 
    \right] \delta z_i.
    \end{equation}
    Setting this variation to zero leads to the partial differential
equation
    \begin{equation}\label{EL}
    \rho_0 \left(\frac{\partial^2 z_i}{\partial t^2}\right) + \rho_0
    \frac{\partial \Phi}{\partial z_i} + \frac{\partial}{\partial
    a_j}\left(P J_{ij}\right) = 0.
    \end{equation}
    From the relations ${\partial J_{ij}}/{\partial a_j} = 0$,
the last term becomes
    \begin{equation}
    \frac{\partial}{\partial a_j}\left(P  J_{ij} \right) =
    \frac{\partial P}{\partial a_j} J_{ij} = \frac{\partial P}{\partial
    z_k}\frac{\partial z_k}{\partial a_j} J_{ij}.
    \end{equation}
    Finally by using $({\partial z_k}/{\partial a_j}) J_{ij} = J
\delta_{ik}$ and $J = \rho_0 / \rho$ we put Eq.~(\ref{EL}) into the
familar form
    \begin{equation}\label{eq:Euler}
    \left(\frac{\partial^2 z_i}{\partial t^2}\right) + \frac{\partial
    \Phi}{\partial z_i} + \frac{1}{\rho} \frac{\partial P}{\partial z_i} =
    0.
    \end{equation}



%%%%%%%%%%%%%%%%%%%%%%%%%%%%%%%%%%%%%%%%%%%%%%%%%%%%%%%%%%%%%%%%%%%%%%
%%%%%%%%%%%%%%%%%%%%%%%%%%%%%%%%%%%%%%%%%%%%%%%%%%%%%%%%%%%%%%%%%%%%%%
%%%%%%%%%%%%%%%%%%%%%%%%%%%%%%%%%%%%%%%%%%%%%%%%%%%%%%%%%%%%%%%%%%%%%%
    \section{Smooth Particle Hydrodynamics}\label{sph}
In this section, we produce some of the standard SPH equations
by making approximations in the variational principle.

The single most important relation that should be stated is the idea of
approximating the equation
    \begin{equation}
    f({\vec z}) = \int\!\! d^3z'\,\delta({\vec z} - \vec{ z'}) f(\vec{z'})
    \end{equation}
by the relation
    \begin{equation}\label{sph_fun}
    \langle f({\vec z}) \rangle \simeq \int\!\! d^3z'\,W({\vec z} 
    - \vec{z'}; h) f(\vec{z'}).
    \end{equation} The essential properties of the smoothing kernel
$W$ are that it is normalized
    \begin{equation}\label{norm}
    \int\!\! d^3z\,W({\vec z};h) = 1
    \end{equation}
and that it has a delta function limit
    \begin{equation}
    \lim_{h \rightarrow 0} W({\vec z} - \vec{z'};h) = \delta({\vec z}
    - \vec{z'})
    \end{equation} In addition, we assume that we are always using a
symmetric, even, non-negative kernel so that the smoothing
approximation is $O(h^2)$.  In other words,
    \begin{equation}
    \langle f({\vec z}) \rangle = f({\vec z}) + c\nabla f({\vec z}) h^2
    \end{equation}
where $c$ is a constant independent of $h$.  

Gingold and Monaghan motivate a derivation of SPH starting from the
Lagrangian for a nondissipative, isentropic gas which in our notation
reads
    \begin{equation}
    L = \int\!\! d^3a\,\rho_0 \left[ \frac{1}{2} {\dot {\vec z}\,}^2 -
        e(\rho) - \Phi(z,t)\right] 
    \end{equation}
where ${\partial e}/{\partial \rho} = {P}/{\rho^2}$ and where $z$
means $z(\a,t)$.  Then they write the integral in discrete form
    \begin{equation}
    L \simeq \sum_{a=1}^{N} m_a \left( \frac{1}{2} {\dot {\vec z_a}}^2 -
    e(\rho_a)  - \Phi(\vec z_a, t) \right) .
    \end{equation}
Calculating the terms in the Euler-Lagrange equations gives the SPH
momentum equation
    \begin{equation}
    m_a \ddot {\vec z_a}  + m_a \left( \frac{\partial \Phi}{\partial
                                    \x} \right)_{\x = \z_a}
    + \sum_{b=1}^{N} \frac{P_b}{\rho_b^2}
    \frac{\partial \rho_b}{\partial {\vec z_a}} m_b 
    = 0
    \end{equation}
Connection with SPH can be made by identifying 
    \begin{equation}\label{eq:smoothed_rho}
    \rho_b = \sum_{c=1}^{N} m_c W ( {\vec z_b} - {\vec z_c}; h )
    \end{equation}
which leads to the symmetric momentum equation
    \begin{equation}
    {\ddot {\vec z_a}}  + \left( \frac{\partial \Phi}{\partial
                                    \x} \right)_{\x = \z_a}
    + \sum_{b=1}^{N} \left( \frac{P_b}{\rho_b^2} +
    \frac{P_a}{\rho_a^2} \right) m_b \nabla_a W ( {\vec z_a} - {\vec
    r_b}; h )
    \end{equation} 
that Monaghan advocates for its ability to conserve linear and angular
momentum.

Note that two approximation steps are invoked here.  One is the
replacement of $\int\!\! d^3a\,\rho_0$ by $\sum_{a=1}^N m_a$ which is a
Monte Carlo sampling of the continuum of fluid elements by a finite
subset.  The second step is the estimate (\ref{eq:smoothed_rho}) of
the continuum density from this finite sample.  The normalization
condition (\ref{norm}) provides a consistency between these two
steps, but we do not see that one can be derived from the other.
%%%%%%%%%%%%%%%%%%%%%%%%%%%%%%%%%%%%%%%%%%%%%%%%%%%%%%%%%%%%%%%%%%%%%%
%%%%%%%%%%%%%%%%%%%%%%%%%%%%%%%%%%%%%%%%%%%%%%%%%%%%%%%%%%%%%%%%%%%%%%
%%%%%%%%%%%%%%%%%%%%%%%%%%%%%%%%%%%%%%%%%%%%%%%%%%%%%%%%%%%%%%%%%%%%%%
    \section{Newtonian Gravity}\label{Newton}
    In the preceding sections Newtonian gravity was included as a
potential $\Phi$ due to external masses.  When the self-gravitation
of the fluid is to be included more care is needed.  Unlike usual
Newtonian SPH, we do not want to think of gravitation as a mutual
interaction of the smoothed particles---this viewpoint, effective
in Newtonian problems, will not provide guidance for the relativistic
problems we aim to formulate.  Instead we write a field theory.  A
variational principle which gives rise to the Poisson equation is 
$\delta I_G = 0$ with
    \begin{equation}\label{eq:I-G}
    I_G = -\int\!\!dt\!\int\!\!d^3x \left[  
            (1/8\pi G)(\nabla \Phi)^2 + \rho(\x,t) \Phi(\x,t) \right] .
    \end{equation}
    Varying $\Phi$ here gives
    \begin{equation}\label{eq:Poisson}
    \nabla^2 \Phi = 4 \pi G \rho 
    \end{equation}
    with solutions such as $\Phi = -GM/r$.

    A continuum variational principle extending that of
Sec.~\ref{cfm} to include the field dynamics is $\delta I = 0$ with 
$I = \int\!L\,dt$ and
    \begin{eqnarray}\label{eq:full-L}
    L = \int\!\! d^3a \left[\frac{\rho_0}{2} \left(
\frac{\partial z_i}{\partial t}\right)^2 
- \rho_0\, e(\rho,s_0) - \rho_0\, \Phi(\z(\a,t),t) \right] 
             \nonumber \\
      - (1/8\pi G)\int\!\!d^3x\,\nabla \Phi(\x,t)^2  .
    \end{eqnarray}
    The terms here containing $\z(\a,t)$ are exactly those considered
in Sec.~\ref{cfm}, so the fluid equations are just (\ref{eq:Euler}).
    But to see that Eq.~(\ref{eq:Poisson}) also results we need to
rewrite the $\rho\Phi$ term to see that it is the same as in
Eq.~(\ref{eq:I-G}).
    This we do by invoking the definition of $\rho$ in a change of
variables $\rho_0\,d^3a = \rho\,d^3z$ in the integral
    \begin{equation}\label{eq:rhoPhi}
    \int\!\! d^3a\, \rho_0(\a)\, \Phi(\z(\a,t),t) =
    \int\!\! d^3z\, \rho(\z,t)\, \Phi(\z,t) =
    \int\!\! d^3x\, \rho(\x,t)\, \Phi(\x,t)
    \end{equation}
    where the last step is a notational change of the dummy variable
of integration.    
    Thus Eq.~(\ref{eq:Poisson}) also results by varying $\Phi(\x)$ in
this combined Lagrangian (\ref{eq:full-L}).

    It is important to note that only the first form of the
interaction term (\ref{eq:rhoPhi}) as given in Eq.~(\ref{eq:full-L})
is acceptable in the fundamental Lagrangian.
    For in making the change of variables from $\a$ to $\z$ in
equation (\ref{eq:rhoPhi}) one has assumed a definite fluid motion
$\z(\a,t)$.
    Since reference to this particular motion disappears in the 
$\int\!\! d^3x\, \rho(\x,t)\, \Phi(\x,t)$ form of this term, it would
not be possible to carry out the $\delta \z$ variations were this
form to be stated as part of the basic Lagrangian.
    We have used it here only to carry out a variation of $\Phi$
while holding the fluid motion $\z(\a,t)$ unchanged.

    When we now approximate this Lagrangian through discretization,
the $\rho\Phi$ interaction term can be treated in (at least) two
different ways.  

    The first is based on the basic $\int\!\!d^3a\,\rho_0 \Phi$ form;
it begins by replacing $\int\!\!d^3a\,\rho_0 \Phi$ by $\sum_1^N m_a
\Phi(\vec z_a)$.  
    Were we to simply make no further approximation and solve the
resulting Poisson equation for $\Phi$ analytically, the Newtonian
$1/r^2$ attraction between point particles $\vec z_a$ would result. 
    But we want to follow the field theory line and therefore assume
that the field Lagrangian $\int\!\!d^3x\,(\nabla \Phi)^2$ is put into
finite difference form with $\Phi$ defined only on a discrete grid of
points $\vec x_\ell$ where $\ell$ is a triplet of integers labelling
the three dimensional grid vertices.  
    Then the factor $\Phi(\vec z_a)$ in the interaction term will be
undefined since $\vec z_a$ will not normally coincide with one of the
lattice points $\vec x_\ell$.  
    Thus some interpolation is required, and we can set
    \begin{equation}\label{eq:interpolate}
    \Phi(\vec z_a) = \sum_\ell h^3 \Phi(\vec x_\ell) I(\vec z_a - \vec x_\ell)
    \end{equation}
    where $h$ is the grid spacing and
    where $I(\vec x)$ is an interpolation kernel normalized so that 
    \begin{equation}\label{eq:normalize}
            \sum_\ell h^3 I(\vec z -\vec x_\ell) = 1 . 
    \end{equation}
    The kernel $I$ chooses a linear combination of nearby grid values
$\Phi_\ell$ to estimate $\Phi$ at any desired point and provides an
approximation to the delta function in 
    \begin{equation}\label{eq:approxDelta}
    \Phi(\z,t)) = \int\!\!d^3x\,\Phi(\x,t) \delta^3(\z-\x) .
    \end{equation}
    With this discretization the source term in the Poisson equation
(\ref{eq:Poisson}) becomes
    \begin{equation}\label{eq:gravSource1}
    4 \pi G \rho(\vec x_\ell) =
          4 \pi G \sum_{a=1}^{N} m_a I({\vec z_a} - {\vec x_\ell}).
    \end{equation}
    Note that the normalization condition (\ref{eq:normalize}) leads
to a conserved gravitational mass
    \begin{equation}\label{eq:gravMass}
    \sum_\ell h^3 G \rho(\vec x_\ell) =
           \sum_{a=1}^{N} G m_a 
    \end{equation}
    which is important since simple discretizations of the Laplacian
satisfy Gauss's theorem and show that this gravitational mass will be
reflected in a discrete surface integral of the gravitational field at
an outer boundary outside the mass distribution.


    The second way to treat the interaction term $L_{\rm int}$ of
Eq.~\ref{eq:rhoPhi} is to begin from its $\int\!d^3x\,\rho \Phi$ form
and introduce first the grid of field points $x_\ell$.  
    This yields $\sum_\ell h^3 \rho(\x_\ell,t) \Phi(\x_\ell, t)$.   
    Now we need a definition for $\rho(\x_\ell,t)$ and can choose the
same smoothing (\ref{eq:smoothed_rho}) considered in the definition
of the pressure gradient forces:
    \begin{equation}\label{eq:rhogrid}
    \rho(\x_\ell,t) = \sum_{c=1}^{N} m_c 
                    W( {\vec x_\ell} - {\vec z_c}; h).
    \end{equation}
    From this definition it follows that 
    \begin{equation}\label{eq:dicreteLint}
    L_{\rm int} = -\sum_\ell h^3  \Phi(\x_\ell, t)
                  \sum_{c=1}^{N} m_c W({\vec x_\ell} - {\vec z_c}; h).
    \end{equation}
    This differs from the interaction term found by the first method
only in the replacement of an interpolation kernel $I$ by a smoothing
kernel $W$.
    
    What we learn by comparing these two approaches to discretization
is that a consistent Lagrangian formulation (with anticipated
benefits from conservation of energy and momentum) is possible using
a different smoothing $I$ for the gravitational interaction than the
one $W$ used for the hydrodynamic interaction.
    From a physical viewpoint is would seem likely that the
appropriate choice is to use the same $W$ smoothing kernel in both
places.
    The right hand side of the Poisson equation
(\ref{eq:Poisson}) then becomes
    \begin{equation}\label{eq:gravSource2}
    4 \pi G \rho(\vec x_\ell) =
          4 \pi G \sum_{c=1}^{N} m_c W({\vec x_\ell} - {\vec z_c}; h).
    \end{equation}
    However this will conserve the gravitational mass only if the
normalization condition for $W$ in Eq.~(\ref{norm}) is discretized to
agree with Eq.~(\ref{eq:normalize}), which we therefore assume.
    The gravitational term in the hydrodynamic equations becomes
    \begin{equation}\label{eq:gravForce}
    \left( \frac{\partial \Phi}{\partial \x} \right)_{\x = \z_a}
    \mapsto
    \sum_\ell h^3  \Phi(\x_\ell, t)
         \frac{\partial W({\vec x_\ell} - {\vec z_a},h)}{\partial \z_a}.
    \end{equation}
    
\end{document}


