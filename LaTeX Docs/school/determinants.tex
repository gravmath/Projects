%Appendix on Determinants
\documentclass[12pt]{article}
%load packages
\usepackage{latexsym}
\usepackage{epic,eepic,graphicx,url}
%spacer commands
\newcommand{\eqc}{\ensuremath{\: ,} }
\newcommand{\eqp}{\ensuremath{\: .} }
\newcounter{bean}
\newcommand{\surp}[1]{\ensuremath{\left( {#1} \right)} }
\newcommand{\surb}[1]{\ensuremath{\left[ {#1} \right]} }
\newcommand{\surc}[1]{\ensuremath{\left\{ {#1} \right\}} }
\newcommand{\sura}[1]{\ensuremath{\left\langle #1 \right\rangle}}
\newcommand{\sure}[1]{({#1})}
\newcommand{\push} {\ensuremath{\:}}
\newcommand{\nudge}{\ensuremath{\,}}
\newcommand{\back}{\ensuremath{\!\!}}
\newcommand{\iback}{\ensuremath{\!\!\!\!}}
%equation commands
\newcommand{\be} { \begin{equation} }
\newcommand{\ee} { \end{equation}   }
\newcommand{\bes} { \[ }
\newcommand{\ees} { \]  }
\newcommand{\bea}{ \begin{eqnarray} }
\newcommand{\eea}{ \end{eqnarray}   }
\newcommand{\beas}{ \begin{eqnarray*} }
\newcommand{\eeas}{ \end{eqnarray*}   }
%index commands
\newcommand{\idx}[2] {\ensuremath{ {#1}{#2}   }}
\newcommand{\up} [1] {\ensuremath{{}^{#1}     } }
\newcommand{\dn} [1] {\ensuremath{{}_{#1}     } }
%array commands
\newcommand{\cv}  [1]{\ensuremath{\vec {#1}} }
\newcommand{\rv}  [1]{\ensuremath{{\vec {#1}}^{\,T}} }
\newcommand{\cf}  [1]{\ensuremath{\tilde {#1}} }
\newcommand{\rf}  [1]{\ensuremath{{\tilde {#1}}{\,^T}} }
\newcommand{\cs}  [1]{\ensuremath{\bar {#1}} }
\newcommand{\rs}  [1]{\ensuremath{{\bar {#1}}{\,^T}} }
\newcommand{\cuv} [1]{\ensuremath{\hat {#1}} }
\newcommand{\ruv} [1]{\ensuremath{{\hat {#1}}{\,^T}} }
\newcommand{\op}  [1]{\ensuremath{\mathbf{{#1}}}}
\newcommand{\dt}  [1]{\ensuremath{\dot {{#1}}} }
\newcommand{\ddt} [1]{\ensuremath{\ddot {{#1}}} }
\newcommand{\trps}[1]{\ensuremath{{#1}^{\,T} } }
\newcommand{\bv}  [1]{\ensuremath{\cs{e}_{#1}} }
\newcommand{\bd}  [1]{\ensuremath{\cf{\gw}^{#1}} }
%\newcommand{\rank}[2]{\ensuremath{\surp{\begin{array}{c}{#1 \\ #2}\end{array}}} }
\newcommand{\rank}[2]{\ensuremath{\surp{#1, #2}}}
%inner products
\newcommand{\ipp}[2]{\ensuremath{ (  {#1} ,     {#2} ) } }
\newcommand{\ipb}[2]{\ensuremath{ \langle {#1} ,     {#2} \rangle } }
\newcommand{\ipD}[2]{\ensuremath{ \langle {#1} |     {#2} \rangle } }
\newcommand{\ipd}[2]{\ensuremath{         {#1} \cdot {#2}         } }
%derivative and integral commands
\newcommand{\dby}   [1]{\ensuremath{ \frac{d}{d #1}} }
\newcommand{\dxby}  [2]{\ensuremath{ \frac{d #1}{d #2}} }
\newcommand{\pdby}  [1]{\ensuremath{ \frac{\partial}{\partial #1}} }
\newcommand{\pdxby} [2]{\ensuremath{ \frac{\partial #1}{\partial #2}} }
\newcommand{\myint} [1]{\ensuremath{ \int #1 \push} }
%variations
\newcommand{\var}[1]{\ensuremath{\delta {#1}}}
\newcommand{\vwrt}[2]{\ensuremath{\idx{\delta {#1}\vert}{\dn{\delta {#2}}}} }
\newcommand{\IBP}{\ensuremath{\stackrel{\mbox{\tiny{IBP}}}{=}}}
%common abreviations
\newcommand{\wrt}{with respect to }
\newcommand{\bdyterms}{boundary-terms }
\newcommand{\ibp}{integration-by-parts }
\newcommand{\dfunc}{\gd\back-function }
\newcommand{\dfuncs}{\gd\back-functions }
%special characters
\newcommand{\ga}{\ensuremath{\alpha} }
\newcommand{\gba}{\ensuremath{\bar \alpha} }
\newcommand{\gha}{\ensuremath{\hat \alpha} }
\newcommand{\gta}{\ensuremath{\tilde \alpha} }
\newcommand{\gb}{\ensuremath{\beta} }
\newcommand{\gbb}{\ensuremath{\bar \beta} }
\newcommand{\ghb}{\ensuremath{\hat \beta} }
\newcommand{\gtb}{\ensuremath{\tilde \beta} }
\newcommand{\gd}{\ensuremath{\delta} }
\newcommand{\gbd}{\ensuremath{\bar \delta} }
\newcommand{\ghd}{\ensuremath{\hat \delta} }
\newcommand{\gtd}{\ensuremath{\tilde \delta} }
\newcommand{\gD}{\ensuremath{\Delta} }
\newcommand{\gbD}{\ensuremath{\bar \Delta} }
\newcommand{\ghD}{\ensuremath{\hat \Delta} }
\newcommand{\gtD}{\ensuremath{\tilde \Delta} }
\newcommand{\get}{\ensuremath{\eta} }
\newcommand{\gbet}{\ensuremath{\bar \eta} }
\newcommand{\ghet}{\ensuremath{\hat \eta} }
\newcommand{\gtet}{\ensuremath{\tilde \eta} }
\newcommand{\gf}{\ensuremath{\phi} }
\newcommand{\gbf}{\ensuremath{\bar \phi} }
\newcommand{\ghf}{\ensuremath{\hat \phi} }
\newcommand{\gtf}{\ensuremath{\tilde \phi} }
\newcommand{\gF}{\ensuremath{\Phi} }
\newcommand{\gbF}{\ensuremath{\bar \Phi} }
\newcommand{\ghF}{\ensuremath{\hat \Phi} }
\newcommand{\gtF}{\ensuremath{\tilde \Phi} }
\newcommand{\gG}{\ensuremath{\Gamma} }
\newcommand{\gbG}{\ensuremath{\bar \Gamma} }
\newcommand{\ghG}{\ensuremath{\hat \Gamma} }
\newcommand{\gtG}{\ensuremath{\tilde \Gamma} }
\newcommand{\ggm}{\ensuremath{\gamma} }
\newcommand{\gbgm}{\ensuremath{\bar \gamma} }
\newcommand{\ghgm}{\ensuremath{\hat \gamma} }
\newcommand{\gtgm}{\ensuremath{\tilde \gamma} }
\newcommand{\gl}{\ensuremath{\lambda} }
\newcommand{\gbl}{\ensuremath{\bar \lambda} }
\newcommand{\ghl}{\ensuremath{\hat \lambda} }
\newcommand{\gtl}{\ensuremath{\tilde \lambda} }
\newcommand{\gL}{\ensuremath{\Lambda} }
\newcommand{\gbL}{\ensuremath{\bar \Lambda} }
\newcommand{\ghL}{\ensuremath{\hat \Lambda} }
\newcommand{\gtL}{\ensuremath{\tilde \Lambda} }
\newcommand{\gm}{\ensuremath{\mu} }
\newcommand{\gbm}{\ensuremath{\bar \mu} }
\newcommand{\ghm}{\ensuremath{\hat \mu} }
\newcommand{\gtm}{\ensuremath{\tilde \mu} }
\newcommand{\gn}{\ensuremath{\nu} }
\newcommand{\gbn}{\ensuremath{\bar \nu} }
\newcommand{\ghn}{\ensuremath{\hat \nu} }
\newcommand{\gtn}{\ensuremath{\tilde \nu} }
\newcommand{\gp}{\ensuremath{\pi} }
\newcommand{\gbp}{\ensuremath{\bar \pi} }
\newcommand{\ghp}{\ensuremath{\hat \pi} }
\newcommand{\gtp}{\ensuremath{\tilde \pi} }
\newcommand{\gq}{\ensuremath{\theta} }
\newcommand{\gbq}{\ensuremath{\bar \theta} }
\newcommand{\ghq}{\ensuremath{\hat \theta} }
\newcommand{\gtq}{\ensuremath{\tilde \theta} }
\newcommand{\gr}{\ensuremath{\rho} }
\newcommand{\gbr}{\ensuremath{\bar \rho} }
\newcommand{\ghr}{\ensuremath{\hat \rho} }
\newcommand{\gtr}{\ensuremath{\tilde \rho} }
\newcommand{\gs}{\ensuremath{\sigma} }
\newcommand{\gbs}{\ensuremath{\bar \sigma} }
\newcommand{\ghs}{\ensuremath{\hat \sigma} }
\newcommand{\gts}{\ensuremath{\tilde \sigma} }
\newcommand{\gt}{\ensuremath{\tau} }
\newcommand{\gbt}{\ensuremath{\bar \tau} }
\newcommand{\ght}{\ensuremath{\hat \tau} }
\newcommand{\gtt}{\ensuremath{\tilde \tau} }
\newcommand{\gw}{\ensuremath{\omega} }
\newcommand{\gbw}{\ensuremath{\bar \omega} }
\newcommand{\ghw}{\ensuremath{\hat \omega} }
\newcommand{\gtw}{\ensuremath{\tilde \omega} }
\newcommand{\gW}{\ensuremath{\Omega} }
\newcommand{\gbW}{\ensuremath{\bar \Omega} }
\newcommand{\ghW}{\ensuremath{\hat \Omega} }
\newcommand{\gtW}{\ensuremath{\tilde \Omega} }
\newcommand{\gy}{\ensuremath{\psi} }
\newcommand{\gby}{\ensuremath{\bar \psi} }
\newcommand{\ghy}{\ensuremath{\hat \psi} }
\newcommand{\gty}{\ensuremath{\tilde \psi} }
\newcommand{\gY}{\ensuremath{\Psi} }
\newcommand{\gbY}{\ensuremath{\bar \Psi} }
\newcommand{\ghY}{\ensuremath{\hat \Psi} }
\newcommand{\gtY}{\ensuremath{\tilde \Psi} }
%Script letters
\newcommand{\mA}{\ensuremath{\mathcal A} }
\newcommand{\mB}{\ensuremath{\mathcal B} }
\newcommand{\mC}{\ensuremath{\mathcal C} }
\newcommand{\mD}{\ensuremath{\mathcal D} }
\newcommand{\mE}{\ensuremath{\mathcal E} }
\newcommand{\mF}{\ensuremath{\mathcal F} }
\newcommand{\mG}{\ensuremath{\mathcal G} }
\newcommand{\mH}{\ensuremath{\mathcal H} }
\newcommand{\mI}{\ensuremath{\mathcal I} }
\newcommand{\mJ}{\ensuremath{\mathcal J} }
\newcommand{\mK}{\ensuremath{\mathcal K} }
\newcommand{\mL}{\ensuremath{\mathcal L} }
\newcommand{\mM}{\ensuremath{\mathcal M} }
\newcommand{\mN}{\ensuremath{\mathcal N} }
\newcommand{\mO}{\ensuremath{\mathcal O} }
\newcommand{\mP}{\ensuremath{\mathcal P} }
\newcommand{\mQ}{\ensuremath{\mathcal Q} }
\newcommand{\mR}{\ensuremath{\mathcal R} }
\newcommand{\mS}{\ensuremath{\mathcal S} }
\newcommand{\mT}{\ensuremath{\mathcal T} }
\newcommand{\mU}{\ensuremath{\mathcal U} }
\newcommand{\mV}{\ensuremath{\mathcal V} }
\newcommand{\mW}{\ensuremath{\mathcal W} }
\newcommand{\mX}{\ensuremath{\mathcal X} }
\newcommand{\mY}{\ensuremath{\mathcal Y} }
\newcommand{\mZ}{\ensuremath{\mathcal Z} }
%references and citations
\newcommand{\refq}[1]{\sure{\ref{#1}}}
\newcommand{\refp}[1]{\ref{#1}}
\newcommand{\refs}[1]{\cite{#1}}
%Thermo, Fluid, and GR objects
%%%particle kinematics
\newcommand{\ptraj}  [1]{\ensuremath{\idx{z}{\up{#1}}}}
\newcommand{\pdtraj} [1]{\ensuremath{\idx{\dt{z}}{\up{#1}}}}
\newcommand{\pvel}   [1]{\ensuremath{\idx{u}{\dn{#1}}}}
\newcommand{\ppos}   [1]{\ensuremath{\idx{z}{\up{#1}}}}
\newcommand{\pdpos}  [1]{\ensuremath{\idx{\dt{z}}{\up{#1}}}}
\newcommand{\ptvel}  [1]{\ensuremath{\idx{u}{\dn{#1}}}}
\newcommand{\pdtvel} [1]{\ensuremath{\idx{\dt{u}}{\dn{#1}}}}
\newcommand{\ptraja} [2]{\ensuremath{\idx{z}{\dn{#1}\up{#2}}}}
\newcommand{\pdtraja}[2]{\ensuremath{\idx{\dt{z}}{\dn{#1}\up{#2}}}}
\newcommand{\pvela}  [2]{\ensuremath{\idx{u}{\dn{#1#2}}}}
\newcommand{\sm}     [2]{\ensuremath{W\!\surp{\cv{#1} - \cv{#2}}}}
%%%thermodynamic parameters
\newcommand{\gtro}{\ensuremath{\idx{\gtr}{\dn{0}}} }
\newcommand{\ierg}{\ensuremath{e\surp{\gr\surp{x}}} }
\newcommand{\erg} {\ensuremath{e} }
%%%metric functions
\newcommand{\bV}[2] {\idx{\cv{#1}}{\dn{#2}} \nudge }
\newcommand{\bF}[2] {\idx{\cf{#1}}{\up{#2}} \nudge }
\newcommand{\Jac}[2]{\idx{\gL}{\up{#1}\dn{#2}} \nudge }
\newcommand{\jac}[2]{\idx{\gL}{\dn{#1}\up{#2}} \nudge }
\newcommand{\Kd}[2] {\idx{\gd \nudge}{\up{#1}\dn{#2}} \nudge }
\newcommand{\kd}[2] {\idx{\gd}{\dn{#1}\up{#2}} \nudge }
\newcommand{\Cnx}[2]{\idx{\gG}{\up{#1}\dn{#2}} \nudge }
\newcommand{\met} [2]{\ensuremath{\idx{g}{\dn{#1 #2}}}}
\newcommand{\imet}[2]{\ensuremath{\idx{g}{\up{#1 #2}}}}
\newcommand{\metd}   {\ensuremath{ g } }
\newcommand{\smet} [1]{\ensuremath{\sura{\idx{g}{\dn{#1}}}} }
\newcommand{\simet}[1]{\ensuremath{\sura{\idx{g}{\up{#1}}}} }
\newcommand{\smetd}   {\ensuremath{\sura{g}}}
%functional arguments
\newcommand{\contarg}{\ensuremath{\surp{\idx{a}{\up{0}};\cv{a}}} }
\newcommand{\harg}   {\ensuremath{\surp{a^0,\cv{r}_1}} }
\newcommand{\parg}   {\ensuremath{\surp{t,\cv{z}}} }
\newcommand{\farg}   {\ensuremath{\surp{t,\cv{x}}} }
%variations
\newcommand{\vIz}[1] {\ensuremath{\vwrt{I}{\ptraj{#1}}} }
%useful relations
\newcommand{\normeq}[2]{\ensuremath{ \imet{#1}{#2} \pvel{#1}
\pvel{#2} + 1} }

%bibtex stuff
\pagenumbering{arabic}
\bibliographystyle{plain}
%backwards compatibility
\def\a{{\vec a}}
\def\x{{\vec x}}
\def\z{{\vec z}}
\def\P{{\Phi}}
\def\l{{\ell}}
\def\Pl{{\Phi_{\ell}}}
\def\D{{\Delta}}
\def\.{{\quad .}}
%\def\_,{{\quad ,}}
\def\half{\mbox{$\frac{1}{2}$}}
%\def\_.{{\quad .}}
\def\po{{\hat \rho}_{0}}

\begin{document}
\title{On Some Useful Properties of Determinants}
\maketitle

In several places in the main body of this text, derivatives of the determinant of the
metric or of the Jacobian of a map are needed.  The algebra of the computations
is involved enough that its inclusion would be a distraction and thus the various
results have been gathered here.

Let \op{A} be an $N \times N$ matrix
\bes
  \op{A} = \left(
              \begin{array}{cccc}
                \idx{a}{\dn{11}} & \idx{a}{\dn{12}} & \cdots & \idx{a}{\dn{1N}} \\
                \idx{a}{\dn{21}} & \idx{a}{\dn{22}} & \vdots & \idx{a}{\dn{2N}} \\
                \vdots           & \vdots           & \ddots & \vdots           \\
                \idx{a}{\dn{N1}} & \idx{a}{\dn{N2}} & \cdots & \idx{a}{\dn{NN}}
              \end{array}
           \right)
\ees
with entries $ \idx{\surb{\op{A}}}{\dn{ij}} = \idx{a}{\dn{ij}}$ and with $i=1,2, \ldots, N$,
and $j=1,2, \ldots, N$.
Define a signed elementary product from \op{A} \refs{anton84} to mean any product
$\pm a_{i_1 j_1} a_{i_2 j_2} \cdots a_{i_N j_N}$ of $N$
entries from \op{A}, no two of which come
from the same row or column.
The plus sign is chosen if both $i_1, i_2, \ldots, i_N$ and
$j_1, j_2, \ldots, j_N$ are either even or odd permutations of $1, 2, \ldots, N$.
The determinant of \op{A}, denoted by $a$, can be expressed as
\be\label{eq:D1}
\det{\surp{\op{A}}} \equiv a = \frac{1}{N!} \surb{i_1, i_2, \ldots, i_N}
       \surb{j_1, j_2, \ldots, j_N} \idx{a}{\dn{i_1 j_1}}
       \idx{a}{\dn{i_2 j_2}} \dots \idx{a}{\dn{i_N j_N}} \eqc
\ee
where the permutation symbol $\surb{i_1, i_2, \ldots, i_N}$ is defined as
\bes
\surb{i_1, i_2, \ldots, i_N}  = \left\{
                                  \begin{array}{rl}
                                  +1 & \mbox{if $i_1, i_2, \ldots, i_N$
                                             is an even permutation of $1, 2, \ldots, N$} \\
                                  -1 & \mbox{if $i_1, i_2, \ldots, i_N$
                                             is an odd permutation of $1, 2, \ldots, N$}  \\
                                  0  & \mbox{otherwise} \eqp
                                  \end{array}
                                \right.
\ees
The first permutation symbol assures that each term
$a_{i_1 j_1} a_{i_2 j_2} \cdots a_{i_N j_N}$ comes from a different row while the
second permutation symbol assures each term comes from a different column.
The determinant is defined to be the sum of the $N!$ different signed
elementary products in \op{A} \refs{anton84}.
The product of the two permutation symbols in \refq{eq:D1} produces a sum
of $N!$ terms, each term being comprised of the $N!$ different signed elementary
products.
The normalization $1/N!$ is included to account for this overcounting.

Differentiating \sure{\ref{eq:D1}} \wrt $a_{r s}$ yields an expression for the
cofactors of the determinant
\be\label{eq:D2}
  C_{r s} = \pdxby{a}{a_{rs}}
          = \frac{1}{\surp{N-1}!} \surb{r, i_2, \ldots, i_N}
            \surb{s, j_2, \ldots, j_N} a_{i_1 j_1} \cdots a_{i_N j_N} \eqp
\ee
Comparing \refq{eq:D2} to \refq{eq:D1} leads to
\bes
C_{r s} a_{t s} = C_{s r} a_{s t} = a \push \delta_{r t}
\ees
of which the familiar Laplace expression \refs{anton84}
\be\label{eq:D3}
 a = C_{r s} a_{r s} \eqc
\ee
is a special case.
These results can be combined to yield the well-known result
\be
  \surb{\op{A}^{-1}}_{rs} = \frac{1}{a} C_{s r}
\ee
for the inverse of an $N \times N$ matrix in terms of the transpose of the
matrix of cofactors \refs{anton84,L_00}.
Using \refq{eq:D3}, the formula for the variation of the determinant is
\be\label{eq:D4}
  \var{a} = C_{r s} \var{a_{r s}} = a \surb{\op{A}^{-1}}_{s r} \var{a_{r s}} \eqp
\ee

\noindent \emph{Variations of the Metric}

It is common, when performing variational principles in general relativity,
to have to compute the variation of the determinant of the metric, denoted
by $g$.
To obtain the desired result, the substitutions $a_{r s} \rightarrow g_{\gm \gn}$
and $\surb{\op{A}^{-1}}_{s r} \rightarrow g^{\gm \gn}$ are used in \refq{eq:D4} to yield
\be
 \var{g} = g \push g^{\gn \gm} \var{g_{\gm \gn}} = g \push g^{\gm \gn} \var{g_{\gm \gn}} \eqp
\ee

\noindent \emph{Variations of a Jacobean}

Consider a general mapping from coordinates $x^j$ to coordinates $q^i$ where the transformation
is given by
\bes
    q^{\gbm} = q^{\gbm} \surp{x^\gn} \eqp
\ees
The Jacobean of the map is defined to be the matrix of partial derivatives
\bes
  \Jac{\gbm}{\gn} = \pdxby{q^{\gbm}}{x^{\gn}} \eqp
\ees
The determinant of this matrix
\bes
   J \equiv \det{\Jac{\gbm}{\gn}} = \det{\surp{\pdxby{q^{\gbm}}{x^{\gn}}}}
\ees
plays a fundamental role in the fluid dynamics variational principles discussed.
The variation of the Jacobean determinant can be expressed in
terms of \refq{eq:D3} as
\be\label{eq:D5}
  \var{J} = J \push \idx{C}{\dn{\gbm}\up{\gn}} \Jac{\gbm}{\gn} \eqp
\ee
where the cofactors \idx{C}{\dn{\gbm}\up{\gn}} are given by
\be\label{eq:D6}
\idx{C}{\dn{\gbm}\up{\gn}} = \frac{1}{\surp{N-1}!} \surb{\gbm, i_2, \ldots, i_N}
                        \surb{\gn, j_2, \ldots, j_N} \idx{q}{\up{i_2}\dn{,j_2}}
                        \cdots \idx{q}{\up{i_N}\dn{,j_N}} \eqp
\ee
An important property of the cofactors of the Jacobean is
\be\label{eq:D7}
\pdby{x^{\gn}} \idx{C}{\dn{\gbm}\up{\gn}} = 0 \eqp
\ee
This property can be easily seen since each term in \refq{eq:D7}
\bes
   \frac{1}{\surp{N-1}!} \surb{\gbm, i_2, \ldots, i_N} \surb{\gn, j_2, \ldots, j_N}
                         \idx{q}{\up{i_2}\dn{,j_2}}   \cdots
                         \idx{q}{\up{i_M}\dn{,j_M \gn}} \cdots
                         \idx{q}{\up{i_N}\dn{,j_N}} \eqc
\ees
is a product between symmetric and antisymmetric arrays.

Other relations involving the determinant of the Jacobian arise when defining the
concept of a tensor density.
The presentation here of both the covariant and Lie derivatives of a tensor density follow
closely the respective presentations in section 4.1 and 4.4 of \refs{LR89}.
To define a tensor density, consider the transformation of the metric between `barred'
and `un-barred' coordinates given by
\be
 \met{\gbm \gbn} \nudge = \nudge \jac{\gbm}{\ga} \jac{\gbn}{\gb} \met{\ga \gb} \eqp
\ee
Taking the determinant of both sides yields
\be\label{eq:D8}
  {\bar g} = \mJ^2 g \eqp
\ee
where the determinant of the inverse Jacobian of the mapping \mJ \nudge is defined as
\be
  \mJ = \det{ \surp{ \pdxby{x^{\gn}}{q^{\gbm}}} } \eqp
\ee
Equation \refq{eq:D8} is the simplest example of a tensor density, in this case a scalar
density, and the power of \mJ \nudge in \refq{eq:D8} is called the weight.
The determinant of the metric is said to be a scalar density
of weight $2$ and the more usual quantity $\sqrt{-g}$ is a scalar density of weight $1$.
Generalizing \refq{eq:D8} to an arbitrary tensor \idx{a}{\dn{\ga\gb}} of
weight $w$ and taking the partial
derivative with respect to \idx{x}{\up{\gba}} yields
\be
  \idx{\bar a}{\dn{,\gba}} = w \mJ^{w-1} \pdxby{\mJ}{\jac{\gbm}{\gn}}
                                         \pdxby{\jac{\gbm}{\gn}}{x^{\gba}}
                             + \mJ^w \pdxby{a}{x^{\gb}} \pdxby{x^{\gb}}{x^{\gba}} \eqp
\ee
Again \refq{eq:D4} can used to rewrite the first term yielding the expression
\be\label{eq:D9}
  \idx{\bar a}{\dn{,\gba}} = w \mJ^w \jac{\gn}{\gbm} \idx{\jac{\gbm}{\gn}}{\dn{,\gba}}
                             + \mJ^w \pdxby{a}{x^{\gb}} \pdxby{x^{\gb}}{x^{\gba}} \eqp
\ee
Using the transformation equation for the connection coefficients (see \textit{e.g.}
equation 10.26 of \refs{MTW})
\bes
\Cnx{\gbr}{\gba\gbt} = \Jac{\gm}{\gba} \Jac{\gn}{\gbt} \Jac{\gbr}{\gs} \Cnx{\gs}{\gm\gn}
                         + \pdxby{\Jac{\gn}{\gba}}{x^{\gbt}} \Jac{\gbr}{\gn} \eqc
\ees
the term involving the partial derivative of the Jacobian in \refq{eq:D9} can be written
as
\bes
   \jac{\gn}{\gbm} \idx{\jac{\gbm}{\gn}}{\dn{,\gba}} =    \Cnx{\gbr}{\gbr\gba}
                                                       - \Jac{\gn}{\gba} \Cnx{\gs}{\gs \gn} \eqp
\ees
Combining leads to
\be\label{eq:D10}
\surp{ \idx{\bar a}{\dn{,\gba}} - w \Cnx{\gbr}{\gbr \gba} {\bar a} } =
  \jac{\gba}{\gb} \mJ^w \surp{ \idx{a}{\dn{,\gb}} - w \Cnx{\gs}{\gs \gb} a }.
\ee
Equation \refq{eq:D10} is the transformation law for a rank \rank{0}{1} tensor density
of weight $w$ and thus defines the covariant derivative of the scalar density of weight
$w$ to be
\be\label{eq:D11}
  \idx{a}{\dn{;\gb}} = \idx{a}{\dn{,\gb}} - w \Cnx{\gs}{\gs \gb} a \eqp
\ee
Arbitrary tensor densities are built by multiplying the desired absolute tensors
by scalar densities of the appropriate weight.

Finally, the Lie derivative of a tensor density may be defined.
Recall that if a manifold is equipped with a vector field then the action of this
field can be interpreted as a mapping between those points in the manifold that
lie on the same integral curve of \cv{V} (see \textit{e.g.} \refs{Frankel}).
\begin{figure}
\centerline{
   \includegraphics[height=2in,width=3in,keepaspectratio]{lie_flows.eps}}
   \caption{Schematic representation of the flow on the manifold \mM \nudge due to
   the vector field \cv{V}.  The point \mP \nudge is mapped downstream an amount
   \gl to the point \mQ.}\label{fig:D1}
\end{figure}
Figure \ref{fig:D1} schematically shows this relationship.

Assuming that points \mP and \mQ \push are separated along a particular integral curve,
then the Lie derivative of a tensor (or tensor density) can be defined symbolically
as
\be\label{eq:D12}
  \pounds_{\cv{V}} \op{T} = \lim_{\gl \rightarrow 0}
                            \frac
                              {\surp{\op{T}(\mQ) - \surb{\op{T}(\mP)}_{\mQ}}}
                              {\gl} \eqc
\ee
where \op{T}(\mQ \back) is the tensor evaluated at point \mQ \push and $\surb{\op{T}(\mP)}_{\mQ}$
is the same tensor, now evaluated at point \mP, and then mapped downstream.
Since the flow generated by \cv{V} is a diffeomorphism, the mapping downstream can be done
for tensors of mixed ranks (see appendix C of \refs{wald} for more details).
All that is now needed is to express \refq{eq:D12} in terms of coordinates to derive
the formulae in question.
However, the general case of \refq{eq:D12} is unwieldy, and following \refs{LR89}
only a \rank{1}{1} tensor density will be examined, from which the general pattern
can be inferred.
In the equations that follow, all terms will be kept to first order in \gl.
To begin, assume that the mapping mediated by \cv{V} has the form
\be
  {\tilde x}^\gm = x^\gm + \gl V^{\gm}(x) \eqp
\ee
The Jacobian of this mapping is given by
\be
  \Jac{\gtm}{\gn} \equiv \pdxby{{\tilde x}^{\gm}}{x^\gn}
\ee
The inverse Jacobian of this mapping is given by
\be
  \jac{\gtn}{\gb} = \kd{\gn}{\gb} - \gl \idx{V}{\up{\gb}\dn{,\gn}} \eqc
\ee
and the corresponding determinant is
\be
  \mJ \simeq 1 - \gl \idx{V}{\up{\gs}\dn{,\gs}} \eqp
\ee
Now consider the \rank{1}{1} tensor density \idx{\mT}{\up{\gm}\dn{\gn}},
which transforms as
\be
  \idx{\mT}{\up{\gtm}\dn{\gtn}} = \mJ^w \Jac{\gtm}{\ga} \jac{\gtn}{\gb}
                                  \idx{\mT}{\up{\ga}\dn{\gb}} \eqp
\ee
The value of the tensor at downstream point \mQ \push is given by
\bea
 \op{T}(\mQ) & \doteq &   \idx{\mT}{\up{\gtm}\dn{\gtn}}(x^\gs + \gl V^{\gs}) \\
             &   =    &   \idx{\mT}{\up{\gtm}\dn{\gtn}}(x^\gs)
                         + \gl  \idx{\mT}{\up{\gtm}\dn{\gtn,\gs}} V^\gs \eqp
\eea
Likewise, the value of the tensor at the upstream point which is mapped downstream
is
\bea
  \surb{\op{T}(\mP)}_\mQ & \doteq & \surp{1 - w \gl \idx{V}{\up{\gs}\dn{,\gs}}}
                                    \surp{\Kd{\gm}{\ga} + \gl \idx{V}{\up{\gm}\dn{,\ga}}}
                                    \surp{\kd{\gn}{\gb} - \gl \idx{V}{\up{\gb}\dn{,\gn}}}
                                    \idx{\mT}{\up{\ga}\dn{\gb}} \\
                         &  = &   \idx{\mT}{\up{\gm}\dn{\gn}}
                                 - \gl \idx{T}{\up{\gm}\dn{\gb}} \idx{V}{\up{\gb}\dn{,\gn}}
                                 + \gl \idx{T}{\up{\ga}\dn{\gn}} \idx{V}{\up{\gm}\dn{,\ga}}
                                 - w \gl \idx{\mT}{\up{\gm}\dn{\gn}} \idx{V}{\up{\gs}\dn{,\gs}} \eqc
\eea
where every term on the right-hand side takes the argument $x^\gs$.
Combining these expressions in \refq{eq:D12} and noting, in the limit as
$\gl \rightarrow 0$, that ${\tilde x}^\gs \rightarrow x^{\gs}$, leads to the desired relation
\be\label{eq:D13}
  \pounds_{\cv{V}} \idx{\mT}{\up{\gm}\dn{\gn}} = \idx{\mT}{\up{\gm}\dn{\gn,\gs}} V^{\gs}
   + \idx{\mT}{\up{\gm}\dn{\gb}} \idx{V}{\up{\gb}\dn{,\gn}}
   - \idx{\mT}{\up{\ga}\dn{\gn}} \idx{V}{\up{\gm}\dn{,\ga}}
   + w \idx{\mT}{\up{\gm}\dn{\gn}} \idx{V}{\up{\gs}\dn{,\gs}} \eqc
\ee
and the obvious generalizations to higher rank tensors.
\bibliography{thesis}
\end{document}
