\documentclass{article}
%load packages
\usepackage{latexsym}
\usepackage{epic,eepic,graphicx,url}
%spacer commands
\newcommand{\eqc}{\ensuremath{\: ,} }
\newcommand{\eqp}{\ensuremath{\: .} }
\newcounter{bean}
\newcommand{\surp}[1]{\ensuremath{\left( {#1} \right)} }
\newcommand{\surb}[1]{\ensuremath{\left[ {#1} \right]} }
\newcommand{\surc}[1]{\ensuremath{\left\{ {#1} \right\}} }
\newcommand{\sura}[1]{\ensuremath{\left\langle #1 \right\rangle}}
\newcommand{\sure}[1]{({#1})}
\newcommand{\push} {\ensuremath{\:}}
\newcommand{\nudge}{\ensuremath{\,}}
\newcommand{\back}{\ensuremath{\!\!}}
\newcommand{\iback}{\ensuremath{\!\!\!\!}}
%equation commands
\newcommand{\be} { \begin{equation} }
\newcommand{\ee} { \end{equation}   }
\newcommand{\bes} { \[ }
\newcommand{\ees} { \]  }
\newcommand{\bea}{ \begin{eqnarray} }
\newcommand{\eea}{ \end{eqnarray}   }
\newcommand{\beas}{ \begin{eqnarray*} }
\newcommand{\eeas}{ \end{eqnarray*}   }
%index commands
\newcommand{\idx}[2] {\ensuremath{ {#1}{#2}   }}
\newcommand{\up} [1] {\ensuremath{{}^{#1}     } }
\newcommand{\dn} [1] {\ensuremath{{}_{#1}     } }
%array commands
\newcommand{\cv}  [1]{\ensuremath{\vec {#1}} }
\newcommand{\rv}  [1]{\ensuremath{{\vec {#1}}^{\,T}} }
\newcommand{\cf}  [1]{\ensuremath{\tilde {#1}} }
\newcommand{\rf}  [1]{\ensuremath{{\tilde {#1}}{\,^T}} }
\newcommand{\cs}  [1]{\ensuremath{\bar {#1}} }
\newcommand{\rs}  [1]{\ensuremath{{\bar {#1}}{\,^T}} }
\newcommand{\cuv} [1]{\ensuremath{\hat {#1}} }
\newcommand{\ruv} [1]{\ensuremath{{\hat {#1}}{\,^T}} }
\newcommand{\op}  [1]{\ensuremath{\mathbf{{#1}}}}
\newcommand{\dt}  [1]{\ensuremath{\dot {{#1}}} }
\newcommand{\ddt} [1]{\ensuremath{\ddot {{#1}}} }
\newcommand{\trps}[1]{\ensuremath{{#1}^{\,T} } }
\newcommand{\bv}  [1]{\ensuremath{\cs{e}_{#1}} }
\newcommand{\bd}  [1]{\ensuremath{\cf{\gw}^{#1}} }
%\newcommand{\rank}[2]{\ensuremath{\surp{\begin{array}{c}{#1 \\ #2}\end{array}}} }
\newcommand{\rank}[2]{\ensuremath{\surp{#1, #2}}}
%inner products
\newcommand{\ipp}[2]{\ensuremath{ (  {#1} ,     {#2} ) } }
\newcommand{\ipb}[2]{\ensuremath{ \langle {#1} ,     {#2} \rangle } }
\newcommand{\ipD}[2]{\ensuremath{ \langle {#1} |     {#2} \rangle } }
\newcommand{\ipd}[2]{\ensuremath{         {#1} \cdot {#2}         } }
%derivative and integral commands
\newcommand{\dby}   [1]{\ensuremath{ \frac{d}{d #1}} }
\newcommand{\dxby}  [2]{\ensuremath{ \frac{d #1}{d #2}} }
\newcommand{\pdby}  [1]{\ensuremath{ \frac{\partial}{\partial #1}} }
\newcommand{\pdxby} [2]{\ensuremath{ \frac{\partial #1}{\partial #2}} }
\newcommand{\myint} [1]{\ensuremath{ \int #1 \push} }
%variations
\newcommand{\var}[1]{\ensuremath{\delta {#1}}}
\newcommand{\vwrt}[2]{\ensuremath{\idx{\delta {#1}\vert}{\dn{\delta {#2}}}} }
\newcommand{\IBP}{\ensuremath{\stackrel{\mbox{\tiny{IBP}}}{=}}}
%common abreviations
\newcommand{\wrt}{with respect to }
\newcommand{\bdyterms}{boundary-terms }
\newcommand{\ibp}{integration-by-parts }
\newcommand{\dfunc}{\gd\back-function }
\newcommand{\dfuncs}{\gd\back-functions }
%special characters
\newcommand{\ga}{\ensuremath{\alpha} }
\newcommand{\gba}{\ensuremath{\bar \alpha} }
\newcommand{\gha}{\ensuremath{\hat \alpha} }
\newcommand{\gta}{\ensuremath{\tilde \alpha} }
\newcommand{\gb}{\ensuremath{\beta} }
\newcommand{\gbb}{\ensuremath{\bar \beta} }
\newcommand{\ghb}{\ensuremath{\hat \beta} }
\newcommand{\gtb}{\ensuremath{\tilde \beta} }
\newcommand{\gd}{\ensuremath{\delta} }
\newcommand{\gbd}{\ensuremath{\bar \delta} }
\newcommand{\ghd}{\ensuremath{\hat \delta} }
\newcommand{\gtd}{\ensuremath{\tilde \delta} }
\newcommand{\gD}{\ensuremath{\Delta} }
\newcommand{\gbD}{\ensuremath{\bar \Delta} }
\newcommand{\ghD}{\ensuremath{\hat \Delta} }
\newcommand{\gtD}{\ensuremath{\tilde \Delta} }
\newcommand{\get}{\ensuremath{\eta} }
\newcommand{\gbet}{\ensuremath{\bar \eta} }
\newcommand{\ghet}{\ensuremath{\hat \eta} }
\newcommand{\gtet}{\ensuremath{\tilde \eta} }
\newcommand{\gf}{\ensuremath{\phi} }
\newcommand{\gbf}{\ensuremath{\bar \phi} }
\newcommand{\ghf}{\ensuremath{\hat \phi} }
\newcommand{\gtf}{\ensuremath{\tilde \phi} }
\newcommand{\gF}{\ensuremath{\Phi} }
\newcommand{\gbF}{\ensuremath{\bar \Phi} }
\newcommand{\ghF}{\ensuremath{\hat \Phi} }
\newcommand{\gtF}{\ensuremath{\tilde \Phi} }
\newcommand{\gG}{\ensuremath{\Gamma} }
\newcommand{\gbG}{\ensuremath{\bar \Gamma} }
\newcommand{\ghG}{\ensuremath{\hat \Gamma} }
\newcommand{\gtG}{\ensuremath{\tilde \Gamma} }
\newcommand{\ggm}{\ensuremath{\gamma} }
\newcommand{\gbgm}{\ensuremath{\bar \gamma} }
\newcommand{\ghgm}{\ensuremath{\hat \gamma} }
\newcommand{\gtgm}{\ensuremath{\tilde \gamma} }
\newcommand{\gl}{\ensuremath{\lambda} }
\newcommand{\gbl}{\ensuremath{\bar \lambda} }
\newcommand{\ghl}{\ensuremath{\hat \lambda} }
\newcommand{\gtl}{\ensuremath{\tilde \lambda} }
\newcommand{\gL}{\ensuremath{\Lambda} }
\newcommand{\gbL}{\ensuremath{\bar \Lambda} }
\newcommand{\ghL}{\ensuremath{\hat \Lambda} }
\newcommand{\gtL}{\ensuremath{\tilde \Lambda} }
\newcommand{\gm}{\ensuremath{\mu} }
\newcommand{\gbm}{\ensuremath{\bar \mu} }
\newcommand{\ghm}{\ensuremath{\hat \mu} }
\newcommand{\gtm}{\ensuremath{\tilde \mu} }
\newcommand{\gn}{\ensuremath{\nu} }
\newcommand{\gbn}{\ensuremath{\bar \nu} }
\newcommand{\ghn}{\ensuremath{\hat \nu} }
\newcommand{\gtn}{\ensuremath{\tilde \nu} }
\newcommand{\gp}{\ensuremath{\pi} }
\newcommand{\gbp}{\ensuremath{\bar \pi} }
\newcommand{\ghp}{\ensuremath{\hat \pi} }
\newcommand{\gtp}{\ensuremath{\tilde \pi} }
\newcommand{\gq}{\ensuremath{\theta} }
\newcommand{\gbq}{\ensuremath{\bar \theta} }
\newcommand{\ghq}{\ensuremath{\hat \theta} }
\newcommand{\gtq}{\ensuremath{\tilde \theta} }
\newcommand{\gr}{\ensuremath{\rho} }
\newcommand{\gbr}{\ensuremath{\bar \rho} }
\newcommand{\ghr}{\ensuremath{\hat \rho} }
\newcommand{\gtr}{\ensuremath{\tilde \rho} }
\newcommand{\gs}{\ensuremath{\sigma} }
\newcommand{\gbs}{\ensuremath{\bar \sigma} }
\newcommand{\ghs}{\ensuremath{\hat \sigma} }
\newcommand{\gts}{\ensuremath{\tilde \sigma} }
\newcommand{\gt}{\ensuremath{\tau} }
\newcommand{\gbt}{\ensuremath{\bar \tau} }
\newcommand{\ght}{\ensuremath{\hat \tau} }
\newcommand{\gtt}{\ensuremath{\tilde \tau} }
\newcommand{\gw}{\ensuremath{\omega} }
\newcommand{\gbw}{\ensuremath{\bar \omega} }
\newcommand{\ghw}{\ensuremath{\hat \omega} }
\newcommand{\gtw}{\ensuremath{\tilde \omega} }
\newcommand{\gW}{\ensuremath{\Omega} }
\newcommand{\gbW}{\ensuremath{\bar \Omega} }
\newcommand{\ghW}{\ensuremath{\hat \Omega} }
\newcommand{\gtW}{\ensuremath{\tilde \Omega} }
\newcommand{\gy}{\ensuremath{\psi} }
\newcommand{\gby}{\ensuremath{\bar \psi} }
\newcommand{\ghy}{\ensuremath{\hat \psi} }
\newcommand{\gty}{\ensuremath{\tilde \psi} }
\newcommand{\gY}{\ensuremath{\Psi} }
\newcommand{\gbY}{\ensuremath{\bar \Psi} }
\newcommand{\ghY}{\ensuremath{\hat \Psi} }
\newcommand{\gtY}{\ensuremath{\tilde \Psi} }
%Script letters
\newcommand{\mA}{\ensuremath{\mathcal A} }
\newcommand{\mB}{\ensuremath{\mathcal B} }
\newcommand{\mC}{\ensuremath{\mathcal C} }
\newcommand{\mD}{\ensuremath{\mathcal D} }
\newcommand{\mE}{\ensuremath{\mathcal E} }
\newcommand{\mF}{\ensuremath{\mathcal F} }
\newcommand{\mG}{\ensuremath{\mathcal G} }
\newcommand{\mH}{\ensuremath{\mathcal H} }
\newcommand{\mI}{\ensuremath{\mathcal I} }
\newcommand{\mJ}{\ensuremath{\mathcal J} }
\newcommand{\mK}{\ensuremath{\mathcal K} }
\newcommand{\mL}{\ensuremath{\mathcal L} }
\newcommand{\mM}{\ensuremath{\mathcal M} }
\newcommand{\mN}{\ensuremath{\mathcal N} }
\newcommand{\mO}{\ensuremath{\mathcal O} }
\newcommand{\mP}{\ensuremath{\mathcal P} }
\newcommand{\mQ}{\ensuremath{\mathcal Q} }
\newcommand{\mR}{\ensuremath{\mathcal R} }
\newcommand{\mS}{\ensuremath{\mathcal S} }
\newcommand{\mT}{\ensuremath{\mathcal T} }
\newcommand{\mU}{\ensuremath{\mathcal U} }
\newcommand{\mV}{\ensuremath{\mathcal V} }
\newcommand{\mW}{\ensuremath{\mathcal W} }
\newcommand{\mX}{\ensuremath{\mathcal X} }
\newcommand{\mY}{\ensuremath{\mathcal Y} }
\newcommand{\mZ}{\ensuremath{\mathcal Z} }
%references and citations
\newcommand{\refq}[1]{\sure{\ref{#1}}}
\newcommand{\refp}[1]{\ref{#1}}
\newcommand{\refs}[1]{\cite{#1}}
%Thermo, Fluid, and GR objects
%%%particle kinematics
\newcommand{\ptraj}  [1]{\ensuremath{\idx{z}{\up{#1}}}}
\newcommand{\pdtraj} [1]{\ensuremath{\idx{\dt{z}}{\up{#1}}}}
\newcommand{\pvel}   [1]{\ensuremath{\idx{u}{\dn{#1}}}}
\newcommand{\ppos}   [1]{\ensuremath{\idx{z}{\up{#1}}}}
\newcommand{\pdpos}  [1]{\ensuremath{\idx{\dt{z}}{\up{#1}}}}
\newcommand{\ptvel}  [1]{\ensuremath{\idx{u}{\dn{#1}}}}
\newcommand{\pdtvel} [1]{\ensuremath{\idx{\dt{u}}{\dn{#1}}}}
\newcommand{\ptraja} [2]{\ensuremath{\idx{z}{\dn{#1}\up{#2}}}}
\newcommand{\pdtraja}[2]{\ensuremath{\idx{\dt{z}}{\dn{#1}\up{#2}}}}
\newcommand{\pvela}  [2]{\ensuremath{\idx{u}{\dn{#1#2}}}}
\newcommand{\sm}     [2]{\ensuremath{W\!\surp{\cv{#1} - \cv{#2}}}}
%%%thermodynamic parameters
\newcommand{\gtro}{\ensuremath{\idx{\gtr}{\dn{0}}} }
\newcommand{\ierg}{\ensuremath{e\surp{\gr\surp{x}}} }
\newcommand{\erg} {\ensuremath{e} }
%%%metric functions
\newcommand{\bV}[2] {\idx{\cv{#1}}{\dn{#2}} \nudge }
\newcommand{\bF}[2] {\idx{\cf{#1}}{\up{#2}} \nudge }
\newcommand{\Jac}[2]{\idx{\gL}{\up{#1}\dn{#2}} \nudge }
\newcommand{\jac}[2]{\idx{\gL}{\dn{#1}\up{#2}} \nudge }
\newcommand{\Kd}[2] {\idx{\gd \nudge}{\up{#1}\dn{#2}} \nudge }
\newcommand{\kd}[2] {\idx{\gd}{\dn{#1}\up{#2}} \nudge }
\newcommand{\Cnx}[2]{\idx{\gG}{\up{#1}\dn{#2}} \nudge }
\newcommand{\met} [2]{\ensuremath{\idx{g}{\dn{#1 #2}}}}
\newcommand{\imet}[2]{\ensuremath{\idx{g}{\up{#1 #2}}}}
\newcommand{\metd}   {\ensuremath{ g } }
\newcommand{\smet} [1]{\ensuremath{\sura{\idx{g}{\dn{#1}}}} }
\newcommand{\simet}[1]{\ensuremath{\sura{\idx{g}{\up{#1}}}} }
\newcommand{\smetd}   {\ensuremath{\sura{g}}}
%functional arguments
\newcommand{\contarg}{\ensuremath{\surp{\idx{a}{\up{0}};\cv{a}}} }
\newcommand{\harg}   {\ensuremath{\surp{a^0,\cv{r}_1}} }
\newcommand{\parg}   {\ensuremath{\surp{t,\cv{z}}} }
\newcommand{\farg}   {\ensuremath{\surp{t,\cv{x}}} }
%variations
\newcommand{\vIz}[1] {\ensuremath{\vwrt{I}{\ptraj{#1}}} }
%useful relations
\newcommand{\normeq}[2]{\ensuremath{ \imet{#1}{#2} \pvel{#1}
\pvel{#2} + 1} }

%bibtex stuff
\pagenumbering{arabic}
\bibliographystyle{plain}
%backwards compatibility
\def\a{{\vec a}}
\def\x{{\vec x}}
\def\z{{\vec z}}
\def\P{{\Phi}}
\def\l{{\ell}}
\def\Pl{{\Phi_{\ell}}}
\def\D{{\Delta}}
\def\.{{\quad .}}
%\def\_,{{\quad ,}}
\def\half{\mbox{$\frac{1}{2}$}}
%\def\_.{{\quad .}}
\def\po{{\hat \rho}_{0}}

\begin{document}
%%
%%Top matter
%%
%%First draft version 0.3 completed on 1/29/02
%%
%%Updates:
%%         3/05/02 - corrected the density definition based on an update to my_commands
%%                   section 1 version 0.3.1
%%         4/12/02 - corrected sign error in sqrt{-g{x}} term in T_\mu\nu equation in
%%                   section 2.3 version 0.3.2
%%         5/6/02  - added initial section on the single fat particle star
\title{Single Fat Particle Systems}
\author{Conrad Schiff \\
        University of Maryland\\
      \texttt{cmschiff@erols.com} \\
      version 0.3.3}
\date{1/29/02  \\ \small{update 5/6/02} }
\maketitle
%%
%%Abstract
%%
%\begin{abstract}
%
%\end{abstract}
\section{Introduction}

In our previous work we developed a Lagrangian variational
principle for a self-gravitating fluid based on the action
\bes
I_{matter} = I_{grav} + \myint{d^4a} \gtro (1 + \erg) \surb{
\pdtraj{\gm} \pvel{\gm} - \gL \mH} \eqc
\ees
where the parameter \gtro is related to the number density \gr by
\bes
  \gtro = \gr \contarg \sqrt{
        \frac{ -{\bar g} \contarg }{ -g_{\bar 0 \bar
        0}\contarg}}
\ees
and, as the notation implies, the metric functions in \gtro are
evaluated in terms of a co-moving frame that is at rest with
respect to the fluid flow. The internal energy $e = e(\gr)$ gives
rise to the pressure required to keep the fluid from collapsing.
The last term leads to the Euler equations determining how the
fluid flows.  The super-Hamiltonian
\bes
  \mH = \frac{1}{2}\surp{ \imet{\gm}{\gn} \pvel{\gm} \pvel{\gn} + 1}
\ees
expresses the standard four-velocity normalization and is identically zero along
the proper geodesic motion \ptraj{\gm} of the fluid.
The Lagrange multiplier \gL enforces this constraint.
All told, this variational principle requires the variation of the metric
components \imet{\gm}{\gn}, the fluid-element trajectories
\ptraj{\gn}, the conjugate four-velocities \pvel{\gm}, and the
Lagrange multiplier \gL \back. These variations are performed subject to
the background constraint of the conservation of baryon number. In
our earlier work we wrote this conservation equation as
\bes
  \gtro = \gtr J
\ees
where \gtr has analogous definition as \gtro  above
\bes
\gtr(z) = \gr(z) \sqrt{\frac{-\metd(z)}{-\met{\gm}{\gn}(z)
\pdtraj{\gm} \pdtraj{\gn}}} \eqc
\ees
$\ptraj{\gm} \equiv \frac{d z^\gm}{d a^0}$, and the metric
functions are evaluated along the fluid flow lines. The Jacobean determinant
is given by $J \equiv \det \surp{
\frac{\partial z^{\mu}}{\partial a^{\nu}}}$.

What we wish to do in this work is to develop a coupled
matter-field system for a single self-gravitating fluid element.
The basic approach will be based on the above action but will be
specialized to a single particle whose extent will be spread out
over a finite coordinate distance using a smoothing kernel.  This
discretization will be affected using the relation $\gtro = m \gd
\surp{\cv{a} - \cv{r}_1}$.  The internal energy will not be set
equal to zero in the action.  At a glance, this may seem like
nonsense since we expect the internal energy to arise from an
interaction amongst several particles.  However, in physics we
often refer to collections of particles like a star, or a lake, or
a glass of water in singular terms.  What we are doing here is to
extend this concept to a single fluid element in General
Relativity.  There will necessarily be an approximation that all
of the constituents of the fluid element, which we are ignoring,
move with the same four-velocity.  Thus the element will resist
shear stresses and remain rigid through its course. Our hope for
this program is that the resulting field equations provide a
computational tool for modelling black holes in 3-d numerical
relativity.  However, before we try to model this difficult
problem we will attempt to see if a single fat particle can
provide a black hole-like solution to the Einstein field
equations.

\section{Deriving the Equations of Motion}

In this section we derive the equations of motion for our single
fluid element, hereafter denoted as a single fat particle. As
discussed in the Introduction, we arrive at a single fat particle
action by making the substitution
\bes
 \gtro = m \gd(\cv{a} - \cv{r}_1) \eqp
\ees
Inserting this expression into the action leads to
\bes
  I = I_{grav} + \int da^0 m (1 + \erg_1) \surp{ \pdtraja{1}{\gm}
  \pvela{1}{\gm} - \gL {\mathcal H}_1 } \eqc
\ees
where we use the following notation
\\ \\
\bea
  \pdtraja{1}{\gm} & = &  \pdtraj{\gm}\harg     \push \nonumber \\
  \pvela{1}{\gm}   & = &  \pvel{\gm} \harg      \push \nonumber \\
  \erg_1              & = &  \erg\surp{\gr\harg}      \push \nonumber \\
  {\mathcal H}_1   & = & \frac{1}{2}(\imet{\gm}{\gn}\harg \pvela{1}{\gm} \pvela{1}{\gn} + 1) \eqp   \nonumber
\eea
The `1' index at this point is excess baggage reminding us of our
humble beginnings with one particle but serving no other purpose.
Thus we drop it hereafter and just keep the single particle
concept in the back of our minds.

The introduction of the four-dimensional \dfunc $\gd(x^\gm -
z^\gm) = \gd(x^0 - z^0) \, \gd(\cv{x} - \cv{z})$ allows the
hydrodynamic action to be written in term of the gravitational
field variables.
\beas
I & = & \frac{1}{16 \pi} \myint{d^4x} R(x) \sqrt{-g(x)} \\
  &   & + m \myint{da^0} d^4x \push (1 + \erg) \surb{ \ptraj{\gm} \pvel{\gm} -
\gL \mathcal{H} } \gd(x^0 - z^0) \gd(\cv{x} - \cv{z})
\eeas
Now use path invariance and say $da^0 = dz^0$.  This allows us to
perform the $da^0$ integration and eliminate one \dfunc. The
remaining three-dimensional \dfunc over the spatial variables is
then approximated by a smoothing kernel leading to:
\bes
I =  \int d^4x \push \surc{ \frac{R(x) \sqrt{-g(x)}}{16 \pi} + m
(1 + \erg) \surb{ \ptraj{\gm} \pvel{\gm} - \gL\mathcal{H} }
\sm{x}{z} }
\ees
A delicate point arises here.  The presence of the \dfuncs meant
that we could be sloppy, allowing ourselves the luxury of
switching back and forth between the fluid trajectory variables
\ptraj{\gm} and the field variables $x^\mu$.  Once the
approximation of the spatial \dfuncs with the smoothing kernel has
been made, we are compelled to pick where our functions live.

In general, the choice for each function appearing in the action
is relatively straightforward.  Only the density, tentatively
defined by
\bes
  \gr = \frac { \sqrt{-\met{\gm}{\gn} \pdtraj{\gm} \pdtraj{\gn}}}
              { \sqrt{-g}} m \sm{x}{z} \eqc
\ees
presents any difficulty.  Here we are confronted with two choices
for where the metric functions are defined.  In great measure, we
are free to pick any convenient definition from the numerous ones
possible. We have this freedom since the prescription for baryon
number conservation can be written as
\bes
  \int d^3x \, \rho \frac{ \sqrt{-g}}
                         { \sqrt{-\met{\gm}{\gn} \ptraj{\gm} \ptraj{\nu}}}
                         m \sm{x}{z}
  = m \eqp
\ees
As long as the smoothing kernel $W$ is normalized any combination
of $\frac{ \sqrt{-\met{\gm}{\gn} \pdtraj{\gm} \pdtraj{\gn} }}{
\sqrt{-g}}$ can be used in the definition of \gr as long as the
same combination is used when calculating the total baryon number.
The primary complication is the `coupling' term
$\sqrt{-\met{\gm}{\gn} \pdtraj{\gm} \pdtraj{\gn} }$. It is
directly related to $u^0$ and is identified with the Lagrange
multiplier \gL in the continuum case by simultaneously using the
four-velocity normalization and the equation $\pdtraj{\gm} = \gL
\imet{\gm}{\gn} \pvel{\gn}$.  We will see below that in order to
make the `smoothed' analogs of the equations for the four-velocity
normalization and the relation between \pdtraj{\gm} and \pvel{\gm}
look natural we will have to adopt the following convention.  Any
time a metric function is used in combination with a fat particle
hydrodynamic function it should be smoothed according to the
prescription
\be\label{eq:inv_sm_met}
  \simet{\gm\gn}\parg = \myint{d^3x} \imet{\gm}{\gn}\farg \sm{x}{z} \eqp
\ee
We define the smoothed metric \sura{\met{\gm}{\gn}} by requiring
\be\label{eq:sm_met}
\simet{\gm\gn} \smet{\gn\gr} = \Kd{\gm}{\gr} \eqp
\ee
The density can be explicitly written as
\be\label{eq:rho_def}
    \gr = m \frac{ \sqrt{-\smet{\gm\gn} \pdtraj{\gm} \pdtraj{\gn}}}
                 { \sqrt{-g(t,\cv{x})}} \sm{x}{z} \eqp
\ee

Also we will be compelled to regard the internal energy, where
this coupling between the field and the hydrodynamic variables is
most entangled, as being a function of both $x^{\mu}$ subject to
the particle trajectories \ptraj{\gm}.  Functionally, we will
denote this dependence, when needed, as $\erg = \ierg$.  Finally we
will take the term $\sqrt{-g} = \sqrt{-g(x)}$.  Again the reason
for this choice is that it will make the field equations derived
by varying $\imet{\ga}{\gb}(x)$ look natural.  However, we
emphasize that once these two choices have been adopted we no
longer have the ability to adjust anything else in the action. The
resulting equations for the fluid motion will then reflect the
viability of these choices.

\subsection{Action Variation With Respect to \gL}
First we look at the variation of the action with respect to \gL
which yields
\bes
\vwrt{I}{\gL} = -\int d^4x \push m(1+\erg) \mathcal{H} \sm{x}{z}
\var{\gL} \eqp
\ees
Using the definition of the Hamiltonian reduces the above equation
to
\bes
\int dt \push d^3x \push \sm{x}{z} (1 + \erg)
\surb{\imet{\gm}{\gn}(x) \pvel{\gm} \pvel{\gn} + 1} = 0 \eqp
\ees
Defining an associated smoothing
\bes
    \langle f(z) \rangle _\erg = \int d^3x \, \sm{x}{z} (1+\erg) f(x)
\ees
where the inclusion of the internal energy term is unavoidable due
to the entanglement of the field and hydrodynamic degrees of
freedom as discussed above.  Ordinarily this would present a
substantial complication, creating a self-coupled variational
principle that would require extensive modification of the simple
equations of motion that resulted from the continuum theory.  In
order to avoid this we employ the standard SPH approximation of
rational functions that states
\bes
  \sura{ \frac{A(\vec r)}{B(\vec r)} }= \frac{ \langle
  A(\vec r) \rangle}{\langle B(\vec r) \rangle} + O(h^2) \eqp
\ees
Writing the equation resulting from $\delta \Lambda$ in terms of
the auxiliary smoothing we arrive at
\bes
  \langle g^{\mu\nu}(t,\vec z) \rangle_e u_{\mu}(t) u_{\nu}(t) +
  \langle 1 \rangle_\erg = 0 \eqp
\ees
Dividing by $\langle 1 \rangle_e$ we arrive at
\bes
  \frac{\langle g^{\mu\nu}(t,\vec z) \rangle_\erg}{\langle 1
  \rangle_\erg}u_{\mu}(t) u_{\nu}(t) + 1 = 0 \eqp
\ees
The first term is simplified using the SPH approximation of
rational functions to yield
\bes
\frac{\langle g^{\mu\nu}(t,\vec z) \rangle_\erg}{\langle 1
  \rangle_\erg}  =   \myint{d^3x} \frac{g^{\mu\nu}(t,\vec x)}{1+\erg} (1+\erg)
  W(\vec z - \vec x) = \myint{d^3x} g^{\mu\nu}(t,\vec x) W(\vec z
  - \vec x)
\ees
Using this relation simplifies the four-velocity normalization
condition to
\be\label{eq:norm_cond}
g^{\mu\nu}(t,\vec z) u_{\mu}(t) u_{\nu}(t) + 1 = 0
\ee
where $g^{\mu\nu}(t,\vec z) = \int d^3x \, g^{\mu\nu}(x) W(\vec x
- \vec z)$.

This equation agrees with what one would obtain by taking the
corresponding equation of the continuum theory and directly
smoothing.  This correspondence is used to justify the claim made
above that the metric functions can be smoothed using solely the
smoothing kernel.

\subsection{Action Variation with Respect to \pvel{\gm}}
Next we look at the variation of the action with respect to the
covariant four-velocity $u_{\mu}$ which yields
\bes
  \delta I_{\vert \delta u_{\mu}} = \myint{dt} m (1 + \erg) \left\{
  {\dot z}^{\mu} - \Lambda g^{\mu\nu}(t,\vec z) u_{\nu}(t)
  \right\} \delta u_{\mu}(t) \eqp
\ees
Again using the SPH approximation as above we arrive at
\be\label{eq:4vel_def}
{\dot z}^{\mu} = \Lambda g^{\mu\nu}(t,\vec z) u_{\nu}(t)
\ee
which is the `smoothed' analog of the continuum equation relating
the coordinate velocities ${\dot z}^{\mu}$ to the covariant
four-velocities $u_{\mu}$. Combining the two equations and solving
for $\Lambda$ yields
\be\label{eq:Lambda}
 \Lambda^2 = -g_{\mu\nu}(t,\vec z) {\dot z}^{\mu}{\dot z}^{\nu}
\ee
where $g_{\mu\nu}(t,\vec z)$ is the inverse matrix of
$g^{\mu\nu}(t,\vec z)$ formally defined by
\bes
g_{\mu\nu}(t,\vec z) g^{\mu\sigma}(t,\vec z) =
{\delta^{\sigma}}_{\nu} \eqp
\ees
Solving the above equation for $\Lambda$ gives us an equation
consistent with our smoothing convention; namely that since
$g_{\mu\nu}$ appears in conjunction with fluid degrees of freedom
it must be the smoothed form of that function.

%%%%%%%%%%%%%%%%%%%%%%%%%%%%%%%%%%%%%%%%%%%%%%%%%%%%%%%%%%%%%%%%%%
\subsection{Action Variation with Respect to \imet{\gm}{\gn}(x)}
%%%%%%%%%%%%%%%%%%%%%%%%%%%%%%%%%%%%%%%%%%%%%%%%%%%%%%%%%%%%%%%%%%

We now turn to the variation of the combined action with respect
to $g^{\mu\nu}(x)$. Taking this variation yields
\bea\label{eq:var_I_g}
 \vwrt{I}{g^{\mu\nu}} & = &
   \frac{1}{16\pi} \myint{d^3x \nudge dt} G_{\mu\nu} \sqrt{-g} \,\,
   \var{\imet{\gm}{\gn}} \nonumber \\
& &
 + \myint{d^3x \nudge dt} m \frac{P}{\gr^2} \vwrt{\gr}{\imet{\gm}{\gn}}
     \surc{ \pdtraj{\gm} u_{\mu} - \gL \mathcal{H} } W(\vec x - \vec z) \nonumber \\
& &
 - \myint{d^3x \nudge dt} m(1 + \erg)\frac{\gL}{2} u_{\mu} u_{\nu} \var{\imet{\gm}{\gn}} \eqp
\eea
The first term in \refq{eq:var_I_g} is the usual term that comes from the
variation of the Hilbert action.  The second term depends on the variation of the density,
as defined in \refq{eq:rho_def}, which is comprised of the two terms
\be\label{eq:var_rho}
\vwrt{\gr}{\imet{\gm}{\gn}} = \frac{-m \sm{x}{z}}{2}
\surc{ \frac{1}{\gL\sqrt{-g}} \var{\smet{\gm\gn}} \pdtraj{\gm}\pdtraj{\gn}
+ \frac{\gL}{\surp{-g}^{3/2}} \var{g}}
\ee
with \gL defined as in \refq{eq:Lambda} and $g(x) = \det{\surp{\met{\gm}{\gn}}}$.
The first term in \refq{eq:var_rho}, \var{\smet{\gm\gn}}, can be related to
the variation of the smoothed inverse metric defined in \refq{eq:inv_sm_met} by
taking the variation of \ref{eq:sm_met}
\bes
  \var{\smet{\gs\gt}} = \smet{\gs\ga} \var{\simet{\ga\gb}} \smet{\gb\gt} \eqp
\ees
The variation of \simet{\ga\gb}
\bes
  \var{\simet{\ga\gb}} = \myint{d^3x} \sm{x}{z} \var{\imet{\ga}{\gb}}
\ees
is easily obtained from the equation \refq{eq:inv_sm_met}.
The variation of the determinant of the metric follows from the standard relation
\bes
  \var{\metd} = - \metd \met{\gm}{\gn} \var{\imet{\gm}{\gn}} \eqp
\ees

Substituting these relations in \refq{eq:var_rho} and collecting terms against the
definition of \gr in \refq{eq:rho_def} yields
\bes
  \vwrt{\gr}{\imet{\gm}{\gn}}    =   \frac{\gr}{2} \surp{\met{\gm}{\gn}
                                     \var{\imet{\gm}{\gn}}
                                   + u_{\gm}(t) u_{\gn}(t)
                                     \myint{d^3y} \var{\imet{\gm}{\gn}}(\cv{y},t) \sm{y}{z}} \eqp
\ees
Using equation \refq{eq:4vel_def}, in the form \pdtraj{\mu} \ptvel{\mu} = -\gL, in
combination with \mH = 0, renders the action variation as
\beas
 \var{I} & = & \myint{d^3x dt}    \surc{ \frac{\idx{G}{\dn{\gm\gn}}}{16 \gp}
                                - \frac{\gr}{2} (1+e) \ptvel{\mu} \ptvel{\gn}
                                - \frac{P}{2} \met{\gm}{\gn}
                                       } \sqrt{-\metd} \nudge \var{\imet{\gm}{\gn}} \\
         & + & \myint{d^3x dt} P \nudge \ptvel{\gm} \ptvel{\gn} \sqrt{-\metd}
               \myint{d^3y} \sm{y}{z} \var{\imet{\gm}{\gn}(\cv{y},t)} \eqp
\eeas
By switching the dummy variables of integration in the latter integral, $x \rightarrow y$ and
$y \rightarrow x$, and collecting terms under one integral, we arrive at
\beas
\var{I} & = & \myint{d^3x dt} \left\{ \frac{\idx{G}{\dn{\gm\gn}}}{16 \gp} - \frac{\gr}{2}
  (1+e) \ptvel{\gm} \ptvel{\gn} - \frac{P}{2} \met{\gm}{\gn} \right. \\
        &   &  \left. + \frac{\sm{x}{z}}{\sqrt{-\metd}}  \ptvel{\gm} \ptvel{\gn}
        \myint{d^3y} P \nudge \sqrt{-\metd} \right\} \sqrt{-\metd} \nudge \var{\imet{\gm}{\gn}} \eqp
\eeas
Setting the variation to zero, we arrive at the `smoothed' form of the Einstein equations
\bea\label{eq:def_T}
   \idx{G}{\dn{\gm\gn}} = 8 \gp \surc{   \gr (1+e) \ptvel{\gm} \ptvel{\gn} + P \met{\gm}{\gn}
                                  + \frac{W(\cv{x} - \cv{z})}{\sqrt{-g(x)}}
                                    \ptvel{\gm} \ptvel{\gn}
                                    \surb{\myint{d^3y} P(y) \sqrt{-g(y)}} }
\eea
with the right-hand side being taken as the definition of the `smoothed' stress-energy tensor.


\subsection{Action Variation with Respect to \ptraj{\gm}}

The final step is to take the variation of the action \wrt
\ptraj{\gm} to obtain the Euler equations for the particle. To
facilitate this end, the separation between space and time will be
observed.  The action, thus becomes
\be\label{eq:smthI}
I = m \int
dt \push \pdtraj{\gm} \pvel{\gm} \sura{1+e} - m \int dt \push
\frac{\gL}{2} \sura{\surp{1+e}\surp{\imet{\gm}{\gn} \pvel{\gm}
\pvel{\gn} + 1}} \eqp
\ee
%
Now taking the variation of \surp{\ref{eq:smthI}} yields
\bes
\vIz{i} = m \int dt \push \surc{ \var{\pdpos{i}} \ptvel{i}
\sura{1+e} + \pdtraj{\gm} \pvel{\gm} \var{\sura{1+e}} -
\frac{\gL}{2} \var{ \sura{ \surp{1+e} \surb{\normeq{\gm}{\gn}}}}}
\eqp
\ees
Integrate the first term by parts to move the time derivative off
of \ptraj{\mu}.  Next use the relation $\pdtraj{\gm}\!\pvel{\gm} =
-\gL$ in the second term.  For the third term, the relation
\be\label{eq:norm}
 \sura{f(x)\surb{\normeq{\gm}{\gn}}} =
\sura{f(x)}\sura{\normeq{\gm}{\gn}} = 0
\ee
%
is used to expand the variation.  Follow this by switching the
derivative on the smoothing kernel from
\idx{\partial}{\dn{\cv{z}}} to \idx{\partial}{\dn{\cv{x}}} and
integrate by parts.  Performing these steps yields \bea \vIz{i} &
= & m \myint{dt} \surc{
   \dby{t} \surp{\pvel{i}\sura{1+\erg}} \var{\ptraj{i}}
 - \gL \var{\sura{1+\erg}} } \nonumber \\
        &   & - m \myint{dt}
   \frac{\gL}{2} \myint{d^3x} \pdby{x^i} \surc{ \surp{1+\erg}
     \surp{\imet{\gm}{\gn} \pvel{\gm} \pvel{\gn} + 1}} \sm{x}{z} \var{\ptraj{i}}
\eea
%
Again use \surp{\ref{eq:norm}} when expanding
\pdby{x^i}\surc{\surp{1+\erg}\surb{\normeq{\gm}{\gn}}} to obtain
%
\bea
\vIz{i} & = & -m \myint{dt} \surc{
   \dby{t} \surp{ \pvel{i} \sura{1+\erg} } \var{\ptraj{i}}
 - \gL \var{\sura{1+\erg}}} \nonumber \\
             &   & - m \myint{dt}
   \frac{\gL}{2} \myint{d^3x} \surp{1+\erg}
   \pdxby{\imet{\gm}{\gn}}{x^i} \pvel{\gm} \pvel{\gm} \sm{x}{z}
   \var{\ptraj{i}} \eqp
\eea
%
Now expand the \var{\sura{1+\erg}} term to obtain
%
\bea
\vIz{i} & = & -m \myint{dt} \dby{t} \surp{\pvel{i}
\sura{1+\erg}} \var{\ptraj{i}} \nonumber \\
             &   & -m \myint{dt} \gL \surp{
             \pdxby{\sura{1+\erg}}{\ptraj{i}}  \var{\ptraj{i}} +
             \pdxby{\sura{1+\erg}}{\pdtraj{i}} \var{\pdtraj{i}} }
             \nonumber \\
             &   & -m \myint{dt} \frac{\gL}{2} \pvel{\gm}
             \pvel{\gn} \sura{ \surp{1+\erg}
             \pdxby{\imet{\gm}{\gn}}{x^i}} \var{\ptraj{i}} \eqc
\eea
%

Integrating the third term by parts and collecting yields
%
\bea
\vIz{i} & = & -m \myint{dt} \left\{ \dby{t} \surp{\pvel{i}
\sura{1+\erg} - \gL \pdxby{\sura{1+\erg}}{\pdtraj{i}} } \right.
\nonumber \\
        &   & \left. \gL \pdxby{\sura{1+\erg}}{\ptraj{i}}
        + \frac{\gL}{2} \pvel{\gm} \pvel{\gn} \sura{ \surp{1+\erg}
        \pdxby{\imet{\gm}{\gn}}{x^i}} \right\} \var{\ptraj{i}}
        \eqp
\eea
%
To proceed further, the terms involving
\pdxby{\sura{1+\erg}}{\ptraj{i}} and
\pdxby{\sura{1+\erg}}{\pdtraj{i}} must be expanded. Several
intermediate results are helpful in tackling the
\pdxby{\sura{1+\erg}}{\ptraj{i}} term first:
\bes
\pdxby{\gr}{\ptraj{i}} = \gr \surb{ \frac{1}{\sm{x}{z}}
\pdxby{\sm{x}{z}}{\ptraj{i}} + \frac{1}{2}
\pdxby{\simet{\ga\gb}}{\ptraj{i}} \pvel{\ga} \pvel{\gb} }
\ees
and
\bes
\pdxby{\gr}{x^i} = \gr \surb{ \frac{-1}{\sm{x}{z}}
\pdxby{\sm{x}{z}}{\ptraj{i}} + \frac{1}{2} \met{\gm}{\gn}
\pdxby{\imet{\gm}{\gn}}{x^i}} \eqp
\ees
Using these relations gives
%
\bea \pdxby{\sura{1+\erg}}{\ptraj{i}} & = & \myint{d^3x}
\pdxby{\erg}{\gr} \surc{\pdxby{\gr}{\ptraj{i}} +
\pdxby{\gr}{x^i}} \sm{x}{z} \nonumber \\
& = & \sura{\frac{P}{\gr}} \frac{1}{2}
\pdxby{\simet{\ga\gb}}{\ptraj{i}} + \frac{1}{2} \sura{
\frac{P}{\gr} \met{\gm}{\gn} \pdxby{\imet{\gm}{\gn}}{x^i}} \eea
%
Similarly, the term of \pdxby{\sura{1+\erg}}{\pdtraj{i}} is
tackled by using the intermediate relation
\bes
\pdxby{\gr}{\pdtraj{i}} = -\frac{\gr \pvel{i}}{\gL}
\ees
to obtain
\bes
\pdxby{\sura{1+\erg}}{\pdtraj{i}} =
-\frac{\pvel{i}}{\gL}\sura{\frac{P}{\gr}} \eqp
\ees
Finally, the term \sura{\surp{1+\erg}\pdxby{\imet{\gm}{\gn}}{x^i}}
can be simplified using the SPH rules used before to yield
\bes
\sura{\surp{1+\erg}\pdxby{\imet{\gm}{\gn}}{x^i}} = \sura{1+\erg}
\sura{\pdxby{\imet{\gm}{\gn}}{x^i}} = \sura{1+\erg}
\pdxby{\simet{\gm\gn}}{\ptraj{i}} \eqp
\ees
Putting all of these pieces together and setting the variation
equal to zero gives the SPH version of the Euler equations
%
\be\label{eq:SPHEuler}
 \frac{1}{\gL} \dby{t} \surb{ \sura{1+\erg+\frac{P}{\gr}}
\pvel{i} } + \frac{1}{2} \sura{1+\erg+\frac{P}{\gr}}
\pdxby{\simet{\gm\gn}}{\ptraj{i}} \pvel{\gm} \pvel{\gn} +
\frac{1}{2} \sura{\frac{P}{\gr} \met{\gm}{\gn}
\pdxby{\imet{\gm}{\gn}}{x^i}} = 0 \eqp
\ee
%
The covariant form of the Euler equations for an ideal fluid
($\idx{T}{\dn{\gm\gn}} = \gr \surp{1+\erg+\frac{P}{\gr}}
\idx{u}{\dn{\gm}} \idx{u}{\dn{\gn}} + P \met{\gm}{\gn}$) take the
form
\bes
\gr\surp{1+\erg+\frac{P}{\gr}} \pvel{\ga;\gn} \idx{u}{\up{\gn}} =
-\idx{P}{\dn{,\ga}} - \pvel{\ga} \idx{u}{\up{\gn}}
\idx{P}{\dn{,\gn}}
\ees
Making the identifications $\pvel{\ga,\gn} \idx{u}{\up{\gn}} =
\dxby{\pvel{\ga}}{\gt}$ and $\idx{P}{\dn{,\gn}} \idx{u}{\up{\gn}}
= \dxby{P}{\gt}$, yields the equivalent form
\bes
 \gr \surp{1 + \erg
+ \frac{P}{\gr}} \surp{\dxby{\pvel{i}}{\gt} -
\idx{\gG}{\up{\ga}\dn{\gn i}} \pvel{\ga} u^{\gn} } =
-\idx{P}{\dn{,i}} - \pvel{i} \dxby{P}{\gt} \eqc
\ees
which is more suitable for comparison.  Continuing along, the identities
$\idx{\gG}{\up{\gm}\dn{\ga\gb}} \pvel{\gm} \idx{u}{\up{\gb}} =
\frac{1}{2} \idx{u}{\up{\gs}} \idx{u}{\up{\gb}} \met{\gs}{\gb , \ga} =
-\frac{1}{2} \pvel{\gs} \pvel{\gb} \idx{\imet{\gs}{\gb}}{\dn{,\ga}}$ and
$\dby{\gt}\surp{1 + \erg + \frac{P}{\gr}} = \frac{1}{\gr} \dxby{P}{\gt}$ allow for
further simplification, yielding
\be\label{eq:Euler_final_form}
\dby{\gt} \surb{ \surp{1 + \erg + \frac{P}{\gr}} \pvel{i}}
+ \frac{1}{2} \surp{1 + \erg + \frac{P}{\gr}} \pdxby{\imet{\ga}{\gn}}{x^i}\pvel{\ga}\pvel{\gn}
+ \frac{1}{\gr} \idx{P}{\dn{,i}} = 0
\ee
as our final form.
Comparison between \sure{\ref{eq:Euler_final_form}} and \sure{\ref{eq:SPHEuler}} shows that
the functional forms of the two equations are very similar.
The predominant difference arises in the pressure gradient term.
In \sure{\ref{eq:Euler_final_form}} the spatial gradient of the pressure is directly
computed while in \sure{\ref{eq:SPHEuler}} the pressure gradient is inferred from the spatial
gradient of the metric.
This coupling between the pressure and the metric is also present in the expression for the
stress-energy tensor.

\section{Analysis of the Fat Particle Equations}

In this section, attention is turned to analyzing the Fat Particle equations for a
variety of elementary physical systems.  The first system is the case of a static,
spherically-symmetric star.

\subsection{Static Spherically Symmetric Star}

Our aim here is to demonstrate that the Fat Particle equations provide a
mathematically complete description of a static spherically-symmetric star
where a barotropic equation of state is assumed.  The Fat Particle will be
assumed to be at rest at the origin of the coordinate system.


\subsubsection{Metric Choice and the Einstein Field Tensor}

Because of the presence of the term $\sqrt{-g(t,\cv{x})}$ in the denominator of
\refq{eq:rho_def}, we desired a form of a static spherically symmetric which can be
expressed in coordinates that are free of singularities at the origin.
The metric chosen is the isotropic coordinates and is given by
\bes
  ds^2 = -e^{2\gF} dt^2 + e^{2\gm} \surb{dr^2 + r^2 d\gW^2}
\ees
or alternatively
\bes
  ds^2 = -e^{2\gF} dt^2 + e^{2\gm} \surb{dx^2 + dy^2 + dz^2} \eqp
\ees

Our strategy will be to use the spherical coordinates to compute the field equations
and then to transform to cartesian coordinates.  In this way we can exploit the
spherical symmetry for the majority of the computations and then only at the last
step, when defining the hydrodynamics, switch to cartesian coordinates to avoid the
singularity at the origin.

The non-zero connection coefficients are:
\bea
  \Cnx{t}{t r}       & = & \gF' \eqc \nonumber \\
  \nonumber \\
  \Cnx{r}{t t}       & = & e^{2 \surp{\gF - \gm} } \gF'      \nonumber \\
  \Cnx{r}{r r}       & = & \gm'                              \nonumber \\
  \Cnx{r}{\gq \gq}   & = & -r \surp{ r \gm' + 1 }            \nonumber \\
  \Cnx{r}{\gf \gf}   & = & -r \sin^2(\gq) \surp{ r \gm' + 1} \eqc \nonumber \\
  \nonumber \\
  \Cnx{\gq}{r \gq}   & = & \gm' + \frac{1}{r}                \nonumber \\
  \Cnx{\gq}{\gf \gf} & = & - \sin(\gq) \cos(\gq)             \eqc \nonumber \\
  \nonumber \\
  \Cnx{\gf}{r \gf}   & = &  \gm' + \frac{1}{r}               \nonumber \\
  \Cnx{\gf}{\gq \gf} & = &  \frac{\cos(\gq)}{\sin(\gq)}      \eqp
\eea

Using GRTensorII the Einstein field equations in $(t,r,\gq,\gf)$ coordinates is
\bes
  \idx{G}{\up{\gm\gn}} = diag \surp{ A, B, C/r^2, C/r^2 \sin(\gq)^2}
\ees
where the functions $A, B, \mbox{and}, C$ are
\bea
 A & = & - e^{-2\surp{\gF + \gm}} \surb{ \frac{4}{r} \gm' + 2 \gm '' + \surp{\gm'}^2 } \nonumber \\
 B & = & e^{-4\gm} \surb{2 \gF' \gm' + \frac{2}{r} \surp{\gm' + \gF'} + \surp{\gm'}^2 } \nonumber \\
 C & = & e^{-4\gm} \surb{ \gm'' + \gF'' + \surp{\gF'}^2 + \frac{1}{r} \surp{\gF' + \gm'}} \eqp
\eea

The next step is to convert the Einstein tensor from spherical to cartesian coordinates.
The transformation equations are given by
\bea
  x & = & r \sin(\gq) \cos(\gf) \\
  y & = & r \sin(\gq) \sin(\gf) \\
  z & = & r \cos(\gq)
\eea
and the corresponding Jacobean \Jac{\gta}{\gm} = \pdxby{(t,x,y,z)}{(t,r,\gq,\gf)}
\bes
\surp{
  \begin{array}{cccc}
    1 & 0                   &  0                     &  0                     \\
    0 & \cos(\gf) \sin(\gq) &  r \cos(\gf) \cos(\gq) & -r \sin(\gf) \sin(\gq) \\
    0 & \sin(\gf) \sin(\gq) &  r \sin(\gf) \cos(\gq) &  r \cos(\gf) \sin(\gq) \\
    0 & \cos(\gq)           & -r \sin(\gq)           &  0
  \end{array}
} \eqp
\ees
The conversion of the Einstein tensor to cartesian coordinates given by
\bes
   \idx{G}{\up{\gta\gtb}} = \Jac{\gta}{\gm} \idx{G}{\up{\gm\gn}} \jac{\gn}{\gtb}
\ees
results in the following system of equations
\bea
  G^{xx}          & = & (B-C) \sin(\gq)^2 \cos(\gf)^2 + C     \eqc \nonumber \\
  G^{xy} = G^{yx} & = & (B-C) \sin(\gq)^2 \sin(\gf) \cos(\gf) \eqc \nonumber \\
  G^{xz} = G^{zx} & = & (B-C) \sin(\gq) \cos(\gq) \cos(\gf)   \eqc \nonumber \\
  G^{yy}          & = & (B-C) \sin(\gq)^2 \sin(\gf)^2 + C     \eqc \nonumber \\
  G^{yz} = G^{zy} & = & (B-C) \sin(\gq) \cos(\gq) \sin(\gf)   \eqc \nonumber \\
  G^{zz}          & = & -(B-C) \sin(\gq)^2 + B \eqp
\eea
The expressions $r$, \gq, and \gf should now be regarded as functions of the coordinates
$x$, $y$, and $z$ given by inverting the transformation from spherical to cartesian
coordinates.  In particular, $r$ should be thought of as a representation for
the function $\sqrt{x^2 + y^2 + z^2}$.
Finally, the square-root of the determinant of the metric, in cartesian coordinates
is given by
\bes
  \sqrt{-g} = e^{\gF + 3 \gm} \eqp
\ees

\subsubsection{Stress-Energy Tensor}

In this section, we construct the specific form of the stress-energy tensor \idx{T}{\up{\gm\gn}}
based on \refq{eq:def_T} and the cartesian form of the metric.

The requirement that the Fat Particle remains at rest at the origin implies that
the coordinate velocity take the form
\bes
  \idx{\dt{z}}{\up{\gm}} = \surp{1,0,0,0} \eqp
\ees
Since the metric is diagonal, the covariant 4-velocity must be given by
\bes
  \idx{u}{\dn{\gm}} = \surp{u_t,0,0,0} \eqp
\ees
The value of $u_t$ is obtained from the normalization condition
\bes
  \idx{\surp{u_t}}{\up{2}} \sura{g^{tt}} = -1
\ees
and the definition of the smoothed metric which
yields
\bes
  u_t = 1/\sqrt{ - \sura{g^{tt}}}
\ees
with \sura{\idx{g}{\up{tt}}} given by
\bes
  \sura{g^{tt}} = -4 \gp \myint{dr} r^2 e^{-2\gF} W(r) \eqp
\ees
The factor \gL is expressed by
\bes
  \gL = \sqrt{ - \sura{g_{\gm\gn}} \dt{z}^{\gm} \dt{z}^{\gn} } = 1 / \sqrt{-\sura{g^{tt}}} \eqp
\ees
Thus the density, defined in \refq{eq:rho_def} is expressed as
\bes
  \gr = m \frac{e^{-\gF-3\gm}}{\sqrt{-\sura{g^{tt}}}} W(r)
\ees
where it should be emphasized that \gF and \gm are still unknown functions.
Now that the rest-mass energy density has been obtained, the pressure and the
internal energy must be expressed.  We will consider only barotropic equations of
state in which the pressure and internal energy are solely functions of \gr.
In particular, we will assume that the mass-energy density is
expressed as a power-law
\bes
  \gr \surp{1+e} = K \gr^{\ggm} \eqp
\ees
in terms of the rest-mass density \gr.
The usual thermodynamic identity $de = P/\gr^2 d\gr$ allow the pressure to be expressed as
\bea
  P & \equiv & \gr \dby{\gr} \surb{ \gr \surp{1+e}} - \gr \surp{1+e} \nonumber \\
    & = & \surp{\ggm - 1} \gr \surp{1 + e}  = K \surp{\ggm - 1} \gr^{\ggm} \eqp
\eea
The constant $K$ is an arbitrary real positive constant.  The speed of sound, which is also
of interest, is given by
\be
  c^2 \equiv \frac{\gr}{\gr\surp{1+e} + P}  \surp{\pdxby{P}{\gr}}_{s}  = \ggm - 1 \eqp
\ee
To have a speed of sound less than the speed of light \ggm could not exceed $2$
and to avoid negative pressure \ggm should not be smaller than $1$.

The stress energy tensor has the form
\beas
  T^{tt} & = & \surb{ \gr(1+e) + W(r) e^{-(\gF+3\gm)} \myint{d^3 x} P e^{\gF+3\gm}}
               \surp{u_{t}}^2 \sura{g^{tt}}^2 - P e^{-2\gF} \nonumber \\
         & = & -\surb{ \gr(1+e) + 4 \gp W(r) e^{-(\gF+3\gm)} \myint{d r} r^2 P e^{\gF+3\gm}}
                \sura{g^{tt}} - P e^{-2\gF}
\eeas
and
\bes
  T^{xx} = T^{yy} = T^{zz}  \equiv T^{CC} = P e^{-2\gm}
\ees
where, as mentioned above, the coordinate $r$ is now regarded as a function of the
cartesian coordinates $x, y, \mbox{and } z$ (\textit{i.e.} $r= \sqrt{x^2 +y^2 + z^2}$).

\subsubsection{Einstein Equations}

The Einstein field equations can now be written in matrix form as:
\bes
  \surp{
   \begin{array}{cccc}
   G^{tt} &   0    & 0      & 0      \\
   0      & G^{xx} & G^{xy} & G^{xz} \\
   0      & G^{yx} & G^{yy} & G^{yz} \\
   0      & G^{zx} & G^{zy} & G^{zz}
   \end{array}
  }
=
  \surp{
   \begin{array}{cccc}
   T^{tt} &   0    & 0      & 0      \\
   0      & T^{CC} & 0      & 0      \\
   0      & 0      & T^{CC} & 0      \\
   0      & 0      & 0      & T^{CC}
   \end{array}
  }  \eqp
\ees
There are two possible solutions for this system of equations.
In the first, we take the above equation at face value an assume that
the only solution possible is if $B = C$.
The resulting equations are
\bea
  A & = & 8 \gp T^{tt} \\
  B & = & 8 \gp T^{CC} \eqc
\eea
with the equation for $G^{\gq\gq}/r^2$ being relegated to a constraint.
The presence of the unknown functions \gF and \gm on both sides
of the equations means implies that Einstein equations are now integro-differential
equations.  We envision solving these equations by using an iterative technique in
which the form of the spacetime will be guessed and substituted into the
right-hand side and treated as a fixed source for the derivative terms on the
left-hand side.  The net effect is to transform the set of integro-differential
equations into a set of differential equations with fixed sources.  The cost
associated with this simplification is that the process must be iterated until
a final form of \gF and \gm is reached.

To begin the recasting of these equations, we define the state of our
system as
\be
 \cs{S} = \trps{\surp{\gF,\gm, \gm'}} \eqp
\ee
Transforming Einstein's equations into a set of first order differential equations
for this state involves isolating $\gF'$ and $\gm''$ on the right-hand side.
It will be convenient to take the terms involving the exponential functions
$\exp(\gF)$ and $\exp(\gm)$ also to the right-hand side making them effectively
additional contributions to the source.
Thus the equations can now be written as
\bea
  \surb{ \frac{4}{r} \gm' + 2 \gm'' + \surp{\gm'}^2} & = &
    -8 \gp e^{2\surp{\gF+\gm}} T^{tt} \nonumber \\
  \surb{2 \gF' \gm' + \frac{2}{r} \surp{\gm' + \gF'} + \surp{\gm'}^2 } & = &
    8 \gp e^{4 \gm} T^{CC} \eqp
\eea
Solving the first equation for $\gm''$ yields
\be
  \gm'' = -4 \gp T^{tt} e^{2\surp{\gF+\gm}} - \gm' \surb{ \frac{2}{r} + \frac{1}{2} \gm'} \eqp
\ee
Solving the second equation for $\gF'$ yields
\be
  \gF' = \frac{ 8 \gp e^{4\gm} T^{CC} - \gm' \surp{\gm' + \frac{2}{r}}}
              {2 \surb{\gm' + \frac{1}{r}}} \eqp
\ee
Evolution of the state of the system is obtained by numerically solving the equation
\be
\dby{r} \cs{S} = \dby{r} \surp{\begin{array}{c} \gF \\ \gm \\ \gm'\end{array}}
               =         \surp{\begin{array}{c}
                              f(\gF,\gm,\gm') \\ \gm' \\ g(\gF,\gm,\gm') \end{array}} \eqp
\ee
where
\be
  f(\gF,\gm,\gm') = \frac{ 8 \gp e^{4\gm} T^{CC} - \gm' \surp{\gm' + \frac{2}{r}}}
              {2 \surb{\gm' + \frac{1}{r}}}
\ee
and
\be
  g(\gF,\gm,\gm') = -4 \gp T^{tt} e^{2\surp{\gF+\gm}} - \gm' \surb{ \frac{2}{r} + \frac{1}{2} \gm'} \eqp
\ee
To see how this arises, consider the Bianchi identity \idx{G}{\up{\gm\gn}\dn{;\gn}} = 0.
Expanding the terms and using the non-zero connection coefficients allows us to express
$G^{\gq\gq}$ as
\be
  \idx{G}{\up{\gq\gq}} = -\frac{ \surb{ \idx{G}{\up{rr}\dn{,r}} + G^{rr} \surp{2 \Cnx{r}{rr} +
                                \Cnx{\gq}{r\gq} + \Cnx{\gf}{r \gf} } + \Cnx{r}{tt} G^{tt}} }
                               {\surb{\Cnx{r}{\gq\gq} + \sin^{-2}(\gq) \Cnx{r}{\gf\gf}}} \eqp
\ee
In the event that the stress-energy tensor is also divergenceless
(i.e. \idx{T}{\up{\gm\gn}\dn{;\gn}} = 0), then the $G^{\gq\gq}$ equation
is guaranteed to be satisfied if the other two equations are satisfied.
The other possibility, is that $B \neq C$ and that the stress-energy
tensor that we are proposing fails to capture all the stresses needed to keep
the Fat Particle structure.  We will assume for now that the first viewpoint is
the correct one and will revisit this point after some numerical simulations
have been constructed.  In any event, since the Fat Particles are intended to
be enable numerical simulations, the presence of 'unmodelled' stresses are not
unreasonable and may simply be the cost we must pay to have such an object
present.

\subsubsection{Euler's Equation}

Finally we have to consider Euler's equation for the motion of the Fat Particle.
The requirement is that if an isolated Fat Particle starts at rest it will
\bes
 \sura{F}_{\mO} = \int_{0}^{2\gp} d\gf \int_0^{\gp} d\gq \sin(\gq) \int_0^R dr \nudge
                   r^2 F(r) W(r)
\ees

\bes
  \sura{\pdxby{F}{x}}_{\mO} = \int_{0}^{2\gp} d\gf \int_0^{\gp}
  d\gq \sin(\gq) \int_0^R dr \nudge r^2 \pdxby{F(r)}{r} \sin(\gq) \cos(\gf) W(r) = 0
\ees

Similar results obtain for \sura{\pdxby{F}{y}} and \sura{\pdxby{F}{z}}.  Thus the Euler
equation is identically zero.
\end{document}
