\documentclass[12pt]{report}
%%%%%%%%%%%%%%%%%%%%%%%%%%%%%%%%%%%%%%%%
%------------------------------------------
% PACKAGES
%------------------------------------------
\usepackage[none,bottom]{draftcopy}
\usepackage{epic,eepic,graphicx,url}
% THIS FOR PDF DOCUMENT GENERATION
%%%% Begin PDF stuff
\usepackage[backref]{hyperref} % USE TO NAVIGATE THROUGH PDF FILE
\special{ pdf: docinfo << /Author(Laurie M. Mailhe)
 /Title( Leonardo-BRDF : Preliminary Primer Vector Analysis Tool - Final
Report)
/Subject(Optimization)
/Keywords(Primer Vector)
>>}
%%%% End PDF stuff
\special{papersize=8.5in,11in}  % DVI SPECIAL
%------------------------------------------
% FOR FLOATS
%------------------------------------------
\renewcommand{\topfraction}{0.4}
\renewcommand{\bottomfraction}{0.5}
\renewcommand{\textfraction}{0.3}
\renewcommand{\intextsep}{2.0mm}
%------------------------------------------
% VERTICAL PARAMETERS
%------------------------------------------
\setlength{\topmargin}{0in} \setlength{\headheight}{0in}
\setlength{\headsep}{0in}
%------------------------------------------
% MARGINS AND TEXT LENGTH
%------------------------------------------
\setlength{\evensidemargin}{0.0truein}%!!!
\setlength{\oddsidemargin}{0.0truein}
\setlength{\textheight}{9.00truein}
\setlength{\textwidth}{6.5truein}
%------------------------------------------
% SOME NEW COMMANDS
%------------------------------------------
\newcommand{\FF}{FreeFlyer$^{\mbox{\tiny TM}}~$}
\newcommand{\MT}{MATLAB}

\setcounter{secnumdepth}{0}

%%%%%%%%%%%%%%%%%%%%%%%%%%%%%%%%%%%%%%%%
\begin{document}
%%%%%%%%%%%%%%%%%%%%%%%%%%%%%%%%%%%%%%%%
\title{ Leonardo-BRDF \\ A Preliminary Primer Vector Analysis Tool \\ Final Report}
\author{Laurie M. Mailhe }
%\hspace{-5.0cm} a.i. solutions, Inc.
%\begin{picture}(13,9)(20,0)
%      \put(-110.0,-100.00){\mbox{ \includegraphics[width=2.0in,clip]{D:/ailogo.eps} }}
%\end{picture}
\maketitle
\tableofcontents
\newpage
\chapter{INTRODUCTION}
This report summarizes a preliminary study in developing a set of
optimization tools for orbital Rendezvous.  While this work was
performed under the Leonardo-BRDF task, it takes part on a larger
scale effort undergoing at GSFC, FDAB to build generic tools,
which will unable mission with tight fuel budget such as
Leonardo-BRDF~\cite{Hugh00}.  The work presented in this report
focuses on trajectory optimization using Lawden Primer Vector
theory \cite{Lawden63}. This optimization scheme based on calculus
of variation is also referred as indirect optimization method as
opposed to genetic algorithms or simulated annealing which are
labeled direct methods.  We chose to first assess Primer Vector,
to get a better understanding of optimal control theory.  Our
ultimate goal is to determine the pros and cons of the main
existing techniques as well as their application domaine
\cite{Jezew91}.  We hope this 'state-of-the-art' survey to help
determining the appropriate optimization technique for a given
mission. The report is divided in two main sections. In
chapter~\ref{chp:pvt},\,we introduce the basics of Primer Vector
theory which were implemented in the \MT \,\,code. In
chapter~\ref{chp:pvat}, we present the $\beta$-version of the
Primer Vector Analysis Tool (PVAT) written in \MT.  First, we
detail the algorithms and the functions composing the current
version of the code. Then, we discuss some test cases and
preliminary results obtained using PVAT. Finally, we present our
conclusions and
our directions for future work.\\

\chapter{PRIMER VECTOR THEORY}\label{chp:pvt}

In this chapter, we recall the main Lawden primer vector rules of
optimality used in PVAT. A more detailed derivation of the primer
vector theory can be found in ref. \cite{Schiff01}.

\section{Minimum Fuel Problem}
First, we need to formulate our problem. We wish to minimize the
mission $\Delta$V expense while still satisfying the mission
constraints. Assuming a N-impulse trajectory, our problem
translates to:
\begin{equation}
\mbox{Find  Min$_{_{t \epsilon [t_{o},t_{f}]}}$ J}= \sum_{i=1}^{N}
\Delta V_{i} \label{eq:mainprob}
\end{equation}
Subject to:
\begin{eqnarray*}
\dot{x}(t)       & = & f(\dot{x},x,t)  \\
x(t_{o})      & = &  x_{o} \\
\dot{x}(t_{o})& = & \dot{x}_{o} \\
x(t_{f})      & = &  x_{f} \\
\dot{x}(t_{f})& = & \dot{x}_{f} \\
\end{eqnarray*}
\noindent Where $f(\dot{x},x,t)$ represents the spacecraft
two-body equations of motion. The problem now defined, we can
express the primer vector equations using calculus of variation
theory. We label the initial trajectory "reference trajectory".

\section{Primer Vector Equations for the Two-Body Problem}

The primer vector is defined as the adjoint to the velocity vector
in the variational Hamiltonian formulation
\cite{Lawden63,Schiff01,Hiday92}.  It obeys the following equation
also known as the second order canonical form of the Euler-Lagrange equation:\\
\begin{equation}
\ddot{\bar{p}} = \mu\,(-\frac{\bar{p}}{r^{3}} + 3
\frac{\bar{r}}{r^{5}}(\sum_{i=1}^{3} p_{i}\,r_{i}))
\label{eq:pveom}
\end{equation}
\noindent where $\bar r$ is the satellite position vector on the
reference trajectory, $\bar p$ is the primer vector and $\mu$ is
the Earth gravitational constant.\\
The satellite position is found using the well-known second-order
form equation of motion:
\begin{equation}
\ddot{\bar{r}} = -\mu\,\frac{\bar{r}}{r^{3}}
\end{equation}
As shown in Eq.~\ref{eq:pveom}, the primer vector state can not be
derived simultaneously with the spacecraft state. Its history must
be post-processed from the spacecraft ephemeris.\\
\noindent The primer vector state $(\bar p, \dot{\bar p})$ can
also be obtained from an initial state $(\bar p_{o}, \dot{\bar
p_{o}})$ using $\underline{\phi}(t,t_{o})$, the satellite
trajectory state transition matrix (STM):\\

\begin{equation}
    \left( \begin{array}{c} \bar p \\ \dot{\bar p}
    \end{array} \right) = \underline{\phi}(t,t_{o})\cdot
    \left( \begin{array}{c} \bar p_{o} \\ \dot{\bar p_{o}}
    \end{array} \right) =
    \left[ \begin{array}{cc} M & N \\ S & T \\ \end{array} \right] \cdot
    \left( \begin{array}{c} \bar p_{o} \\ \dot{\bar p_{o}}
    \end{array} \right)
    \label{eq:stm}
\end{equation}

\noindent Where M, N, S and T are 3x3 matrices, partitions of
$\underline{\phi}(t,t_{o})$. In this report, we use the STM
expression developed by Der~\cite{Der97},which is valid for
arbitrary conic.\\

\section{Lawden Necessary Conditions for an Optimal Trajectory}

Based on primer vector defined by Eq.~\ref{eq:pveom}, Lawden
derived the 4 necessary conditions for an optimal trajectory:
\begin{enumerate}
\item The primer vector must be $\mathcal{C}^{1}$ for the
entire history.
\item The magnitude of the primer must be less than one during a coasting phase and equal
to one when an impulse is performed.
\item At the impulse time, the primer is a unit vector equal to the thrust direction.
\item The derivative of the primer vector magnitude must be zero for all interior impulses
( i.e. not the initial or final impulse).
\end{enumerate}

\section{Computing the Primer Vector History}
\noindent Solving for the primer vector history is equivalent to
solving a two-point boundary value problem. Lets first assume that
$N=2$ in Eq.~\ref{eq:mainprob}. We know from Lawden that, at an
impulse, the primer vector equals the thrust direction. Thus, we
can express the initial and final primer vector as :
\begin{equation}
\bar p_{o} = \frac{\Delta\bar V_{o}}{|\Delta\bar V_{o}|}
\end{equation}
\begin{equation}
\bar p_{f} = \frac{\Delta\bar V_{f}}{|\Delta\bar V_{f}|}
\end{equation}\\

\noindent The initial and final primer vectors and the
time-of-flight $(t_{f}-t_{o})$ are given to the problem. However,
to propagate Eq.~\ref{eq:pveom} we need the complete primer
initial state $(\bar p_{o}, \dot{\bar p_{o}})$. To obtain
$\dot{\bar p_{o}}$, we can either use a shooting method or use the
STM formulation from Eq.~\ref{eq:stm}. In general, using the STM
is faster and sufficiently accurate. However, it is worthwhile
noting that for coplanar, 180$^{\circ}$ transfer, the STM breaks down~\cite{PrusCon}.\\
Using Eq.~\ref{eq:stm}, the initial derivative of the primer
vector can be expressed as:

\begin{equation}
\dot{\bar p_{o}} = N^{-1} \cdot (\bar p_{f} - M \cdot \bar p_{o})
\label{eq:pdot}
\end{equation}

\noindent Eq.~\ref{eq:pdot} can be generalized for a N-impulse
trajectory by examining each individual 2-burn transfer. Lets
consider the transfer between impulse(i) and impulse(i+1) where i
varies from 1 to N. For this leg, the primer vector derivatives
are:

\begin{equation}
\dot{\bar p_{i}}^{+} = N_{i}^{-1} \cdot (\bar p_{i+1} - M_{i}
\cdot \bar p_{i}) \label{eq:pdot2}
\end{equation}

\begin{equation}
\dot{\bar p}_{i+1}^{-} = S_{i} \cdot \bar p_{i} + T_{i} \cdot
\dot{\bar p_{i}}^{+} \label{eq:pdot3}
\end{equation}

\noindent Where:
\begin{eqnarray}
M_{i}& = & M(t_{i+1},t_{i}) \\
N_{i}& = & N(t_{i+1},t_{i})\nonumber\\
S_{i}& = & S(t_{i+1},t_{i})\nonumber\\
T_{i}& = & T(t_{i+1},t_{i}) \nonumber
\end{eqnarray}


\noindent Once the primer vector history is computed, we can
determine the optimality of the trajectory using the 4 Lawden
necessary conditions. For a non-optimal primer vector history, two
type of actions can be inferred:
\begin{enumerate}
\item Moving the initial or final impulse
\item Adding or/and moving an interior impulse
\end{enumerate}

\section{Moving an Initial or a Final Impulse}
An improvement in total $\Delta$V can be achieved, if the initial
or final primer vector magnitude velocity $\frac{d|\bar{p}|}{dt}$
is different from zero~\cite{Schiff01}. We assume that the
departure from the initial orbit and the arrival to the final
orbit can be varied. However, additional constraint to the problem
might require the boundary burns to be fixed or to be within a
certain time interval. If so, there might be no or very little way
of improving the $\Delta$V budget using the initial and final
impulses.  In general, there are four different cases:
\begin{itemize}
  \item The slope of the initial primer vector magnitude is positive : $\frac{d|\bar{p}|}{dt}|_{t_{o}}>0$ \\
  \noindent Right after the first impulse, the magnitude of the
  primer vector is over one, which violates the Lawden law
  no. 2. In this case, the trajectory can be improved by coasting in
  the initial orbit before performing the departure maneuver.

  \item The slope of the initial primer vector magnitude is negative : $\frac{d|\bar{p}|}{dt}|_{t_{o}}<0$ \\
  \noindent Right after the first impulse, the magnitude of the
  primer vector is below one, which does not violate any optimality law.
  However, this is an indication of a late departure and the total fuel expense
  can decrease by burning earlier in the initial orbit
  \cite{Hiday92}. In this case the primer vector slope is in agreement with the Lawden rules.
  Therefore, the initial burn is moved only if the reference trajectory has no interior
  impulse and we are not concerned with any primer vector slope
  discontinuity at the midcourse impulse.

  \item The slope of the final primer vector magnitude is negative : $\frac{d|\bar{p}|}{dt}|_{t_{f}}<0$ \\
  \noindent Right before the last impulse, the magnitude of the
  primer vector is over one, which violates the Lawden law
  no. 2. In this case, the trajectory can be improved by adding  a coast in
  the final orbit (i.e. performing the arrival maneuver earlier in the final
  orbit and then coasting to the final epoch).

  \item The slope of the final primer vector magnitude is positive : $\frac{d|\bar{p}|}{dt}|_{t_{f}}>0$ \\
  \noindent Right before the last impulse, the magnitude of the
  primer vector is below one, which does not violate any optimality law.
  However, this is an indication of a early arrival and the total fuel expense
  can decrease by burning later in the initial orbit
  \cite{Hiday92}. Again, this scenario is considered only if the
  reference trajectory has no interior impulse.
\end{itemize}

\noindent All the derivations which correspond to the statement
listed above can be found in ref. \cite{Schiff01}.

\section{Moving or Adding an Interior Impulse}
As stated above, if the trajectory departure and arrival states
are optimal, we then examine the primer vector at interior
impulses \,(if any) and during the coasting phases. There are two
major scenarios:
\begin{itemize}
\item There exist midcourse impulses at which the primer slope is
discontinuous. Then, we can decrease the cost function by moving
the impulse position and time so that Lawden law no.1 is
satisfied.\\
\item Lawden rule no.1 is not violated but the magnitude of the
primer vector goes above 1 during a coast phase. Then, we can
decrease the cost function by adding an impulse.\\
\end{itemize}

\subsection{Moving an Impulse}
If the primer vector derivative is not continuous at an interior
impulse, then the cost function J (i.e. total trajectory
$\Delta$V) defined in Eq.~\ref{eq:mainprob} can be decreased by
moving the midcourse position and time ($t_{m}$,$\bar R_{m}$). The
midcourse state is varied using the following first-order gradient
of J:

\begin{equation}
\bar \nabla J = \left[ \begin{array}{c}
 \dot{\bar p}_{m}^{+} - \dot{\bar p}_{m}^{-} \\
 -(H_{m}^{+} - H_{m}^{-}) \\
\end{array} \right]
\label{eq:grad1}
\end{equation}

\noindent Where H is the variational Hamiltonian corresponding to
the problem stated in Eq.~\ref{eq:mainprob}. At the interior
impulse, the Hamiltonian is expressed as:
\begin{equation}
H = \dot{\bar p^{T}}\cdot\bar v - {\bar p}^{T}\cdot\bar g
\end{equation}
\noindent Since $\bar g$ is continuous at ($t_{m}$,$R_{m}$),
Eq.~\ref{eq:grad1} becomes:

\begin{equation}
\bar \nabla J = \left[ \begin{array}{c}
 \dot{\bar p}_{m}^{+} - \dot{\bar p}_{m}^{-} \\
 -(\tilde{H}_{m}^{+} - \tilde{H}_{m}^{-}) \\
\end{array} \right]
\label{eq:grad2}
\end{equation}

\noindent Where
\begin{equation}
\tilde{H}_{m}^{+} = \dot{\bar p_{m}^{+}}^{T}\cdot{\bar v_{m}}^{+}
\end{equation}
\begin{equation}
\tilde{H}_{m}^{-} = \dot{\bar p_{m}^{-}}^{T}\cdot{\bar v_{m}}^{-}
\end{equation}

\noindent The gradient of J in Eq.~\ref{eq:grad2} is used in
conjunction with a Broyden-Fletcher-Goldberg-Shanno minimization
technique which will be detailed in chapter~\ref{chp:pvat}. At the
local minimum, $\nabla J$ equals 0, thus satisfying Lawden rule
no.1. Finally, when all interior impulse $\dot{\bar p}$ are
continuous and the boundary impulses are optimal, we must check
that law no.2 holds for the coasting phases.

\subsection{Adding an Impulse}
When the primer vector magnitude is over 1 during a coasting phase
on a given leg , an improvement in the total $\Delta$V is obtained
by adding an impulse. To get the most efficient decrease in cost,
the impulse should be added at the time, $t_{max}$, where $|\bar
p|$ is maximum, $\bar p_{max}$.

\noindent We know from Lawden rules that the impulse direction
should equal the primer vector:
\begin{equation}
\Delta \bar V_{m} = c \cdot \frac{\bar p_{max}}{|\bar
p_{max}|}\label{eq:dv}
\end{equation}

\noindent Using a second-order approximation, a first-guess for
the magnitude $c$ is derived:

\begin{equation}
c = \frac{\beta \cdot \frac{ \Delta V_{f}}{|\Delta V_{f}|} -
\alpha \cdot \frac{\Delta V_{o}}{|\Delta V_{o}|} - 1}{\frac{\alpha
\cdot \alpha - \frac{(\alpha \cdot \Delta V_{o})^{2}}{|\Delta
V_{o}|^{2}}}{|\Delta V_{o}|}+ \frac{\beta \cdot \beta -
\frac{(\beta \cdot \Delta V_{f})^{2}}{|\Delta V_{f}|^{2}}}{|\Delta
V_{f}|}}\label{eq:c}
\end{equation}

\noindent Where $\alpha$ and $\beta$ are:
\begin{equation}
\alpha = N^{-1}(t_{max},t_{o}) \cdot \underline{A}^{-1} \cdot
\frac{\bar p_{max}}{|\bar p_{max}|}
\end{equation}

\begin{equation}
\beta = N^{-1}(t_{max},t_{f}) \cdot \underline{A}^{-1} \cdot
\frac{\bar p_{max}}{|\bar p_{max}|}
\end{equation}


\noindent $\Delta V_{o}$ and $\Delta V_{f}$ are the initial 2
impulses on the selected leg of the reference trajectory before
any burn is added. The matrix A is defined as:

\begin{equation}
\underline{A} = T(t_{max},t_{f}) \cdot N^{-1}(t_{max},t_{f}) -
T(t_{max},t_{o}) \cdot N^{-1}(t_{max},t_{o}) \label{eq:a}
\end{equation}

\noindent The matrices T and N are partitions of the STM defined
in Eq.~\ref{eq:stm}. Note that Eq.~\ref{eq:c} is an approximation
of the solution to the following equation:


\begin{eqnarray}
\frac{d}{dc}[\sqrt{\bar \Delta V_{o}^{T} \cdot \bar \Delta
V_{o}^{T} + 2c(\bar\alpha^{T}\cdot \bar \Delta V_{o}) + c^{2}
(\bar\alpha^{T}\cdot\bar\alpha)}\\
+\sqrt{\bar \Delta V_{f}^{T} \cdot \bar \Delta V_{f}^{T} +
2c(\bar\beta^{T}\cdot \bar \Delta V_{f})+c^{2}
(\bar\beta^{T}\cdot\bar\beta)}& +c ]=0 \nonumber \label{eq:ceq}
\end{eqnarray}

\noindent Using Eqs.~\ref{eq:dv}, \ref{eq:c}, \ref{eq:a}, we can
express the perturbed position $\bar R_{m}$:

\begin{equation}
\bar R_{m} = \bar R_{ref}(t_{max}) + \underline{A}^{-1} \cdot
\Delta \bar V_{m} \label{eq:Rpert}
\end{equation}

\noindent Finally, using a Lambert solver, we compute the
trajectory data, $\underline{R}$, $\underline{V}_{p}$,
$\underline{V}_{m}$, and $\overline{TOF}$, for the two new
transfer leg.



\chapter{PRIMER VECTOR APPLICATION : Primer Vector Analysis Tool
(PVAT)}\label{chp:pvat} This chapter is intended as an
introduction to a $\beta$-version of the Primer Vector Analysis
Tool written in \MT. First, we detail the major functions and
algorithm of PVAT. Then, we present some test cases and
preliminary results obtained using PVAT.

\section{Algorithm and Functions}
\subsection{Overall Algorithm}
Figure~\ref{main} shows the overall primer vector analysis
technique. Since th is algorithm is \underline{not} implemented
yet, we manually call the functions. First, we need to provide a
reference trajectory ,which initially satisfy the mission
constraint, for PVAT to analyze . This reference trajectory
summary file can be created manually, using FreeFlyer or any other
orbit analysis tool. By calling the subroutine PVAT, the
trajectory data is loaded and stored in the proper variables.
Then, we call ComputePrimer to obtain the major primer vector data
needed to evaluate whether the primer vector is optimal and if
not, how it can be improved. This subroutine needs to be called
every time the trajectory is changed as it will also change the
primer vector history. Once the primer vector data is generated,
we have to perform a serie of checks. First, we check the initial
primer magnitude slope, $\dot{|p|}_{o}$. If the initial slope is
positive for $N>2$ or non-zero for $N=2$ then, the total $\Delta
V$ will be decreased by calling MoveInitialBurn. The final primer
magnitude slope,$\dot{|p|}_{f}$ is also checked and this process
is repeated until both the initial \underline{and} final slopes
meet their stopping condition. Note that if both the initial and
final primer vector magnitude slopes violates Lawden laws, there
is another option : adding an impulse at the maximum primer
magnitude by calling AddImpulseN. This option is not shown in the
schematic in Figure~\ref{main} as we opted for a sequential check
of the primer vector criteria. Once the boundary impulses are
optimal according to Lawden laws, we need to check the interior
impulses (if any) and the primer vector magnitude during the
transfer. If $N>2$, we compute the gradient of J according to
Eq.~\ref{eq:grad2}. If the gradient is different from null, we
call MoveImpulseN. This function moves the interior impulses
position and time using the BFGS minimization
technique~\cite{Rao96}. If $N=2$ or the gradient of J is zero, we
check the maximum primer vector magnitude. If it is above one,
then adding an impulse by calling AddImpulseN will decrease the
total trajectory cost J. Each time, the trajectory is changed, we
need to recompute the primer vector data and re-iterate the entire
checking process until no improvement can be made according to
Lawden rules.

\begin{figure}
\centering
\includegraphics[scale=0.55]{MainAlg2.eps}
\caption{\textbf{PVAT} Overall Flowchart} \label{main}
\end{figure}


\subsection{Creating/Loading the Reference Trajectory : PVAT}

If the option of loading a file is chosen, it must have the
following format where each line represents a burn:
\begin{tabbing}
Co\= Column222 \= Colu 333  \= Colu 444  \= Colu 55  \= Colu 6 \=
Colu 7 \= Colu 8  \= Colu 9  \= Colu 10  \= Colu 11  \kill \>
\textsc{$\;\;$Epoch} \> \textsc{$\;\;\;$ X} \>
\textsc{$\;\;\;\;$Y} \> \textsc{$\;\;\;$Z} \> \textsc{$VX^{-}$} \>
\textsc{$VY^{-}$} \> \textsc{$VZ^{-}$} \> \textsc{$VX^{+}$} \>
\textsc{$VY^{+}$}
\> \textsc{$VZ^{+}$} \\
 \> \textit{21544.98} \> \textit{1986.09} \>
\textit{-6497.48} \> \textit{27.97} \> \textit{7.32} \>
\textit{2.23} \> \textit{0.19} \> \textit{7.31} \> \textit{2.17}
\> \textit{0.47} \\
\> \textit{21545.00} \> \textit{6769.33} \> \textit{619.08} \>
\textit{422.84} \> \textit{-0.729} \> \textit{7.58} \>
\textit{0.063} \> \textit{-0.70} \> \textit{7.61}
\> \textit{0.075}\\
 \> ... \> ... \> ... \> ... \> ... \> ... \> ... \> ... \> ...
\> etc ...\\
\end{tabbing}
\noindent The default name of the file described above is
"Summary.txt". However, the user can also load other filename with
the format described above, choose to enter manually the data or
load a Matlab workspace file "*.mat". Based on the Epoch
information in units of days, the time-of-flight is generated for
each transfer leg and stored in a (N-1) vector, $TOF$, converted
in seconds. The cartesian position,$\;\underline{R} = [ X \; Y \;Z
]$ and the velocities before an impulse,$\underline{V}m = [ VX^{-}
\; VY^{-} \; VZ^{-}]$, and after an impulse,$\; \underline{V}p= [
VX^{+} \; VY^{+} \; VZ^{+}]$, are stored in a (3xN) matrix. The
rows correspond to the 3 cartesian coordinates and the columns
correspond to the N impulses. Once the initial trajectory is
loaded, we can compute the primer vector data using
"ComputePrimer".



\subsection{Computing the Primer Vector Data : ComputePrimer}
\begin{description}
\item[\underline{syntax}]$[ \dot{|p|}_{o}, \dot{|p|}_{f}, |p|_{max}, t_{max}, \bar p_{max}, filename, leg\underline{\;\;}number, \Delta \dot{\bar p}_{m}, \Delta \tilde{H} ]$
\item[]$\; \; \; \; \; \; \; \; \; \; \; \;\;\;\;\;\;\;\;\;\;\;\;\; = \textbf{ComputePrimer}(R, Vp, Vm, TOF,plot\underline{\;\;}output)$
\end{description}

\noindent This subroutine computes the major primer vector data
needed to evaluate whether the primer vector is optimal and if
not, how it can be improved. $\textbf{ComputePrimer}$ requires 5
inputs. The first 4 inputs are the trajectory time-of flight
vector,$TOF$,the position matrix,$\;\underline{R}$, the velocity
matrices before and after the impulses,$\;\underline{V}m$ and
$\;\underline{V}p$. In addition, the user also has to specify
whether he wishes to plot the data by setting the output flag,$\;
plot\underline{\;\;}output $,$\;$ to 1 or 0 otherwise.
Figure~\ref{CompP} summarizes the function algorithm. First using
Lawden law no. 3, we can compute the primer vector at each
impulse. Then, a trajectory ephemeris is generated for each leg
between the impulse. As explained in the previous section, to
propagate Eq.~\ref{eq:pveom} ,we need the primer initial state
$(\bar p_{i}, \dot{\bar {p_{i}}^{+}})$ for each transfer leg
$[t_{i},t_{i+1}]$ where $i=1,...,N$. To find the primer velocity,
we use the State Transition Matrix (STM) from
Eqs.~\ref{eq:pdot2}-\ref{eq:pdot3}. However,if the transfer is
coplanar with a transfer angle of 180$^{\circ}$, we use a shooting
method~\cite{Guzman01}. If there are any interior impulse, we
record the discontinuity $ \Delta \dot{\bar p}_{m}$. Once, the
primer state is determined the equations of motion are propagated
using the ephemeris previously stored. From the primer vector
ephemeris, we can generate all the outputs necessary to evaluate
the optimality of the trajectory. There are 9 outputs: the initial
primer magnitude velocity,$\dot{|p|}_{o}$, the final primer
magnitude velocity,$\dot{|p|}_{f}$, the maximum primer
magnitude,$|p|_{max}$, its corresponding epoch, $t_{max}$,
ephemeris filename, and leg number, the difference in primer
velocity at the interior impulse,$\Delta \dot{\bar p}_{m}$, and
the difference in Hamiltonian at the interior impulse,$\Delta
\tilde{H}$.

\begin{figure}
\centering
\includegraphics[scale=0.5]{CPFig.eps}
\caption{\textbf{ComputePrimer} Algorithm Flowchart} \label{CompP}
\end{figure}

\subsection{Moving Initial/Final Impulse : MoveInitialBurn / MoveFinalBurn}
\begin{description}
\item[\underline{syntax}]$[ \Delta V, \underline{R}^{*}, {\underline{V}p}^{*},{\underline{V}m}^{*},{\overline{TOF}}^{\,*}]$
\item[]$\; \; \; \; \; \; \; \; \; \; \; \;\;\;\;\;\;\;\;\;\;\;\;\; = \textbf{MoveInitialBurn}(\dot{|\bar p|}_{o}, Tolerance, \underline{R}, \underline{V}p,\underline{V}m,\bar TOF)$
\end{description}
MoveInitialBurn (MoveFinalBurn) moves the departure (arrival)
state to zero-out the initial (final) slope of the primer
magnitude to within a certain tolerance. The MoveInitialBurn and
MoveFinalBurn subroutines are identical in their structure so only
one is presented in Figure~\ref{MIni}. MoveInitialBurn requires 6
inputs: the initial primer magnitude slope,$\dot{|\bar p|}_{o}$,
the stopping condition or tolerance and the trajectory data
$\underline{R}, \underline{V}p,\underline{V}m$ and $\bar TOF$. To
determine the departure state which corresponds to a zero initial
slope, we implemented a bissection method. If the slope is
positive then the departure state is propagated forward, otherwise
it is propagated backward. To quickly recompute the trajectory
data, we use a Lambert solver. Two Lambert solvers are
implemented: one fast formulation based on the Lagrangian
equations for transfer with no singularity and one used when there
exist singularities such as 180$^{\circ}$-Coplanar
transfer~\cite{Bat84}. The primer magnitude slope is then updated
for the new trajectory and the process is repeated until the slope
is below the tolerance. When, the step size is too big, the
$\Delta V$ sometimes suffer a sudden increase. There exist two
remedies: decrease the step size or switch the type of the Lambert
arc. Two type of Lambert arc can pass through 2 given positions,
$\bar R_{1}$ and $\bar R_{2}$ for a specified time-of-flight : a
type I defined as a transfer with a transfer angle smaller than
180$^{\circ}$ and a type II defined as transfer with a transfer
angle bigger than 180$^{\circ}$. Switching from a type I to a type
II or vice-versa can allow convergence to a different kind of
solution than if one had decreased the step size. Example of
switching type of transfer is shown in the results section. Once
the subroutine has converged on a solutions, it outputs the
$\Delta V$ history and the new trajectory states,
$\underline{R}^{*}, {\underline{V}p}^{*}, {\underline{V}m}^{*},
{\overline{TOF}}^{\,*}$.\\
\\
\\


\begin{figure}
\centering
\includegraphics[scale=0.5]{MIni.eps}
\caption{\textbf{MoveInitialBurn} Algorithm} \label{MIni}
\end{figure}

\subsection{Moving Midcourse Impulse : MoveImpulseN}
\begin{description}
\item[\underline{syntax}]$[ S^{*}, \underline{R}^{*}, {\underline{V}p}^{*},{\underline{V}m}^{*},{\overline{TOF}}^{\,*},S_{iter},\Delta V, \|Grad_{rel} J\|]$
\item[]$\; \; \; \; \; \; \; \; \; \; \; \;\;\;\;\;\;\;\;\;\;\;\;\; = \textbf{MoveImpulseN}(\underline{R}, \underline{V}p,\underline{V}m,\overline{TOF},Tolerance)$
\end{description}
\noindent This subroutine moves the interior impulses position and
time so that the gradient of the cost function expressed in
Eq.~\ref{eq:grad2} is null. It requires 5 inputs: the trajectory
data, $\underline{R},
\underline{V}p,\underline{V}m,\overline{TOF}$ and the stopping
condition or tolerance. To move the midcourse impulses states, we
use the Broyden-Fletcher-Goldberg-Shanno (BFGS) minimization
technique. This Quasi-Newton method uses an approximation of the
Hessian matrix instead of its direct evaluation. The approximation
formula is called BFGS update. We are trying to solve for the
vector $\bar S$ which contains the position and time of all the
interior impulse so that the gradient J is null:

\begin{equation}
\mbox{Find $\bar S^{*}$  / Min [J}(\bar S^{*})= \sum_{i=1}^{N}
\Delta V_{i}] \label{eq:gradprob}
\end{equation}

Where:

\begin{equation}
\bar S = \left( \begin{array}{c}
\bar R_{2} \\
 t_{2} \\
. \\
\bar R_{k} \\
 t_{k} \\
. \\
\bar R_{N-1} \\
 t_{N-1} \end{array} \right)
\label{eq:S}
\end{equation}

And,

\begin{equation}
\bar \nabla J = \left[ \begin{array}{c}
\dot{\bar p}_{2}^{+} - \dot{\bar p}_{2}^{-} \\
-(\tilde{H}_{2}^{+} - \tilde{H}_{2}^{-}) \\
. \\
\dot{\bar p}_{k}^{+} - \dot{\bar p}_{k}^{-} \\
-(\tilde{H}_{k}^{+} - \tilde{H}_{k}^{-}) \\
. \\
\dot{\bar p}_{N-1}^{+} - \dot{\bar p}_{N-1}^{-} \\
-(\tilde{H}_{N-1}^{+} - \tilde{H}_{N-1}^{-}) \end{array} \right]
\label{eq:gradJ}
\end{equation}

\noindent The line search direction is defined as:
\begin{equation}
\overline{SD} = - \overline{Hess} \cdot \bar \nabla J
\end{equation}

\noindent We are trying to solve a multi-dimensional problem,
thus, we need to normalize the search direction:
\begin{equation}
\overline{SD}_{norm} = \frac{\overline{SD}}{\|\overline{SD}\|}
\end{equation}

\noindent Using quadratic interpolation, we find the optimal step
size, $\lambda^{*}$, along the search direction $\overline{SD}$
~\cite{Rao96} and we update $\bar S$ as:

\begin{equation}
\bar S^{*} = \bar S + \lambda^{*} \cdot \overline {SD}_{norm}
\end{equation}

\noindent Then, we compute the new gradient of J at $\bar S^{*}$
and repeat the process until the stopping condition is met. We
achieve convergence on a local minimum when the norm of the
relative gradient is below a given tolerance. The relative
gradient is defined as follow:

\begin{equation}
\bar \nabla J_{rel} = \left[ \begin{array}{c}
...\\
\frac{\dot{\bar p1}_{k}^{+} - \dot{\bar p1}_{k}^{-}}{|\dot{\bar p1}_{k}^{+}| + |\dot{\bar p1}_{k}^{-}|} \\
\; \\
\frac{\dot{\bar p2}_{k}^{+} - \dot{\bar p2}_{k}^{-}}{|\dot{\bar p2}_{k}^{+}| + |\dot{\bar p2}_{k}^{-}|} \\
\; \\
\frac{\dot{\bar p3}_{k}^{+} - \dot{\bar p3}_{k}^{-}}{|\dot{\bar p3}_{k}^{+}| + |\dot{\bar p3}_{k}^{-}|} \\
\; \\
\frac{-(\tilde{H}_{k}^{+} - \tilde{H}_{k}^{-})}{|\tilde{H}_{k}^{+}| + |\tilde{H}_{k}^{-}| } \\
...\\
\end{array} \right] k = 2,...,(N-1)
\label{eq:gJrel}
\end{equation}

\noindent Unlike the gradient, directly re-computed every
iteration, the Hessian matrix is approximated using the BFGS
update~\cite{Rao96}. The updated matrix is an approximation of the
Hessian, thus we have to check that it remains positive
definitive. If not, we re-initialize it to the identity matrix.
When the convergence is achieves, MoveImpulseN outputs 8
variables: the final midcourse position and time vector, $S^{*}$,
the final trajectory data, $\underline{R}^{*},
{\underline{V}p}^{*},{\underline{V}m}^{*},{\overline{TOF}}^{\,*}$,
an history of $\bar S$,$\Delta V$, $\|Grad_{rel} J\|$ at each
iteration.

\begin{figure}
\centering
\includegraphics[scale=0.5]{MImp.eps}
\caption{\textbf{MoveImpulseN} Algorithm} \label{MImp}
\end{figure}

\subsection{Adding an Impulse : AddImpulseN}

\begin{description}
\item[\underline{syntax}]$[ \underline{R}^{*}, {\underline{V}p}^{*},{\underline{V}m}^{*},{\overline{TOF}}^{\,*}, Gain ]$
\item[]$\; \; \; \; \; \; = \textbf{AddImpulseN}( \underline{R}, \underline{V}p,\underline{V}m,\overline{TOF},t_{max},\bar p_{max},filename,leg\underline{\;\;}number)$
\end{description}

\noindent This subroutine add an impulse at the maximum primer
vector magnitude. It requires 8 inputs: the four trajectory data,
the time at which the primer is maximum, $t_{max}$, the primer
vector at $t_{max}$, $\bar p_{max}$, the name of the ephemeris
file corresponding to the leg at which an impulse will be added,
and the leg number, $leg\underline{\;\;}number$. AddImpulseN
implements Eqs.~\ref{eq:dv}, \ref{eq:c} and \ref{eq:Rpert} seen in
the previous section and retarget for the new trajectory state
using a Lambert algorithm. Note that in some cases where we need
accuracy in determining $c$, we must solve directly
Eq.~\ref{eq:ceq} instead of using the first order approximation
from Eq.~\ref{eq:c}. In a future version, an option will be
prompted for the user to decide which method he chooses for
determining the added impulse magnitude. Once the impulse is added
to the selected leg and the trajectory updated, AddImpulseN
returns the new trajectory data as well as the gain in $\Delta V$
as a result of the added impulse.

\begin{figure}
\centering
\includegraphics[scale=0.5]{AddImp.eps}
\caption{\textbf{AddImpulseN} Algorithm} \label{AddImp}
\end{figure}

\section{Test Cases and Preliminary Results}
In this section, we present four test cases : a Hohmann and a
Bi-Elliptical transfer for two different transfer radii ratios
1.43 and 17. In addition, we show some preliminary result for
initializing one Leonardo-BRDF spacecraft from its launching
orbit.

\subsection{Test Cases : Hohmann/Bi-elliptical Transfers}
To test the validity of the primer vector computation, we choose
to first look into the "well-known" Hohmann and Bi-elliptical
transfers. The Hohmann transfer was our first choice as it is the
known optimum for a circle-to-circle coplanar transfer. For ratio
$R = \frac{r2}{r1}$ sufficiently large, Bi-elliptical transfer
were shown to be cheaper than the Hohmann transfer\cite{PrusCon}.
Figures~\ref{Ho7x10} and~\ref{Bi7x9x10} illustrate the primer
vector data for a Hohmann and a Bi-elliptical transfer with a
ratio of 1.43. Figures ~\ref{Ho7x119} and ~\ref{Bi7x140x119}
present a Hohmann and a Bi-elliptical transfer for a transfer with
a ratio of 17.

\begin{figure}[h]
\centerline{\includegraphics[scale=0.6]{Ho7x10.eps}}
\caption{Primer Vector Data for a 7,000-km to 10,000-km Hohmann
Transfer, $\Delta V = 1.22881 km/s$} \label{Ho7x10}
\end{figure}

\noindent Both Bi-elliptical and Hohmann transfers are optimal in
term of the Lawden Laws, which was expected. For $R = 1.43$, the
total trajectory cost are identical. For $R= 17$, the
Bi-elliptical transfer is 7 m/s cheaper. Even though this gain in
$\Delta V$ might not seem significant, it points to a very
important characteristic of the primer vector theory: it is a
local optima theory. In other words, the primer vector analysis
tool will converge on a neighborhood local
solution not on the global optimum.% A good example of this limitation is shown in
%ref.~\cite{Jezew}.

\begin{figure}[ht]
\centerline{\includegraphics[scale=0.6]{Bi7x9x10.eps}}
\caption{Primer Vector Data for a 7,000-km to 10,000-km
Bi-elliptical Transfer (Midradius of 9,000-km), $\Delta V =
1.22883 km/s$ } \label{Bi7x9x10}
\end{figure}

\begin{figure}[ht]
\centerline{\includegraphics[scale=0.6]{Ho7x119.eps}}
\caption{Primer Vector Data for a 7,000-km to 119,000-km Hohmann
Transfer, $\Delta V = 4.04513 km/s$ } \label{Ho7x119}
\end{figure}

\begin{figure}[ht]
\centerline{\includegraphics[scale=0.6]{Bi7x140x119.eps}}
\caption{Primer Vector Data for a 7,000-km to 119,000-km
Bi-elliptical Transfer (Midradius of 140,000-km), $\Delta V =
4.03801 km/s$} \label{Bi7x140x119}
\end{figure}



\subsection{Initializing a Leonardo-BRDF Trajectory}
\noindent After checking the validity of the primer vector
computation with known optimal transfers such as a Hohmann, we ran
PVAT to initialize one leonardo-type trajectory. Using PVAT, we
achieved a decrease of 40 m/s (i.e. 12$\%$). Below is a table
summarizing the orbital elements for the launching orbit and the
final Leonardo orbit to achieve:

\begin{table}[htb]
\centering \caption{Keplerian orbital elements} \vspace{0 pt}
\begin{tabular}{ccccccc} \hline \hline
& a (km) & e & $I (^{\circ})$ & $\Omega (^{\circ})$ & $\omega
(^{\circ})$ & $TA (^{\circ})$ \\ \hline
 Launch & 6803.1 &  0.0017  & 1.5084 & 278.4  & 329.59 & TBD \\
 LEO1   & 6802.8 &  0.0012  & 3.6042 & 256.79 & 284.27 & 184.19\\
\hline \hline
\end{tabular}
\label{table:orbitelemt}
\end{table}

\noindent This scenario initialize one spacecraft out of the six
composing the total formation with a two-burn lambert arc as an
initial guess. Using FreeFlyer, we vary the position of the first
maneuver in the departure orbit and the transfer time-of-flight
such that the total $\Delta V$ of the transfer to the final orbit
is reasonable($< 1 km/s)$. We arbitrarily picked an initial
trajectory with an initial true anomaly of xx , a time-of-flight
of xx and an initial cost of 323 m/s. Figure~\ref{leo1ini} shows
the primer vector data for this reference trajectory.

\begin{figure}[htb]
\centerline{\includegraphics{Leo1_initial.eps}}\label{leo1ini}
\end{figure}

\noindent By inspection of Figure~\ref{leo1ini}, we know from
chapter xx that a decrease in $\Delta V$ can be achieved by moving
the initial departure. After re-computing the primer vector data,
we successively vary the initial and final impulses.
Figure~\ref{dvhist} summarizes the $\Delta V$ history for each
iteration. The final optimal primer vector history is plotted in
Figure~\ref{leo1final}.

\begin{figure}
\centerline{\includegraphics{Final_Leo1_Type1.eps}}\label{leo1final}
\end{figure}

\begin{figure}
\centerline{\includegraphics{TypeDVHist.eps}}\label{dvhist}
\end{figure}

\noindent While varying the initial and final impulse, there was
an option to switch from a type 1 to a type 2. The results
presented above are for a two-burn type 1 transfer where a choice
was made to reduce the step size instead of switching the Lambert
type. Below we present the results obtained when we switched to a
type 2. Figure~\ref{swt1t2} is a plot of the primer vector data
right after switching to a type 2. The primer vector history
indicates that adding an impulse will improve the total trajectory
cost. Adding an impulse leads to a non-optimal 3-burn trajectory
shown in Figure~\ref{nopt3}. After several iterations, we converge
to a 3-burn optimal trajectory with a total cost of 283 m/s
(Figure~\ref{leo1fin2}). Figure~\ref{dvhist2} summarize the
$\Delta V$ history for this scenario.
\begin{figure}
\centerline{\includegraphics{SwitchT1_to_T2.eps}}\label{swt1t2}
\end{figure}

\begin{figure}
\centerline{\includegraphics{Type2_NonOpt_3.eps}}\label{nopt3}
\end{figure}

\begin{figure}
\centerline{\includegraphics{Final_Leo1_Type2.eps}}\label{leo1fin2}
\end{figure}

\begin{figure}
\centerline{\includegraphics{Type2DVHist.eps}}\label{dvhist2}
\end{figure}




\chapter{CONCLUSIONS and RECOMMENDATIONS}

\bibliographystyle{unsrt}
\bibliography{primer_vector}


%%%%%%%%%%%%%%%%%%%%%%%%%%%%%%%%%%%%%%%
\end{document}
%%%%%%%%%%%%%%%%%%%%%%%%%%%%%%%%%%%%%%%
