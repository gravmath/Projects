%\documentstyle[eqsecnum,aps]{revtex}
\documentclass{article}
\def\a{{\vec a}}
\def\x{{\vec x}}
\def\z{{\vec z}}

\def\P{{\Phi}}
\def\l{{\ell}}
\def\Pl{{\Phi_{\ell}}}
\def\D{{\Delta}}
\def\.{{\quad .}}
\def\_,{{\quad ,}}


%%Modification History:   Version 1.2  - clean-up of the language and
%%including references
%
%

\begin{document}

\title{Discretized Variational Smooth Particle Hydrodynamics}
%
\author{   \sc
             Conrad Schiff {\rm and} Charles W. Misner\\
           \em
            Department of Physics, University of Maryland,
           \\
            College Park MD 20742-4111 USA\\
           \rm
         e-mail: \tt  cmschiff@erols.com\\
                      misner@umail.umd.edu
%
%   Version 1.2
}
\date{\today}
\maketitle

\begin{abstract}
We present a Lagrangian formulation of the dynamics of a classical perfect
self-gravitating fluid derived from a discrete action via a variational
principle.  In our approach, the gravitational potential is represented as a
distinct field defined on a discrete grid of points while the perfect fluid is
modeled as a set of fluid elements able to move anywhere within the boundaries
of the grid. Connection between the two 'fields` is made by using a the
equations of smoothed particle hydrodynamics (SPH).  Our approach provides a
step toward the derivation a full numerical integration scheme that models the
interaction between gravity and fluid dynamics in general relativity,
\end{abstract}

\section{Introduction}

Misner presented a Lagrangian hydrodynamic variational principle, as a first
step toward a full numerical integration of Einstein's equations for gravity
interacting with a relativistic perfect fluid \cite{A:CWM96}.  The approach,
based on the earlier work in classical fluid dynamics by Mittag, Stephen, and
Yourgrau (MSY) \cite{B:MSY79}, yielded continuum equations for both the
gravitational and fluid degrees of freedom.  However, any numerical integration
scheme will necessarily have to be expressed in terms of discrete equations.
To ensure that energy and momentum are better conserved, the equations should
ideally come from an action which has been discretized prior to carrying out
the variation \cite{A:CWM96,A:NP94}.

In this paper, we formulate such a discrete action principle within the simple
context of classical Newtonian fluids.  In order to prepare for the transition
to general relativity, where the fluid self-gravity cannot be treated as a
particle-particle interaction, we assume that the Newtonian gravitational
potential lives on a discrete grid.  In addition, we assume that the fluid is
made up of discrete elements whose position will generally not coincide with
a grid point.  We believe that this hybrid scheme best represents the physics of
the inspiral of two neutron stars within the framework of general relativity,
which is expected to be a prominent source of weak gravitational waves
(see section 36.6 of \cite{B:MTW}).  In this scenario, the matter distribution
is concentrated near the center-of-mass of the stars for most of the motion.
A full Eulerian description using finite-difference techniques requires
sophisticated reorganization or adaptation of the computational mesh while
the Lagrangian description puts the mesh where the material is \cite{A:GM82}.
Connection between the physical quantities define at the grid-points and the
particle is provided by techniques from the smoothed particle hydrodynamics
(SPH) numerical scheme \cite{A:NP94,A:JJM92,A:WB89,A:GM82,A:PJM91}.

%%%%%%%%%%%%%%%%%%%%%%%%%%%%%%%%%%%%%%%%%%%%%%%%%%%%%%%%%%%%%%%%%%%%%
%%%%%%%%%%%%%%%%%%%%%%%%%%%%%%%%%%%%%%%%%%%%%%%%%%%%%%%%%%%%%%%%%%%%%%%
%%%%%%%%%%%%%%%%%%%%%%%%%%%%%%%%%%%%%%%%%%%%%%%%%%%%%%%%%%%%%%%%%%%%%%
\section{Classical Fluid Mechanics}\label{cfm}

Start with a fluid and label each fluid element by its position ${\vec a}$ at
$t=0$.  MSY then introduce what we call the trajectory function
${\vec z}({\vec a},t)$ which gives any fluid element's position at some
later time $t$ by

\begin{equation}
    {\vec x} = {\vec z}({\vec a},t) \.
\end{equation}

In discrete approximations later there will only be a finite number of particle
trajectories $\z$ computed, and fields such as the gravitational potential
$\Phi(\x)$ will be computed only on a finite set of grid locations $\x$.

The goal of making and combining these two discretization approximations makes
it important that a clear distinction be kept in mind between the particle
trajectories $\z$ and the field points $\x$, even though the equation
$\x = \z(t)$ imbeds a particle trajectory within the grid where field values
will be known.

The trajectory function has the obvious boundary condition
${\vec z}({\vec a},0) = {\vec a}$.  Assuming the fluid to be ideal, its motion
is subject to two constraints.  The first is the conservation of mass which
takes the form

\begin{equation}
    \rho({\vec z},t)\,d^3 z = \rho({\vec a},0)\,d^3 a \.
\end{equation}

Introducing the Jacobian $J ={\partial(z_1,z_2,z_3)}/{\partial(a_1,a_2,a_3)}$
allows the conservation of mass equation to be given by

\begin{equation}\label{eq:rho}
    \rho J({\vec z}(\a,t)) = \rho_0(\a) \.
\end{equation}

Thus $\rho$ depends on the family of trajectories $\z(\a,t)$ under
consideration, and is entirely determined by it.  The second constraint is the
conservation of entropy which takes on the form

\begin{equation}
    s(\a,t) = s({\vec a},0) = s_0(\vec a) \.
\end{equation}

Although MSY incorporate these constraints in a variational principle by the
use of the Lagrange multipliers, we find the equations shorter and easier to
read if one takes $\rho$ to be defined by Eqn.~(\ref{eq:rho}).
The MSY Lagrangian then is:

\begin{equation}\label{eq:MSY}
    I  =  \int\!\! dt \int\!\! d^3a \left[\frac{\rho_0}{2} \left(
          \frac{\partial z_i}{\partial t}\right)^2 - \rho_0(e+\Phi) \right] \_,
\end{equation}
%
where $e(\rho,s)$ is the specific internal energy and $\Phi(\z(\a,t),t)$ is the
gravitational potential energy per unit mass at the position of particle $\a$.
In this formalism, ${\vec z}(\a,t)$ are the dynamical variables that are varied
to produce the equations of motion.

The variation of the action with respect to the trajectory function ${\vec z}$
requires the various properties of the Jacobian that MSY employ in their
formulation. We must remember that since $\rho$ is defined by
Eqn.~(\ref{eq:rho}) both $\rho$ and $e(\rho,s)$ will vary when $\z$ is varied.
Thus we find

\begin{equation}
    \delta I = \int\!\! dt \int\!\! d^3a \left[ \rho_0
    \left(\frac{\partial z_i}{\partial t}\right)\delta
    \left(\frac{\partial z_i}{\partial t} \right)
    - \rho_0 \frac{P}{\rho^2} \delta \rho
    - \rho_0 \frac{\partial \Phi}{\partial z_i} \delta {z_i}
    \right] \.
\end{equation}
%
Here we have used the thermodynamic relationship $de = T\,ds -P\,d(1/\rho)$ in
the form $(\partial e/\partial \rho)_s = P/\rho^2$ in evaluating how the
internal energy $e$ changes when the trajectory variations cause changes in
$\rho$ but not in $s$ at a particular fluid element $\a$.  The problematic
term here is that containing $\delta \rho$.  But from $\rho J = \rho_0(\a)$
where the right hand side is independent of $\z$ one has
$J \delta \rho + \rho \delta J = 0$.  Thus the term
$-(\rho_0 {P}/{\rho^2}) \delta \rho$ becomes
$-(P/\rho) J \delta \rho = + P \delta J$.  Using $\delta J = ({\partial J}/
{\partial z_{i,j}}) \delta z_{i,j} = J_{ij} \delta z_{i,j}$,
where $z_{i,j} = {\partial z_i}/{\partial a_j}$ and $J_{ij}$ is the (i,j)
minor of the Jacobian, the second ($P$) term can be simplified.  Exchanging
the order of the variation and the partial differentiation in the first and
second terms and integrating these terms by parts gives

\begin{equation}
    \delta I = -\int\!\! dt \int\!\! d^3a \left[ \rho_0
    \left(\frac{\partial^2 z_i}{\partial t^2}\right)
    + \frac{\partial}{\partial
    a_j}\left(P J_{ij}\right)
    + \rho_0 \frac{\partial \Phi}{\partial z_i}
    \right] \delta z_i \.
\end{equation}

Setting this variation to zero leads to the partial differential equation

\begin{equation}\label{EL}
    \rho_0 \left(\frac{\partial^2 z_i}{\partial t^2}\right) + \rho_0
    \frac{\partial \Phi}{\partial z_i} + \frac{\partial}{\partial
    a_j}\left(P J_{ij}\right) = 0 \.
\end{equation}
%
From the relations ${\partial J_{ij}}/{\partial a_j} = 0$, the last term becomes

\begin{equation}
    \frac{\partial}{\partial a_j}\left(P  J_{ij} \right) =
    \frac{\partial P}{\partial a_j} J_{ij} = \frac{\partial P}{\partial
    z_k}\frac{\partial z_k}{\partial a_j} J_{ij} \.
\end{equation}
%
Finally by using $({\partial z_k}/{\partial a_j}) J_{ij} = J \delta_{ik}$ and
$J = \rho_0 / \rho$ we put Eq.~(\ref{EL}) into the familar form

\begin{equation}\label{eq:Euler}
    \left(\frac{\partial^2 z_i}{\partial t^2}\right) + \frac{\partial
    \Phi}{\partial z_i} + \frac{1}{\rho} \frac{\partial P}{\partial z_i} =
    0 \.
\end{equation}



%%%%%%%%%%%%%%%%%%%%%%%%%%%%%%%%%%%%%%%%%%%%%%%%%%%%%%%%%%%%%%%%%%%%%%
%%%%%%%%%%%%%%%%%%%%%%%%%%%%%%%%%%%%%%%%%%%%%%%%%%%%%%%%%%%%%%%%%%%%%%
%%%%%%%%%%%%%%%%%%%%%%%%%%%%%%%%%%%%%%%%%%%%%%%%%%%%%%%%%%%%%%%%%%%%%%
\section{Smooth Particle Hydrodynamics}\label{sph}

In this section, we review some of the standard SPH equations
\cite{A:NP94,A:JJM92,A:WB89,A:GM82,A:PJM91} that we used in approximating the
continuum variational principle in discrete form.

The single most important relation that should be stated is the idea of
approximating the equation

\begin{equation}
    f({\vec z}) = \int\!\! d^3z'\,\delta({\vec z} - \vec{ z'}) f(\vec{z'})
\end{equation}
%
by the relation
\begin{equation}\label{sph_fun}
    \langle f({\vec z}) \rangle \simeq \int\!\! d^3z'\,W({\vec z}
    - \vec{z'}; h) f(\vec{z'}) \.
\end{equation}

The essential properties of the smoothing kernel $W$ are that it is normalized
\begin{equation}\label{norm}
    \int\!\! d^3z\,W({\vec z};h) = 1 \_,
\end{equation}
%
and that it has a delta function limit

\begin{equation}
    \lim_{h \rightarrow 0} W({\vec z} - \vec{z'};h) = \delta({\vec z}
    - \vec{z'}) \.
\end{equation}

In addition, we assume that we are always using a symmetric, even,
non-negative kernel so that the smoothing approximation is $O(h^2)$.
In other words,

\begin{equation}
    \langle f({\vec z}) \rangle = f({\vec z}) + c \left( \nabla^2 f({\vec z}) \right) h^2 \_,
\end{equation}
%
where $c$ is a constant independent of $h$.

Gingold and Monaghan \cite{A:GM82} motivate a derivation of SPH starting from
the Lagrangian for a nondissipative, isentropic gas which in our notation reads

\begin{equation}
    L = \int\!\! d^3a\,\rho_0 \left[ \frac{1}{2} {\dot {\vec z}\,}^2 -
        e(\rho) - \Phi(z,t)\right] \_,
\end{equation}
%
where ${\partial e}/{\partial \rho} = {P}/{\rho^2}$ and where $z$ means
$z(\a,t)$.  Then they write the integral in discrete form

\begin{equation}
    L \simeq \sum_{a=1}^{N} m_a \left( \frac{1}{2} {\dot {\vec z_a}}^2 -
    e(\rho_a)  - \Phi(\vec z_a, t) \right) \.
\end{equation}
%
Calculating the terms in the Euler-Lagrange equations gives the SPH
momentum equation

\begin{equation}
    m_a \ddot {\vec z_a}  + m_a \left( \frac{\partial \Phi}{\partial
                                    \x} \right)_{\x = \z_a}
    + \sum_{b=1}^{N} \frac{P_b}{\rho_b^2}
    \frac{\partial \rho_b}{\partial {\vec z_a}} m_b
    = 0 \.
\end{equation}


Connection with SPH can be made by identifying

\begin{equation}\label{eq:smoothed_rho}
    \rho_b = \sum_{c=1}^{N} m_c W ( {\vec z_b} - {\vec z_c}; h ) \_,
\end{equation}
%
which leads to the symmetric momentum equation

\begin{equation}\label{eq:sym_sph}
    {\ddot {\vec z_a}}  + \left( \frac{\partial \Phi}{\partial
                                    \x} \right)_{\x = \z_a}
    + \sum_{b=1}^{N} \left( \frac{P_b}{\rho_b^2} +
    \frac{P_a}{\rho_a^2} \right) m_b \nabla_a W ( {\vec z_a} - {\vec
    r_b}; h ) \_,
\end{equation}
%
that Monaghan \cite{A:JJM92,A:GM82,A:PJM91} advocates for its ability to
conserve linear and angular momentum.

Note that two approximation steps are invoked here.  One is the replacement of
$\int\!\! d^3a\,\rho_0$ by $\sum_{a=1}^N m_a$ which is a Monte Carlo sampling
of the continuum of fluid elements by a finite subset.  The second step is the
estimate (\ref{eq:smoothed_rho}) of the continuum density from this finite
sample.  The normalization condition (\ref{norm}) provides a consistency
between these two steps, but we do not see that one can be derived from
the other.

%%%%%%%%%%%%%%%%%%%%%%%%%%%%%%%%%%%%%%%%%%%%%%%%%%%%%%%%%%%%%%%%%%%%%%
%%%%%%%%%%%%%%%%%%%%%%%%%%%%%%%%%%%%%%%%%%%%%%%%%%%%%%%%%%%%%%%%%%%%%%
%%%%%%%%%%%%%%%%%%%%%%%%%%%%%%%%%%%%%%%%%%%%%%%%%%%%%%%%%%%%%%%%%%%%%%
\section{Newtonian Gravity}\label{Newton}

In the preceding sections Newtonian gravity was included as a potential $\Phi$
due to external masses.  When the self-gravitation of the fluid is to be
included more care is needed.  Unlike usual Newtonian SPH, we do not want to
think of gravitation as a mutual interaction of the smoothed particles---
this viewpoint, effective in Newtonian problems, will not provide guidance
for the relativistic problems we aim to formulate.  Instead we write a
field theory.\\

A variational principle which gives rise to the Poisson equation is
$\delta I_G = 0$ with

\begin{equation}\label{eq:I-G}
    I_G = -\int\!\!dt\!\int\!\!d^3x \left[
            (1/8\pi G)(\nabla \Phi)^2 + \rho(\x,t) \Phi(\x,t) \right] \.
\end{equation}
%
Varying $\Phi$ here gives

\begin{equation}\label{eq:Poisson}
    \nabla^2 \Phi = 4 \pi G \rho  \_,
\end{equation}
%
with solutions such as $\Phi = -GM/r$.

A continuum variational principle extending that of Sec.~\ref{cfm} to include
the field dynamics is $\delta I = 0$ with $I = \int\!L\,dt$ and

\begin{eqnarray}\label{eq:full-L}
    L & = & \int\!\! d^3a \left[\frac{\rho_0}{2} \left(
        \frac{\partial z_i}{\partial t}\right)^2
      - \rho_0\, e(\rho,s_0) - \rho_0\, \Phi(\z(\a,t),t) \right]
       \nonumber \\
      & & - (1/8\pi G)\int\!\!d^3x\,\nabla \Phi(\x,t)^2  \.
\end{eqnarray}

The terms here containing $\z(\a,t)$ are exactly those considered in
Sec.~\ref{cfm}, so the fluid equations are just (\ref{eq:Euler}).  But to see
that Eq.~(\ref{eq:Poisson}) also results we need to rewrite the $\rho_0\Phi$
term to see that it is the same as in Eq.~(\ref{eq:I-G}).  This we do by
invoking the definition of $\rho$ in a change of variables $\rho_0\,d^3a
= \rho\,d^3z$ in the integral

\begin{equation}\label{eq:rhoPhi}
    \int\!\! d^3a\, \rho_0(\a)\, \Phi(\z(\a,t),t) =
    \int\!\! d^3z\, \rho(\z,t)\, \Phi(\z,t) =
    \int\!\! d^3x\, \rho(\x,t)\, \Phi(\x,t) \_,
\end{equation}
%
where the last step is a notational change of the dummy variable
of integration.  Thus Eq.~(\ref{eq:Poisson}) also results by varying
$\Phi(\x)$ in this combined Lagrangian (\ref{eq:full-L}).

It is important to note that only the first form of the interaction term
(\ref{eq:rhoPhi}) as given in Eq.~(\ref{eq:full-L}) is acceptable in the
fundamental Lagrangian.  For in making the change of variables from $\a$ to
$\z$ in equation (\ref{eq:rhoPhi}) one has assumed a definite fluid motion
$\z(\a,t)$. Since reference to this particular motion disappears in the
$\int\!\! d^3x\, \rho(\x,t)\, \Phi(\x,t)$ form of this term, it would not be
possible to carry out the $\delta \z$ variations were this form to be stated
as part of the basic Lagrangian.  We have used it here only to carry out a
variation of $\Phi$ while holding the fluid motion $\z(\a,t)$ unchanged.


%%%%%%%%%%%%%%%%%%%%%%%%%%%%%%%%%%%%%%%%%%%%%%%%%%%%%%%%%%%%%%%%%%%%%%
%%%%%%%%%%%%%%%%%%%%%%%%%%%%%%%%%%%%%%%%%%%%%%%%%%%%%%%%%%%%%%%%%%%%%%
%%%%%%%%%%%%%%%%%%%%%%%%%%%%%%%%%%%%%%%%%%%%%%%%%%%%%%%%%%%%%%%%%%%%%%
\section{Discretization and Numerical Formulation}\label{Disc.}

At this point, the basic pieces are in place and all that remains is to combine
and interpret them.  Start from the complete action integral, which is written
as
\begin{eqnarray}
I  & = & - \int\!\! dt \int\!\! d^3x
           \frac{\left[\nabla \Phi(\x,t)\right]^2}{8 \pi G}
         + \int\!\! dt \int\!\! d^3a \rho_0 \left[ \left(\frac{\partial \z}
          {\partial t}\right)^2 - e \right] \nonumber \\
   &   & - \int\!\! dt \int\!\! d^3 a \rho_0 \Phi(\z,t) \_,
\end{eqnarray}
%
subject to the constraint of $\rho J = \rho_0$ and $s = s_0$.  Assume, as
discussed above, that the fields will be numerically determined on a
computational grid and that the particles will be labeled and their individual
motion followed.  The discretization of the field becomes
$\Phi(\x,t) \rightarrow \Phi_\l(t)$ where $\l$ represents the triple integer
set $\left\{\l_x, \l_y, \l_z\right\}$.  Likewise the particle discretization
becomes $ \int d^3 \rho_0 \rightarrow \sum_{a} m_a$.

Under these definitions, the action becomes
\begin{eqnarray}
I & = & - \int\!\! dt \sum_{l} \frac{1}{8 \pi G}
        \left\{
         \left[ \D_x \Pl - \Pl \right]^2 +
         \left[ \D_y \Pl - \Pl \right]^2 +
         \left[ \D_z \Pl - \Pl \right]^2
        \right\} \nonumber \\
  &   &
      + \int\!\! dt \sum_{a} m_a \left[ \left(\frac{\partial \z_a}
         {\partial t}\right)^2 - e(\rho_a) \right]
      - \int\!\! dt \sum_{a} m_a \Phi(\z_a,t) \_,
\end{eqnarray}
where the $\Delta$ operator changes the corresponding '$l$'-label on any object
possessing it by one ({\it e.g.} $\Delta_x \Pl - \Pl = \Phi_{\l_x+1,\l_y,\l_z} -
\Phi_{\l_x,\l_y,\l_z}$).

The factor $\Phi(\vec z_a)$ in the interaction term is undefined since
$\vec z_a$ will not normally coincide with one of the lattice points
$\vec x_\ell$.  Thus some interpolation is required, and we can set
\begin{equation}\label{eq:interpolate}
      \Phi(\z_a) = \frac{\sum_\l h^3 \Phi_\l I(\z_a - \x_\l)}
                  {\sum_n h^3 I(\z_a -\x_n)} \_,
\end{equation}
%
where the denominator ensures that interpolation of a constant function yields
that constant value.  The kernel $I$ chooses a linear combination of nearby
grid values $\Pl$ to estimate $\Phi$ at any desired point and provides an
approximation to the delta function in

\begin{equation}\label{eq:approxDelta}
    \Phi(\z,t) = \int\!\!d^3x\,\Phi(\x,t) \delta^3(\z-\x) \.
\end{equation}

Setting the variation of this discrete action with respect to $\Pl(t)$ gives

\begin{eqnarray}\label{eq:dis_Poi}
  \left( 2 \P_m - \D_{-x} \P_m - \D_x \P_m \right) +
  \left( 2 \P_m - \D_{-y} \P_m - \D_x \P_m \right) \nonumber & & \\
+ \left( 2 \P_m - \D_{-z} \P_m - \D_x \P_m \right) & = & 4 \pi G \sum_{a} m_a
  \frac{ h^3 I(\z_a - \x_m)}{\sum_n h^3 I(\z_a -\x_n)} \_,
\end{eqnarray}
%
which is the expected discrete form of the Poisson equation with the central
differencing approximation to the second derivative.  The source term then
gives an effective definition of the density

\begin{equation}\label{eq:dis_den}
   \rho(\x_l) = \sum_{a} m_a \frac{ h^3 I(\z_a - \x_\l)}{\sum_n h^3 I(\z_a -\x_n)} \_,
\end{equation}
which is analogous to the one invoked in (\ref{eq:smoothed_rho}).  We will take
this as our definition of an interpolated density.

It is instructive to note that summing (\ref{eq:dis_Poi}) over $m$ gives

\begin{equation}
  \sum_{\partial m} \P_m = 4 \pi G \sum_{a} m_{a} \_,
\end{equation}

where the $\sum_{\partial m}$ represents a summation over the boundary grid
points.  This is just a discrete form of Gauss's law.

Finally, the variation with respect to the trajectory function $\z_b(t)$ must
be performed.  Setting the functional derivative of the discrete action with
respect to $\z_b(t)$ equal to zero yields:

\begin{eqnarray}\label{eq:dis_particle}
  m_b {\ddot \z}_b + \sum_a \left[ m_a \frac{P(\z_a)}{\rho^2(\z_a)}
                                       \frac{\partial \rho(\z_a)}{\partial \z_b}
                                 + m_a \frac{\partial \P(\z_a)}{\partial \z_b}
                            \right] = 0
\end{eqnarray}

After some straightforward but lengthy calculations one arrives at

\begin{eqnarray}\label{eq:dis_drho}
  \sum_a m_a \frac{P(\z_a)}{\rho^2(\z_a)}
             \frac{\partial \rho(\z_a)}{\partial \z_b} & = &
    m_b \sum_a m_a
      \left(
        \frac{P(\z_a)}{\rho^2(\z_a)} + \frac{P(\z_b)}{\rho^2(\z_b)}
      \right)
      \frac{ \nabla_b I(\z_b - \z_a) h^3}{\sum_\l h^3 I(\z_b - \x_\l)}
    \nonumber \\
  & &   -
    m_b \sum_a m_a
      \frac{P(\z_a)}{\rho^2(\z_a)} h^3 I(\z_b - \z_a)
      \frac{\sum_m h^3 \nabla_b I(\z_b - \x_m)}
        {\left[ \sum_\l h^3 I( \z_b - \x_\l) \right] ^2}
\end{eqnarray}
%
and
%
\begin{eqnarray}\label{eq:dis_dphi}
  \sum_a m_a \frac{\partial \P(\z_a)}{\partial \z_b}
       = m_b \frac{ \sum_\l h^3 \nabla_b I (\z_b - \x_\l) \Pl
             \sum_n h^3 I(\z_b - \x_n)}{\left[
             \sum_m h^3 I( \z_b - \x_m) \right] ^2} \nonumber \\
           - \frac{\sum_\l h^3 \nabla_b I(\z_b - \x_\l)
             \sum_n h^3 I(\z_b - \x_n) \P_n}
            {\left[ \sum_m h^3 I( \z_b - \x_m) \right] ^2}
\end{eqnarray}
for the second and third terms respectively.  Despite their initial appearance,
there is much that is familiar.  Equation (\ref{eq:dis_drho}) is essentially
the symmetric momentum equation (\ref{eq:sym_sph}) discussed in Section
\ref{sph}.  Equation (\ref{eq:dis_dphi}) contains the derivatives of the
gravitational potential one expects from interpolating from the grid

\begin{equation}\label{eq:gravForce}
    \left( \frac{\partial \Phi}{\partial \x} \right)_{\x = \z_a}
    \mapsto
    \sum_\l h^3  \Pl(t)
         \frac{\partial I({\x_\l} - {\z_a})}{\partial \z_a} \.
\end{equation}
Both terms are complicated by the presence of derivitives with respect to the
kernel normalization but these extra terms are analogous to those terms arising
when $\nabla h$ terms are accounted for in conventional SPH \cite{A:NP94}.

The implementation scheme is now clear (if not concise).  Specify an initial
distribution of matter ({\it i.e.} $m_a$ and $\z_a(t=0)$).  Use
(\ref{eq:dis_Poi}) to determine the gravitational potential at the grid
locations.  Use (\ref{eq:dis_drho}) and (\ref{eq:dis_dphi}) in conjunction with
(\ref{eq:dis_particle}) to calculate the accelerations of the particles in
terms of known quantities.  Integrate the particles equations of motion on time
step and repeat the process.  Note that the causality inherent in this scheme
is Galilean; absolute time has meaning, the gravitational force is essentially
instantaneous, and there is no mechanism to propagate a solution that satisfies
the initial-value problem forward (or backward) in time.  It is for this reason
that the Poisson equation must be solved at each integration step.

\begin{thebibliography}{9}

\bibitem{A:CWM96}
  C. W. Misner, \emph{Variational Principle for Lagrangian Hydrodynamics},
  Proceedings of the Second International Sakharov Conference of Physics, 1996,
  I. Dremin and A. Semikhatov, eds., World Scientific Pubs.

\bibitem{B:MSY79}
  L. Mittag, M. J. Stephen, and W. Yourgrau ``Variational Principles in Hydrodynamics'',
  Chapter 13 in W. Yourgrau and S. Mandelstam, \emph{Variational Principles in Dynamics
  and Quantum Theory}, (Dover Publications Inc., New York 1979)

\bibitem{A:NP94}
  R. P. Nelson and J. C. B. Papaloizou, \emph{Mon. Not. R. Astron. Soc.}
  \textbf{270}, 1--20 (1994)

\bibitem{B:MTW}
  C. W. Misner, K. S. Thorne, and J. A. Wheeler, \emph{Gravitation},
  (W. H. Freeman, San Francisco, 1973)

\bibitem{A:JJM92}
  J. J. Monaghan, \emph{Annu. Rev. Astron. Astrophys.} \textbf{30}, 543--74 (1992)

\bibitem{A:WB89}
  W. Benz, ``Smooth Particle Hydrodynamics: A Review'', in \emph{Numerical Modeling of
  Stellar Pulsation}, NATO Workshop, Les Arcs, France (1989)

\bibitem{A:GM82}
  R. A. Gingold and J. J. Monaghan, \emph{J. Comp. Phys.} \textbf{46}, 429--453 (1982)

\bibitem{A:PJM91}
  J. J. Monaghan, \emph{Comp. Phys. Comm.} \textbf{48}, 89--96 (1988)

\end{thebibliography}

\end{document}
