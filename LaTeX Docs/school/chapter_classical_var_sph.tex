%section classical var_sph

\section{Introduction}
This chapter concerns itself with the development of the particle-grid method
for a classical ideal fluid.  The aim is to show the framework of theory within the
simpler context of the classical hydrodynamics of an ideal fluid.  The first section
briefly reviews the derivation of the equations of motion for an ideal fluid within
the Lagrangian picture.  The second section presents a Lagrangian variational principle
based on the work of Mittag, Stephen, and Yourgrau \refs{MSY79}.  This variational principle or
its relativistic analog form the base upon which the particle method is built in classical
mechanics or general relativity, respectively.
The Smoothed Particle Hydrodynamics (SPH) formalism is presented in
the third section and the SPH equations for an ideal fluid are derived.  The final section
synthesizes these various ideas to obtain the particle-grid method for a classical ideal fluid.


%%%%%%%%%%%%%%%%%%%%%%%%%%%%%%%%%%%%%%%%%%%%%%%%%%%%%%%%%%%%%%%%%%%%%
%%%%%%%%%%%%%%%%%%%%%%%%%%%%%%%%%%%%%%%%%%%%%%%%%%%%%%%%%%%%%%%%%%%%%%%
%%%%%%%%%%%%%%%%%%%%%%%%%%%%%%%%%%%%%%%%%%%%%%%%%%%%%%%%%%%%%%%%%%%%%%
\section{Equations of Motion for an Ideal Fluid}\label{cfm}

In this section, the equations of motion for an ideal fluid are
derived from elementary considerations of the basic physics.

As is usually done in fluid dynamics, the continuum approximation is the starting point.
The basic concept involved in this approximation is that of a fluid element.  A fluid element
is any volume in the fluid that is small enough to be idealized as a
single point (\textit{i.e.} small compared to the physical dimensions of the problem)
and yet large enough that is still contains a very large number of the
molecules \refs{HB67,LL59}.

Fluid dynamics is then classified according to whether the equations of motion are
expressed as integral relations or differential equations.  Within the latter category,
the further distinction is made between Eulerian and Lagrangian pictures
\refs{liggett94,granger95,HB67}.  The Eulerian picture concerns itself with what is
happening within a fixed volume located at specified point in space.
It naturally leads to a field theory in which the functions describing
the fluid flow depend on both the position of the observation point and the time of the
observation.  Thus the velocity of the fluid would be given by a functional relationship
$\cv{v} = \cv{v}(x,y,z,t)$.  The Eulerian approach leads directly to the finite-difference
methods traditionally used in numerical solutions.  In the Lagrangian picture,
individual fluid elements are followed as they evolve along their trajectories.
The position and velocity of each element depends only on their initial conditions and on
the elapsed time.
Since the Smoothed Particle Hydrodynamics (SPH) method is based on the Lagrangian picture,
it is this picture which will be used throughout the chapter.

Consider a given fluid and divide it up into elements, with element labelled by its
initial position ${\vec a}$.  Subsequently, introduce the trajectory function
$\cv{z}(\cv{a},t)$ which gives any fluid element's position at some
later time $t$ by
\begin{equation}
    \cv{x} = \cv{z}(\cv{a},t) \eqp
\end{equation}
The trajectory function has the obvious boundary condition
${\vec z}({\vec a},0) = {\vec a}$.
The action of the trajectory function is shown schematically in \refp{fig:fluid_elements}.

\begin{figure}
\centerline{
   \includegraphics[height=2in,width=3in,keepaspectratio]{chap2_fluid_elements.eps}}
   \caption{Sketch of the action of the trajectory function \cv{z}(\cv{a},t).  Each of
   the fluid elements move along their respective trajectories as time evolves.  Each
   trajectory is labelled by its initial position \cv{a}.  The two fluid elements shown
   start at \cv{a} = \surp{a,b,c} and \cv{a} = \surp{a',b',c'} and move to positions
   \cv{z} = \surp{x,y,z} and \cv{z} = \surp{x',y',z'}, respectively.}\label{fig:fluid_elements}
\end{figure}

Assume that the mass of the fluid element $m$, is constant along its flow.
This conservation equation can be expressed in terms of the fluid element's density \gr
and volume $dV$ as
\be\label{eq:mass_con}
  \dby{t} m = \dby{t} \surp{\gr dV} = \dby{t} \surp{\gr \push d^3x} = 0 \eqp
\ee

This equation is one of the two constraints to which the fluid flow is subject.
Both the density and the volume will generally change as the fluid element moves.
The time derivative of the volume $dV$ can be determined once the velocity of
the fluid element is known.
Consider the unit volume $dV = \gd x \push \gd y \push \gd z$, shown in
Figure \refp{fig:volume_evolution}, that is moving with velocity \cv{V}.
\begin{figure}
\centerline{
   \includegraphics[height=2in,width=3in,keepaspectratio]{volume_evolution_b.eps}}
   \caption{A moving fluid element with velocity \cv{V}.  The action of the
   trajectory function on two of the vertices $A$ and $B$ are also shown.
   During its movement, the volume of the fluid element translates, rotates and shears and
   the vertices move from their positions $A_t$ and $B_t$ to their new
   positions $A_{t+dt}$ and $B_{t+dt}$.
   }\label{fig:volume_evolution}
\end{figure}
The change in the volume can be determined by looking at how the vertices of the
volume element change under the action of the fluid flow (i.e. how they are
mapped downstream by the trajectory function).
To begin, consider the vertices $A$ and $B$.  At time $t$, the distance
between them is given by $dx_t = B_t - A_t$.  After the element has moved, the distance
is now $dx_{t+dt} = B_{t+dt} - A_{t+dt}$.  However, in the elapsed time $dt$, the difference
in position of the vertex $A$ is $A_{t+dt} - A_t = V_x(x,y,z)$, where
$V_x(x,y,z)$ is the $x$-component of the fluid's velocity at position $(x,y,z)$.
Likewise, the analogous relation for the vertex $B$ is $B_{t+dt} - B_t = V_x(x+dx,y,z)$,
where now the $x$-component of the velocity is evaluated at the position $(x+dx,y,z)$.
Putting these relationships together, gives the change in the spacing between the vertices
caused by the fluid flow as
$ \gD \surp{dx} \equiv dx_{t+dt} - dx_{t} = \partial V_x / \partial x  \push dx$.
Repeating the same procedure for both the $y$- and $z$-directions and substituting
the resulting relations into the expression for the volume gives the change in the
volume in the time $dt$ as
\bea
  \dby{t} dV & = &  \surp{ \pdxby{V_x}{x} + \pdxby{V_y}{y} + \pdxby{V_z}{z} } dV \nonumber \\
             & = &  \surp{ \ipd{\cv{\nabla}}{\cv{V}} } dV \eqp
\eea

the conservation of mass can also be written as either
\be
    \gr(\cv{z},t) \push d^3z = \gr(\cv{a},0) \push d^3 a
\ee
or by introducing the Jacobian determinant
$J ={\partial(z_1,z_2,z_3)}/{\partial(a_1,a_2,a_3)}$ as
\begin{equation}\label{eq:rho}
    \rho J({\vec z}(\a,t)) = \rho_0(\a) \eqp
\end{equation}

The density \gr depends on the family of trajectories $\cv{z}(\cv{a},t)$ under
consideration and is entirely determined by it.

The conservation of mass is a general law that applies to all classical fluids.  However, a full
description of the fluid requires knowledge of five basic variables; three velocity components
and two thermodynamic properties \refs{HB67}.  Knowledge of any two thermodynamic quantities
along with an equation of state fully specifies the remaining parameters.  For the current
purposes, the fluid density and the entropy are the thermodynamic functions of choice and
the fluid is assumed to be ideal.  An ideal fluid is a model of fluid motion for those
fluids in which thermal conductivity and viscosity are unimportant \refs{LL59}.
The absence of viscosity means that an ideal fluid suffers no shearing stress, even
when the fluid is in motion, and the sole stress is due to the pressure \refs{symon71}.
In addition, an ideal fluid does not take into account heat transfer between elements
and the motion throughout the fluid is adiabatic \refs{LL59}.

Applying this condition yields the second and final constraint of ideal fluid motion.
From thermodynamics, the adiabatic assumption is expressed as $\gD q = T ds = 0$, where
$\gD q$ is the heat flow per unit mass and $s$ is the specific entropy \refs{callen85}.
Thus the requirement of adiabatic flow means that the entropy is conserved.
For simplicity, the composition ideal fluid will be modelled as only one chemical constituent
with a fixed number of particles.
Under this assumption, the first law of thermodynamics can be written as \refs{callen85}
\bes
  de = T ds - P d \surp{\frac{1}{\gr}} \eqp
\ees
In this form of the first law, $e$ is the specific internal energy (internal energy
per unit mass) and the quantities $T$ and $P$ are temperature and pressure, respectively.
Substituting in the adiabatic requirement simplifies the first law
\be
  d e =   \frac{P}{\gr^2} d \gr = \surp{ \pdxby{e}{\gr} }_{s} d \gr \eqp
\ee

The work done by the fluid as it expands is


The final consideration is to determine the equation of motion for the ideal fluid.
Consider Figure \refp{fig:pressure_forces}, which shows the pressures on the faces of
a fluid element located at the point $\mP = (x,y,z)$.
The net pressure force in the x-direction, \idx{\cv{F}}{\dn{P_x}} is given by the difference between
the pressure forces on the left and right faces.
This difference is given by
\beas
\idx{\cv{F}}{\dn{P_x}} & = &  \idx{\cv{F}}{\dn{\idx{P}{\dn{L.F.}}}} -
                            \idx{\cv{F}}{\dn{\idx{P}{\dn{R.F.}}}}  \\
                     & = &   P(x,y,z) \nudge \gd y \nudge \gd z \nudge \cuv{\imath}
                           - P(x+\gd x,y,z) \nudge \gd y  \nudge \gd z \nudge \cuv{\imath} \\
                     & = & - \pdxby{P(x,y,z)}{x} \gd x \nudge \gd y
                             \nudge \gd z \nudge \cuv{\imath} \eqp
\eeas
Repetition of the computation in the y- and z-directions gives the pressure force on
the ideal fluid element, \idx{\cv{F}}{\dn{P}}
\bes
  \idx{\cv{F}}{\dn{P}} = - \cv{\nabla} P(x,y,z) \nudge \gd x \nudge \gd y \nudge \gd z
                       = - \cv{\nabla} P(x,y,z) \nudge \gd V \eqp
\ees
Newton's second law for the constant mass fluid element is now written as
\bes
  m \dxby{\cv{v}}{t} = - \cv{\nabla} P \nudge \gd V \eqp
\ees
Using the relation $m = \gr \nudge \gd \nudge V$, the equation can be simplified and
the expression
\bes
  \dxby{\cv{v}}{t} = - \frac{\cv{\nabla P}}{\gr}
\ees
obtains.
If a body force, \cv{f}, such as the gravitational attraction of the fluid element
by an external body, is present the equation is modified to read
\be\label{eq:Euler_eq}
  \dxby{\cv{v}}{t} = -\frac{\cv{\nabla} P}{\gr} + \cv{f} \eqp
\ee
Equation \refq{eq:Euler_eq} is known as Euler's equation for an ideal fluid in the
Lagrangian picture.  Note that the velocity of the fluid element in \refq{eq:Euler_eq}
is labelled by the coordinates $(x,y,z)$.

Finally,

\begin{equation}
    s(\a,t) = s({\vec a},0) = s_0(\vec a) \eqp
\end{equation}


\begin{figure}
\centerline{
   \includegraphics[height=2in,width=3in,keepaspectratio]{pressure_force.eps}}
   \caption{The pressure forces on a test element located at the point $\mP = (x,y,z)$
   in an ideal fluid.  Gradients of the pressure lead to an imbalance of contact
   forces on the test element which cause the fluid to accelerate.}\label{fig:pressure_forces}
\end{figure}


Although MSY incorporate these constraints in a variational principle by the
use of the Lagrange multipliers, we find the equations shorter and easier to
read if one takes $\rho$ to be defined by Eqn.~(\ref{eq:rho}).
The MSY Lagrangian then is:

\begin{equation}\label{eq:MSY}
    I  =  \int\!\! dt \int\!\! d^3a \left[\frac{\rho_0}{2} \left(
          \frac{\partial z_i}{\partial t}\right)^2 - \rho_0(e+\Phi) \right] \eqc
\end{equation}
%
where $e(\rho,s)$ is the specific internal energy and $\Phi(\z(\a,t),t)$ is the
gravitational potential energy per unit mass at the position of particle $\a$.
In this formalism, ${\vec z}(\a,t)$ are the dynamical variables that are varied
to produce the equations of motion.

The variation of the action with respect to the trajectory function ${\vec z}$
requires the various properties of the Jacobian that MSY employ in their
formulation. We must remember that since $\rho$ is defined by
Eqn.~(\ref{eq:rho}) both $\rho$ and $e(\rho,s)$ will vary when $\z$ is varied.
Thus we find

\begin{equation}
    \delta I = \int\!\! dt \int\!\! d^3a \left[ \rho_0
    \left(\frac{\partial z_i}{\partial t}\right)\delta
    \left(\frac{\partial z_i}{\partial t} \right)
    - \rho_0 \frac{P}{\rho^2} \delta \rho
    - \rho_0 \frac{\partial \Phi}{\partial z_i} \delta {z_i}
    \right] \eqp
\end{equation}
%
Here we have used the thermodynamic relationship $de = T\,ds -P\,d(1/\rho)$ in
the form $(\partial e/\partial \rho)_s = P/\rho^2$ in evaluating how the
internal energy $e$ changes when the trajectory variations cause changes in
$\rho$ but not in $s$ at a particular fluid element $\a$.  The problematic
term here is that containing $\delta \rho$.  But from $\rho J = \rho_0(\a)$
where the right hand side is independent of $\z$ one has
$J \delta \rho + \rho \delta J = 0$.  Thus the term
$-(\rho_0 {P}/{\rho^2}) \delta \rho$ becomes
$-(P/\rho) J \delta \rho = + P \delta J$.  Using $\delta J = ({\partial J}/
{\partial z_{i,j}}) \delta z_{i,j} = J_{ij} \delta z_{i,j}$,
where $z_{i,j} = {\partial z_i}/{\partial a_j}$ and $J_{ij}$ is the (i,j)
minor of the Jacobian, the second ($P$) term can be simplified.  Exchanging
the order of the variation and the partial differentiation in the first and
second terms and integrating these terms by parts gives

\begin{equation}
    \delta I = -\int\!\! dt \int\!\! d^3a \left[ \rho_0
    \left(\frac{\partial^2 z_i}{\partial t^2}\right)
    + \frac{\partial}{\partial
    a_j}\left(P J_{ij}\right)
    + \rho_0 \frac{\partial \Phi}{\partial z_i}
    \right] \delta z_i \eqp
\end{equation}

Setting this variation to zero leads to the partial differential equation

\begin{equation}\label{EL}
    \rho_0 \left(\frac{\partial^2 z_i}{\partial t^2}\right) + \rho_0
    \frac{\partial \Phi}{\partial z_i} + \frac{\partial}{\partial
    a_j}\left(P J_{ij}\right) = 0 \eqp
\end{equation}
%
From the relations ${\partial J_{ij}}/{\partial a_j} = 0$, the last term becomes

\begin{equation}
    \frac{\partial}{\partial a_j}\left(P  J_{ij} \right) =
    \frac{\partial P}{\partial a_j} J_{ij} = \frac{\partial P}{\partial
    z_k}\frac{\partial z_k}{\partial a_j} J_{ij} \eqp
\end{equation}
%
Finally by using $({\partial z_k}/{\partial a_j}) J_{ij} = J \delta_{ik}$ and
$J = \rho_0 / \rho$ we put Eq.~(\ref{EL}) into the familar form

\begin{equation}\label{eq:Euler}
    \left(\frac{\partial^2 z_i}{\partial t^2}\right) + \frac{\partial
    \Phi}{\partial z_i} + \frac{1}{\rho} \frac{\partial P}{\partial z_i} =
    0 \eqp
\end{equation}



%%%%%%%%%%%%%%%%%%%%%%%%%%%%%%%%%%%%%%%%%%%%%%%%%%%%%%%%%%%%%%%%%%%%%%
%%%%%%%%%%%%%%%%%%%%%%%%%%%%%%%%%%%%%%%%%%%%%%%%%%%%%%%%%%%%%%%%%%%%%%
%%%%%%%%%%%%%%%%%%%%%%%%%%%%%%%%%%%%%%%%%%%%%%%%%%%%%%%%%%%%%%%%%%%%%%
In discrete approximations later there will only be a finite number of particle
trajectories $\z$ computed, and fields such as the gravitational potential
$\Phi(\x)$ will be computed only on a finite set of grid locations $\x$.

The goal of making and combining these two discretization approximations makes
it important that a clear distinction be kept in mind between the particle
trajectories $\z$ and the field points $\x$, even though the equation
$\x = \z(t)$ imbeds a particle trajectory within the grid where field values
will be known.

\section{Smooth Particle Hydrodynamics}\label{sph}

In this section, we review some of the standard SPH equations
\cite{A:NP94,A:JJM92,A:WB89,A:GM82,A:PJM91} that we used in approximating the
continuum variational principle in discrete form.

The single most important relation that should be stated is the idea of
approximating the equation

\begin{equation}
    f({\vec z}) = \int\!\! d^3z'\,\delta({\vec z} - \vec{ z'}) f(\vec{z'})
\end{equation}
%
by the relation
\begin{equation}\label{sph_fun}
    \langle f({\vec z}) \rangle \simeq \int\!\! d^3z'\,W({\vec z}
    - \vec{z'}; h) f(\vec{z'}) \eqp
\end{equation}

The essential properties of the smoothing kernel $W$ are that it is normalized
\begin{equation}\label{norm}
    \int\!\! d^3z\,W({\vec z};h) = 1 \eqc
\end{equation}
%
and that it has a delta function limit

\begin{equation}
    \lim_{h \rightarrow 0} W({\vec z} - \vec{z'};h) = \delta({\vec z}
    - \vec{z'}) \eqp
\end{equation}

In addition, we assume that we are always using a symmetric, even,
non-negative kernel so that the smoothing approximation is $O(h^2)$.
In other words,

\begin{equation}
    \langle f({\vec z}) \rangle = f({\vec z}) + c \left( \nabla^2 f({\vec z}) \right) h^2 \eqc
\end{equation}
%
where $c$ is a constant independent of $h$.

Gingold and Monaghan \cite{A:GM82} motivate a derivation of SPH starting from
the Lagrangian for a nondissipative, isentropic gas which in our notation reads

\begin{equation}
    L = \int\!\! d^3a\,\rho_0 \left[ \frac{1}{2} {\dot {\vec z}\,}^2 -
        e(\rho) - \Phi(z,t)\right] \eqc
\end{equation}
%
where ${\partial e}/{\partial \rho} = {P}/{\rho^2}$ and where $z$ means
$z(\a,t)$.  Then they write the integral in discrete form

\begin{equation}
    L \simeq \sum_{a=1}^{N} m_a \left( \frac{1}{2} {\dot {\vec z_a}}^2 -
    e(\rho_a)  - \Phi(\vec z_a, t) \right) \eqp
\end{equation}
%
Calculating the terms in the Euler-Lagrange equations gives the SPH
momentum equation

\begin{equation}
    m_a \ddot {\vec z_a}  + m_a \left( \frac{\partial \Phi}{\partial
                                    \x} \right)_{\x = \z_a}
    + \sum_{b=1}^{N} \frac{P_b}{\rho_b^2}
    \frac{\partial \rho_b}{\partial {\vec z_a}} m_b
    = 0 \eqp
\end{equation}


Connection with SPH can be made by identifying

\begin{equation}\label{eq:smoothed_rho}
    \rho_b = \sum_{c=1}^{N} m_c W ( {\vec z_b} - {\vec z_c}; h ) \eqc
\end{equation}
%
which leads to the symmetric momentum equation

\begin{equation}\label{eq:sym_sph}
    {\ddot {\vec z_a}}  + \left( \frac{\partial \Phi}{\partial
                                    \x} \right)_{\x = \z_a}
    + \sum_{b=1}^{N} \left( \frac{P_b}{\rho_b^2} +
    \frac{P_a}{\rho_a^2} \right) m_b \nabla_a W ( {\vec z_a} - {\vec
    r_b}; h ) \eqc
\end{equation}
%
that Monaghan \cite{A:JJM92,A:GM82,A:PJM91} advocates for its ability to
conserve linear and angular momentum.

Note that two approximation steps are invoked here.  One is the replacement of
$\int\!\! d^3a\,\rho_0$ by $\sum_{a=1}^N m_a$ which is a Monte Carlo sampling
of the continuum of fluid elements by a finite subset.  The second step is the
estimate (\ref{eq:smoothed_rho}) of the continuum density from this finite
sample.  The normalization condition (\ref{norm}) provides a consistency
between these two steps, but we do not see that one can be derived from
the other.

%%%%%%%%%%%%%%%%%%%%%%%%%%%%%%%%%%%%%%%%%%%%%%%%%%%%%%%%%%%%%%%%%%%%%%
%%%%%%%%%%%%%%%%%%%%%%%%%%%%%%%%%%%%%%%%%%%%%%%%%%%%%%%%%%%%%%%%%%%%%%
%%%%%%%%%%%%%%%%%%%%%%%%%%%%%%%%%%%%%%%%%%%%%%%%%%%%%%%%%%%%%%%%%%%%%%
\section{Newtonian Gravity}\label{Newton}

In the preceding sections Newtonian gravity was included as a potential $\Phi$
due to external masses.  When the self-gravitation of the fluid is to be
included more care is needed.  Unlike usual Newtonian SPH, we do not want to
think of gravitation as a mutual interaction of the smoothed particles---
this viewpoint, effective in Newtonian problems, will not provide guidance
for the relativistic problems we aim to formulate.  Instead we write a
field theory.\\

A variational principle which gives rise to the Poisson equation is
$\delta I_G = 0$ with

\begin{equation}\label{eq:I-G}
    I_G = -\int\!\!dt\!\int\!\!d^3x \left[
            (1/8\pi G)(\nabla \Phi)^2 + \rho(\x,t) \Phi(\x,t) \right] \eqp
\end{equation}
%
Varying $\Phi$ here gives

\begin{equation}\label{eq:Poisson}
    \nabla^2 \Phi = 4 \pi G \rho  \eqc
\end{equation}
%
with solutions such as $\Phi = -GM/r$.

A continuum variational principle extending that of Sec.~\ref{cfm} to include
the field dynamics is $\delta I = 0$ with $I = \int\!L\,dt$ and

\begin{eqnarray}\label{eq:full-L}
    L & = & \int\!\! d^3a \left[\frac{\rho_0}{2} \left(
        \frac{\partial z_i}{\partial t}\right)^2
      - \rho_0\, e(\rho,s_0) - \rho_0\, \Phi(\z(\a,t),t) \right]
       \nonumber \\
      & & - (1/8\pi G)\int\!\!d^3x\,\nabla \Phi(\x,t)^2  \eqp
\end{eqnarray}

The terms here containing $\z(\a,t)$ are exactly those considered in
Sec.~\ref{cfm}, so the fluid equations are just (\ref{eq:Euler}).  But to see
that Eq.~(\ref{eq:Poisson}) also results we need to rewrite the $\rho_0\Phi$
term to see that it is the same as in Eq.~(\ref{eq:I-G}).  This we do by
invoking the definition of $\rho$ in a change of variables $\rho_0\,d^3a
= \rho\,d^3z$ in the integral

\begin{equation}\label{eq:rhoPhi}
    \int\!\! d^3a\, \rho_0(\a)\, \Phi(\z(\a,t),t) =
    \int\!\! d^3z\, \rho(\z,t)\, \Phi(\z,t) =
    \int\!\! d^3x\, \rho(\x,t)\, \Phi(\x,t) \eqc
\end{equation}
%
where the last step is a notational change of the dummy variable
of integration.  Thus Eq.~(\ref{eq:Poisson}) also results by varying
$\Phi(\x)$ in this combined Lagrangian (\ref{eq:full-L}).

It is important to note that only the first form of the interaction term
(\ref{eq:rhoPhi}) as given in Eq.~(\ref{eq:full-L}) is acceptable in the
fundamental Lagrangian.  For in making the change of variables from $\a$ to
$\z$ in equation (\ref{eq:rhoPhi}) one has assumed a definite fluid motion
$\z(\a,t)$. Since reference to this particular motion disappears in the
$\int\!\! d^3x\, \rho(\x,t)\, \Phi(\x,t)$ form of this term, it would not be
possible to carry out the $\delta \z$ variations were this form to be stated
as part of the basic Lagrangian.  We have used it here only to carry out a
variation of $\Phi$ while holding the fluid motion $\z(\a,t)$ unchanged.


%%%%%%%%%%%%%%%%%%%%%%%%%%%%%%%%%%%%%%%%%%%%%%%%%%%%%%%%%%%%%%%%%%%%%%
%%%%%%%%%%%%%%%%%%%%%%%%%%%%%%%%%%%%%%%%%%%%%%%%%%%%%%%%%%%%%%%%%%%%%%
%%%%%%%%%%%%%%%%%%%%%%%%%%%%%%%%%%%%%%%%%%%%%%%%%%%%%%%%%%%%%%%%%%%%%%
\section{Discretization and Numerical Formulation}\label{Disc.}

At this point, the basic pieces are in place and all that remains is to combine
and interpret them.  Start from the complete action integral, which is written
as
\begin{eqnarray}
I  & = & - \int\!\! dt \int\!\! d^3x
           \frac{\left[\nabla \Phi(\x,t)\right]^2}{8 \pi G}
         + \int\!\! dt \int\!\! d^3a \rho_0 \left[ \left(\frac{\partial \z}
          {\partial t}\right)^2 - e \right] \nonumber \\
   &   & - \int\!\! dt \int\!\! d^3 a \rho_0 \Phi(\z,t) \eqc
\end{eqnarray}
%
subject to the constraint of $\rho J = \rho_0$ and $s = s_0$.  Assume, as
discussed above, that the fields will be numerically determined on a
computational grid and that the particles will be labeled and their individual
motion followed.  The discretization of the field becomes
$\Phi(\x,t) \rightarrow \Phi_\l(t)$ where $\l$ represents the triple integer
set $\left\{\l_x, \l_y, \l_z\right\}$.  Likewise the particle discretization
becomes $ \int d^3 \rho_0 \rightarrow \sum_{a} m_a$.

Under these definitions, the action becomes
\begin{eqnarray}
I & = & - \int\!\! dt \sum_{l} \frac{1}{8 \pi G}
        \left\{
         \left[ \D_x \Pl - \Pl \right]^2 +
         \left[ \D_y \Pl - \Pl \right]^2 +
         \left[ \D_z \Pl - \Pl \right]^2
        \right\} \nonumber \\
  &   &
      + \int\!\! dt \sum_{a} m_a \left[ \left(\frac{\partial \z_a}
         {\partial t}\right)^2 - e(\rho_a) \right]
      - \int\!\! dt \sum_{a} m_a \Phi(\z_a,t) \eqc
\end{eqnarray}
where the $\Delta$ operator changes the corresponding '$l$'-label on any object
possessing it by one ({\it e.g.} $\Delta_x \Pl - \Pl = \Phi_{\l_x+1,\l_y,\l_z} -
\Phi_{\l_x,\l_y,\l_z}$).

The factor $\Phi(\vec z_a)$ in the interaction term is undefined since
$\vec z_a$ will not normally coincide with one of the lattice points
$\vec x_\ell$.  Thus some interpolation is required, and we can set
\begin{equation}\label{eq:interpolate}
      \Phi(\z_a) = \frac{\sum_\l h^3 \Phi_\l I(\z_a - \x_\l)}
                  {\sum_n h^3 I(\z_a -\x_n)} \eqc
\end{equation}
%
where the denominator ensures that interpolation of a constant function yields
that constant value.  The kernel $I$ chooses a linear combination of nearby
grid values $\Pl$ to estimate $\Phi$ at any desired point and provides an
approximation to the delta function in

\begin{equation}\label{eq:approxDelta}
    \Phi(\z,t) = \int\!\!d^3x\,\Phi(\x,t) \delta^3(\z-\x) \eqp
\end{equation}

Setting the variation of this discrete action with respect to $\Pl(t)$ gives

\begin{eqnarray}\label{eq:dis_Poi}
  \left( 2 \P_m - \D_{-x} \P_m - \D_x \P_m \right) +
  \left( 2 \P_m - \D_{-y} \P_m - \D_x \P_m \right) \nonumber & & \\
+ \left( 2 \P_m - \D_{-z} \P_m - \D_x \P_m \right) & = & 4 \pi G \sum_{a} m_a
  \frac{ h^3 I(\z_a - \x_m)}{\sum_n h^3 I(\z_a -\x_n)} \eqc
\end{eqnarray}
%
which is the expected discrete form of the Poisson equation with the central
differencing approximation to the second derivative.  The source term then
gives an effective definition of the density

\begin{equation}\label{eq:dis_den}
   \rho(\x_l) = \sum_{a} m_a \frac{ h^3 I(\z_a - \x_\l)}{\sum_n h^3 I(\z_a -\x_n)} \eqc
\end{equation}
which is analogous to the one invoked in (\ref{eq:smoothed_rho}).  We will take
this as our definition of an interpolated density.

It is instructive to note that summing (\ref{eq:dis_Poi}) over $m$ gives

\begin{equation}
  \sum_{\partial m} \P_m = 4 \pi G \sum_{a} m_{a} \eqc
\end{equation}

where the $\sum_{\partial m}$ represents a summation over the boundary grid
points.  This is just a discrete form of Gauss's law.

Finally, the variation with respect to the trajectory function $\z_b(t)$ must
be performed.  Setting the functional derivative of the discrete action with
respect to $\z_b(t)$ equal to zero yields:

\begin{eqnarray}\label{eq:dis_particle}
  m_b {\ddot \z}_b + \sum_a \left[ m_a \frac{P(\z_a)}{\rho^2(\z_a)}
                                       \frac{\partial \rho(\z_a)}{\partial \z_b}
                                 + m_a \frac{\partial \P(\z_a)}{\partial \z_b}
                            \right] = 0
\end{eqnarray}

After some straightforward but lengthy calculations one arrives at

\begin{eqnarray}\label{eq:dis_drho}
  \sum_a m_a \frac{P(\z_a)}{\rho^2(\z_a)}
             \frac{\partial \rho(\z_a)}{\partial \z_b} & = &
    m_b \sum_a m_a
      \left(
        \frac{P(\z_a)}{\rho^2(\z_a)} + \frac{P(\z_b)}{\rho^2(\z_b)}
      \right)
      \frac{ \nabla_b I(\z_b - \z_a) h^3}{\sum_\l h^3 I(\z_b - \x_\l)}
    \nonumber \\
  & &   -
    m_b \sum_a m_a
      \frac{P(\z_a)}{\rho^2(\z_a)} h^3 I(\z_b - \z_a)
      \frac{\sum_m h^3 \nabla_b I(\z_b - \x_m)}
        {\left[ \sum_\l h^3 I( \z_b - \x_\l) \right] ^2}
\end{eqnarray}
%
and
%
\begin{eqnarray}\label{eq:dis_dphi}
  \sum_a m_a \frac{\partial \P(\z_a)}{\partial \z_b}
       = m_b \frac{ \sum_\l h^3 \nabla_b I (\z_b - \x_\l) \Pl
             \sum_n h^3 I(\z_b - \x_n)}{\left[
             \sum_m h^3 I( \z_b - \x_m) \right] ^2} \nonumber \\
           - \frac{\sum_\l h^3 \nabla_b I(\z_b - \x_\l)
             \sum_n h^3 I(\z_b - \x_n) \P_n}
            {\left[ \sum_m h^3 I( \z_b - \x_m) \right] ^2}
\end{eqnarray}
for the second and third terms respectively.  Despite their initial appearance,
there is much that is familiar.  Equation (\ref{eq:dis_drho}) is essentially
the symmetric momentum equation (\ref{eq:sym_sph}) discussed in Section
\ref{sph}.  Equation (\ref{eq:dis_dphi}) contains the derivatives of the
gravitational potential one expects from interpolating from the grid

\begin{equation}\label{eq:gravForce}
    \left( \frac{\partial \Phi}{\partial \x} \right)_{\x = \z_a}
    \mapsto
    \sum_\l h^3  \Pl(t)
         \frac{\partial I({\x_\l} - {\z_a})}{\partial \z_a} \eqp
\end{equation}
Both terms are complicated by the presence of derivatives with respect to the
kernel normalization but these extra terms are analogous to those terms arising
when $\nabla h$ terms are accounted for in conventional SPH \cite{A:NP94}.

The implementation scheme is now clear (if not concise).  Specify an initial
distribution of matter ({\it i.e.} $m_a$ and $\z_a(t=0)$).  Use
(\ref{eq:dis_Poi}) to determine the gravitational potential at the grid
locations.  Use (\ref{eq:dis_drho}) and (\ref{eq:dis_dphi}) in conjunction with
(\ref{eq:dis_particle}) to calculate the accelerations of the particles in
terms of known quantities.  Integrate the particles equations of motion on time
step and repeat the process.  Note that the causality inherent in this scheme
is Galilean; absolute time has meaning, the gravitational force is essentially
instantaneous, and there is no mechanism to propagate a solution that satisfies
the initial-value problem forward (or backward) in time.  It is for this reason
that the Poisson equation must be solved at each integration step.
