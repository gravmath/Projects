%Simple document on the Riemann tensor
\documentclass[12pt]{article}
%load packages
\usepackage{latexsym}
\usepackage{epic,eepic,graphicx,url}
%spacer commands
\newcommand{\eqc}{\ensuremath{\: ,} }
\newcommand{\eqp}{\ensuremath{\: .} }
\newcounter{bean}
\newcommand{\surp}[1]{\ensuremath{\left( {#1} \right)} }
\newcommand{\surb}[1]{\ensuremath{\left[ {#1} \right]} }
\newcommand{\surc}[1]{\ensuremath{\left\{ {#1} \right\}} }
\newcommand{\sura}[1]{\ensuremath{\left\langle #1 \right\rangle}}
\newcommand{\sure}[1]{({#1})}
\newcommand{\push} {\ensuremath{\:}}
\newcommand{\nudge}{\ensuremath{\,}}
\newcommand{\back}{\ensuremath{\!\!}}
\newcommand{\iback}{\ensuremath{\!\!\!\!}}
%equation commands
\newcommand{\be} { \begin{equation} }
\newcommand{\ee} { \end{equation}   }
\newcommand{\bes} { \[ }
\newcommand{\ees} { \]  }
\newcommand{\bea}{ \begin{eqnarray} }
\newcommand{\eea}{ \end{eqnarray}   }
\newcommand{\beas}{ \begin{eqnarray*} }
\newcommand{\eeas}{ \end{eqnarray*}   }
%index commands
\newcommand{\idx}[2] {\ensuremath{ {#1}{#2}   }}
\newcommand{\up} [1] {\ensuremath{{}^{#1}     } }
\newcommand{\dn} [1] {\ensuremath{{}_{#1}     } }
%array commands
\newcommand{\cv}  [1]{\ensuremath{\vec {#1}} }
\newcommand{\rv}  [1]{\ensuremath{{\vec {#1}}^{\,T}} }
\newcommand{\cf}  [1]{\ensuremath{\tilde {#1}} }
\newcommand{\rf}  [1]{\ensuremath{{\tilde {#1}}{\,^T}} }
\newcommand{\cs}  [1]{\ensuremath{\bar {#1}} }
\newcommand{\rs}  [1]{\ensuremath{{\bar {#1}}{\,^T}} }
\newcommand{\cuv} [1]{\ensuremath{\hat {#1}} }
\newcommand{\ruv} [1]{\ensuremath{{\hat {#1}}{\,^T}} }
\newcommand{\op}  [1]{\ensuremath{\mathbf{{#1}}}}
\newcommand{\dt}  [1]{\ensuremath{\dot {{#1}}} }
\newcommand{\ddt} [1]{\ensuremath{\ddot {{#1}}} }
\newcommand{\trps}[1]{\ensuremath{{#1}^{\,T} } }
\newcommand{\bv}  [1]{\ensuremath{\cs{e}_{#1}} }
\newcommand{\bd}  [1]{\ensuremath{\cf{\gw}^{#1}} }
%\newcommand{\rank}[2]{\ensuremath{\surp{\begin{array}{c}{#1 \\ #2}\end{array}}} }
\newcommand{\rank}[2]{\ensuremath{\surp{#1, #2}}}
%inner products
\newcommand{\ipp}[2]{\ensuremath{ (  {#1} ,     {#2} ) } }
\newcommand{\ipb}[2]{\ensuremath{ \langle {#1} ,     {#2} \rangle } }
\newcommand{\ipD}[2]{\ensuremath{ \langle {#1} |     {#2} \rangle } }
\newcommand{\ipd}[2]{\ensuremath{         {#1} \cdot {#2}         } }
%derivative and integral commands
\newcommand{\dby}   [1]{\ensuremath{ \frac{d}{d #1}} }
\newcommand{\dxby}  [2]{\ensuremath{ \frac{d #1}{d #2}} }
\newcommand{\pdby}  [1]{\ensuremath{ \frac{\partial}{\partial #1}} }
\newcommand{\pdxby} [2]{\ensuremath{ \frac{\partial #1}{\partial #2}} }
\newcommand{\myint} [1]{\ensuremath{ \int #1 \push} }
%variations
\newcommand{\var}[1]{\ensuremath{\delta {#1}}}
\newcommand{\vwrt}[2]{\ensuremath{\idx{\delta {#1}\vert}{\dn{\delta {#2}}}} }
\newcommand{\IBP}{\ensuremath{\stackrel{\mbox{\tiny{IBP}}}{=}}}
%common abreviations
\newcommand{\wrt}{with respect to }
\newcommand{\bdyterms}{boundary-terms }
\newcommand{\ibp}{integration-by-parts }
\newcommand{\dfunc}{\gd\back-function }
\newcommand{\dfuncs}{\gd\back-functions }
%special characters
\newcommand{\ga}{\ensuremath{\alpha} }
\newcommand{\gba}{\ensuremath{\bar \alpha} }
\newcommand{\gha}{\ensuremath{\hat \alpha} }
\newcommand{\gta}{\ensuremath{\tilde \alpha} }
\newcommand{\gb}{\ensuremath{\beta} }
\newcommand{\gbb}{\ensuremath{\bar \beta} }
\newcommand{\ghb}{\ensuremath{\hat \beta} }
\newcommand{\gtb}{\ensuremath{\tilde \beta} }
\newcommand{\gd}{\ensuremath{\delta} }
\newcommand{\gbd}{\ensuremath{\bar \delta} }
\newcommand{\ghd}{\ensuremath{\hat \delta} }
\newcommand{\gtd}{\ensuremath{\tilde \delta} }
\newcommand{\gD}{\ensuremath{\Delta} }
\newcommand{\gbD}{\ensuremath{\bar \Delta} }
\newcommand{\ghD}{\ensuremath{\hat \Delta} }
\newcommand{\gtD}{\ensuremath{\tilde \Delta} }
\newcommand{\get}{\ensuremath{\eta} }
\newcommand{\gbet}{\ensuremath{\bar \eta} }
\newcommand{\ghet}{\ensuremath{\hat \eta} }
\newcommand{\gtet}{\ensuremath{\tilde \eta} }
\newcommand{\gf}{\ensuremath{\phi} }
\newcommand{\gbf}{\ensuremath{\bar \phi} }
\newcommand{\ghf}{\ensuremath{\hat \phi} }
\newcommand{\gtf}{\ensuremath{\tilde \phi} }
\newcommand{\gF}{\ensuremath{\Phi} }
\newcommand{\gbF}{\ensuremath{\bar \Phi} }
\newcommand{\ghF}{\ensuremath{\hat \Phi} }
\newcommand{\gtF}{\ensuremath{\tilde \Phi} }
\newcommand{\gG}{\ensuremath{\Gamma} }
\newcommand{\gbG}{\ensuremath{\bar \Gamma} }
\newcommand{\ghG}{\ensuremath{\hat \Gamma} }
\newcommand{\gtG}{\ensuremath{\tilde \Gamma} }
\newcommand{\ggm}{\ensuremath{\gamma} }
\newcommand{\gbgm}{\ensuremath{\bar \gamma} }
\newcommand{\ghgm}{\ensuremath{\hat \gamma} }
\newcommand{\gtgm}{\ensuremath{\tilde \gamma} }
\newcommand{\gl}{\ensuremath{\lambda} }
\newcommand{\gbl}{\ensuremath{\bar \lambda} }
\newcommand{\ghl}{\ensuremath{\hat \lambda} }
\newcommand{\gtl}{\ensuremath{\tilde \lambda} }
\newcommand{\gL}{\ensuremath{\Lambda} }
\newcommand{\gbL}{\ensuremath{\bar \Lambda} }
\newcommand{\ghL}{\ensuremath{\hat \Lambda} }
\newcommand{\gtL}{\ensuremath{\tilde \Lambda} }
\newcommand{\gm}{\ensuremath{\mu} }
\newcommand{\gbm}{\ensuremath{\bar \mu} }
\newcommand{\ghm}{\ensuremath{\hat \mu} }
\newcommand{\gtm}{\ensuremath{\tilde \mu} }
\newcommand{\gn}{\ensuremath{\nu} }
\newcommand{\gbn}{\ensuremath{\bar \nu} }
\newcommand{\ghn}{\ensuremath{\hat \nu} }
\newcommand{\gtn}{\ensuremath{\tilde \nu} }
\newcommand{\gp}{\ensuremath{\pi} }
\newcommand{\gbp}{\ensuremath{\bar \pi} }
\newcommand{\ghp}{\ensuremath{\hat \pi} }
\newcommand{\gtp}{\ensuremath{\tilde \pi} }
\newcommand{\gq}{\ensuremath{\theta} }
\newcommand{\gbq}{\ensuremath{\bar \theta} }
\newcommand{\ghq}{\ensuremath{\hat \theta} }
\newcommand{\gtq}{\ensuremath{\tilde \theta} }
\newcommand{\gr}{\ensuremath{\rho} }
\newcommand{\gbr}{\ensuremath{\bar \rho} }
\newcommand{\ghr}{\ensuremath{\hat \rho} }
\newcommand{\gtr}{\ensuremath{\tilde \rho} }
\newcommand{\gs}{\ensuremath{\sigma} }
\newcommand{\gbs}{\ensuremath{\bar \sigma} }
\newcommand{\ghs}{\ensuremath{\hat \sigma} }
\newcommand{\gts}{\ensuremath{\tilde \sigma} }
\newcommand{\gt}{\ensuremath{\tau} }
\newcommand{\gbt}{\ensuremath{\bar \tau} }
\newcommand{\ght}{\ensuremath{\hat \tau} }
\newcommand{\gtt}{\ensuremath{\tilde \tau} }
\newcommand{\gw}{\ensuremath{\omega} }
\newcommand{\gbw}{\ensuremath{\bar \omega} }
\newcommand{\ghw}{\ensuremath{\hat \omega} }
\newcommand{\gtw}{\ensuremath{\tilde \omega} }
\newcommand{\gW}{\ensuremath{\Omega} }
\newcommand{\gbW}{\ensuremath{\bar \Omega} }
\newcommand{\ghW}{\ensuremath{\hat \Omega} }
\newcommand{\gtW}{\ensuremath{\tilde \Omega} }
\newcommand{\gy}{\ensuremath{\psi} }
\newcommand{\gby}{\ensuremath{\bar \psi} }
\newcommand{\ghy}{\ensuremath{\hat \psi} }
\newcommand{\gty}{\ensuremath{\tilde \psi} }
\newcommand{\gY}{\ensuremath{\Psi} }
\newcommand{\gbY}{\ensuremath{\bar \Psi} }
\newcommand{\ghY}{\ensuremath{\hat \Psi} }
\newcommand{\gtY}{\ensuremath{\tilde \Psi} }
%Script letters
\newcommand{\mA}{\ensuremath{\mathcal A} }
\newcommand{\mB}{\ensuremath{\mathcal B} }
\newcommand{\mC}{\ensuremath{\mathcal C} }
\newcommand{\mD}{\ensuremath{\mathcal D} }
\newcommand{\mE}{\ensuremath{\mathcal E} }
\newcommand{\mF}{\ensuremath{\mathcal F} }
\newcommand{\mG}{\ensuremath{\mathcal G} }
\newcommand{\mH}{\ensuremath{\mathcal H} }
\newcommand{\mI}{\ensuremath{\mathcal I} }
\newcommand{\mJ}{\ensuremath{\mathcal J} }
\newcommand{\mK}{\ensuremath{\mathcal K} }
\newcommand{\mL}{\ensuremath{\mathcal L} }
\newcommand{\mM}{\ensuremath{\mathcal M} }
\newcommand{\mN}{\ensuremath{\mathcal N} }
\newcommand{\mO}{\ensuremath{\mathcal O} }
\newcommand{\mP}{\ensuremath{\mathcal P} }
\newcommand{\mQ}{\ensuremath{\mathcal Q} }
\newcommand{\mR}{\ensuremath{\mathcal R} }
\newcommand{\mS}{\ensuremath{\mathcal S} }
\newcommand{\mT}{\ensuremath{\mathcal T} }
\newcommand{\mU}{\ensuremath{\mathcal U} }
\newcommand{\mV}{\ensuremath{\mathcal V} }
\newcommand{\mW}{\ensuremath{\mathcal W} }
\newcommand{\mX}{\ensuremath{\mathcal X} }
\newcommand{\mY}{\ensuremath{\mathcal Y} }
\newcommand{\mZ}{\ensuremath{\mathcal Z} }
%references and citations
\newcommand{\refq}[1]{\sure{\ref{#1}}}
\newcommand{\refp}[1]{\ref{#1}}
\newcommand{\refs}[1]{\cite{#1}}
%Thermo, Fluid, and GR objects
%%%particle kinematics
\newcommand{\ptraj}  [1]{\ensuremath{\idx{z}{\up{#1}}}}
\newcommand{\pdtraj} [1]{\ensuremath{\idx{\dt{z}}{\up{#1}}}}
\newcommand{\pvel}   [1]{\ensuremath{\idx{u}{\dn{#1}}}}
\newcommand{\ppos}   [1]{\ensuremath{\idx{z}{\up{#1}}}}
\newcommand{\pdpos}  [1]{\ensuremath{\idx{\dt{z}}{\up{#1}}}}
\newcommand{\ptvel}  [1]{\ensuremath{\idx{u}{\dn{#1}}}}
\newcommand{\pdtvel} [1]{\ensuremath{\idx{\dt{u}}{\dn{#1}}}}
\newcommand{\ptraja} [2]{\ensuremath{\idx{z}{\dn{#1}\up{#2}}}}
\newcommand{\pdtraja}[2]{\ensuremath{\idx{\dt{z}}{\dn{#1}\up{#2}}}}
\newcommand{\pvela}  [2]{\ensuremath{\idx{u}{\dn{#1#2}}}}
\newcommand{\sm}     [2]{\ensuremath{W\!\surp{\cv{#1} - \cv{#2}}}}
%%%thermodynamic parameters
\newcommand{\gtro}{\ensuremath{\idx{\gtr}{\dn{0}}} }
\newcommand{\ierg}{\ensuremath{e\surp{\gr\surp{x}}} }
\newcommand{\erg} {\ensuremath{e} }
%%%metric functions
\newcommand{\bV}[2] {\idx{\cv{#1}}{\dn{#2}} \nudge }
\newcommand{\bF}[2] {\idx{\cf{#1}}{\up{#2}} \nudge }
\newcommand{\Jac}[2]{\idx{\gL}{\up{#1}\dn{#2}} \nudge }
\newcommand{\jac}[2]{\idx{\gL}{\dn{#1}\up{#2}} \nudge }
\newcommand{\Kd}[2] {\idx{\gd \nudge}{\up{#1}\dn{#2}} \nudge }
\newcommand{\kd}[2] {\idx{\gd}{\dn{#1}\up{#2}} \nudge }
\newcommand{\Cnx}[2]{\idx{\gG}{\up{#1}\dn{#2}} \nudge }
\newcommand{\met} [2]{\ensuremath{\idx{g}{\dn{#1 #2}}}}
\newcommand{\imet}[2]{\ensuremath{\idx{g}{\up{#1 #2}}}}
\newcommand{\metd}   {\ensuremath{ g } }
\newcommand{\smet} [1]{\ensuremath{\sura{\idx{g}{\dn{#1}}}} }
\newcommand{\simet}[1]{\ensuremath{\sura{\idx{g}{\up{#1}}}} }
\newcommand{\smetd}   {\ensuremath{\sura{g}}}
%functional arguments
\newcommand{\contarg}{\ensuremath{\surp{\idx{a}{\up{0}};\cv{a}}} }
\newcommand{\harg}   {\ensuremath{\surp{a^0,\cv{r}_1}} }
\newcommand{\parg}   {\ensuremath{\surp{t,\cv{z}}} }
\newcommand{\farg}   {\ensuremath{\surp{t,\cv{x}}} }
%variations
\newcommand{\vIz}[1] {\ensuremath{\vwrt{I}{\ptraj{#1}}} }
%useful relations
\newcommand{\normeq}[2]{\ensuremath{ \imet{#1}{#2} \pvel{#1}
\pvel{#2} + 1} }

%bibtex stuff
\pagenumbering{arabic}
\bibliographystyle{plain}
%backwards compatibility
\def\a{{\vec a}}
\def\x{{\vec x}}
\def\z{{\vec z}}
\def\P{{\Phi}}
\def\l{{\ell}}
\def\Pl{{\Phi_{\ell}}}
\def\D{{\Delta}}
\def\.{{\quad .}}
%\def\_,{{\quad ,}}
\def\half{\mbox{$\frac{1}{2}$}}
%\def\_.{{\quad .}}
\def\po{{\hat \rho}_{0}}

\begin{document}
\title{On the Riemann Tensor}
\maketitle


The purpose of this short note is to demonstrate that the Riemann tensor is
in fact a tensor (i.e. a linear machine).  The starting point will be
the general formula for the curvature operator given by
\bes
  \mR\surp{\cs{B},\cs{C}} = \surb{\nabla_{\cs{B}},\nabla_{\cs{C}}}
      - \nabla_{\surb{\cs{B},\cs{C}}} \eqp
\ees
In the process, the formulas for the Riemann tensor in both a coordinate and non-
coordinate basis will be found.  The notational style is a unique blend of
the abstract and index approaches.

\section{Coordinate Basis}

Start by examining the action of the curvature operator on a vector \cs{A}.
To set the notation, the term $\nabla_{\cs{C}} \cs{A}$ will be examined first.
\bea
  \nabla_{\cs{C}} \cs{A} & = & \sura{ d \surp{A^a \bv{a}},C^b \bv{b} }\nonumber \\
                         & = & \sura{\idx{A}{\up{a}\dn{,m}} \bv{a} \bd{m}
                                     + A^a \Cnx{n}{am} \bv{n} \bd{m},C^b \bv{b}} \nonumber \\
                         & = & \idx{A}{\up{a}\dn{,m}} \bv{a} C^b \ipb{\bd{m}}{\bv{b}}
                                     + A^a \Cnx{n}{am}\bv{n} C^b \ipb{\bd{m}}{\bv{b}} \nonumber \\
                         & = & \idx{A}{\up{a}\dn{,b}} \bv{a} C^b
                                     + A^a \Cnx{n}{ab} \bv{n} C^b \eqp
\eea
For simplicity in the next steps, the three pieces of the curvature operator will be
examined separately.
The first term yields
\bea
  \nabla_{\cs{B}} \nabla_{\cs{C}} \cs{A} & = & \surb{ \surp{\idx{A}{\up{a}\dn{;b}} C^{b}}_{,d}
                                             B^d + \idx{A}{\up{m}\dn{;b}} C^b \Cnx{a}{md} B^d }
                                             \bv{a} \eqp
\eea
The next term yields
\bea
  \nabla_{\cs{C}} \nabla_{\cs{B}} \cs{A} & = & \surb{ \surp{\idx{A}{\up{a}\dn{;b}} B^{b}}_{,d}
                                             C^d + \idx{A}{\up{m}\dn{;b}} B^b \Cnx{a}{md} C^d }
                                             \bv{a} \eqp
\eea
The third term yields
\be
 \nabla_{\surb{\cs{B},\cs{C}}} \cs{A} = \idx{A}{\up{a}\dn{;b}} \surb{\cs{B},\cs{C}}^b \bv{a} \eqp
\ee
Thus the action of the curvature operator is expressed as follows
\bea
 \mR\surp{\cs{B},\cs{C}}\cs{A}  & = &  \left\{   \surp{\idx{A}{\up{a}\dn{;b}}C^b}_{,d} B^d
                                               + \idx{A}{\up{m}\dn{;b}} C^b \Cnx{a}{md} B^d
                                               - \surp{\idx{A}{\up{a}\dn{;b}} B^b}_{,d} C^d
                                       \right. \nonumber \\
                                &    & \left.
                                         - \idx{A}{\up{m}\dn{;b}} B^b \Cnx{a}{md} C^d
                                         - \idx{A}{\up{a}\dn{;b}} \surb{\cs{B},\cs{C}}^b
                                        \right\}
                                         \bv{a} \eqp
\eea
Using the standard definition of the covariant derivative, expanding the partial derivative,
and using various index gymnastics the final result is
\be
 \mR\surp{\cs{B},\cs{C}}\cs{A}   =   \surp{   \Cnx{a}{bd,c} - \Cnx{a}{bc,d}
                                            + \Cnx{a}{sc}\Cnx{s}{bd} - \Cnx{a}{sd}\Cnx{s}{bc} }
                                     A^b B^c C^d \eqp
\ee
It is clear that in fact the Riemann tensor is a tensor; it is a linear operator on \cs{A},
\cs{B}, and \cs{C}.  The components of the Riemann tensor is then obtained from
\bea\label{eq:riemann}
  \idx{R}{\up{a}\dn{bcd}} & = & \ipb{\bd{a}}{\mR\surp{\bv{c},\bv{d}} \bv{b}} \nonumber \\
                          & = &   \Cnx{a}{bd,c} - \Cnx{a}{bc,d}
                                + \Cnx{a}{sc}\Cnx{s}{bd} - \Cnx{a}{sd}\Cnx{s}{bc} \eqp
\eea
\section{Non-Coordinate Basis}
As a final exercise, consider the components of Riemann in a non-coordinate basis.
Such a basis is characterized by a non-vanishing of the Lie bracket of the basis vectors
\be
\surb{\bv{a},\bv{b}} = \idx{C}{\dn{ab}\up{c}} \bv{c} \eqp
\ee
but with the standard relationship
\be
  d \bv{a} = \Cnx{c}{ab} \bd{b} \bv{c}
\ee
preserved.
Plugging these relationships into \refq{eq:riemann} gives
\bea
  \idx{R}{\up{a}\dn{bcd}} & = & \ipb{\bd{a}}{\mR\surp{\bv{c},\bv{d}} \bv{b}} \nonumber \\
                          & = & \ipb{\bd{a}}
                                {   \nabla_c \surp{ \ipb{d\bv{b}}{\bv{d}}}
                                  - \nabla_d \surp{ \ipb{d\bv{b}}{\bv{c}}} }
                                -\ipb{\bd{a}}{\ipb{d\bv{b}}{\idx{C}{\dn{cd}\up{m}}\bv{m}}} \eqp
\eea
Expanding and simplifying leads to
\be
  \idx{R}{\up{a}\dn{bcd}} =   \Cnx{a}{bd,c} - \Cnx{a}{bc,d}
                            + \Cnx{a}{sc}\Cnx{s}{bd} - \Cnx{a}{sd}\Cnx{s}{bc}
                            - \Cnx{a}{bp} \idx{C}{\dn{cd}\up{p}}
\ee
\end{document}
