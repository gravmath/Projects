%Classical Mechanics - filename class_mech.tex

\null
%%%%%%%%%%%%%%%%%%%%%%%%%%%%%%%%%%%%%%%%%%%%%%%%%%%%%%%%%%%%%%%
%          Classical Mechanics
%%%%%%%%%%%%%%%%%%%%%%%%%%%%%%%%%%%%%%%%%%%%%%%%%%%%%%%%%%%%%%%
\textblockcolor{test}
\begin{textblock*}{7.5in}(0mm,0mm)
\begin{tabular*}{7.5in}{c @{\extracolsep{\fill}} c }
       \tiny ~ & ~\\
       \multicolumn{2}{c}{\normalsize \bf Classical Mechanics} \\
       \tiny~ & ~\\
\end{tabular*}
\end{textblock*}

%%%%%%%%%%%%%%%%%%%%%%%%%%%%%%%%%%%%%%%%%%%%%%%%%%%%%%%%%%%%%%%
%          Invariance of the EL Equations
%%%%%%%%%%%%%%%%%%%%%%%%%%%%%%%%%%%%%%%%%%%%%%%%%%%%%%%%%%%%%%%
\scriptsize
\textblockcolor{LightYellow}
\begin{textblock*}{65mm}(0mm,12.77mm)
The Euler Lagrange equations are invariant to a basic change in coordinates from $q^i$ to
$y^j$ of the form 
\begin{equation}\label{ELinv_b}
  q^i = q^i \left(y^j \right) .
\end{equation}
To see this first note that the from the 
form of the transformation equation (\ref{ELinv_b}) we get
\begin{equation}\label{ELinv_a}
  {\dot q}^i = \frac{\partial q^i}{\partial y^j} {\dot y}^j \,
\end{equation}
where $\dot f = \frac{d}{dt} f$. Next note that the Lagrangian 
$\tilde L \left(y^j,{\dot y}^j;t \right)$ in the $y^j$ coordinates 
is related to the Lagrangian $L \left(q^i,{\dot q}^i;t \right)$ 
in $q^i$ coordinates by virtue of a substitution of (\ref{ELinv_b}) 
and (\ref{ELinv_a}) yielding
\begin{equation}\label{ELinv_c}
  \tilde L \left(y^j,{\dot y}^j;t \right) = L \left(q^i(y^j),{\dot q}^i(y^j,{\dot y}^j);t \right) .
\end{equation}
The parts of the Euler-Lagrange equation in terms
of the $y^j$ coordinates in relation to the $q^i$ coordinates are
\begin{equation}\label{ELinv_d}
  \frac{\partial \tilde L}{\partial y^j} =    \frac{\partial L}{\partial q^i}        \frac{\partial q^i}{\partial y^j} 
                                        +  \frac{\partial L}{\partial {\dot q}^i} \frac{\partial {\dot q}^i}{\partial y^j}
\end{equation}
and
\begin{equation}\label{ELinv_e}
  \frac{\partial \tilde L}{\partial {\dot y}^j} = \frac{\partial L}{\partial {\dot q}^i} \frac{\partial {\dot q}^i}{\partial {\dot y}^j} .
\end{equation}
Now substituting (\ref{ELinv_d}) and (\ref{ELinv_e}) into the Euler-\\Lagrange equations
yields
\begin{eqnarray}\label{ELinv_f}
  \frac{d}{dt} \left( \frac{\partial \tilde L}{\partial {\dot y}^j } \right) - \frac{\partial \tilde L}{\partial y^j} 
    & = &   \frac{d}{d t} \left( \frac{\partial L}{\partial {\dot q}^i} \right) \frac{\partial {\dot q}^i}{\partial {\dot y}^j} \\ \nonumber
	&   & + \frac{\partial L}{\partial {\dot q}^i} \frac{d}{dt} \left( \frac{\partial {\dot q}^i}{\partial {\dot y}^j} \right) \\ \nonumber
    & = & - \frac{\partial L}{\partial q^i}        \frac{\partial q^i}{\partial y^j} 
	      - \frac{\partial L}{\partial {\dot q}^i} \frac{\partial {\dot q}^i}{\partial y^j} .\nonumber
\end{eqnarray}
But from (\ref{ELinv_a}) $ {\partial {\dot q}^i}/{\partial {\dot y}^j} = {\partial q^i}/{\partial y^j}$ and thus the second and
fourth terms in (\ref{ELinv_f}) cancel, leaving
\begin{equation}\label{ELinv_g}
  \frac{d}{dt} \left[ \frac{\partial \tilde L}{\partial {\dot y}^j } \right) - \frac{\partial \tilde L}{\partial y^j} = 
  \left( \frac{d}{d t} \left( \frac{\partial L}{\partial {\dot q}^i} \right) - \frac{\partial L}{\partial q^i} \right] 
  \frac{\partial q^i}{\partial y^j} ,
\end{equation}
which shows that the Euler-Lagrange equations transform like the components of a covariant vector
\end{textblock*}



%%%%%%%%%%%%%%%%%%%%%%%%%%%%%%%%%%%%%%%%%%%%%%%%%%%%%%%%%%%%%%%
%          EL Equations in 1st order form
%%%%%%%%%%%%%%%%%%%%%%%%%%%%%%%%%%%%%%%%%%%%%%%%%%%%%%%%%%%%%%%
\begin{textblock*}{65mm}(64.85mm,12.77mm)
The Euler-Lagrange equations can be cast into 1$^{st}$-order form by first making the identification
\[
  \frac{d}{d t} q^{\alpha} = {\dot q}^{\alpha}
\]
and then by expanding
\[
  \frac{d}{dt} \left( \frac{\partial L}{\partial {\dot q}^{\alpha}} \right) = 
    \frac{\partial^2 L}{\partial q^{\beta} \partial {\dot q}^{\alpha}} {\dot q}^{\beta} + 
	\frac{\partial^2 L}{\partial {\dot q}^{\alpha} \partial {\dot q}^{\beta}} {\ddot q^{\beta}}         +
	\frac{\partial^2 L}{\partial t \partial q^{\alpha}}                 
\]
Substituting this form in the EL equations and solving for the ${\ddot q}^{\alpha}$ yields
\begin{eqnarray*}
  \frac{d}{dt} {\dot q}^{\alpha} & = & {\ddot q}^{\alpha} \\ 
                                 & = & \left( \frac{\partial^2 L}{\partial {\dot q}^{\alpha} \partial {\dot q}^{\beta}} \right)^{-1} \\
								 &   & \times \left( \frac{\partial L}  {\partial q^{\beta}}                    -
							                         \frac{\partial^2 L}{\partial t \partial {\dot q}^{\beta}}  - 
									                 \frac{\partial^2 L}{\partial q^{\beta} \partial {\dot q}^{\gamma}} {\dot q}^{\gamma} 
									   \right)										  
\end{eqnarray*}
which requires that the Hessian, defined by
\[
  \left( \frac{\partial^2 L}{\partial {\dot q}^{\alpha} \partial {\dot q}^{\beta}} \right)
\]
be invertible.
\end{textblock*}
\newpage