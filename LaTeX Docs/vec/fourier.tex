%Fourier Analysis - filename fourier.tex

\null
%%%%%%%%%%%%%%%%%%%%%%%%%%%%%%%%%%%%%%%%%%%%%%%%%%%%%%%%%%%%%%%
%          Fourier Analysis
%%%%%%%%%%%%%%%%%%%%%%%%%%%%%%%%%%%%%%%%%%%%%%%%%%%%%%%%%%%%%%%
\textblockcolor{test}
\begin{textblock*}{7.5in}(0mm,0mm)
\begin{tabular*}{7.5in}{c @{\extracolsep{\fill}} c }
       \tiny ~ & ~\\
       \multicolumn{2}{c}{\normalsize \bf Fourier Analysis} \\
       \tiny~ & ~\\
\end{tabular*}
\end{textblock*}

\begin{textblock*}{90mm}(0mm,15mm)
Consider a periodic function $f(x-2\ell) = f(x) = f(x + 2\ell)$ defined on the interval $[-\ell,\ell]$.  
Because of the underlying periodicity of $f(x)$, both the derivative and the integral also inherit 
periodic properties.

In the case of the derivative, the observation is evident simply from 
the standard definition as follows:
\begin{eqnarray*}
  \frac{d f}{d x}(x) & = & \lim_{dx \rightarrow 0} \frac{ f(x+dx)-f(x)}{dx}\\
                     & = & \lim_{dx \rightarrow 0} \frac{ f(x + 2\ell + dx) - f(x + 2\ell) }{dx} \\
                     & = & \frac{d f}{d x}(x + 2\ell).
\end{eqnarray*}
In the case of the integral, a bit more thinking and manipulation is needed.
To begin, consider the integral of $f(x)$ over an interval $[a,b]$.  The value of this integral 
is invariant if the entire interval is shifted by $2 \ell$.  This is proven by direct substitution
as 
\begin{eqnarray}\label{eq_per_ab_int}
  \int_{a}^{b} f(x) dx & = & \int_{a}      ^{b}       f(x - 2\ell) dx            \\
                       & = & \int_{a+2\ell}^{b+2\ell} f(y)         dy  \nonumber \\
                       & = & \int_{a+2\ell}^{b+2\ell} f(x)         dx. \nonumber
\end{eqnarray}
With this result, it is easy (although somewhat subtle) to show that the integral
of $f(x)$ over the interval $[c-\ell,c+\ell]$ is equal to the integral over 
the interval $[-\ell,\ell]$.  To do so, start by letting $a = c - \ell$ and $b = -\ell$
in (\ref{eq_per_ab_int}), which yields the identity 
\begin{equation}\label{eq_per_int}
  \int_{c-\ell}^{-\ell} f(x) dx  =  \int_{c+\ell}^{\ell} f(x) dx,
\end{equation}
which is then used to manipulate the integral over $[c-\ell,c+\ell]$ as follows,
\begin{eqnarray} 
  \int_{c-\ell}^{c+\ell} f(x) dx  & = &  \int_{c-\ell}^{-\ell} f(x) dx + \int_{-\ell}^{c+\ell} f(x) dx \\
                                  & = &  \int_{c+\ell}^{\ell}  f(x) dx + \int_{-\ell}^{c+\ell} f(x) dx \nonumber \\
                                  & = &  \int_{-\ell}^{c+\ell} f(x) dx + \int_{c+\ell}^{\ell}  f(x) dx \nonumber \\
                                  & = &  \int_{-\ell}^{\ell}   f(x) dx                                 \nonumber
\end{eqnarray}
where (\ref{eq_per_int}) was used in going from the first to the second line.

\[
  -\frac{d}{d x}\left( p(x) \frac{d y}{d x} \right) + q(x) y = \lambda w(x) y
\]
\end{textblock*}

\begin{textblock*}{100mm}(89.85mm,15mm)
\[
  \int_{c}^{c+2L} \cos \left( \frac{m \pi x}{L} \right) dx = 0
 \]
 \[
  \int_{c}^{c+2L} \sin \left( \frac{m \pi x}{L} \right) dx = 0
 \]
 
\[
  \int_{c}^{c+2L} dx \, \cos \left( \frac{m \pi x}{L} \right) \cos \left( \frac{n \pi x}{L} \right) = 
      \left\{ \begin{array}{cc} 2 L \, \delta_{mn} & m = 0 \\ L \, \delta_{mn} & m \neq 0 \end{array} \right.
\]
\[
  \int_{c}^{c+2L} dx \, \sin \left( \frac{m \pi x}{L} \right) \sin \left( \frac{n \pi x}{L} \right) = 
      \left\{ \begin{array}{cc} 0 \, \delta_{mn} & m = 0 \\ L \, \delta_{mn} & m \neq 0 \end{array} \right.
\]
\[
 {\tilde f}(x) = \frac{a_0}{2} + 
    \sum_{n=1}^{\infty} \left( a_n \cos \left( \frac{n \pi x}{L} \right) + b_n \sin \left( \frac{n \pi x}{L} \right) \right)
\]
\[
  a_n = \frac{1}{L} \int_{c}^{c+2L} f(x) \cos \left( \frac{n \pi x}{L} \right) dx
\]
\[
  b_n = \frac{1}{L} \int_{c}^{c+2L} f(x) \sin \left( \frac{n \pi x}{L} \right) dx
\]

\[
  {\tilde f}(x) = \sum_{n = - \infty}^{\infty} c_n e^{i n \pi x/L}
\]
\[
  c_n = \frac{1}{2 L} \int_{c}^{c+2L} f(x) e^{-i n \pi x/L} dx
\]
\end{textblock*}
\newpage