%Unit Step (in its many guises) - filename unit_step.tex

\null
%%%%%%%%%%%%%%%%%%%%%%%%%%%%%%%%%%%%%%%%%%%%%%%%%%%%%%%%%%%%%%%
%          Unit Step
%%%%%%%%%%%%%%%%%%%%%%%%%%%%%%%%%%%%%%%%%%%%%%%%%%%%%%%%%%%%%%%
\textblockcolor{test}
\begin{textblock*}{7.5in}(0mm,0mm)
\begin{tabular*}{7.5in}{c @{\extracolsep{\fill}} c }
       \tiny ~ & ~\\
       \multicolumn{2}{c}{\normalsize \bf Unit Step} \\
       \tiny~ & ~\\
\end{tabular*}
\end{textblock*}

%%%%%%%%%%%%%%%%%%%%%%%%%%%%%%%%%%%%%%%%%%%%%%%%%%%%%%%%%%%%%%%
%          Unit Step - Basic Results
%%%%%%%%%%%%%%%%%%%%%%%%%%%%%%%%%%%%%%%%%%%%%%%%%%%%%%%%%%%%%%%
\scriptsize
\textblockcolor{LightYellow}
\begin{textblock*}{75mm}(0mm,12.77mm)
\begin{tabular*}{73mm}{l @{\extracolsep{\fill}} l}
   & ~\\
\multicolumn{2}{c}{\bf Unit Step} \\
   & ~\\
definition:         & $U(x - q) = \left\{ \begin{array}{ll} 0 & x < q \\ 1 & x > q \end{array} \right.$\\
   & ~\\
integral property:  & $\int_{a}^{b} dx \, U(x-q) = \int_{a}^{q} d x \quad a < q < b$\\
   & ~\\
reversal:           & $ \int_{a}^{b} dx \, U(q - x)	= \int_{q}^{b} dx \quad a < q < b$ \\
   & ~\\
$\epsilon$ 
representation:     & $\frac{1}{2 \pi i} \int_{-\infty}^{\infty} \frac{e^{i x t}}{x - i \epsilon} dx$\\
   & ~\\
PV representation:  & $\frac{1}{2} + PV \int_{-\infty}^{\infty} \frac{ e^{i x t}}{x} dx$\\
\end{tabular*}
\begin{eqnarray*}
 \int_{a}^{b} dx \, \theta(q - x)	& = & \int_{a}^{b} dx \left\{1 - \theta(x-q) \right\} \\
								& = & \int_{a}^{b} dx - \int_{a}^{q} dx \\
								& = & \int_{a}^{q} dx + \int_{q}^{b} dx - \int_{a}^{q} dx\\
								& = & \int_{q}^{b} dx
\end{eqnarray*}
\end{textblock*}

%%%%%%%%%%%%%%%%%%%%%%%%%%%%%%%%%%%%%%%%%%%%%%%%%%%%%%%%%%%%%%%
%          FT of the Unit Step
%%%%%%%%%%%%%%%%%%%%%%%%%%%%%%%%%%%%%%%%%%%%%%%%%%%%%%%%%%%%%%%
\scriptsize
\textblockcolor{LightYellow}
\begin{textblock*}{75mm}(76.85mm,12.77mm)
The Fourier transform (FT) of the unit step can be obtained by considering
the Schwartz function approximation
\begin{equation}
  U_{\alpha}(t) = \left\{ \begin{array}{cc} e^{-\alpha t} & t \geq 0 \\ 0 & t < 0 \end{array} \right.  
\end{equation}
in the limit as $\alpha \rightarrow \infty$.
The FT of $U_{\alpha}$ is given by
\begin{eqnarray}
{\mathcal F} U_{\alpha} & = & \int_{-\infty}^{\infty} U_{\alpha}(t) e^{-2 \pi i s t} dt \\ \nonumber
                        & = & \int_{0}^{\infty} e^{-\alpha t} e^{-2 \pi i s t} dt \\ \nonumber
                        & = & \left. \frac{-1}{\alpha + 2 \pi i s} e^{-(\alpha + 2 \pi i s) t}  \right|_{0}^{\infty} \\ \nonumber
                        & = & \frac{1}{\alpha + 2 \pi i s}.
\end{eqnarray}
The next step is to take the limit, but to make the limit meaningful, ${\mathcal F}U_{\alpha}$ must
be separated into real and imaginary parts
\begin{equation}
  {\mathcal F} U_{\alpha} = \frac{\alpha}{\alpha^2 + 4 \pi^2 s^2} - \frac{2 \pi s}{\alpha^2 + 4 \pi^3 s^2} i.
\end{equation}
The limit of the imaginary part is straightforward and yields $\frac{1}{2 \pi i s}$ but the real
part requires more care.  First calculate the integral of the real part $s \in [-\infty,\infty]$
\begin{equation}
I =  \int_{-\infty}^{\infty} \frac{\alpha}{\alpha^2 + 4 \pi^2 s^2} ds .
\end{equation}
The value of $I$ is obtained by making the subsitution $s = \frac{\alpha}{2 \pi} \tan(\theta)$ to yield
\[
  I = \frac{1}{2 \pi} \int_{-\pi/2}^{\pi/2} \frac{ \sec^2{\theta} }{1 + \tan^2(\theta)} d \theta = \frac{1}{2} .
\]
Now note that the real part is zero as $s = \pm \infty$ and $1/\alpha$ at $s=0$, which means that the real part
is strongly-peaked at $s=0$.  The only function that integrates to a constant over $[-\infty,\infty]$, is 
strongly-peaked at the origin, where it tends to infinity, and is zero everywhere else is a function
proportional to the $\delta$ function.  Thus
\begin{equation}
 \lim_{\alpha \rightarrow 0} {\mathcal F} U_{\alpha} = {\mathcal F}U = \frac{1}{2} \delta(s) + \frac{1}{2 \pi i s} .
\end{equation}
\end{textblock*}
\newpage