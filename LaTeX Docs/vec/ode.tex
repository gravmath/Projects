\[
  F = m a
\]
\[
  L[y] =  \left( \frac{d^2}{dt^2} + p(t) \frac{d}{dt} + q(t) \right) y
\]

\[
  L[y] = g(t) \quad y(t_0) = y_0 \quad \frac{dy}{dt} (t_0) = {\dot y}_0
\]
Let $p(t)$ and $q(t)$ be continuous in the open interval $(\alpha,\beta)$ then
there exists one and only one function $y(t)$ which satisfies $L[y] = 0$ and
$y(t_0) = y_0$ and $\frac{dy}{dt}(t_0) = {\dot y}_0$.

Wronskian:
\[
  W[y_1,y_2](t) \equiv y_1 {\dot y}_2 - {\dot y}_1 y_2
\]
\[
  {\dot W} + p(t) W = 0 \Rightarrow W(t) = W_0 \exp \left( -\int_{t0}^{t} p(s) ds \right)
\]

Constant Coefficients:
\[
  p(t) = b \quad q(t) = c
\]
Reduction of order:

\[
  y_2 =v(t) y_1
\]
\[
  v(t) = \int u(s) ds \quad u(t) = \frac{1}{{y_1}^2} exp\left( -\int p(s) ds \right)
\]

distinct complex roots
\[
  \gamma_{1\\2} = \frac{-b \pm \sqrt{b^2 - 4c}}{2}
\]
repeated roots
\[
  y_1 = e^{-\frac{b t}{2}}
\]
\[
  y_2 = t y_1
\]

VOP:
\[
  \psi(t) = u_1(t) y_1(t) + u_2(t) y_2(t)
\]
\[
  \frac{d}{dt} u_1(t) = - \frac{ g(t) y_2(t) }{ W\left[y_1,y_2\right](t) }
\]
\[
  \frac{d}{dt} u_2(t) =   \frac{ g(t) y_1(t) }{ W\left[y_1,y_2\right](t) }
\]

One-sided Green's functions
\[
  g(t,\tau) = \theta(t-\tau) g_1(t,\tau)
\]
\[
  g_1(t,\tau) = \frac{ y_1(\tau) y_2(t) - y_1(t) y_2(\tau) }{ W[y_1,y_2](\tau) }
\]
\[
  N[g(t,t\tau)] = \delta(t-\tau) \quad y(t_0) = 0 \; {\dot y}(t_0) = 0 
\]
\[
  g(\tau^{-},\tau) = g(\tau^{+},\tau)
\]
\[
  \left. \frac{\partial g}{\partial t} \right|^{t=\tau^{+}}_{t=\tau^{-}} = 1
\]
Operators
\[
  M = A_2(x) D^2 + A_1(x) D + A_0(x)
\]
\[
  L = D[ p(x) D] + q(x) \quad L[y] + \lambda r(x) y = 0
\]
\[
  N = D^2 + a_1(x) D + a_0(x)
\]
\[
  B_i[y_k] = a_{ijk\ell} D^j y_k(x_ell)
\]
The conversion $M \rightarrow L$ can be affected by multiplying $\mu(x) = 
p(x)/A_2(x)$ with $p(x) = \exp\left( \int \frac{A_1}{A_2} \right)(x)$.  
Likewise, the conversion $M \rightarrow N$ by dividing by $A_2(x)$.
\[
  N[ \theta(t-\tau) g_1(t,\tau) ] =  \left( \frac{d^2}{d t^2} + a_1 \frac{d}{dt} + a_0 \right) 
                                     [\theta(t-\tau) g_1(t-\tau) ] 
\]
Using the short-hand $g_1 = g_1(t-\tau)$, $\theta = \theta(t-\tau)$, 
$D = \frac{d}{dt}$ the equation becomes
\begin{eqnarray*}
  N[\theta g_1] & = & (D^2 + a_1 D + a_0)[\theta g_1] \\
                & = & D^2(\theta g_1) + a_1 D(\theta g_1) + a_0 \theta g_1\\
                & = & D(g_1 D\theta + \theta D g_1) + a_1 (g_1 D \theta + \theta D g_1 ) + a_0 \theta g_1 \\
                & = & D g_1 D \theta + g_1 D^2 \theta + D\theta Dg_1 
                      + \theta D^2 g_1 + a_1 g_1 D\theta + a_1 \theta Dg_1 + a_0 \theta g_1
\end{eqnarray*}
Now when $t \neq \tau$, $\theta$ is a constant and its derivative is zero.
So it is only at the point when $t = \tau$ where care must be exercised. At 
this time, $g_1 = 0$, $D \theta = \delta$, $Dg_1 = 1$, $g_1 D\delta = -\delta g_1 = -\delta$.
Plugging these results in yields
\begin{eqnarray*}
  N[\theta g_1] & = &   \delta - \delta + \delta 
                      + \theta (D^2 g_1 + a_1 g_1 + a_0 g_1 ) + a_1 g_1 \delta \\
                & = & \delta \theta N[g_1] + a_1 g_1 \delta \, .
\end{eqnarray*}
Finally note that $N[g_1] = 0$ and that the $x*\delta(x) = 0$.  Since $g_1(\tau,\tau) = 0$, 
then $g_1 \delta = 0$.  The final result is (restoring the arguments)
\[
  N[\theta(t-\tau) g_1(t,\tau)] = \delta(t-\tau) \, .
\] 

\begin{figure}[htp]
\centering
\scalebox{1} % Change this value to rescale the drawing.
{
\begin{pspicture}(0,-1.5)(6.479455,1.5)
\definecolor{color540}{rgb}{1.0,0.2,0.2}
\definecolor{color544}{rgb}{0.4,0.4,0.4}
\definecolor{color546}{rgb}{0.2,0.2,1.0}
\pscustom[linewidth=0.04,linecolor=color540]
{
\newpath
\moveto(0.12,-1.24)
\lineto(0.84004784,-0.43351853)
\curveto(1.2000717,-0.03027771)(1.934378,0.61370367)(2.3086603,0.85444444)
\curveto(2.6829424,1.0951852)(3.0679188,1.3419445)(3.1,1.36)
}
\pscustom[linewidth=0.04,linecolor=blue]
{
\newpath
\moveto(0.08,-1.3)
\lineto(0.9208612,-0.8781482)
\curveto(1.3412918,-0.66722226)(2.1988037,-0.33037034)(2.635885,-0.20444442)
\curveto(3.0729663,-0.078518525)(3.522536,0.05055542)(3.56,0.06)
}
\pscustom[linewidth=0.04,linecolor=color540,linestyle=dashed,dash=0.16cm 0.16cm]
{
\newpath
\moveto(1.96,-1.4)
\lineto(2.510909,-0.5066667)
\curveto(2.7863636,-0.06)(3.3481817,0.6533333)(3.6345453,0.92)
\curveto(3.920909,1.1866667)(4.2154546,1.46)(4.24,1.48)
}
\pscustom[linewidth=0.04,linecolor=blue,linestyle=dashed,dash=0.16cm 0.16cm]
{
\newpath
\moveto(2.02,-1.38)
\lineto(3.0926914,-1.0847037)
\curveto(3.629037,-0.93705565)(4.7229695,-0.70125943)(5.2805567,-0.61311126)
\curveto(5.8381433,-0.5249631)(6.411661,-0.4346112)(6.459455,-0.42800003)
}
\psdots[dotsize=0.2,linecolor=color544](1.96,-1.38)
\psdots[dotsize=0.2](0.1,-1.3)
\psline[linewidth=0.04cm,linecolor=color546](3.54,0.06)(3.34,0.1)
\psline[linewidth=0.04cm,linecolor=color546](3.54,0.06)(3.4,-0.08)
\psline[linewidth=0.04cm,linecolor=color546](6.4429126,-0.42200005)(6.213286,-0.38000005)
\psline[linewidth=0.04cm,linecolor=color546](6.4429126,-0.42200005)(6.2643137,-0.52000004)
\psline[linewidth=0.04cm,linecolor=red](3.1,1.36)(2.9,1.4)
\psline[linewidth=0.04cm,linecolor=red](3.1,1.36)(2.98,1.18)
\psline[linewidth=0.04cm,linecolor=red](4.228301,1.4572893)(4.04,1.44)
\psline[linewidth=0.04cm,linecolor=red](4.2227187,1.4772893)(4.16,1.28)
\psline[linewidth=0.04cm,arrowsize=0.05291667cm 2.0,arrowlength=1.4,arrowinset=0.4]{<->}(1.2990625,-0.0115625)(1.6790625,-0.5315625)
\psline[linewidth=0.04cm,arrowsize=0.05291667cm 2.0,arrowlength=1.4,arrowinset=0.4]{<->}(1.7590625,0.4084375)(2.8390625,-0.1315625)
\psline[linewidth=0.04cm,arrowsize=0.05291667cm 2.0,arrowlength=1.4,arrowinset=0.4]{<->}(2.4990625,0.9484375)(5.8790627,-0.5115625)
\end{pspicture} 
}
\end{figure}