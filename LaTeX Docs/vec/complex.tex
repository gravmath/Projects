%Complex Analysis - filename complex.tex

\null
%%%%%%%%%%%%%%%%%%%%%%%%%%%%%%%%%%%%%%%%%%%%%%%%%%%%%%%%%%%%%%%
%          Complex Analysis
%%%%%%%%%%%%%%%%%%%%%%%%%%%%%%%%%%%%%%%%%%%%%%%%%%%%%%%%%%%%%%%
\textblockcolor{test}
\begin{textblock*}{190.5mm}(0mm,0mm)
\begin{tabular*}{190.5mm}{c @{\extracolsep{\fill}} c }
       \tiny ~ & ~\\
       \multicolumn{2}{c}{\normalsize \bf Complex Analysis} \\
       \tiny~ & ~\\
\end{tabular*}
\end{textblock*}

%%%%%%%%%%%%%%%%%%%%%%%%%%%%%%%%%%%%%%%%%%%%%%%%%%%%%%%%%%%%%%%
%          Basic Results
%%%%%%%%%%%%%%%%%%%%%%%%%%%%%%%%%%%%%%%%%%%%%%%%%%%%%%%%%%%%%%%
\scriptsize
\textblockcolor{LightYellow}
\begin{textblock*}{100mm}(0mm,12.54mm)
\begin{tabular*}{98mm}{l @{\extracolsep{\fill}} l}
   & ~\\
\multicolumn{2}{c}{\bf Basic Results} \\
   & ~\\
mapping:            & $\phi : z \equiv (x+iy) \rightarrow w \equiv \left(u(x,y) + iv(x,y)\right)$\\
                    & \\
Cauchy-Riemann:     & $ \frac{\partial u}{\partial x} =  \frac{\partial v}{\partial v} 
                        \quad \& \quad 
                        \frac{\partial u}{\partial y} = -\frac{\partial v}{\partial x} \iff w$ analytic\\
& \\
contour integral:   & $\int_{z_1}^{z_2} f(z) dz = \int_{(x_1,y_1)}^{(x_2,y_2)}\left[ u(x,y) + iv(x,y)\right]\left[dx + idy\right]$\\
                    & with the path from $(x_1,y_1)$ to $(x_2,y_2)$ specified\\
& \\
Cauchy's theorem:   & $\oint_{\mathcal C} f(z) dz = 0 \iff f(z)$ is analytic throughout some\\
                    &  simply connected region ${\mathcal R}$ bound by ${\mathcal C}$\\
& \\                    
Cauchy's formula:   & $\oint_{\mathcal C} \frac{f(z)}{z-z_0} dz = 2 \pi i f(z_0)$ where $f(z)$ is analytic\\
& \\
Cauchy's formula:   & $f^{(n)}(z_0) = \frac{n!}{2\pi i} \oint_{\mathcal C} \frac{f(z) dz}{\left(z-z_0\right)^n}$\\
& \\
residue primitve:   & $\oint_{\mathcal C} \left(z-z_0\right)^n dz = \left\{ \begin{array}{ccc}  2 \pi i & n = -1    & {\mathcal C} \quad ccw \\
                                                                                               -2 \pi i & n = -1    & {\mathcal C} \quad cw  \\
                                                                                                      0 & n \neq -1 &                   
                                                                           \end{array} \right. $\\
& \\
Laurent series:     & $q(z) = \sum_{n=-\infty}^{n=\infty} a_n \left(z - z_0\right)^n$\\
& \\
residue:            & $a_{-1} = \left. \left(z-z_0\right) q(z) \right|_{z=z_0}$\\
& \\
\end{tabular*}
\end{textblock*}

%%%%%%%%%%%%%%%%%%%%%%%%%%%%%%%%%%%%%%%%%%%%%%%%%%%%%%%%%%%%%%%
%          Residue Primitive
%%%%%%%%%%%%%%%%%%%%%%%%%%%%%%%%%%%%%%%%%%%%%%%%%%%%%%%%%%%%%%%
\textblockcolor{LightYellow}
\begin{textblock*}{90.65mm}(99.85mm,12.54mm)

\begin{tabular*}{88.65mm}{l @{\extracolsep{\fill}} l}
   & ~\\
\multicolumn{2}{c}{\bf Residue Primitive} \\
   & ~\\
\end{tabular*}
The point of this article is to establish the residue primitive result by explicitly
evaluating the contour integral
\[
   I = \oint_{\mathcal C} \left( z - z_0 \right)^n dz,
\]
where ${\mathcal C}$ is a circular contour around $z_0$ in a counterclock-wise (ccw)
direction.  Assume on ${\mathcal C}$ that $z - z_0 = R e^{i \theta}$ where $\theta$ 
goes from $0$ to $2 \pi$. With this parametrization $dz = R i e^{i \theta} d \theta$ and
\[
  I = i R^{n+1} \int_0^{2 \pi} e^{i \theta \left( n + 1 \right)} d \theta .
\]
There are two cases to consider: $n \neq -1$ and $n = -1$. 
In the first case
\begin{eqnarray*}
  I_{n \neq -1}  & = & i R^{n+1} \int_0^{2 \pi} e^{i \theta \left(n+1\right)} d \theta \\
                & = & \frac{R^{n+1}}{n+1} \left. e^{i \theta \left(n+1\right)} \right|^{2 \pi}_0 \\
                & = & 0
\end{eqnarray*}
and in the second
\[
  I_{n = -1}   = i \int_0^{2 \pi} d \theta = 2 \pi i
\]
To determine the value of the integral if the contour were traversed in the clockwise direction,
the limits of the integral can be reversed so that it went from $2 \pi$ to $0$ but ir is more
instructive to consider the closed contour ${\mathcal C} \rightarrow {\mathcal C}_{in} 
\rightarrow {\mathcal C}_{out} \rightarrow {\mathcal C}_{2}$. 
\begin{figure}[htp]
\centering
\psscalebox{2}{
\begin{pspicture}(-1.5,-1.5)(1.5,1.5)
  \pscircle[fillstyle=solid,fillcolor=LightGreen,linestyle=none](0,0){1.005}
  \psarc[linewidth=0.5pt]{->}(0,0){1}{32.96709734}{212}
  \psarc[linewidth=0.5pt]{-}(0,0){1}{210}{27.03290266}
  \pscircle[fillstyle=solid,fillcolor=LightYellow,linestyle=none](0,0){0.21}  
  \psarc[linewidth=0.5pt]{-}(0,0){0.205}{45}{190}
  \psarc[linewidth=0.5pt]{<-}(0,0){0.205}{180}{15}
  \pspolygon[fillstyle=solid,fillcolor=LightYellow,linestyle=none,linewidth=0.5pt](0.19125331,0.05124617)(0.89067955,0.45446836)(0.83892092,0.54411694)(0.14140721,0.14140721)
  \psline[linewidth=0.5pt]{-}(0.19318517,0.05176381)(0.89076863,0.45451381)
  \psline[linewidth=0.5pt]{-}(0.14142136,0.14142136)(0.83900482,0.54417136)
  \psline[linewidth=0.5pt]{<-}(0.5419769,0.25313881)(0.89076863,0.45451381)
  \psline[linewidth=0.5pt]{->}(0.14142136,0.14142136)(0.49021309,0.34279636)
  \psline[linewidth=0.2pt]{->}(0,0)(1.29903811,0.75)
  \pscircle[fillstyle=solid,fillcolor=black,linestyle=none](0,0){0.05}
  \uput[45](0.6,0.6){${\scriptsize {\mathcal C}}$}
  \uput[1](-0.8,-0.05){${z_0}$}
  \uput[1](-0.4,-0.4){${{\mathcal C}_2}$}
  \uput[1](-0.2,0.45){${{\mathcal C}_{in}}$}  
  \uput[1](0.2,0){${{\mathcal C}_{out}}$}  
  %\psgrid(-1.5,-1.5)(1.5,1.5)
\end{pspicture}}
\end{figure}
From Cauchy's theorem $\left(z - z_0 \right)^n$ is analytic in the region bound by this combined
contour and the integrals over ${\mathcal C}_{in}$ and ${\mathcal C}_{out}$ cancel leaving
$I_{\mathcal C} = -I_{ {\mathcal C}_2 }$.
\end{textblock*}
\newpage