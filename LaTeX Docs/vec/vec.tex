%vec
\documentclass[10pt]{article}
\usepackage{pstricks}
\usepackage{pst-plot}
\usepackage{pst-node}
\usepackage{pst-tree}
\usepackage{pst-coil}

\usepackage[absolute]{textpos}
\setlength{\parindent}{0pt}
\usepackage{color}
\usepackage{amssymb}
\usepackage{graphicx}
\textblockorigin{0.5in}{0.5in}
\definecolor{MintGreen}      {rgb}{0.2,0.85,0.5}
\definecolor{LightGreen}     {rgb}{0.9,1,0.9}
\definecolor{LightYellow}    {rgb}{1,1,0.6}
\definecolor{test}           {rgb}{0.8,0.95,0.95}

\usepackage{hyperref}

\TPshowboxestrue
\TPMargin{2mm}
\begin{document}


%Euclidean Vector Analysis - filename eucl_geo_anal.tex

\null
%%%%%%%%%%%%%%%%%%%%%%%%%%%%%%%%%%%%%%%%%%%%%%%%%%%%%%%%%%%%%%%%%%%%%%%%%%%%%%%
%          Euclidean Geometric Analysis
%%%%%%%%%%%%%%%%%%%%%%%%%%%%%%%%%%%%%%%%%%%%%%%%%%%%%%%%%%%%%%%%%%%%%%%%%%%%%%%
\textblockcolor{test}
\begin{textblock*}{7.5in}(0mm,0mm)
\begin{tabular*}{7.5in}{c @{\extracolsep{\fill}} c }
       \tiny ~ & ~\\
       \multicolumn{2}{c}{\normalsize \bf Euclidean Geometric
                                          Analysis} \\
       \tiny~ & ~\\
\end{tabular*}
\end{textblock*}

%%%%%%%%%%%%%%%%%%%%%%%%%%%%%%%%%%%%%%%%%%%%%%%%%%%%%%%%%%%%%%%
%          Basic Geometric Objects
%%%%%%%%%%%%%%%%%%%%%%%%%%%%%%%%%%%%%%%%%%%%%%%%%%%%%%%%%%%%%%%
\scriptsize
\textblockcolor{LightYellow}
\begin{textblock*}{60mm}(0mm,12.54mm)
\begin{tabular*}{58mm}{l @{\extracolsep{\fill}} l}
   & ~\\
\multicolumn{2}{c}{\bf Basic Geometric Objects} \\
   & ~\\
point:              & ${\mathcal P},{\mathcal Q},\dots$\\
bounding points:    & $\partial \, {\mathcal C}$\\
curve:              & ${\mathcal C}$\\
bounding curve:     & $\partial \, {\mathcal S}$\\
surface:            & ${\mathcal S}$\\
bounding surface:   & $\partial \, {\mathcal V}$\\
                    & ~\\
volume:             & ${\mathcal V}$\\
scalar:             & $m$, $n$, etc.\\
indices:            & $i, j = 1, 2, \dots, N$\\
coordinates:        & $x^i$, $q^{i'}$ \, $i$, $j'$ = 1,2,3\\
transformations:    & $x^i = x^i \left( q^{j'} \right)$\\
field:              & $\phi = \phi \left( q^i \right)$\\
parametric curve:   & $q^i \left( s \right)$\\
implicit surface:   & $\phi_{\mathcal S}\left(q^i \right) 
                      - c = 0$\\
parametric surface: & $q^i\left(u,v\right)$\\
                    & ~\\
Kronecker delta:    & $\delta_{ij}$\\
permutation symb:   & $[i,j,k]$\\
column array:       & $| {\mathbf q} \rangle$\\
row array:          & $\langle {\mathbf q} |$\\
\end{tabular*}
\vspace{11.56mm}
\end{textblock*}

%%%%%%%%%%%%%%%%%%%%%%%%%%%%%%%%%%%%%%%%%%%%%%%%%%%%%%%%%%%%%%%
%          Defining Vectors
%%%%%%%%%%%%%%%%%%%%%%%%%%%%%%%%%%%%%%%%%%%%%%%%%%%%%%%%%%%%%%%
\scriptsize
\textblockcolor{LightYellow}
\begin{textblock*}{60mm}(59.85mm,12.54mm)
\begin{tabular*}{58mm}{l @{\extracolsep{\fill}} l}
   & ~\\
\multicolumn{2}{c}{\bf Defining Vectors} \\
   & ~\\
A01:                  & ${\mathbf A} + {\mathbf B} = 
                         {\mathbf B} + {\mathbf A}$\\
A02:                  & ${\mathbf A} + \left( {\mathbf B} 
                        + {\mathbf C} \right) 
                      = \left( {\mathbf A} + {\mathbf B} \right) 
                        + {\mathbf C}$\\
A03:                  & $ m {\mathbf A} = {\mathbf A} m$\\
A04:                  & $ m \left( n {\mathbf A} \right) 
                        = \left( m n \right) {\mathbf A}$\\
A05:                  & $\left( m + n \right) {\mathbf A} 
                        = m {\mathbf A} + m {\mathbf A}$\\
A06:                  & $m \left( {\mathbf A} + {\mathbf B} \right) 
                        = m {\mathbf A} + m {\mathbf B}$\\
A07:${}^\dagger$      & ${\mathbf A} \cdot {\mathbf B} 
                        = {\mathbf B} \cdot {\mathbf A}$\\
A08:${}^\dagger$      & ${\mathbf A} \cdot \left( {\mathbf B} 
                        + {\mathbf C} \right) 
                        = {\mathbf A} \cdot {\mathbf B} 
                        + {\mathbf A} \cdot {\mathbf C}$\\
A09:${}^\dagger$      & $m \left( {\mathbf A} \cdot 
                        {\mathbf B} \right) 
                        = \left( m {\mathbf A} \right) 
                        \cdot {\mathbf B}$\\
					  & \quad $ = {\mathbf A} \cdot 
                        \left( m {\mathbf B} \right)
                        = \left( {\mathbf A} \cdot 
                        {\mathbf B} \right) m$\\
A10:${}^\dagger$      & ${\mathbf e}_i \cdot {\mathbf e}_j 
                        = g_{ij}$\\
A11:${}^\dagger$      & ${\mathbf A} = A^i {\mathbf e}_i$ 
                        with $A^i = {\mathbf A} 
                        \cdot {\mathbf e}_i$\\
A12:${}^\dagger$      & ${\mathbf A} \cdot {\mathbf A} 
                        = |{\mathbf A}|^2$\\
A13:${}^\dagger$      & ${\mathbf A} \cdot {\mathbf B} 
                        = | {\mathbf A} | | {\mathbf B} | 
                        \cos \left( \theta \right)$\\
A14:${}^\ddag$        & ${\mathbf A} \times {\mathbf B} 
                        = - {\mathbf B} \times {\mathbf A}$\\
A15:${}^\ddag$        & ${\mathbf A} \times \left( {\mathbf B}
                        + {\mathbf C} \right) 
                        = {\mathbf A} \times {\mathbf B} 
                        + {\mathbf A} \times {\mathbf C}$\\
A16:${}^\ddag$        & $m \left( {\mathbf A} \times 
                        {\mathbf B} \right) 
                         = \left( m {\mathbf A} \right) 
                        \times {\mathbf B}$\\
                      & $\quad = {\mathbf A} \times 
                        \left( m {\mathbf B} \right) 
				  	    = \left( {\mathbf A} \times 
                        {\mathbf B} \right) m$\\
A17:${}^\ddag$        & $ {\mathbf e}_i \times 
                       {\mathbf e}_j = [ijk] {\mathbf e}_k$\\
A18:${}^\ddag$        & ${\mathbf A} \times 
                        \left( {\mathbf B} \times 
                        {\mathbf C} \right) 
                        \neq \left( {\mathbf A} \times 
                        {\mathbf B} \right) \times 
                        {\mathbf C}$\\
A19:${}^\dag{}^\ddag$ & $| {\mathbf A} \times {\mathbf B} | 
                        = |{\mathbf A}| |{\mathbf B}| 
                        \sin \left( \theta \right)$\\
A20:${}^\ddag$       & ${\mathbf A} \cdot( {\mathbf B} \times {\mathbf C} )
                       ={\mathbf B} \cdot( {\mathbf C} \times {\mathbf A} )$\\
\end{tabular*}
\vspace{7.85mm}
\end{textblock*}

%%%%%%%%%%%%%%%%%%%%%%%%%%%%%%%%%%%%%%%%%%%%%%%%%%%%%%%%%%%%%%%
%          Defining Vector Fields
%%%%%%%%%%%%%%%%%%%%%%%%%%%%%%%%%%%%%%%%%%%%%%%%%%%%%%%%%%%%%%%
\scriptsize
\textblockcolor{LightYellow}
\begin{textblock*}{70.8mm}(119.7mm,12.54mm)
\begin{tabular*}{70mm}{l @{\extracolsep{\fill}} l}
   & ~\\
\multicolumn{2}{c}{\bf Defining Vector Fields} \\
   & ~\\
F01:                  & vector field ${\mathbf F} = {\mathbf F} \left(q^i \right)$\\
F02:                  & fundamental field ${\mathbf r} = x {\mathbf e}_x + y {\mathbf e}_y + z {\mathbf e}_z$\\
                      & \\
F03:${}^\dag$         & covariant basis ${\mathbf e}_i = \frac{ \partial_{q^i} {\mathbf r}}{|\partial_{q^i} {\mathbf r}|}$\\
                      & \\
F04:                  & curve tangent ${\mathbf e}_{\mathcal C} = \frac{ d {\mathbf r}_{\mathcal C}}{ d s} \equiv {\hat t}$\\
                      & \\
F05:                  & surface tangents ${\mathbf e}_{u} = \frac{ \partial {\mathbf r}_{\mathcal S}}{\partial u}$,
                                         ${\mathbf e}_{v} = \frac{ \partial {\mathbf r}_{\mathcal S}}{\partial v}$\\
                      & \\
F06:${}^\dag{}^\ddag$ & surface normal (parametric): ${\mathbf n}_{\mathcal S} = \frac{ {\mathbf e}_u \times {\mathbf e}_v }
                                                                                     {|{\mathbf e}_u \times {\mathbf e}_v|}$\\
                      & \\
F07:${}^\dag$         & surface normal (implicit): ${\mathbf n}_{\mathcal S} = \frac{\nabla \phi_{\mathcal S}}{|\nabla \phi_{\mathcal S}|}$\\
                      & \\
F08:${}^\dag$         & $div({\mathbf F}) = \lim_{{\mathcal V} -> 0} 
                            \frac{ \int_{\partial {\mathcal V}} {\mathbf F} \cdot {\mathbf n} d {\mathcal S}}
							{\mathcal V}$\\
                      & \\
F09:${}^\dag{}^\ddag$ & $curl({\mathbf F}) \cdot {\mathbf n} = \lim_{{\mathcal S} -> 0}
                              \frac{ \int_{\partial {\mathcal S}} {\mathbf F} \cdot {\hat t} dl}{\mathcal S}$\\
                      & \\
F10:${}^\dag$         & $grad(\phi) = \lim_{{\mathcal C}->0}
                            \frac{ \int_{\partial {\mathcal C}} \phi {\hat t} dl}{\mathcal C}$\\
                      & \\
F11:${}^\dag$         & $\int_{{\mathcal V}} div({\mathbf F}) d {\mathcal V} 
                         = \int_{\partial {\mathcal V}} {\mathbf F} \cdot {\mathbf n} d {\mathcal S}$\\
                      & \\
F12:${}\dag$          & $\int_{\mathcal S} curl({\mathbf F}) \cdot {\mathbf n} d {\mathcal S} 
                           = \int_{\partial {\mathcal S}} {\mathbf F} \cdot d {\mathbf r}$\\
\end{tabular*}
\vspace{2mm}
\end{textblock*}

%%%%%%%%%%%%%%%%%%%%%%%%%%%%%%%%%%%%%%%%%%%%%%%%%%%%%%%%%%%%%%%
%          Orthogonal Coordinates
%%%%%%%%%%%%%%%%%%%%%%%%%%%%%%%%%%%%%%%%%%%%%%%%%%%%%%%%%%%%%%%
\scriptsize
\textblockcolor{LightYellow}
\begin{textblock*}{65mm}(0mm,95.3mm)
\begin{tabular*}{64mm}{l @{\extracolsep{\fill}} l}
   & ~\\
\multicolumn{2}{c}{\bf Orthogonal Coordinates$
                  {}^\dag{}^\ddag$} \\
   & \\
N01:  & $h_i = |\partial {\mathbf r}/\partial q^i|$\\
N02:  & ${\mathbf e}_i = \frac{1}{h_i} \partial 
        {\mathbf r}/\partial q^i$ (no sum)\\
N03:  & $\Omega = h_1 h_2 h_3$\\
      & \\
N04:  & $grad(\phi) = \sum_i {\mathbf e}_i \frac{1}{h_i} \
        \partial_{q^i} \phi$\\
N05:  & $curl({\mathbf F}) = \frac{1}{\Omega} \sum_{ijk} 
        {\mathbf e}_i [ijk] h_i \partial_{q^j} 
        \left( h_k F_k \right)$\\
N06:  & $div({\mathbf F}) = \sum_i \frac{1}{\Omega} 
        \partial_{q^i} \left( \frac{ \Omega F_i }{h_i} 
        \right)$\\
      & \\
N07:  & $laplacian(\phi) = \frac{1}{\Omega} \sum_i 
        \partial_{q^i} \left( \frac{\Omega}{ {h_i}^2 } 
        \partial_{q^i} \phi \right)$\\
      & \\
\multicolumn{2}{l}{$grad$, $div$, and $curl$ can all 
                    be represented}\\
\multicolumn{2}{l}{by a single operator $\nabla$, 
                    which gives}\\      
\multicolumn{2}{l}{functionally the relations 
                    $grad(\phi) = \nabla \phi$,}\\
\multicolumn{2}{l}{$div({\mathbf F}) = \nabla \cdot 
                  {\mathbf F}$, $curl({\mathbf F})= 
                  \nabla \times {\mathbf F}$, and}\\
\multicolumn{2}{l}{$laplacian(\phi) =  \nabla \cdot \nabla 
                    \phi = \nabla^2 \phi$.}\\
\multicolumn{2}{l}{In Cartesian components, $\nabla$ 
                   takes on}\\
\multicolumn{2}{l}{the form:}\\
	  & \\
N08:  & $\nabla = {\mathbf e}_x \partial_x + 
                  {\mathbf e}_y \partial_y + 
                  {\mathbf e}_z \partial_z$\\
      & \\
\end{tabular*}
\end{textblock*}

%%%%%%%%%%%%%%%%%%%%%%%%%%%%%%%%%%%%%%%%%%%%%%%%%%%%%%%%%%%%%%%
%          Derivative Theorems
%%%%%%%%%%%%%%%%%%%%%%%%%%%%%%%%%%%%%%%%%%%%%%%%%%%%%%%%%%%%%%%
\scriptsize
\textblockcolor{LightYellow}
\begin{textblock*}{64mm}(64.85mm,95.3mm)
\begin{tabular*}{63mm}{l @{\extracolsep{\fill}} l}
   & ~\\
\multicolumn{2}{c}{\bf Derivative Theorems${}^\dag{}^\ddag$} \\
   & \\
D01: & $\nabla \cdot \nabla \phi = \nabla^2 \phi$\\
D02: & $\nabla( \phi \psi ) = (\nabla \phi) \psi 
        + \phi (\nabla \psi)$\\
D03: & $\nabla \times \nabla \phi = 0$\\
D04: & $ \nabla( {\mathbf F} \cdot {\mathbf G} ) = 
        ({\mathbf F} \cdot \nabla) {\mathbf G} 
        + {\mathbf F} \times (\nabla \times {\mathbf G})$\\
     & $+ ( {\mathbf G} \cdot \nabla ) {\mathbf F} + {\mathbf G} 
        \times ( \nabla \times {\mathbf F} )$\\
D05: & $\frac{1}{2} \nabla^2 {\mathbf F} = {\mathbf F} 
        \times (\nabla \times {\mathbf F}) 
        + ({\mathbf F} \cdot \nabla) {\mathbf F}$\\
D06: & $\nabla \cdot ( \nabla \times {\mathbf F} ) = 0$\\
D07: & $\nabla \cdot ( \phi {\mathbf F}) 
        = {\mathbf F} \cdot \nabla \phi 
        + \phi \nabla \cdot {\mathbf F}$\\
D08: & $\nabla \cdot ( {\mathbf F} \times {\mathbf G} ) 
        = (\nabla \times {\mathbf F})\cdot {\mathbf G} 
        - (\nabla \times {\mathbf G})\cdot {\mathbf F}$\\
D09: & $\nabla \times (\phi {\mathbf F}) 
        = (\nabla \phi) \times {\mathbf F} 
        + \phi \nabla \times {\mathbf F}$\\
D10: & $ \nabla \times (\nabla \times {\mathbf F}) 
        = \nabla( \nabla \cdot {\mathbf F} ) 
        - \nabla ^2 {\mathbf F}$\\
D11: & $\nabla \times ({\mathbf F} \times {\mathbf G}) 
        = (\nabla \cdot {\mathbf G}) {\mathbf F} 
        - (\nabla \cdot {\mathbf F}) {\mathbf G}$\\
     & $\quad + ({\mathbf G} \cdot \nabla ){\mathbf F}  
     - ({\mathbf F} \cdot \nabla ) {\mathbf G}$\\
	 & \\
D12: & $\nabla \cdot {\mathbf r} = 3$\\
D13: & $\nabla \times {\mathbf r} = 0$\\
D14: & $({\mathbf G} \cdot \nabla) {\mathbf r} = {\mathbf G}$\\
D15: & $\nabla^2 {\mathbf r} = 0 $\\
D16: & $\phi(r)$ or ${\mathbf F}(r)$: $\nabla 
        = \frac{{\mathbf r}}{r} \frac{d}{d r}$\\
D17: & $\nabla( {\mathbf A} \cdot {\mathbf r} ) = {\mathbf A}$\\
D18: & $\phi({\mathbf A}\cdot{\mathbf r})$ 
        or ${\mathbf F}({\mathbf A}\cdot {\mathbf r})$: 
        $\nabla = {\mathbf A} \frac{d}{d ({\mathbf A}
        \cdot {\mathbf r})}$\\
\end{tabular*}
\vspace{0.96mm}
\end{textblock*}

%%%%%%%%%%%%%%%%%%%%%%%%%%%%%%%%%%%%%%%%%%%%%%%%%%%%%%%%%%%%%%%
%          Integral Theorems
%%%%%%%%%%%%%%%%%%%%%%%%%%%%%%%%%%%%%%%%%%%%%%%%%%%%%%%%%%%%%%%
\scriptsize
\textblockcolor{LightYellow}
\begin{textblock*}{61.8mm}(128.7mm,95.3mm)
\begin{tabular*}{60.8mm}{l @{\extracolsep{\fill}} l}
   & ~\\
\multicolumn{2}{c}{\bf Integral Theorems${}^\dag{}^\ddag$} \\
   & \\
I01: & $\int_{\mathcal V} \nabla \times {\mathbf F} \, 
        d {\mathcal V} = \int_{\partial {\mathcal V}} 
        {\hat n} \times {\mathbf F} \, d {\mathcal S}$ \vspace{1.5mm}\\
I02: & $ \int_{\mathcal V} \nabla \phi \, d {\mathcal V} 
    = \int_{\partial {\mathcal V}} \phi \, {\hat n} 
      \, d {\mathcal S}$ \vspace{1.5mm}\\
I03: & $\int_{\mathcal V} ( \phi \nabla^2 \psi 
        + \nabla \phi \cdot \nabla \psi) d {\mathcal V}$ \\
     & $ = \int_{\partial {\mathcal V}} \phi\nabla\psi 
        \cdot {\hat n} d{\mathcal S}$\vspace{1.5mm}\\
I04: & $\int_{\mathcal V} ( \phi \nabla^2 \psi 
       - \psi \nabla^2 \phi) d {\mathcal V}$\\ 
     & $\quad  = \int_{\partial {\mathcal V}}
       ( \phi\nabla\psi -\psi\nabla\phi) \cdot 
       {\hat n} d{\mathcal S}$\vspace{1.5mm}\\
I05: & $\int_{\mathcal V} d {\mathcal V} \nabla \phi \cdot {\mathbf F} 
       = \int_{\partial {\mathcal V}} \phi {\mathbf F} 
       \cdot {\hat n} \, d {\mathcal S} -$ \\
     & $\int_{\partial {\mathcal V}} \phi \nabla 
       \cdot {\mathbf F} d {\mathcal S}$\vspace{1.5mm}\\
I06: & $\int_{\partial {\mathcal S}} \phi \,{\hat t} 
     d \ell = \int_{\mathcal S} ({\hat n} 
     \times \nabla \phi )\, d{\mathcal S}$\vspace{1.5mm}\\
I07: & $\int_{\partial {\mathcal S}} d \ell \, 
       {\hat t}\times {\mathbf F} = \int_{\mathcal S} 
       ({\hat n} \times \nabla )\times {\mathbf F} 
       d {\mathcal S}$\vspace{1.5mm}\\
I08: & $\int_{\partial {\mathcal V}} d{\mathcal S} 
       \, {\hat n} \circ  = \int_{\mathcal V} 
       \, d{\mathcal V} \nabla \circ$\vspace{1.5mm}\\
I09: & $ \int_{\partial {\mathcal S}} d \ell \, 
      {\hat t} \circ = \int_{\mathcal S} d \, 
      {\mathcal S} ( {\hat n} \times \nabla )\circ$\vspace{1.5mm}\\
I10: & $\frac{d \Phi}{dt} = \int_{{\mathcal S}} 
                           \left[  \frac{\partial {\mathbf F}}{\partial t} 
                                 + ( \nabla \cdot {\mathbf F} ) {\mathbf V}
                           \right] \cdot d {\mathcal S}$ \\
     & $ + \quad \int_{\partial {\mathcal S}} 
           {\mathbf F} \times {\mathbf V} \cdot {\hat t} d \ell$\vspace{1.5mm}\\
I11: & $\frac{d }{dt} \int_{\mathcal V} \rho d {\mathbf V} = 
        \int_{\mathcal V} \frac{\partial \rho}{\partial t} d {\mathcal V} 
        + \int_{\partial {\mathcal V}} \rho {\mathbf v} \cdot d {\mathcal S}$\\ 
\end{tabular*}
\vspace{1.5mm}
\end{textblock*}

\TPshowboxesfalse
\begin{textblock*}{70mm}(0.2mm,170mm)
{\centering \bf Jacobian}\\
  Let a $n$-dimensional space be spanned by a
  set of vectors $\{ | {\mathbf e}_i \rangle\}$.
  Then the column-wise collection
    \[
      {\mathbf J} = \left[\, |{\mathbf e}_1 \rangle, \, | {\mathbf e}_2 \rangle, \,  \dots , \, | {\mathbf e}_n \rangle \, \vspace{10mm} \right]
    \]
  is know as the \emph{Jacobian matrix}.  Its inverse
    \[
      {\mathbf J}^{-1} = \left[ \begin{array}{c} \langle {\mathbf e}^1 | \vspace{1mm}\\ \langle {\mathbf e}^2 | \\ \vdots \\ \langle {\mathbf e}^n | \end{array} \right]
    \]
    gives a vector space spanned by the \emph{dual vectors} that obey the contraction rule
\end{textblock*}
\begin{textblock*}{70mm}(70mm,170mm)
\begin{eqnarray*}
  {\mathbf J}^{-1} {\mathbf J} & = &  \left[ \begin{array}{cccc}
                 \langle {\mathbf e}^1 | {\mathbf e}_1 \rangle & \langle {\mathbf e}^1 | {\mathbf e}_2 \rangle & \dots & \langle {\mathbf e}^1 | {\mathbf e}_n \rangle \vspace{1mm}\\
                 \langle {\mathbf e}^2 | {\mathbf e}_1 \rangle & \langle {\mathbf e}^2 | {\mathbf e}_2 \rangle & \dots & \langle {\mathbf e}^2 | {\mathbf e}_n \rangle \\
	    	     \vdots                   &     \vdots                 & \ddots &   \vdots \\
                 \langle {\mathbf e}^n | {\mathbf e}_1 \rangle & \langle {\mathbf e}^n | {\mathbf e}_2 \rangle & \dots & \langle {\mathbf e}^n | {\mathbf e}_n \rangle \\
			   \end{array} \right] \\
		   & = & \langle {\mathbf e}^i | {\mathbf e}_j \rangle = {\delta^i}_j
\end{eqnarray*}
and the completeness relation
\[
  {\mathbf J} {\mathbf J}^{-1} = \sum_i | {\mathbf e}_i \rangle \langle {\mathbf e}^i | = \mbox{Id}_{n \times n} .
\]
\end{textblock*}

\TPshowboxestrue
\begin{textblock*}{50.5mm}(140mm,170mm)
  \begin{tabular*}{50mm}{l @{\extracolsep{\fill}} l}
    & \\
  \multicolumn{2}{c}{\bf Permutation Tensor}\\
  \end{tabular*}
  \[
    \epsilon_{ijk} = \det({\mathbf J}) [ijk] = J [ijk]
  \]
  \[
    \epsilon^{ijk} = \det \left( {\mathbf J}^{-1} \right) [ijk] = J^{-1} [ijk]
  \]
  \[
    \epsilon^{ijk} \epsilon_{pqr} = \left| 
	   \begin{array}{ccc}
	   {\delta^i}_p & {\delta^i}_q & {\delta^i}_p\\
	   {\delta^j}_p & {\delta^j}_q & {\delta^j}_r\\
	   {\delta^k}_p & {\delta^k}_q & {\delta^k}_r\\
	   \end{array}
	  \right|
  \]
  \[
    \epsilon^{ijk} \epsilon_{pqk} = {\delta^i}_p {\delta^j}_q - {\delta^i}_q {\delta^j}_p
  \]
  \[
    \epsilon^{ijk} \epsilon_{pjk} = 2 {\delta^i}_p
  \]
  \[
      \epsilon^{ijk} \epsilon_{ijk} = 6
  \]
\end{textblock*}

\begin{textblock*}{140.15mm}(0mm,170mm)
 \vspace{51.05mm}
\end{textblock*}

%\begin{textblock*}
%A vector is uniquely specified by givings its divergence and curl within a reqion and its 
%normal component over the boundary.

%\emph{Helmholtz's Theorem:} A vector with both source and circulation densities vanishing at inifinity
%may be written as the sum of two parts, one of which is irrotational, the other solenoidal --- 
%$\mathbf{V} = - {\mathbf \nabla} \phi + \nabla \times {\mathbf A}$.
%\end{textblock*}

\begin{textblock*}{7.5in}(0mm,224.90mm)
  $\dag$ = inner product required \quad \quad $\ddag$ = valid only in 3 dimensions 
\end{textblock*}
\newpage
%Integrals - filename integral.tex

\null
%%%%%%%%%%%%%%%%%%%%%%%%%%%%%%%%%%%%%%%%%%%%%%%%%%%%%%%%%%%%%%%%%%%%%%%%%%%%%%%
%          Integrals
%%%%%%%%%%%%%%%%%%%%%%%%%%%%%%%%%%%%%%%%%%%%%%%%%%%%%%%%%%%%%%%%%%%%%%%%%%%%%%%
\textblockcolor{test}
\begin{textblock*}{7.5in}(0mm,0mm)
\begin{tabular*}{7.5in}{c @{\extracolsep{\fill}} c }
       \tiny ~ & ~\\
       \multicolumn{2}{c}{\normalsize \bf Integrals} \\
       \tiny~ & ~\\
\end{tabular*}
\end{textblock*}

%%%%%%%%%%%%%%%%%%%%%%%%%%%%%%%%%%%%%%%%%%%%%%%%%%%%%%%%%%%%%%%
%          Improper Integrals
%%%%%%%%%%%%%%%%%%%%%%%%%%%%%%%%%%%%%%%%%%%%%%%%%%%%%%%%%%%%%%%
\scriptsize
\textblockcolor{LightYellow}
\begin{textblock*}{60mm}(0mm,12.54mm)
\begin{tabular*}{58mm}{l @{\extracolsep{\fill}} l}
   & ~\\
\multicolumn{2}{c}{\bf Improper Integrals} \\
   & ~\\
\end{tabular*}
An integral is \emph{singular} or \emph{improper} if one or both limits are 
infinte or the integrand becomes unbounded at one or more points in the 
interval.  The plan of attack is to: 1) decide on integrability, 2) test
for convergence, and 3) calculate the integral (if integrable and convergent).
Case 1: Integral limits approach $\pm \infty$.
\end{textblock*}


\newpage
%Complex Analysis - filename complex.tex

\null
%%%%%%%%%%%%%%%%%%%%%%%%%%%%%%%%%%%%%%%%%%%%%%%%%%%%%%%%%%%%%%%
%          Complex Analysis
%%%%%%%%%%%%%%%%%%%%%%%%%%%%%%%%%%%%%%%%%%%%%%%%%%%%%%%%%%%%%%%
\textblockcolor{test}
\begin{textblock*}{190.5mm}(0mm,0mm)
\begin{tabular*}{190.5mm}{c @{\extracolsep{\fill}} c }
       \tiny ~ & ~\\
       \multicolumn{2}{c}{\normalsize \bf Complex Analysis} \\
       \tiny~ & ~\\
\end{tabular*}
\end{textblock*}

%%%%%%%%%%%%%%%%%%%%%%%%%%%%%%%%%%%%%%%%%%%%%%%%%%%%%%%%%%%%%%%
%          Basic Results
%%%%%%%%%%%%%%%%%%%%%%%%%%%%%%%%%%%%%%%%%%%%%%%%%%%%%%%%%%%%%%%
\scriptsize
\textblockcolor{LightYellow}
\begin{textblock*}{100mm}(0mm,12.54mm)
\begin{tabular*}{98mm}{l @{\extracolsep{\fill}} l}
   & ~\\
\multicolumn{2}{c}{\bf Basic Results} \\
   & ~\\
mapping:            & $\phi : z \equiv (x+iy) \rightarrow w \equiv \left(u(x,y) + iv(x,y)\right)$\\
                    & \\
Cauchy-Riemann:     & $ \frac{\partial u}{\partial x} =  \frac{\partial v}{\partial v} 
                        \quad \& \quad 
                        \frac{\partial u}{\partial y} = -\frac{\partial v}{\partial x} \iff w$ analytic\\
& \\
contour integral:   & $\int_{z_1}^{z_2} f(z) dz = \int_{(x_1,y_1)}^{(x_2,y_2)}\left[ u(x,y) + iv(x,y)\right]\left[dx + idy\right]$\\
                    & with the path from $(x_1,y_1)$ to $(x_2,y_2)$ specified\\
& \\
Cauchy's theorem:   & $\oint_{\mathcal C} f(z) dz = 0 \iff f(z)$ is analytic throughout some\\
                    &  simply connected region ${\mathcal R}$ bound by ${\mathcal C}$\\
& \\                    
Cauchy's formula:   & $\oint_{\mathcal C} \frac{f(z)}{z-z_0} dz = 2 \pi i f(z_0)$ where $f(z)$ is analytic\\
& \\
Cauchy's formula:   & $f^{(n)}(z_0) = \frac{n!}{2\pi i} \oint_{\mathcal C} \frac{f(z) dz}{\left(z-z_0\right)^n}$\\
& \\
residue primitve:   & $\oint_{\mathcal C} \left(z-z_0\right)^n dz = \left\{ \begin{array}{ccc}  2 \pi i & n = -1    & {\mathcal C} \quad ccw \\
                                                                                               -2 \pi i & n = -1    & {\mathcal C} \quad cw  \\
                                                                                                      0 & n \neq -1 &                   
                                                                           \end{array} \right. $\\
& \\
Laurent series:     & $q(z) = \sum_{n=-\infty}^{n=\infty} a_n \left(z - z_0\right)^n$\\
& \\
residue:            & $a_{-1} = \left. \left(z-z_0\right) q(z) \right|_{z=z_0}$\\
& \\
\end{tabular*}
\end{textblock*}

%%%%%%%%%%%%%%%%%%%%%%%%%%%%%%%%%%%%%%%%%%%%%%%%%%%%%%%%%%%%%%%
%          Residue Primitive
%%%%%%%%%%%%%%%%%%%%%%%%%%%%%%%%%%%%%%%%%%%%%%%%%%%%%%%%%%%%%%%
\textblockcolor{LightYellow}
\begin{textblock*}{90.65mm}(99.85mm,12.54mm)

\begin{tabular*}{88.65mm}{l @{\extracolsep{\fill}} l}
   & ~\\
\multicolumn{2}{c}{\bf Residue Primitive} \\
   & ~\\
\end{tabular*}
The point of this article is to establish the residue primitive result by explicitly
evaluating the contour integral
\[
   I = \oint_{\mathcal C} \left( z - z_0 \right)^n dz,
\]
where ${\mathcal C}$ is a circular contour around $z_0$ in a counterclock-wise (ccw)
direction.  Assume on ${\mathcal C}$ that $z - z_0 = R e^{i \theta}$ where $\theta$ 
goes from $0$ to $2 \pi$. With this parametrization $dz = R i e^{i \theta} d \theta$ and
\[
  I = i R^{n+1} \int_0^{2 \pi} e^{i \theta \left( n + 1 \right)} d \theta .
\]
There are two cases to consider: $n \neq -1$ and $n = -1$. 
In the first case
\begin{eqnarray*}
  I_{n \neq -1}  & = & i R^{n+1} \int_0^{2 \pi} e^{i \theta \left(n+1\right)} d \theta \\
                & = & \frac{R^{n+1}}{n+1} \left. e^{i \theta \left(n+1\right)} \right|^{2 \pi}_0 \\
                & = & 0
\end{eqnarray*}
and in the second
\[
  I_{n = -1}   = i \int_0^{2 \pi} d \theta = 2 \pi i
\]
To determine the value of the integral if the contour were traversed in the clockwise direction,
the limits of the integral can be reversed so that it went from $2 \pi$ to $0$ but ir is more
instructive to consider the closed contour ${\mathcal C} \rightarrow {\mathcal C}_{in} 
\rightarrow {\mathcal C}_{out} \rightarrow {\mathcal C}_{2}$. 
\begin{figure}[htp]
\centering
\psscalebox{2}{
\begin{pspicture}(-1.5,-1.5)(1.5,1.5)
  \pscircle[fillstyle=solid,fillcolor=LightGreen,linestyle=none](0,0){1.005}
  \psarc[linewidth=0.5pt]{->}(0,0){1}{32.96709734}{212}
  \psarc[linewidth=0.5pt]{-}(0,0){1}{210}{27.03290266}
  \pscircle[fillstyle=solid,fillcolor=LightYellow,linestyle=none](0,0){0.21}  
  \psarc[linewidth=0.5pt]{-}(0,0){0.205}{45}{190}
  \psarc[linewidth=0.5pt]{<-}(0,0){0.205}{180}{15}
  \pspolygon[fillstyle=solid,fillcolor=LightYellow,linestyle=none,linewidth=0.5pt](0.19125331,0.05124617)(0.89067955,0.45446836)(0.83892092,0.54411694)(0.14140721,0.14140721)
  \psline[linewidth=0.5pt]{-}(0.19318517,0.05176381)(0.89076863,0.45451381)
  \psline[linewidth=0.5pt]{-}(0.14142136,0.14142136)(0.83900482,0.54417136)
  \psline[linewidth=0.5pt]{<-}(0.5419769,0.25313881)(0.89076863,0.45451381)
  \psline[linewidth=0.5pt]{->}(0.14142136,0.14142136)(0.49021309,0.34279636)
  \psline[linewidth=0.2pt]{->}(0,0)(1.29903811,0.75)
  \pscircle[fillstyle=solid,fillcolor=black,linestyle=none](0,0){0.05}
  \uput[45](0.6,0.6){${\scriptsize {\mathcal C}}$}
  \uput[1](-0.8,-0.05){${z_0}$}
  \uput[1](-0.4,-0.4){${{\mathcal C}_2}$}
  \uput[1](-0.2,0.45){${{\mathcal C}_{in}}$}  
  \uput[1](0.2,0){${{\mathcal C}_{out}}$}  
  %\psgrid(-1.5,-1.5)(1.5,1.5)
\end{pspicture}}
\end{figure}
From Cauchy's theorem $\left(z - z_0 \right)^n$ is analytic in the region bound by this combined
contour and the integrals over ${\mathcal C}_{in}$ and ${\mathcal C}_{out}$ cancel leaving
$I_{\mathcal C} = -I_{ {\mathcal C}_2 }$.
\end{textblock*}
\newpage
%Fourier Analysis - filename fourier.tex

\null
%%%%%%%%%%%%%%%%%%%%%%%%%%%%%%%%%%%%%%%%%%%%%%%%%%%%%%%%%%%%%%%
%          Fourier Analysis
%%%%%%%%%%%%%%%%%%%%%%%%%%%%%%%%%%%%%%%%%%%%%%%%%%%%%%%%%%%%%%%
\textblockcolor{test}
\begin{textblock*}{7.5in}(0mm,0mm)
\begin{tabular*}{7.5in}{c @{\extracolsep{\fill}} c }
       \tiny ~ & ~\\
       \multicolumn{2}{c}{\normalsize \bf Fourier Analysis} \\
       \tiny~ & ~\\
\end{tabular*}
\end{textblock*}

%%%%%%%%%%%%%%%%%%%%%%%%%%%%%%%%%%%%%%%%%%%%%%%%%%%%%%%%%%%%%%%
%          Periodicity of Functions
%%%%%%%%%%%%%%%%%%%%%%%%%%%%%%%%%%%%%%%%%%%%%%%%%%%%%%%%%%%%%%%
\begin{textblock*}{90mm}(0mm,15mm)
Consider a periodic function $f(x-2\ell) = f(x) = f(x + 2\ell)$ defined on the interval $[-\ell,\ell]$.  
Because of the underlying periodicity of $f(x)$, both the derivative and the integral also inherit 
periodic properties.

In the case of the derivative, the observation is evident simply from 
the standard definition as follows:
\begin{eqnarray*}
  \frac{d f}{d x}(x) & = & \lim_{dx \rightarrow 0} \frac{ f(x+dx)-f(x)}{dx}\\
                     & = & \lim_{dx \rightarrow 0} \frac{ f(x + 2\ell + dx) - f(x + 2\ell) }{dx} \\
                     & = & \frac{d f}{d x}(x + 2\ell).
\end{eqnarray*}
In the case of the integral, a bit more thinking and manipulation is needed.
To begin, consider the integral of $f(x)$ over an interval $[a,b]$.  The value of this integral 
is invariant if the entire interval is shifted by $2 \ell$.  This is proven by direct substitution
as 
\begin{eqnarray}\label{eq_per_ab_int}
  \int_{a}^{b} f(x) dx & = & \int_{a}      ^{b}       f(x - 2\ell) dx            \\
                       & = & \int_{a+2\ell}^{b+2\ell} f(y)         dy  \nonumber \\
                       & = & \int_{a+2\ell}^{b+2\ell} f(x)         dx. \nonumber
\end{eqnarray}
With this result, it is easy (although somewhat subtle) to show that the integral
of $f(x)$ over the interval $[c-\ell,c+\ell]$ is equal to the integral over 
the interval $[-\ell,\ell]$.  To do so, start by letting $a = c - \ell$ and $b = -\ell$
in (\ref{eq_per_ab_int}), which yields the identity 
\begin{equation}\label{eq_per_int}
  \int_{c-\ell}^{-\ell} f(x) dx  =  \int_{c+\ell}^{\ell} f(x) dx,
\end{equation}
which is then used to manipulate the integral over $[c-\ell,c+\ell]$ as follows,
\begin{eqnarray} 
  \int_{c-\ell}^{c+\ell} f(x) dx  & = &  \int_{c-\ell}^{-\ell} f(x) dx + \int_{-\ell}^{c+\ell} f(x) dx \\
                                  & = &  \int_{c+\ell}^{\ell}  f(x) dx + \int_{-\ell}^{c+\ell} f(x) dx \nonumber \\
                                  & = &  \int_{-\ell}^{c+\ell} f(x) dx + \int_{c+\ell}^{\ell}  f(x) dx \nonumber \\
                                  & = &  \int_{-\ell}^{\ell}   f(x) dx                                 \nonumber
\end{eqnarray}
where (\ref{eq_per_int}) was used in going from the first to the second line.
\end{textblock*}



%%%%%%%%%%%%%%%%%%%%%%%%%%%%%%%%%%%%%%%%%%%%%%%%%%%%%%%%%%%%%%%
%          Fourier Analysis on a Finite Interval
%%%%%%%%%%%%%%%%%%%%%%%%%%%%%%%%%%%%%%%%%%%%%%%%%%%%%%%%%%%%%%%
\begin{textblock*}{100mm}(89.85mm,15mm)
\[
  \int_{c}^{c+2L} \cos \left( \frac{m \pi x}{L} \right) dx = 0
 \]
 \[
  \int_{c}^{c+2L} \sin \left( \frac{m \pi x}{L} \right) dx = 0
 \]
 
\[
  \int_{c}^{c+2L} dx \, \cos \left( \frac{m \pi x}{L} \right) \cos \left( \frac{n \pi x}{L} \right) = 
      \left\{ \begin{array}{cc} 2 L \, \delta_{mn} & m = 0 \\ L \, \delta_{mn} & m \neq 0 \end{array} \right.
\]
\[
  \int_{c}^{c+2L} dx \, \sin \left( \frac{m \pi x}{L} \right) \sin \left( \frac{n \pi x}{L} \right) = 
      \left\{ \begin{array}{cc} 0 \, \delta_{mn} & m = 0 \\ L \, \delta_{mn} & m \neq 0 \end{array} \right.
\]
\[
  u_n \equiv \frac{1}{\sqrt{L}} \sin\left( \frac{n \pi x}{L} \right) \quad n = 1, 2, \dots
\]
\[
  v_n \equiv \frac{1}{\sqrt{L}} \cos\left( \frac{n \pi x}{L} \right) \quad n = 0, 1, 2, \dots
\]
\[
  f = \sum_{n=1}^{\infty} \left\{ v_n (v_n \cdot f) + u_n (u_n \cdot f) \right\} + \frac{1}{2} (v_0 \cdot f) v_0
\]
\[
 {\tilde f}(x) = \frac{a_0}{2} + 
    \sum_{n=1}^{\infty} \left( a_n \cos \left( \frac{n \pi x}{L} \right) + b_n \sin \left( \frac{n \pi x}{L} \right) \right)
\]
\[
  a_n = \frac{1}{L} \int_{c}^{c+2L} f(x) \cos \left( \frac{n \pi x}{L} \right) dx \equiv \frac{1}{\sqrt{L}} (f \cdot v_n)
\]
\[
  b_n = \frac{1}{L} \int_{c}^{c+2L} f(x) \sin \left( \frac{n \pi x}{L} \right) dx
\]

\[
  {\tilde f}(x) = \sum_{n = - \infty}^{\infty} c_n e^{i n \pi x/L}
\]
\[
  c_n = \frac{1}{2 L} \int_{c}^{c+2L} f(x) e^{-i n \pi x/L} dx
\]
\end{textblock*}

\newpage
\null

%%%%%%%%%%%%%%%%%%%%%%%%%%%%%%%%%%%%%%%%%%%%%%%%%%%%%%%%%%%%%%%
%          Fourier Transform
%%%%%%%%%%%%%%%%%%%%%%%%%%%%%%%%%%%%%%%%%%%%%%%%%%%%%%%%%%%%%%%
\begin{textblock*}{190.5mm}(0mm,0mm)
\begin{tabular*}{7.5in}{c @{\extracolsep{\fill}} c }
       \tiny ~ & ~\\
       \multicolumn{2}{c}{\normalsize \bf Fourier Transform} \\
       \tiny~ & ~\\
\end{tabular*}
\end{textblock*}

%%%%%%%%%%%%%%%%%%%%%%%%%%%%%%%%%%%%%%%%%%%%%%%%%%%%%%%%%%%%%%%
%          Definitions
%%%%%%%%%%%%%%%%%%%%%%%%%%%%%%%%%%%%%%%%%%%%%%%%%%%%%%%%%%%%%%%
\scriptsize
\textblockcolor{LightYellow}
\begin{textblock*}{80mm}(0mm,12.77mm)
\begin{tabular*}{78mm}{l @{\extracolsep{\fill}} l}
\multicolumn{2}{c}{\bf Definitions} \\
  & \\
Constants                        & $a, b$\\
  & \\
General function                 & $f, g$\\
  & \\
Schwartz function                & $\phi$, $\psi$\\
  & \\
Tempered distribution            & ${\mathcal T}, {\mathcal T}_{f}, {\mathcal S}$\\
  & \\
Shift Operator                   & $( \tau_{\pm b} f)(x) \equiv f(x \mp b) $\\  
  & \\
Distributational pairing         & $\langle {\mathcal T}_{f} , \phi \rangle \equiv \int_{-\infty}^{\infty} f(x) \phi(x) dx$ \\
  & \\
Distributional linearity         & $\langle {\mathcal T}, a \phi + b \psi \rangle \equiv a \langle {\mathcal T},\phi \rangle + b \langle {\mathcal T}, \psi \rangle$\\
  & \\
Distributional derivative        & $\langle {\mathcal T}' , \phi \rangle \equiv - \langle {\mathcal T} , \phi' \rangle$ \\
  & \\
Distributional reversal          & $\langle {\mathcal T}^{-} , \phi \rangle \equiv \langle {\mathcal T} , \phi^{-} \rangle$ \\
  & \\  
Convolution                      & $(g * f)(t) \equiv \int_{-\infty}^{\infty} g(t-\tau) f(\tau) d \tau$\\
  & \\
Dist. Fourier Transform          & $\langle {\mathcal F} {\mathcal T} , \phi \rangle \equiv \langle {\mathcal T} , {\mathcal F}  \phi \rangle$ \\
  & \\
Dist. Inverse FT                 & $\langle {\mathcal F}^{-1} {\mathcal T} , \phi \rangle \equiv \langle {\mathcal T} , {\mathcal F}^{-1}  \phi \rangle$  \\
\end{tabular*}
\end{textblock*}

%%%%%%%%%%%%%%%%%%%%%%%%%%%%%%%%%%%%%%%%%%%%%%%%%%%%%%%%%%%%%%%
%          Specifc Distributions
%%%%%%%%%%%%%%%%%%%%%%%%%%%%%%%%%%%%%%%%%%%%%%%%%%%%%%%%%%%%%%%
\begin{textblock*}{80mm}(0mm,90mm)
\begin{tabular*}{78mm}{l @{\extracolsep{\fill}} l}
\multicolumn{2}{c}{\bf Specific Distributions} \\
 & \\
 delta            & $\delta = \left\{ \begin{array}{l} \delta(x) = 0 \; x \neq 0 \\  \\ \int_{-\infty}^{\infty} \delta(x) dx = 1 \end{array} \right.$ \\
 & \\
 signum           & $sgn(x) = \left\{ \begin{array}{ll} -1 & x < 0 \\ 1 & x > 0 \end{array} \right.$ \\
 & \\
 unit step        & $H(x) = \left\{ \begin{array}{ll} 0 & x < 0 \\ 1 & x > 0 \end{array} \right.$    \\
 & \\
 unit ramp        & $R(x) = \left\{ \begin{array}{ll} 0 & x \le 0 \\ x & x > 0 \end{array} \right.$  \\  
 & \\
 delta pairing    & $\langle \delta , \phi \rangle = \phi(0)$   \\
 & \\
 delta pairing    & $\langle \delta_a , \phi \rangle = \phi(a)$ \\
 & \\
 unit step deriv. & $H' = \delta$ \\
 & \\
 signum deriv.    & $sgn' = 2 \delta$ \\
 & \\
 delta deriv.     & $\langle \delta' , \phi \rangle = - \langle \delta , \phi' \rangle =  -\phi'(0)$ \\
\end{tabular*}
\end{textblock*}

%%%%%%%%%%%%%%%%%%%%%%%%%%%%%%%%%%%%%%%%%%%%%%%%%%%%%%%%%%%%%%%
%          Specific Fourier Transforms
%%%%%%%%%%%%%%%%%%%%%%%%%%%%%%%%%%%%%%%%%%%%%%%%%%%%%%%%%%%%%%%
\begin{textblock*}{75mm}(0mm,180mm)
\begin{tabular*}{73mm}{l @{\extracolsep{\fill}} l}
\multicolumn{2}{c}{\bf Specific Fourier Transforms} \\
 & \\
 delta            & ${\mathcal F} \delta = 1$ \\
 & \\
 1                & ${\mathcal F} 1 = \delta$ \\
 & \\
 unit step        & ${\mathcal F} \delta_{\pm a} = e^{\mp 2 \pi i s a}$    \\
 & \\
 unit ramp        & ${\mathcal F} e^{\pm 2 \pi i x a} = \delta_{\pm a}$  \\  
 & \\
 delta pairing    & ${\mathcal F} \left[ \frac{1}{2} \left( \delta_a + \delta_{-a} \right) \right] = \cos 2 \pi s a$   \\
 & \\
 delta pairing    & ${\mathcal F} cos 2 \pi a x = \frac{1}{2} \left[ \delta_a + \delta_{-a} \right]$ \\
 & \\
 unit step deriv. & $H' = \delta$ \\
 & \\
 signum deriv.    & $sgn' = 2 \delta$ \\
 & \\
 delta deriv.     & ${\mathcal F}\left[ \frac{1}{2 i} \left( \delta_{-a} - \delta_{a} \right) \right] = \sin 2 \pi s a$ \\
\end{tabular*}
\end{textblock*}

%%%%%%%%%%%%%%%%%%%%%%%%%%%%%%%%%%%%%%%%%%%%%%%%%%%%%%%%%%%%%%%
%          Fourier Transform Notation
%%%%%%%%%%%%%%%%%%%%%%%%%%%%%%%%%%%%%%%%%%%%%%%%%%%%%%%%%%%%%%%
\begin{textblock*}{110.5mm}(79.85mm,12.77mm)
\begin{tabular*}{108.5mm}{l @{\extracolsep{\fill}} l}
\multicolumn{2}{c}{\bf Fourier Transform Notation} \\
 & \\
\end{tabular*}
There are many different conventions and notations for the Fourier transform.  
The convention used here will be that the complex exponentials will
have the $2 \pi$ explicitly indicated and paired with the transform
variable rather than the usual physics standard of angular frequency,
(\emph{e.g.}, $2 \pi s$ instead of $\omega$).  This usage
supresses the needs for $1/\sqrt{2 \pi}$ or similar terms outside the 
integral.

For notation, two types are employed.  The first is this simple
pairing $f(t) \rightleftharpoons F(s)$ which tends to emphasize 
the two domains (time and frequency) but which clouds the duality.
The second is the script notation ${\mathcal F}f(s)$ which allows
for greater insight into the transform but which should be interpretted as 
follows:
\begin{enumerate}
  \item ${\mathcal F} f$ means substitute $f$ into a Fourier integral
  \item the absence or presence of the inverse symbol ${}^{-1}$ indicates
        which sign to use ($-$ - absence, $+$ - presence)
  \item the variable that follows $(s)$ indicates the free variable in the 
        integral (can be suppressed), the other being dummy. 
\end{enumerate}
Finally, there will be occasions when the reversal of a signal,
defiend as $f(t) \rightarrow f(-t)$, will be examined.  When pairing
this $f(-t)$ with a Fourier integral in the script notation, a 
convenient way to suppress the dummy variable (\emph{i.e.} $t$)
is with the notation $f^{-}$.  It can also be applied to the 
transform itself - ${\mathcal F}f(-s) \equiv ({\mathcal F} f)^{-}$.
\end{textblock*}



%%%%%%%%%%%%%%%%%%%%%%%%%%%%%%%%%%%%%%%%%%%%%%%%%%%%%%%%%%%%%%%
%          Basic Fourier Transform Results
%%%%%%%%%%%%%%%%%%%%%%%%%%%%%%%%%%%%%%%%%%%%%%%%%%%%%%%%%%%%%%%
\begin{textblock*}{110.5mm}(79.85mm,84.7mm)
\begin{tabular*}{108.5mm}{l @{\extracolsep{\fill}} l}
\multicolumn{2}{c}{\bf Basic Fourier Transform Results} \\
 & \\
       Pairing Notation 1  & $f(t) \rightleftharpoons F(s)$ \\
& \\      
       Pairing Notation 2  & $f(t) \rightleftharpoons {\mathcal F}f(s)$ \\
& \\
 FT01:  Forward Transform  & $F(s) \equiv \int_{-\infty}^{\infty} f(t) e^{-2 \pi i s t} dt$\\
  & \\
 FT02:  Inverse Transform  & $f(t) \equiv \int_{-\infty}^{\infty} F(s) e^{2 \pi i s t} ds$ \\
  & \\
 FT03:  Forward Reversal   & $({\mathcal F}{\mathcal T})^{-} = {\mathcal F}^{-1} {\mathcal T}$\\
  & \\
 FT04:  Inverse Reversal   & $({\mathcal F}^{-1} {\mathcal T})^{-} = {\mathcal F} {\mathcal T}$ \\
  & \\
 FT05:  Reversed Forward   & ${\mathcal F} {\mathcal T}^{-}  = {\mathcal F}^{-1} {\mathcal T}$ \\
  & \\
 FT06:  Reversed Inverse   & ${\mathcal F}^{-1} {\mathcal T}^{-} = {\mathcal F} {\mathcal T}$ \\
  & \\
 FT07:  Linearity          & ${\mathcal F}(a {\mathcal T} + b {\mathcal S} ) = a {\mathcal F}{\mathcal T} + b {\mathcal F}{\mathcal S}$\\
  & \\
 FT08:  Shift \& Pairing   & $\langle \tau_{\pm b} f, \phi \rangle = \langle f, \tau_{\mp b} \phi \rangle$\\
  & \\
 FT09:  Time Shift         & ${\mathcal F}( \tau_{\pm b} f) = e^{\mp 2 \pi i s b} {\mathcal F} f$ \\
  & \\
 FT10: Frequency Shift     & $\tau_{\pm b} {\mathcal F}f (s) \rightleftharpoons f(t) e^{\pm 2 \pi i b t}$ \\
  & \\
 FT11: Shift Theorem       & ${\mathcal F}(\tau_{\pm b} {\mathcal T}) = e^{\mp 2 \pi i b x} {\mathcal T}$\\
  & \\
 FT12: Scaling             & $f(at) \rightleftharpoons \frac{1}{|a|} {\mathcal F}f\left( \frac{s}{a} \right)$ \\
  & \\
 FT13: Deriv. Func.        & $f' \rightleftharpoons (2 \pi i s) F(s)$\\
  & \\
 FT14: Power Rule Func.    & $(-2 \pi i t) f(t) \rightleftharpoons F'(s)$\\  
  & \\
 FT15: Deriv. Dist.        & ${\mathcal T}' \rightleftharpoons 2 \pi i x {\mathcal F}{\mathcal T}$\\
  & \\
 FT16: Power Rule Dist.    & $-2 \pi i t {\mathcal T} \rightleftharpoons ( {\mathcal F} {\mathcal T} )'$\\
  & \\
 FT17: Convolution Theorem & $f*g \rightleftharpoons F(s) G(s)$\\
  & \\
 FT18: Convolution Assoc.  & $(k*g)*f = k*(g*f)$\\
\end{tabular*}
\end{textblock*}
\newpage
%Differential Forms Version of Classical Vector Analysis - filename diff_forms_classical_vec.tex

\null
%%%%%%%%%%%%%%%%%%%%%%%%%%%%%%%%%%%%%%%%%%%%%%%%%%%%%%%%%%%%%%%%%%%%%%%%%%%%%%%
%          Differential Forms Version of Classical Vector Analysis
%%%%%%%%%%%%%%%%%%%%%%%%%%%%%%%%%%%%%%%%%%%%%%%%%%%%%%%%%%%%%%%%%%%%%%%%%%%%%%%
\textblockcolor{test}
\begin{textblock*}{7.5in}(0mm,0mm)
\begin{tabular*}{7.5in}{c @{\extracolsep{\fill}} c }
       \tiny ~ & ~\\
       \multicolumn{2}{c}{\normalsize \bf Differential Forms Version of
                                          Classical Vector Analysis} \\
       \tiny~ & ~\\
\end{tabular*}
\end{textblock*}

%%%%%%%%%%%%%%%%%%%%%%%%%%%%%%%%%%%%%%%%%%%%%%%%%%%%%%%%%%%%%%%
%          Conversions Between
%%%%%%%%%%%%%%%%%%%%%%%%%%%%%%%%%%%%%%%%%%%%%%%%%%%%%%%%%%%%%%%
\scriptsize
\textblockcolor{LightYellow}
\begin{textblock*}{60mm}(0mm,12.54mm)
\begin{tabular*}{58mm}{l @{\extracolsep{\fill}} l}
   & ~\\
\multicolumn{2}{c}{\bf Basic Geometric Objects1} \\
   & ~\\
\end{tabular*}
\vspace{11.56mm}
\end{textblock*}

\[
  {\vec A} <=> \phi_{A}
\]
\[
  \phi_{A} = A^x dx + A^y dy + A^z dz
\]
\[
  {\vec \nabla} \times {\vec A} <=> *d \phi_{A}
\]
\[
 div({\vec A}) <=> *d* \phi_{A}
\]
\[
  {\vec \nabla} \cdot {\vec \nabla} \times {\vec A} <=> (*d*)(*d \phi_{A})
\]
\[
  {\vec \nabla} \times {\vec \nabla} f = *d(df) = *d^2 f = 0
\]
\[
  {\vec \nabla} \times ( {\vec \nabla} {\vec A} ) <=> (*d)(*d \phi_{A} ) = *d*d \phi_{A}
\]
This sucks!

\newpage
%Classical Mechanics - filename class_mech.tex

\null
%%%%%%%%%%%%%%%%%%%%%%%%%%%%%%%%%%%%%%%%%%%%%%%%%%%%%%%%%%%%%%%
%          Classical Mechanics
%%%%%%%%%%%%%%%%%%%%%%%%%%%%%%%%%%%%%%%%%%%%%%%%%%%%%%%%%%%%%%%
\textblockcolor{test}
\begin{textblock*}{7.5in}(0mm,0mm)
\begin{tabular*}{7.5in}{c @{\extracolsep{\fill}} c }
       \tiny ~ & ~\\
       \multicolumn{2}{c}{\normalsize \bf Classical Mechanics} \\
       \tiny~ & ~\\
\end{tabular*}
\end{textblock*}

%%%%%%%%%%%%%%%%%%%%%%%%%%%%%%%%%%%%%%%%%%%%%%%%%%%%%%%%%%%%%%%
%          Invariance of the EL Equations
%%%%%%%%%%%%%%%%%%%%%%%%%%%%%%%%%%%%%%%%%%%%%%%%%%%%%%%%%%%%%%%
\scriptsize
\textblockcolor{LightYellow}
\begin{textblock*}{75mm}(0mm,12.77mm)
The Euler Lagrange equations are invariant to a basic change in coordinates from $q^i$ to
$y^j$ of the form 
\begin{equation}\label{ELinv_b}
  q^i = q^i(y^j) .
\end{equation}
To see this first note that the from the 
form of the transformation equation (\ref{ELinv_b}) we get
\begin{equation}\label{ELinv_a}
  {\dot q}^i = \frac{\partial q^i}{\partial y^j} {\dot y}^j \,
\end{equation}
where $\dot f = \frac{d}{dt} f$. Next note that the Lagrangian 
$\tilde L (y^j,{\dot y}^j;t )$ in the $y^j$ coordinates 
is related to the Lagrangian $L (q^i,{\dot q}^i;t )$ 
in $q^i$ coordinates by virtue of a substitution of (\ref{ELinv_b}) 
and (\ref{ELinv_a}) yielding
\begin{equation}\label{ELinv_c}
  \tilde L (y^j,{\dot y}^j;t ) = L (q^i(y^j),{\dot q}^i(y^j,{\dot y}^j);t ) .
\end{equation}
The parts of the Euler-Lagrange equation in terms
of the $y^j$ coordinates in relation to the $q^i$ coordinates are
\begin{equation}\label{ELinv_d}
  \frac{\partial \tilde L}{\partial y^j} =    \frac{\partial L}{\partial q^i}        \frac{\partial q^i}{\partial y^j} 
                                        +  \frac{\partial L}{\partial {\dot q}^i} \frac{\partial {\dot q}^i}{\partial y^j}
\end{equation}
and
\begin{equation}\label{ELinv_e}
  \frac{\partial \tilde L}{\partial {\dot y}^j} = \frac{\partial L}{\partial {\dot q}^i} \frac{\partial {\dot q}^i}{\partial {\dot y}^j} .
\end{equation}
Now substituting (\ref{ELinv_d}) and (\ref{ELinv_e}) into the Euler-\\Lagrange equations
yields
\begin{eqnarray}\label{ELinv_f}
  \frac{d}{dt} \left( \frac{\partial \tilde L}{\partial {\dot y}^j } \right) - \frac{\partial \tilde L}{\partial y^j} 
    & = &   \frac{d}{d t} \left( \frac{\partial L}{\partial {\dot q}^i} \right) \frac{\partial {\dot q}^i}{\partial {\dot y}^j} \\ \nonumber
	&   & + \frac{\partial L}{\partial {\dot q}^i} \frac{d}{dt} \left( \frac{\partial {\dot q}^i}{\partial {\dot y}^j} \right) \\ \nonumber
    & = & - \frac{\partial L}{\partial q^i}        \frac{\partial q^i}{\partial y^j} 
	      - \frac{\partial L}{\partial {\dot q}^i} \frac{\partial {\dot q}^i}{\partial y^j} .\nonumber
\end{eqnarray}
But from (\ref{ELinv_a}) $ {\partial {\dot q}^i}/{\partial {\dot y}^j} = {\partial q^i}/{\partial y^j}$ and thus the second and
fourth terms in (\ref{ELinv_f}) cancel, leaving
\begin{equation}\label{ELinv_g}
  \frac{d}{dt} \left( \frac{\partial \tilde L}{\partial {\dot y}^j } \right) - \frac{\partial \tilde L}{\partial y^j} = 
  \left[ \frac{d}{d t} \left( \frac{\partial L}{\partial {\dot q}^i} \right) - \frac{\partial L}{\partial q^i} \right] 
  \frac{\partial q^i}{\partial y^j} .
\end{equation}
Using the definition $\Xi_i = \frac{d}{dt}\left(\frac{\partial L}{\partial \dot q^i} \right) - \frac{\partial L}{\partial q^i}$,
Eq. (\ref{ELinv_g}) takes on the more obvious form of
\begin{equation}\label{ELinv_h}
  \Xi_{\tilde j} = \Xi_i {\Lambda^i}_{\tilde j}
\end{equation}
which shows that the Euler-Lagrange equations transform like the components of a covariant vector.
\end{textblock*}



%%%%%%%%%%%%%%%%%%%%%%%%%%%%%%%%%%%%%%%%%%%%%%%%%%%%%%%%%%%%%%%
%          EL Equations in 1st order form
%%%%%%%%%%%%%%%%%%%%%%%%%%%%%%%%%%%%%%%%%%%%%%%%%%%%%%%%%%%%%%%
\begin{textblock*}{65mm}(74.85mm,12.77mm)
The Euler-Lagrange equations can be cast into 1$^{st}$-order form by first making the identification
\[
  \frac{d}{d t} q^{\alpha} = {\dot q}^{\alpha}
\]
and then by expanding
\[
  \frac{d}{dt} \left( \frac{\partial L}{\partial {\dot q}^{\alpha}} \right) = 
    \frac{\partial^2 L}{\partial q^{\beta} \partial {\dot q}^{\alpha}} {\dot q}^{\beta} + 
	\frac{\partial^2 L}{\partial {\dot q}^{\alpha} \partial {\dot q}^{\beta}} {\ddot q^{\beta}}         +
	\frac{\partial^2 L}{\partial t \partial q^{\alpha}}                 
\]
Substituting this form in the EL equations and solving for the ${\ddot q}^{\alpha}$ yields
\begin{eqnarray*}
  \frac{d}{dt} {\dot q}^{\alpha} & = & {\ddot q}^{\alpha} \\ 
                                 & = & \left( \frac{\partial^2 L}{\partial {\dot q}^{\alpha} \partial {\dot q}^{\beta}} \right)^{-1} \\
								 &   & \times \left( \frac{\partial L}  {\partial q^{\beta}}                    -
							                         \frac{\partial^2 L}{\partial t \partial {\dot q}^{\beta}}  - 
									                 \frac{\partial^2 L}{\partial q^{\beta} \partial {\dot q}^{\gamma}} {\dot q}^{\gamma} 
									   \right)										  
\end{eqnarray*}
which requires that the Hessian, defined by
\[
  \left( \frac{\partial^2 L}{\partial {\dot q}^{\alpha} \partial {\dot q}^{\beta}} \right)
\]
be invertible.
\end{textblock*}
\newpage
%Classical Electromagnetism- filename em.tex

\null
%%%%%%%%%%%%%%%%%%%%%%%%%%%%%%%%%%%%%%%%%%%%%%%%%%%%%%%%%%%%%%%%%%%%%%%%%%%%%%%
%          Classical Electromagnetism
%%%%%%%%%%%%%%%%%%%%%%%%%%%%%%%%%%%%%%%%%%%%%%%%%%%%%%%%%%%%%%%%%%%%%%%%%%%%%%%
\textblockcolor{test}
\begin{textblock*}{7.5in}(0mm,0mm)
\begin{tabular*}{7.5in}{c @{\extracolsep{\fill}} c }
       \tiny ~ & ~\\
       \multicolumn{2}{c}{\normalsize \bf Classical Electromagnetism} \\
       \tiny~ & ~\\
\end{tabular*}
\end{textblock*}

\begin{tabular}{lll}
   \centering{Point Form} & Integral Form & Label and Name \\
              &               &                \\
    $\vec \nabla \times \vec H = {\vec J}_c + \frac{\partial {\vec D}}{\partial t}$ & 
    $\oint {\vec H} \cdot d {\vec \ell} = \int_S \left( {\vec J}_c + \frac{\partial {\vec D}}{\partial t} \right) \cdot d {\vec S}$ &
    ME1 - Ampere's Law\\
              &               &                \\     
    $\vec \nabla \times \vec E = -\frac{\partial \vec B}{\partial t}$ &
    $\oint \vec E \cdot d \vec \ell = \int_S \left(-\frac{\partial \vec B}{\partial t} \right) \cdot d \vec S$ &
    ME2 - Faraday's Law ($S$ fixed) \\
              &               &                \\     
    $\vec \nabla \cdot \vec D = \rho$ &
    $\oint_S \vec D \cdot d \vec S = \int_V \rho d V$ &
    ME3 - Gauss' Law\\
              &               &                \\     
    $\vec \nabla \cdot \vec B = 0$ &
    $\oint_S \vec B \cdot d \vec S = 0 $ &
    ME4 - nonexistance of monopoles    \\
              &               &                \\
    ${\vec D} = \epsilon {\vec E}$ & & \\
    ${\vec B} = \mu      {\vec H}$ & & \\
    ${\vec F} = q \left( {\vec E} + {\vec v} \times {\vec B} \right)$    & &    \\
    $\epsilon = \epsilon_r \epsilon_0$ & & \\
    $\mu      = \mu_r      \mu_0$      & & \\
    $\vec H$   & magnetic field strength & \\
    $\vec E$   & electric field strength & \\
    $\vec B$   & magnetic flux density   & \\
    $\vec D$   & electric flux density   & \\
    $\vec J$   & current density         & \\
    $\rho  $   & charge density          & \\
    $\vec F$   & force                   & \\
    $q$        & charge                  & \\
    $\vec v$   & velocity                & \\
    $\mu$      & permeability            & \\
    $\epsilon$ & permittivity            & \\
               &                         & \\
    ${\vec E} = \frac{\partial \vec A}{\partial t} - \nabla \phi$ & & \\
    ${\vec B} = \nabla \times {\vec A}$ & & \\
\end{tabular}
\vspace{10mm}
simplifying assumptions:  $EMA1:  \epsilon = const \& \mu = const$

\begin{tabular}{ll}
  \emph{Magnetic Fields}                                                                     &
  \emph{Electric Fields}                                                                     \\
                                                                                             &
                                                                                             \\
  $B_{n1} = B_{n2}$                                                                          & 
  $\left( {\vec D}_1 - {\vec D}_2 \right) \cdot {\vec a}_{n12} = - \rho_s$                   \\
                                                                                             &
                                                                                             \\
  $\left( {\vec H}_1 - {\vec H}_2 \right) \times {\vec a}_{n12}  = {\vec K}$                 &
  $E_{t1} = E_{t2}$                                                                          \\
                                                                                             &
                                                                                             \\
  $\frac{\tan \theta_1}{\tan \theta_2} = \frac{\mu_{r2}}     {\mu_{r1}}$      (current-free) &
  $\frac{\tan \theta_1}{\tan \theta_2} = \frac{\epsilon_{r2}}{\epsilon_{r1}}$ (charge-free)  \\
\end{tabular}

\null

\[
  \frac{d}{d t} {\mathbf S}_\phi = {\mathbf F}_\phi \left( {\mathbf S}_{\phi} ; t \right)
\]
\[
  \frac{d}{d t} {\mathbf S}_\eta = {\mathbf F}_\eta \left( {\mathbf S}_\eta ; t \right)
\]
\[
  \frac{d}{d t} {\mathbf T}_\phi = {\mathbf F}_\phi \left( {\mathbf T}_\phi ; t \right)
\]
\[
  \frac{d}{d t} {\mathbf T}_\eta = {\mathbf F}_\eta \left( {\mathbf T}_\eta ; t \right)
\]

\[
  \theta(x - q) = \left\{ \begin{array}{ll} 0 & x < q \\ 1 & x > q \end{array} \right.
\]
\[
 \int_{a}^{b} dx \, \theta(x-q) = \int_{a}^{q} d x
\]
\begin{eqnarray*}
 \int_{a}^{b} dx \, \theta(q - x)	& = & \int_{a}^{b} dx \left\{1 - \theta(x-q) \right\} \\
								& = & \int_{a}^{b} dx - \int_{a}^{q} dx \\
								& = & \int_{a}^{q} dx + \int_{q}^{b} dx - \int_{a}^{q} dx\\
								& = & \int_{q}^{b} dx
\end{eqnarray*}


\newpage
\null

Define an integrand
\[
  L(a,b)[f] = a f(x)^2 + b f'(x)^2
\]

\[
  I_v(a,b)[f] = \int_0^1 L(a,b)[f] dx = \int_0^1 \left( a f(x)^2 + b f'(x)^2 \right) dx
\]

\begin{tabular}{cc}
  $I_p(1,1)\left[x\right]$   & 1.33333 \\
  $I_p(1,1)\left[x^2\right]$ & 1.53333 \\
  $I_p(1,1)\left[x^3\right]$ & 1.94286
\end{tabular}
subject to the boundary conditions $f(0) = 0$ and $f(1) = 1$.
$I_p$ has as its domain any well-behaved function that is continuous and differentiable (this may be 
too restrictive).  Let the set of such functions be called $\mathcal D$.  Then let $\mathcal D'$ be
the subset of $\mathcal D$ that satisfies the boundary conditions.  Examples of $\mathcal D$ are:
$const, 0, \cos(x), x, x^2, (1-x^3), ln(x), \sin \left( \frac{\pi}{2} x \right), e^x$.  Examples
of $\mathcal D'$ are: $x, x^2, \sin \left( \frac{\pi}{2} \right)$.  


The goal is to find
$q \backepsilon {\mathcal D'}$ extremizes $I_p(a,b)[f]$ by taking the variation $\delta I_p$
and solve for $f$
\[
  \delta I_p = 2 \int_0^1 d\,x \left( a f \delta f + b f' \delta f' \right)
\]
\normalsize
%\begin{figure}[htpb!]
%\centering
%\includegraphics[width=170mm,height=150mm]{var_1.eps}
%\scriptsize\caption{Blah Blag}\label{fig:erptsqfit}
%\end{figure}
\newpage

\[
  \theta(x - q) = \left\{ \begin{array}{ll} 0 & x < q \\ 1 & x > q \end{array} \right.
\]
\[
 \int_{a}^{b} dx \, \theta(x-q) = \int_{a}^{q} d x
\]
\begin{eqnarray*}
 \int_{a}^{b} dx \, \theta(q - x)	& = & \int_{a}^{b} dx \left\{1 - \theta(x-q) \right\} \\
								& = & \int_{a}^{b} dx - \int_{a}^{q} dx \\
								& = & \int_{a}^{q} dx + \int_{q}^{b} dx - \int_{a}^{q} dx\\
								& = & \int_{q}^{b} dx
\end{eqnarray*}

$\bullet$  


\newpage
\null

%%%%%%%%%%%%%%%%%%%%%%%%%%%%%%%%%%%%%%%%%%%%%%%%%%%%%%%%%%%%%%%%%%%%%%%%%%%%%%%
%          Fourier Analysis
%%%%%%%%%%%%%%%%%%%%%%%%%%%%%%%%%%%%%%%%%%%%%%%%%%%%%%%%%%%%%%%%%%%%%%%%%%%%%%%
\textblockcolor{test}
\begin{textblock*}{7.5in}(0mm,0mm)
\begin{tabular*}{7.5in}{c @{\extracolsep{\fill}} c }
       \tiny ~ & ~\\
       \multicolumn{2}{c}{\normalsize \bf Periodic Functions, Fourier Analysis, and Sturm-Liouville Theory} \\
       \tiny~ & ~\\
\end{tabular*}
\end{textblock*}

\begin{textblock*}{90mm}(0mm,15mm)
Consider a periodic function $f(x-2\ell) = f(x) = f(x + 2\ell)$ defined on the interval $[-\ell,\ell]$.  
Because of the underlying periodicity of $f(x)$, both the derivative and the integral also inherit 
periodic properties.

In the case of the derivative, the observation is evident simply from 
the standard definition as follows:
\begin{eqnarray*}
  \frac{d f}{d x}(x) & = & \lim_{dx \rightarrow 0} \frac{ f(x+dx)-f(x)}{dx}\\
                     & = & \lim_{dx \rightarrow 0} \frac{ f(x + 2\ell + dx) - f(x + 2\ell) }{dx} \\
                     & = & \frac{d f}{d x}(x + 2\ell).
\end{eqnarray*}
In the case of the integral, a bit more thinking and manipulation is needed.
To begin, consider the integral of $f(x)$ over an interval $[a,b]$.  The value of this integral 
is invariant if the entire interval is shifted by $2 \ell$.  This is proven by direct substitution
as 
\begin{eqnarray}\label{eq_per_ab_int}
  \int_{a}^{b} f(x) dx & = & \int_{a}      ^{b}       f(x - 2\ell) dx            \\
                       & = & \int_{a+2\ell}^{b+2\ell} f(y)         dy  \nonumber \\
                       & = & \int_{a+2\ell}^{b+2\ell} f(x)         dx. \nonumber
\end{eqnarray}
With this result, it is easy (although somewhat subtle) to show that the integral
of $f(x)$ over the interval $[c-\ell,c+\ell]$ is equal to the integral over 
the interval $[-\ell,\ell]$.  To do so, start by letting $a = c - \ell$ and $b = -\ell$
in (\ref{eq_per_ab_int}), which yields the identity 
\begin{equation}\label{eq_per_int}
  \int_{c-\ell}^{-\ell} f(x) dx  =  \int_{c+\ell}^{\ell} f(x) dx,
\end{equation}
which is then used to manipulate the integral over $[c-\ell,c+\ell]$ as follows,
\begin{eqnarray} 
  \int_{c-\ell}^{c+\ell} f(x) dx  & = &  \int_{c-\ell}^{-\ell} f(x) dx + \int_{-\ell}^{c+\ell} f(x) dx \\
                                  & = &  \int_{c+\ell}^{\ell}  f(x) dx + \int_{-\ell}^{c+\ell} f(x) dx \nonumber \\
                                  & = &  \int_{-\ell}^{c+\ell} f(x) dx + \int_{c+\ell}^{\ell}  f(x) dx \nonumber \\
                                  & = &  \int_{-\ell}^{\ell}   f(x) dx                                 \nonumber
\end{eqnarray}
where (\ref{eq_per_int}) was used in going from the first to the second line.

\[
  -\frac{d}{d x}\left( p(x) \frac{d y}{d x} \right) + q(x) y = \lambda w(x) y
\]
\end{textblock*}

\begin{textblock*}{100mm}(89.85mm,15mm)
\[
  \int_{c}^{c+2L} \cos \left( \frac{m \pi x}{L} \right) dx = 0
 \]
 \[
  \int_{c}^{c+2L} \sin \left( \frac{m \pi x}{L} \right) dx = 0
 \]
 
\[
  \int_{c}^{c+2L} dx \, \cos \left( \frac{m \pi x}{L} \right) \cos \left( \frac{n \pi x}{L} \right) = 
      \left\{ \begin{array}{cc} 2 L \, \delta_{mn} & m = 0 \\ L \, \delta_{mn} & m \neq 0 \end{array} \right.
\]
\[
  \int_{c}^{c+2L} dx \, \sin \left( \frac{m \pi x}{L} \right) \sin \left( \frac{n \pi x}{L} \right) = 
      \left\{ \begin{array}{cc} 0 \, \delta_{mn} & m = 0 \\ L \, \delta_{mn} & m \neq 0 \end{array} \right.
\]
\[
 {\tilde f}(x) = \frac{a_0}{2} + 
    \sum_{n=1}^{\infty} \left( a_n \cos \left( \frac{n \pi x}{L} \right) + b_n \sin \left( \frac{n \pi x}{L} \right) \right)
\]
\[
  a_n = \frac{1}{L} \int_{c}^{c+2L} f(x) \cos \left( \frac{n \pi x}{L} \right) dx
\]
\[
  b_n = \frac{1}{L} \int_{c}^{c+2L} f(x) \sin \left( \frac{n \pi x}{L} \right) dx
\]

\[
  {\tilde f}(x) = \sum_{n = - \infty}^{\infty} c_n e^{i n \pi x/L}
\]
\[
  c_n = \frac{1}{2 L} \int_{c}^{c+2L} f(x) e^{-i n \pi x/L} dx
\]
\end{textblock*}
\newpage
\null
%%%%%%%%%%%%%%%%%%%%%%%%%%%%%%%%%%%%%%%%%%%%%%%%%%%%%%%%%%%%%%%
%          Harmonic Oscillators
%%%%%%%%%%%%%%%%%%%%%%%%%%%%%%%%%%%%%%%%%%%%%%%%%%%%%%%%%%%%%%%
\textblockcolor{test}
\begin{textblock*}{7.5in}(0mm,0mm)
\begin{tabular*}{7.5in}{c @{\extracolsep{\fill}} c }
       \tiny ~ & ~\\
       \multicolumn{2}{c}{\normalsize \bf Harmonic Oscillators} \\
       \tiny~ & ~\\
\end{tabular*}
\end{textblock*}

\begin{textblock*}{2.5in}(0mm,30mm)
The equation for a damped simple harmonic oscillator is
\[
  m {\ddot x} + \Gamma {\dot x} + k x = 0 .
\]
Assume a solution of the form $x = A e^{\imath \omega t}$ and then substitute into the equation of motion.
Factoring out $A e^{i \omega t}$, yields the characteristic equation
\[
  -m \omega^2 + i \Gamma \omega + k = 0 .
\]
Before solving the charateristic equation, divide by $m$ and then define the parameters ${\omega_0}^2 = k/m$ and 
$\gamma = \Gamma/2$.  The equation then becomes
\[
  \omega^2 - i \gamma \omega + {\omega_0}^2 = 0
\]
with corresponding roots
\[
  \omega = \frac{i \gamma}{2} \pm \sqrt{ {\omega_0}^2 - \frac{\gamma^2}{4} } 
         = \frac{i \gamma}{2} \pm \omega_D
\]
Thus the final solution looks like
\[
  x = e^{-i \gamma/2} \left[ x_0 \cos(\omega_D t) + \frac{v_0}{\omega_D} \sin(\omega_D t) \right]
\]
\end{textblock*}
\newpage
\null

\end{document} 