%vec
\documentclass[10pt]{article}
\usepackage{pstricks}
\usepackage{pst-plot}
\usepackage{pst-node}
\usepackage{pst-tree}
\usepackage{pst-coil}

\usepackage[absolute]{textpos}
\setlength{\parindent}{0pt}
\usepackage{color}
\usepackage{amssymb}
\usepackage{graphicx}
\textblockorigin{0.5in}{0.5in}
\definecolor{MintGreen}      {rgb}{0.2,0.85,0.5}
\definecolor{LightGreen}     {rgb}{0.9,1,0.9}
\definecolor{LightYellow}    {rgb}{1,1,0.6}
\definecolor{test}           {rgb}{0.8,0.95,0.95}


\TPshowboxestrue
\TPMargin{2mm}
\begin{document}


%Euclidean Vector Analysis - filename eucl_geo_anal.tex

\null
%%%%%%%%%%%%%%%%%%%%%%%%%%%%%%%%%%%%%%%%%%%%%%%%%%%%%%%%%%%%%%%%%%%%%%%%%%%%%%%
%          Euclidean Geometric Analysis
%%%%%%%%%%%%%%%%%%%%%%%%%%%%%%%%%%%%%%%%%%%%%%%%%%%%%%%%%%%%%%%%%%%%%%%%%%%%%%%
\textblockcolor{test}
\begin{textblock*}{7.5in}(0mm,0mm)
\begin{tabular*}{7.5in}{c @{\extracolsep{\fill}} c }
       \tiny ~ & ~\\
       \multicolumn{2}{c}{\normalsize \bf Euclidean Geometric
                                          Analysis} \\
       \tiny~ & ~\\
\end{tabular*}
\end{textblock*}

%%%%%%%%%%%%%%%%%%%%%%%%%%%%%%%%%%%%%%%%%%%%%%%%%%%%%%%%%%%%%%%
%          Basic Geometric Objects
%%%%%%%%%%%%%%%%%%%%%%%%%%%%%%%%%%%%%%%%%%%%%%%%%%%%%%%%%%%%%%%
\scriptsize
\textblockcolor{LightYellow}
\begin{textblock*}{60mm}(0mm,12.54mm)
\begin{tabular*}{58mm}{l @{\extracolsep{\fill}} l}
   & ~\\
\multicolumn{2}{c}{\bf Basic Geometric Objects} \\
   & ~\\
point:              & ${\mathcal P},{\mathcal Q},\dots$\\
bounding points:    & $\partial \, {\mathcal C}$\\
curve:              & ${\mathcal C}$\\
bounding curve:     & $\partial \, {\mathcal S}$\\
surface:            & ${\mathcal S}$\\
bounding surface:   & $\partial \, {\mathcal V}$\\
                    & ~\\
volume:             & ${\mathcal V}$\\
scalar:             & $m$, $n$, etc.\\
indices:            & $i, j = 1, 2, \dots, N$\\
coordinates:        & $x^i$, $q^{i'}$ \, $i$, $j'$ = 1,2,3\\
transformations:    & $x^i = x^i \left( q^{j'} \right)$\\
field:              & $\phi = \phi \left( q^i \right)$\\
parametric curve:   & $q^i \left( s \right)$\\
implicit surface:   & $\phi_{\mathcal S}\left(q^i \right) 
                      - c = 0$\\
parametric surface: & $q^i\left(u,v\right)$\\
                    & ~\\
Kronecker delta:    & $\delta_{ij}$\\
permutation symb:   & $[i,j,k]$\\
column array:       & $| {\mathbf q} \rangle$\\
row array:          & $\langle {\mathbf q} |$\\
\end{tabular*}
\vspace{11.56mm}
\end{textblock*}

%%%%%%%%%%%%%%%%%%%%%%%%%%%%%%%%%%%%%%%%%%%%%%%%%%%%%%%%%%%%%%%
%          Defining Vectors
%%%%%%%%%%%%%%%%%%%%%%%%%%%%%%%%%%%%%%%%%%%%%%%%%%%%%%%%%%%%%%%
\scriptsize
\textblockcolor{LightYellow}
\begin{textblock*}{60mm}(59.85mm,12.54mm)
\begin{tabular*}{58mm}{l @{\extracolsep{\fill}} l}
   & ~\\
\multicolumn{2}{c}{\bf Defining Vectors} \\
   & ~\\
A01:                  & ${\mathbf A} + {\mathbf B} = 
                         {\mathbf B} + {\mathbf A}$\\
A02:                  & ${\mathbf A} + \left( {\mathbf B} 
                        + {\mathbf C} \right) 
                      = \left( {\mathbf A} + {\mathbf B} \right) 
                        + {\mathbf C}$\\
A03:                  & $ m {\mathbf A} = {\mathbf A} m$\\
A04:                  & $ m \left( n {\mathbf A} \right) 
                        = \left( m n \right) {\mathbf A}$\\
A05:                  & $\left( m + n \right) {\mathbf A} 
                        = m {\mathbf A} + m {\mathbf A}$\\
A06:                  & $m \left( {\mathbf A} + {\mathbf B} \right) 
                        = m {\mathbf A} + m {\mathbf B}$\\
A07:${}^\dagger$      & ${\mathbf A} \cdot {\mathbf B} 
                        = {\mathbf B} \cdot {\mathbf A}$\\
A08:${}^\dagger$      & ${\mathbf A} \cdot \left( {\mathbf B} 
                        + {\mathbf C} \right) 
                        = {\mathbf A} \cdot {\mathbf B} 
                        + {\mathbf A} \cdot {\mathbf C}$\\
A09:${}^\dagger$      & $m \left( {\mathbf A} \cdot 
                        {\mathbf B} \right) 
                        = \left( m {\mathbf A} \right) 
                        \cdot {\mathbf B}$\\
					  & \quad $ = {\mathbf A} \cdot 
                        \left( m {\mathbf B} \right)
                        = \left( {\mathbf A} \cdot 
                        {\mathbf B} \right) m$\\
A10:${}^\dagger$      & ${\mathbf e}_i \cdot {\mathbf e}_j 
                        = g_{ij}$\\
A11:${}^\dagger$      & ${\mathbf A} = A^i {\mathbf e}_i$ 
                        with $A^i = {\mathbf A} 
                        \cdot {\mathbf e}_i$\\
A12:${}^\dagger$      & ${\mathbf A} \cdot {\mathbf A} 
                        = |{\mathbf A}|^2$\\
A13:${}^\dagger$      & ${\mathbf A} \cdot {\mathbf B} 
                        = | {\mathbf A} | | {\mathbf B} | 
                        \cos \left( \theta \right)$\\
A14:${}^\ddag$        & ${\mathbf A} \times {\mathbf B} 
                        = - {\mathbf B} \times {\mathbf A}$\\
A15:${}^\ddag$        & ${\mathbf A} \times \left( {\mathbf B}
                        + {\mathbf C} \right) 
                        = {\mathbf A} \times {\mathbf B} 
                        + {\mathbf A} \times {\mathbf C}$\\
A16:${}^\ddag$        & $m \left( {\mathbf A} \times 
                        {\mathbf B} \right) 
                         = \left( m {\mathbf A} \right) 
                        \times {\mathbf B}$\\
                      & $\quad = {\mathbf A} \times 
                        \left( m {\mathbf B} \right) 
				  	    = \left( {\mathbf A} \times 
                        {\mathbf B} \right) m$\\
A17:${}^\ddag$        & $ {\mathbf e}_i \times 
                       {\mathbf e}_j = [ijk] {\mathbf e}_k$\\
A18:${}^\ddag$        & ${\mathbf A} \times 
                        \left( {\mathbf B} \times 
                        {\mathbf C} \right) 
                        \neq \left( {\mathbf A} \times 
                        {\mathbf B} \right) \times 
                        {\mathbf C}$\\
A19:${}^\dag{}^\ddag$ & $| {\mathbf A} \times {\mathbf B} | 
                        = |{\mathbf A}| |{\mathbf B}| 
                        \sin \left( \theta \right)$\\
A20:${}^\ddag$       & ${\mathbf A} \cdot( {\mathbf B} \times {\mathbf C} )
                       ={\mathbf B} \cdot( {\mathbf C} \times {\mathbf A} )$\\
\end{tabular*}
\vspace{7.85mm}
\end{textblock*}

%%%%%%%%%%%%%%%%%%%%%%%%%%%%%%%%%%%%%%%%%%%%%%%%%%%%%%%%%%%%%%%
%          Defining Vector Fields
%%%%%%%%%%%%%%%%%%%%%%%%%%%%%%%%%%%%%%%%%%%%%%%%%%%%%%%%%%%%%%%
\scriptsize
\textblockcolor{LightYellow}
\begin{textblock*}{70.8mm}(119.7mm,12.54mm)
\begin{tabular*}{70mm}{l @{\extracolsep{\fill}} l}
   & ~\\
\multicolumn{2}{c}{\bf Defining Vector Fields} \\
   & ~\\
F01:                  & vector field ${\mathbf F} = {\mathbf F} \left(q^i \right)$\\
F02:                  & fundamental field ${\mathbf r} = x {\mathbf e}_x + y {\mathbf e}_y + z {\mathbf e}_z$\\
                      & \\
F03:${}^\dag$         & covariant basis ${\mathbf e}_i = \frac{ \partial_{q^i} {\mathbf r}}{|\partial_{q^i} {\mathbf r}|}$\\
                      & \\
F04:                  & curve tangent ${\mathbf e}_{\mathcal C} = \frac{ d {\mathbf r}_{\mathcal C}}{ d s} \equiv {\hat t}$\\
                      & \\
F05:                  & surface tangents ${\mathbf e}_{u} = \frac{ \partial {\mathbf r}_{\mathcal S}}{\partial u}$,
                                         ${\mathbf e}_{v} = \frac{ \partial {\mathbf r}_{\mathcal S}}{\partial v}$\\
                      & \\
F06:${}^\dag{}^\ddag$ & surface normal (parametric): ${\mathbf n}_{\mathcal S} = \frac{ {\mathbf e}_u \times {\mathbf e}_v }
                                                                                     {|{\mathbf e}_u \times {\mathbf e}_v|}$\\
                      & \\
F07:${}^\dag$         & surface normal (implicit): ${\mathbf n}_{\mathcal S} = \frac{\nabla \phi_{\mathcal S}}{|\nabla \phi_{\mathcal S}|}$\\
                      & \\
F08:${}^\dag$         & $div({\mathbf F}) = \lim_{{\mathcal V} -> 0} 
                            \frac{ \int_{\partial {\mathcal V}} {\mathbf F} \cdot {\mathbf n} d {\mathcal S}}
							{\mathcal V}$\\
                      & \\
F09:${}^\dag{}^\ddag$ & $curl({\mathbf F}) \cdot {\mathbf n} = \lim_{{\mathcal S} -> 0}
                              \frac{ \int_{\partial {\mathcal S}} {\mathbf F} \cdot {\hat t} dl}{\mathcal S}$\\
                      & \\
F10:${}^\dag$         & $grad(\phi) = \lim_{{\mathcal C}->0}
                            \frac{ \int_{\partial {\mathcal C}} \phi {\hat t} dl}{\mathcal C}$\\
                      & \\
F11:${}^\dag$         & $\int_{{\mathcal V}} div({\mathbf F}) d {\mathcal V} 
                         = \int_{\partial {\mathcal V}} {\mathbf F} \cdot {\mathbf n} d {\mathcal S}$\\
                      & \\
F12:${}\dag$          & $\int_{\mathcal S} curl({\mathbf F}) \cdot {\mathbf n} d {\mathcal S} 
                           = \int_{\partial {\mathcal S}} {\mathbf F} \cdot d {\mathbf r}$\\
\end{tabular*}
\vspace{2mm}
\end{textblock*}

%%%%%%%%%%%%%%%%%%%%%%%%%%%%%%%%%%%%%%%%%%%%%%%%%%%%%%%%%%%%%%%
%          Orthogonal Coordinates
%%%%%%%%%%%%%%%%%%%%%%%%%%%%%%%%%%%%%%%%%%%%%%%%%%%%%%%%%%%%%%%
\scriptsize
\textblockcolor{LightYellow}
\begin{textblock*}{65mm}(0mm,95.3mm)
\begin{tabular*}{64mm}{l @{\extracolsep{\fill}} l}
   & ~\\
\multicolumn{2}{c}{\bf Orthogonal Coordinates$
                  {}^\dag{}^\ddag$} \\
   & \\
N01:  & $h_i = |\partial {\mathbf r}/\partial q^i|$\\
N02:  & ${\mathbf e}_i = \frac{1}{h_i} \partial 
        {\mathbf r}/\partial q^i$ (no sum)\\
N03:  & $\Omega = h_1 h_2 h_3$\\
      & \\
N04:  & $grad(\phi) = \sum_i {\mathbf e}_i \frac{1}{h_i} \
        \partial_{q^i} \phi$\\
N05:  & $curl({\mathbf F}) = \frac{1}{\Omega} \sum_{ijk} 
        {\mathbf e}_i [ijk] h_i \partial_{q^j} 
        \left( h_k F_k \right)$\\
N06:  & $div({\mathbf F}) = \sum_i \frac{1}{\Omega} 
        \partial_{q^i} \left( \frac{ \Omega F_i }{h_i} 
        \right)$\\
      & \\
N07:  & $laplacian(\phi) = \frac{1}{\Omega} \sum_i 
        \partial_{q^i} \left( \frac{\Omega}{ {h_i}^2 } 
        \partial_{q^i} \phi \right)$\\
      & \\
\multicolumn{2}{l}{$grad$, $div$, and $curl$ can all 
                    be represented}\\
\multicolumn{2}{l}{by a single operator $\nabla$, 
                    which gives}\\      
\multicolumn{2}{l}{functionally the relations 
                    $grad(\phi) = \nabla \phi$,}\\
\multicolumn{2}{l}{$div({\mathbf F}) = \nabla \cdot 
                  {\mathbf F}$, $curl({\mathbf F})= 
                  \nabla \times {\mathbf F}$, and}\\
\multicolumn{2}{l}{$laplacian(\phi) =  \nabla \cdot \nabla 
                    \phi = \nabla^2 \phi$.}\\
\multicolumn{2}{l}{In Cartesian components, $\nabla$ 
                   takes on}\\
\multicolumn{2}{l}{the form:}\\
	  & \\
N08:  & $\nabla = {\mathbf e}_x \partial_x + 
                  {\mathbf e}_y \partial_y + 
                  {\mathbf e}_z \partial_z$\\
      & \\
\end{tabular*}
\end{textblock*}

%%%%%%%%%%%%%%%%%%%%%%%%%%%%%%%%%%%%%%%%%%%%%%%%%%%%%%%%%%%%%%%
%          Derivative Theorems
%%%%%%%%%%%%%%%%%%%%%%%%%%%%%%%%%%%%%%%%%%%%%%%%%%%%%%%%%%%%%%%
\scriptsize
\textblockcolor{LightYellow}
\begin{textblock*}{64mm}(64.85mm,95.3mm)
\begin{tabular*}{63mm}{l @{\extracolsep{\fill}} l}
   & ~\\
\multicolumn{2}{c}{\bf Derivative Theorems${}^\dag{}^\ddag$} \\
   & \\
D01: & $\nabla \cdot \nabla \phi = \nabla^2 \phi$\\
D02: & $\nabla( \phi \psi ) = (\nabla \phi) \psi 
        + \phi (\nabla \psi)$\\
D03: & $\nabla \times \nabla \phi = 0$\\
D04: & $ \nabla( {\mathbf F} \cdot {\mathbf G} ) = 
        ({\mathbf F} \cdot \nabla) {\mathbf G} 
        + {\mathbf F} \times (\nabla \times {\mathbf G})$\\
     & $+ ( {\mathbf G} \cdot \nabla ) {\mathbf F} + {\mathbf G} 
        \times ( \nabla \times {\mathbf F} )$\\
D05: & $\frac{1}{2} \nabla^2 {\mathbf F} = {\mathbf F} 
        \times (\nabla \times {\mathbf F}) 
        + ({\mathbf F} \cdot \nabla) {\mathbf F}$\\
D06: & $\nabla \cdot ( \nabla \times {\mathbf F} ) = 0$\\
D07: & $\nabla \cdot ( \phi {\mathbf F}) 
        = {\mathbf F} \cdot \nabla \phi 
        + \phi \nabla \cdot {\mathbf F}$\\
D08: & $\nabla \cdot ( {\mathbf F} \times {\mathbf G} ) 
        = (\nabla \times {\mathbf F})\cdot {\mathbf G} 
        - (\nabla \times {\mathbf G})\cdot {\mathbf F}$\\
D09: & $\nabla \times (\phi {\mathbf F}) 
        = (\nabla \phi) \times {\mathbf F} 
        + \phi \nabla \times {\mathbf F}$\\
D10: & $ \nabla \times (\nabla \times {\mathbf F}) 
        = \nabla( \nabla \cdot {\mathbf F} ) 
        - \nabla ^2 {\mathbf F}$\\
D11: & $\nabla \times ({\mathbf F} \times {\mathbf G}) 
        = (\nabla \cdot {\mathbf G}) {\mathbf F} 
        - (\nabla \cdot {\mathbf F}) {\mathbf G}$\\
     & $\quad + ({\mathbf G} \cdot \nabla ){\mathbf F}  
     - ({\mathbf F} \cdot \nabla ) {\mathbf G}$\\
	 & \\
D12: & $\nabla \cdot {\mathbf r} = 3$\\
D13: & $\nabla \times {\mathbf r} = 0$\\
D14: & $({\mathbf G} \cdot \nabla) {\mathbf r} = {\mathbf G}$\\
D15: & $\nabla^2 {\mathbf r} = 0 $\\
D16: & $\phi(r)$ or ${\mathbf F}(r)$: $\nabla 
        = \frac{{\mathbf r}}{r} \frac{d}{d r}$\\
D17: & $\nabla( {\mathbf A} \cdot {\mathbf r} ) = {\mathbf A}$\\
D18: & $\phi({\mathbf A}\cdot{\mathbf r})$ 
        or ${\mathbf F}({\mathbf A}\cdot {\mathbf r})$: 
        $\nabla = {\mathbf A} \frac{d}{d ({\mathbf A}
        \cdot {\mathbf r})}$\\
\end{tabular*}
\vspace{0.96mm}
\end{textblock*}

%%%%%%%%%%%%%%%%%%%%%%%%%%%%%%%%%%%%%%%%%%%%%%%%%%%%%%%%%%%%%%%
%          Integral Theorems
%%%%%%%%%%%%%%%%%%%%%%%%%%%%%%%%%%%%%%%%%%%%%%%%%%%%%%%%%%%%%%%
\scriptsize
\textblockcolor{LightYellow}
\begin{textblock*}{61.8mm}(128.7mm,95.3mm)
\begin{tabular*}{60.8mm}{l @{\extracolsep{\fill}} l}
   & ~\\
\multicolumn{2}{c}{\bf Integral Theorems${}^\dag{}^\ddag$} \\
   & \\
I01: & $\int_{\mathcal V} \nabla \times {\mathbf F} \, 
        d {\mathcal V} = \int_{\partial {\mathcal V}} 
        {\hat n} \times {\mathbf F} \, d {\mathcal S}$ \vspace{1.5mm}\\
I02: & $ \int_{\mathcal V} \nabla \phi \, d {\mathcal V} 
    = \int_{\partial {\mathcal V}} \phi \, {\hat n} 
      \, d {\mathcal S}$ \vspace{1.5mm}\\
I03: & $\int_{\mathcal V} ( \phi \nabla^2 \psi 
        + \nabla \phi \cdot \nabla \psi) d {\mathcal V}$ \\
     & $ = \int_{\partial {\mathcal V}} \phi\nabla\psi 
        \cdot {\hat n} d{\mathcal S}$\vspace{1.5mm}\\
I04: & $\int_{\mathcal V} ( \phi \nabla^2 \psi 
       - \psi \nabla^2 \phi) d {\mathcal V}$\\ 
     & $\quad  = \int_{\partial {\mathcal V}}
       ( \phi\nabla\psi -\psi\nabla\phi) \cdot 
       {\hat n} d{\mathcal S}$\vspace{1.5mm}\\
I05: & $\int_{\mathcal V} d {\mathcal V} \nabla \phi \cdot {\mathbf F} 
       = \int_{\partial {\mathcal V}} \phi {\mathbf F} 
       \cdot {\hat n} \, d {\mathcal S} -$ \\
     & $\int_{\partial {\mathcal V}} \phi \nabla 
       \cdot {\mathbf F} d {\mathcal S}$\vspace{1.5mm}\\
I06: & $\int_{\partial {\mathcal S}} \phi \,{\hat t} 
     d \ell = \int_{\mathcal S} ({\hat n} 
     \times \nabla \phi )\, d{\mathcal S}$\vspace{1.5mm}\\
I07: & $\int_{\partial {\mathcal S}} d \ell \, 
       {\hat t}\times {\mathbf F} = \int_{\mathcal S} 
       ({\hat n} \times \nabla )\times {\mathbf F} 
       d {\mathcal S}$\vspace{1.5mm}\\
I08: & $\int_{\partial {\mathcal V}} d{\mathcal S} 
       \, {\hat n} \circ  = \int_{\mathcal V} 
       \, d{\mathcal V} \nabla \circ$\vspace{1.5mm}\\
I09: & $ \int_{\partial {\mathcal S}} d \ell \, 
      {\hat t} \circ = \int_{\mathcal S} d \, 
      {\mathcal S} ( {\hat n} \times \nabla )\circ$\vspace{1.5mm}\\
I10: & $\frac{d \Phi}{dt} = \int_{{\mathcal S}} 
                           \left[  \frac{\partial {\mathbf F}}{\partial t} 
                                 + ( \nabla \cdot {\mathbf F} ) {\mathbf V}
                           \right] \cdot d {\mathcal S}$ \\
     & $ + \quad \int_{\partial {\mathcal S}} 
           {\mathbf F} \times {\mathbf V} \cdot {\hat t} d \ell$\vspace{1.5mm}\\
I11: & $\frac{d }{dt} \int_{\mathcal V} \rho d {\mathbf V} = 
        \int_{\mathcal V} \frac{\partial \rho}{\partial t} d {\mathcal V} 
        + \int_{\partial {\mathcal V}} \rho {\mathbf v} \cdot d {\mathcal S}$\\ 
\end{tabular*}
\vspace{1.5mm}
\end{textblock*}

\TPshowboxesfalse
\begin{textblock*}{70mm}(0.2mm,170mm)
{\centering \bf Jacobian}\\
  Let a $n$-dimensional space be spanned by a
  set of vectors $\{ | {\mathbf e}_i \rangle\}$.
  Then the column-wise collection
    \[
      {\mathbf J} = \left[\, |{\mathbf e}_1 \rangle, \, | {\mathbf e}_2 \rangle, \,  \dots , \, | {\mathbf e}_n \rangle \, \vspace{10mm} \right]
    \]
  is know as the \emph{Jacobian matrix}.  Its inverse
    \[
      {\mathbf J}^{-1} = \left[ \begin{array}{c} \langle {\mathbf e}^1 | \vspace{1mm}\\ \langle {\mathbf e}^2 | \\ \vdots \\ \langle {\mathbf e}^n | \end{array} \right]
    \]
    gives a vector space spanned by the \emph{dual vectors} that obey the contraction rule
\end{textblock*}
\begin{textblock*}{70mm}(70mm,170mm)
\begin{eqnarray*}
  {\mathbf J}^{-1} {\mathbf J} & = &  \left[ \begin{array}{cccc}
                 \langle {\mathbf e}^1 | {\mathbf e}_1 \rangle & \langle {\mathbf e}^1 | {\mathbf e}_2 \rangle & \dots & \langle {\mathbf e}^1 | {\mathbf e}_n \rangle \vspace{1mm}\\
                 \langle {\mathbf e}^2 | {\mathbf e}_1 \rangle & \langle {\mathbf e}^2 | {\mathbf e}_2 \rangle & \dots & \langle {\mathbf e}^2 | {\mathbf e}_n \rangle \\
	    	     \vdots                   &     \vdots                 & \ddots &   \vdots \\
                 \langle {\mathbf e}^n | {\mathbf e}_1 \rangle & \langle {\mathbf e}^n | {\mathbf e}_2 \rangle & \dots & \langle {\mathbf e}^n | {\mathbf e}_n \rangle \\
			   \end{array} \right] \\
		   & = & \langle {\mathbf e}^i | {\mathbf e}_j \rangle = {\delta^i}_j
\end{eqnarray*}
and the completeness relation
\[
  {\mathbf J} {\mathbf J}^{-1} = \sum_i | {\mathbf e}_i \rangle \langle {\mathbf e}^i | = \mbox{Id}_{n \times n} .
\]
\end{textblock*}

\TPshowboxestrue
\begin{textblock*}{50.5mm}(140mm,170mm)
  \begin{tabular*}{50mm}{l @{\extracolsep{\fill}} l}
    & \\
  \multicolumn{2}{c}{\bf Permutation Tensor}\\
  \end{tabular*}
  \[
    \epsilon_{ijk} = \det({\mathbf J}) [ijk] = J [ijk]
  \]
  \[
    \epsilon^{ijk} = \det \left( {\mathbf J}^{-1} \right) [ijk] = J^{-1} [ijk]
  \]
  \[
    \epsilon^{ijk} \epsilon_{pqr} = \left| 
	   \begin{array}{ccc}
	   {\delta^i}_p & {\delta^i}_q & {\delta^i}_p\\
	   {\delta^j}_p & {\delta^j}_q & {\delta^j}_r\\
	   {\delta^k}_p & {\delta^k}_q & {\delta^k}_r\\
	   \end{array}
	  \right|
  \]
  \[
    \epsilon^{ijk} \epsilon_{pqk} = {\delta^i}_p {\delta^j}_q - {\delta^i}_q {\delta^j}_p
  \]
  \[
    \epsilon^{ijk} \epsilon_{pjk} = 2 {\delta^i}_p
  \]
  \[
      \epsilon^{ijk} \epsilon_{ijk} = 6
  \]
\end{textblock*}

\begin{textblock*}{140.15mm}(0mm,170mm)
 \vspace{51.05mm}
\end{textblock*}

%\begin{textblock*}
%A vector is uniquely specified by givings its divergence and curl within a reqion and its 
%normal component over the boundary.

%\emph{Helmholtz's Theorem:} A vector with both source and circulation densities vanishing at inifinity
%may be written as the sum of two parts, one of which is irrotational, the other solenoidal --- 
%$\mathbf{V} = - {\mathbf \nabla} \phi + \nabla \times {\mathbf A}$.
%\end{textblock*}

\begin{textblock*}{7.5in}(0mm,224.90mm)
  $\dag$ = inner product required \quad \quad $\ddag$ = valid only in 3 dimensions 
\end{textblock*}
\newpage
%Differential Forms Version of Classical Vector Analysis - filename diff_forms_classical_vec.tex

\null
%%%%%%%%%%%%%%%%%%%%%%%%%%%%%%%%%%%%%%%%%%%%%%%%%%%%%%%%%%%%%%%%%%%%%%%%%%%%%%%
%          Differential Forms Version of Classical Vector Analysis
%%%%%%%%%%%%%%%%%%%%%%%%%%%%%%%%%%%%%%%%%%%%%%%%%%%%%%%%%%%%%%%%%%%%%%%%%%%%%%%
\textblockcolor{test}
\begin{textblock*}{7.5in}(0mm,0mm)
\begin{tabular*}{7.5in}{c @{\extracolsep{\fill}} c }
       \tiny ~ & ~\\
       \multicolumn{2}{c}{\normalsize \bf Differential Forms Version of
                                          Classical Vector Analysis} \\
       \tiny~ & ~\\
\end{tabular*}
\end{textblock*}

%%%%%%%%%%%%%%%%%%%%%%%%%%%%%%%%%%%%%%%%%%%%%%%%%%%%%%%%%%%%%%%
%          Conversions Between
%%%%%%%%%%%%%%%%%%%%%%%%%%%%%%%%%%%%%%%%%%%%%%%%%%%%%%%%%%%%%%%
\scriptsize
\textblockcolor{LightYellow}
\begin{textblock*}{60mm}(0mm,12.54mm)
\begin{tabular*}{58mm}{l @{\extracolsep{\fill}} l}
   & ~\\
\multicolumn{2}{c}{\bf Basic Geometric Objects1} \\
   & ~\\
\end{tabular*}
\vspace{11.56mm}
\end{textblock*}

\[
  {\vec A} <=> \phi_{A}
\]
\[
  \phi_{A} = A^x dx + A^y dy + A^z dz
\]
\[
  {\vec \nabla} \times {\vec A} <=> *d \phi_{A}
\]
\[
 div({\vec A}) <=> *d* \phi_{A}
\]
\[
  {\vec \nabla} \cdot {\vec \nabla} \times {\vec A} <=> (*d*)(*d \phi_{A})
\]
\[
  {\vec \nabla} \times {\vec \nabla} f = *d(df) = *d^2 f = 0
\]
\[
  {\vec \nabla} \times ( {\vec \nabla} {\vec A} ) <=> (*d)(*d \phi_{A} ) = *d*d \phi_{A}
\]
This sucks!

\newpage
%Classical Mechanics - filename class_mech.tex

\null
%%%%%%%%%%%%%%%%%%%%%%%%%%%%%%%%%%%%%%%%%%%%%%%%%%%%%%%%%%%%%%%
%          Classical Mechanics
%%%%%%%%%%%%%%%%%%%%%%%%%%%%%%%%%%%%%%%%%%%%%%%%%%%%%%%%%%%%%%%
\textblockcolor{test}
\begin{textblock*}{7.5in}(0mm,0mm)
\begin{tabular*}{7.5in}{c @{\extracolsep{\fill}} c }
       \tiny ~ & ~\\
       \multicolumn{2}{c}{\normalsize \bf Classical Mechanics} \\
       \tiny~ & ~\\
\end{tabular*}
\end{textblock*}

%%%%%%%%%%%%%%%%%%%%%%%%%%%%%%%%%%%%%%%%%%%%%%%%%%%%%%%%%%%%%%%
%          Invariance of the EL Equations
%%%%%%%%%%%%%%%%%%%%%%%%%%%%%%%%%%%%%%%%%%%%%%%%%%%%%%%%%%%%%%%
\scriptsize
\textblockcolor{LightYellow}
\begin{textblock*}{75mm}(0mm,12.77mm)
The Euler Lagrange equations are invariant to a basic change in coordinates from $q^i$ to
$y^j$ of the form 
\begin{equation}\label{ELinv_b}
  q^i = q^i(y^j) .
\end{equation}
To see this first note that the from the 
form of the transformation equation (\ref{ELinv_b}) we get
\begin{equation}\label{ELinv_a}
  {\dot q}^i = \frac{\partial q^i}{\partial y^j} {\dot y}^j \,
\end{equation}
where $\dot f = \frac{d}{dt} f$. Next note that the Lagrangian 
$\tilde L (y^j,{\dot y}^j;t )$ in the $y^j$ coordinates 
is related to the Lagrangian $L (q^i,{\dot q}^i;t )$ 
in $q^i$ coordinates by virtue of a substitution of (\ref{ELinv_b}) 
and (\ref{ELinv_a}) yielding
\begin{equation}\label{ELinv_c}
  \tilde L (y^j,{\dot y}^j;t ) = L (q^i(y^j),{\dot q}^i(y^j,{\dot y}^j);t ) .
\end{equation}
The parts of the Euler-Lagrange equation in terms
of the $y^j$ coordinates in relation to the $q^i$ coordinates are
\begin{equation}\label{ELinv_d}
  \frac{\partial \tilde L}{\partial y^j} =    \frac{\partial L}{\partial q^i}        \frac{\partial q^i}{\partial y^j} 
                                        +  \frac{\partial L}{\partial {\dot q}^i} \frac{\partial {\dot q}^i}{\partial y^j}
\end{equation}
and
\begin{equation}\label{ELinv_e}
  \frac{\partial \tilde L}{\partial {\dot y}^j} = \frac{\partial L}{\partial {\dot q}^i} \frac{\partial {\dot q}^i}{\partial {\dot y}^j} .
\end{equation}
Now substituting (\ref{ELinv_d}) and (\ref{ELinv_e}) into the Euler-\\Lagrange equations
yields
\begin{eqnarray}\label{ELinv_f}
  \frac{d}{dt} \left( \frac{\partial \tilde L}{\partial {\dot y}^j } \right) - \frac{\partial \tilde L}{\partial y^j} 
    & = &   \frac{d}{d t} \left( \frac{\partial L}{\partial {\dot q}^i} \right) \frac{\partial {\dot q}^i}{\partial {\dot y}^j} \\ \nonumber
	&   & + \frac{\partial L}{\partial {\dot q}^i} \frac{d}{dt} \left( \frac{\partial {\dot q}^i}{\partial {\dot y}^j} \right) \\ \nonumber
    & = & - \frac{\partial L}{\partial q^i}        \frac{\partial q^i}{\partial y^j} 
	      - \frac{\partial L}{\partial {\dot q}^i} \frac{\partial {\dot q}^i}{\partial y^j} .\nonumber
\end{eqnarray}
But from (\ref{ELinv_a}) $ {\partial {\dot q}^i}/{\partial {\dot y}^j} = {\partial q^i}/{\partial y^j}$ and thus the second and
fourth terms in (\ref{ELinv_f}) cancel, leaving
\begin{equation}\label{ELinv_g}
  \frac{d}{dt} \left( \frac{\partial \tilde L}{\partial {\dot y}^j } \right) - \frac{\partial \tilde L}{\partial y^j} = 
  \left[ \frac{d}{d t} \left( \frac{\partial L}{\partial {\dot q}^i} \right) - \frac{\partial L}{\partial q^i} \right] 
  \frac{\partial q^i}{\partial y^j} .
\end{equation}
Using the definition $\Xi_i = \frac{d}{dt}\left(\frac{\partial L}{\partial \dot q^i} \right) - \frac{\partial L}{\partial q^i}$,
Eq. (\ref{ELinv_g}) takes on the more obvious form of
\begin{equation}\label{ELinv_h}
  \Xi_{\tilde j} = \Xi_i {\Lambda^i}_{\tilde j}
\end{equation}
which shows that the Euler-Lagrange equations transform like the components of a covariant vector.
\end{textblock*}



%%%%%%%%%%%%%%%%%%%%%%%%%%%%%%%%%%%%%%%%%%%%%%%%%%%%%%%%%%%%%%%
%          EL Equations in 1st order form
%%%%%%%%%%%%%%%%%%%%%%%%%%%%%%%%%%%%%%%%%%%%%%%%%%%%%%%%%%%%%%%
\begin{textblock*}{65mm}(74.85mm,12.77mm)
The Euler-Lagrange equations can be cast into 1$^{st}$-order form by first making the identification
\[
  \frac{d}{d t} q^{\alpha} = {\dot q}^{\alpha}
\]
and then by expanding
\[
  \frac{d}{dt} \left( \frac{\partial L}{\partial {\dot q}^{\alpha}} \right) = 
    \frac{\partial^2 L}{\partial q^{\beta} \partial {\dot q}^{\alpha}} {\dot q}^{\beta} + 
	\frac{\partial^2 L}{\partial {\dot q}^{\alpha} \partial {\dot q}^{\beta}} {\ddot q^{\beta}}         +
	\frac{\partial^2 L}{\partial t \partial q^{\alpha}}                 
\]
Substituting this form in the EL equations and solving for the ${\ddot q}^{\alpha}$ yields
\begin{eqnarray*}
  \frac{d}{dt} {\dot q}^{\alpha} & = & {\ddot q}^{\alpha} \\ 
                                 & = & \left( \frac{\partial^2 L}{\partial {\dot q}^{\alpha} \partial {\dot q}^{\beta}} \right)^{-1} \\
								 &   & \times \left( \frac{\partial L}  {\partial q^{\beta}}                    -
							                         \frac{\partial^2 L}{\partial t \partial {\dot q}^{\beta}}  - 
									                 \frac{\partial^2 L}{\partial q^{\beta} \partial {\dot q}^{\gamma}} {\dot q}^{\gamma} 
									   \right)										  
\end{eqnarray*}
which requires that the Hessian, defined by
\[
  \left( \frac{\partial^2 L}{\partial {\dot q}^{\alpha} \partial {\dot q}^{\beta}} \right)
\]
be invertible.
\end{textblock*}
\newpage
%Classical Electromagnetism- filename em.tex

\null
%%%%%%%%%%%%%%%%%%%%%%%%%%%%%%%%%%%%%%%%%%%%%%%%%%%%%%%%%%%%%%%%%%%%%%%%%%%%%%%
%          Classical Electromagnetism
%%%%%%%%%%%%%%%%%%%%%%%%%%%%%%%%%%%%%%%%%%%%%%%%%%%%%%%%%%%%%%%%%%%%%%%%%%%%%%%
\textblockcolor{test}
\begin{textblock*}{7.5in}(0mm,0mm)
\begin{tabular*}{7.5in}{c @{\extracolsep{\fill}} c }
       \tiny ~ & ~\\
       \multicolumn{2}{c}{\normalsize \bf Classical Electromagnetism} \\
       \tiny~ & ~\\
\end{tabular*}
\end{textblock*}

\begin{tabular}{lll}
   \centering{Point Form} & Integral Form & Label and Name \\
              &               &                \\
    $\vec \nabla \times \vec H = {\vec J}_c + \frac{\partial {\vec D}}{\partial t}$ & 
    $\oint {\vec H} \cdot d {\vec \ell} = \int_S \left( {\vec J}_c + \frac{\partial {\vec D}}{\partial t} \right) \cdot d {\vec S}$ &
    ME1 - Ampere's Law\\
              &               &                \\     
    $\vec \nabla \times \vec E = -\frac{\partial \vec B}{\partial t}$ &
    $\oint \vec E \cdot d \vec \ell = \int_S \left(-\frac{\partial \vec B}{\partial t} \right) \cdot d \vec S$ &
    ME2 - Faraday's Law ($S$ fixed) \\
              &               &                \\     
    $\vec \nabla \cdot \vec D = \rho$ &
    $\oint_S \vec D \cdot d \vec S = \int_V \rho d V$ &
    ME3 - Gauss' Law\\
              &               &                \\     
    $\vec \nabla \cdot \vec B = 0$ &
    $\oint_S \vec B \cdot d \vec S = 0 $ &
    ME4 - nonexistance of monopoles    \\
              &               &                \\
    ${\vec D} = \epsilon {\vec E}$ & & \\
    ${\vec B} = \mu      {\vec H}$ & & \\
    ${\vec F} = q \left( {\vec E} + {\vec v} \times {\vec B} \right)$    & &    \\
    $\epsilon = \epsilon_r \epsilon_0$ & & \\
    $\mu      = \mu_r      \mu_0$      & & \\
    $\vec H$   & magnetic field strength & \\
    $\vec E$   & electric field strength & \\
    $\vec B$   & magnetic flux density   & \\
    $\vec D$   & electric flux density   & \\
    $\vec J$   & current density         & \\
    $\rho  $   & charge density          & \\
    $\vec F$   & force                   & \\
    $q$        & charge                  & \\
    $\vec v$   & velocity                & \\
    $\mu$      & permeability            & \\
    $\epsilon$ & permittivity            & \\
               &                         & \\
    ${\vec E} = \frac{\partial \vec A}{\partial t} - \nabla \phi$ & & \\
    ${\vec B} = \nabla \times {\vec A}$ & & \\
\end{tabular}
\vspace{10mm}
simplifying assumptions:  $EMA1:  \epsilon = const \& \mu = const$

\begin{tabular}{ll}
  \emph{Magnetic Fields}                                                                     &
  \emph{Electric Fields}                                                                     \\
                                                                                             &
                                                                                             \\
  $B_{n1} = B_{n2}$                                                                          & 
  $\left( {\vec D}_1 - {\vec D}_2 \right) \cdot {\vec a}_{n12} = - \rho_s$                   \\
                                                                                             &
                                                                                             \\
  $\left( {\vec H}_1 - {\vec H}_2 \right) \times {\vec a}_{n12}  = {\vec K}$                 &
  $E_{t1} = E_{t2}$                                                                          \\
                                                                                             &
                                                                                             \\
  $\frac{\tan \theta_1}{\tan \theta_2} = \frac{\mu_{r2}}     {\mu_{r1}}$      (current-free) &
  $\frac{\tan \theta_1}{\tan \theta_2} = \frac{\epsilon_{r2}}{\epsilon_{r1}}$ (charge-free)  \\
\end{tabular}

\null
%%%%%%%%%%%%%%%%%%%%%%%%%%%%%%%%%%%%%%%%%%%%%%%%%%%%%%%%%%%%%%%
%          Stuff
%%%%%%%%%%%%%%%%%%%%%%%%%%%%%%%%%%%%%%%%%%%%%%%%%%%%%%%%%%%%%%%
\[
  {\mathbf u} \times {\mathbf v} = u^i v^j \epsilon_{ijk} {\mathbf e}^k
\]
\[
   ( {\mathbf e}_i \times {\mathbf e}_j ) \cdot {\mathbf e}_k =  \epsilon_{ijk} = J [ijk]
\]
\[
  \epsilon^{ijk} = ({\mathbf e}^i \times {\mathbf e}^j) \cdot {\mathbf e}^k 
\]

\[
(*d*d + d*d*) f \Leftrightarrow \nabla^2 f
\]
\[
  \phi_{\mathbf A} \Leftrightarrow \left( A^i \delta_{ij} \right) dq^j \equiv A_j dq^j
\]
\[
  * d \phi_{\mathbf A} \Leftrightarrow \nabla \times {\mathbf A}
\]
\[
  *d* \phi_{\mathbf A} \Leftrightarrow div({\mathbf A})
\]
\[
(*d*)(*d \phi_{\mathbf A} ) \Leftrightarrow \nabla \cdot \nabla \times {\mathbf A}
\]
\newpage
\null

\begin{tabular}{cc}
  1: & $Q = \left[ \begin{array}{cc}
               1 & 2 \\
			   3 & 4
			 \end{array} \right]$
\end{tabular}

\[
  M = \left[ \begin{array}{cccc}
		|{\mathbf e}_1 \rangle_1 &  | {\mathbf e}_2 \rangle_1 & \dots & | {\mathbf e}_n \rangle_1 \vspace{1mm}\\
		|{\mathbf e}_1 \rangle_2 &  | {\mathbf e}_2 \rangle_2 & \dots & | {\mathbf e}_n \rangle_2 \\
		\vdots                   &     \vdots                 & \ddots &   \vdots \\
		|{\mathbf e}_1 \rangle_n &  | {\mathbf e}_2 \rangle_2 & \dots & | {\mathbf e}_n \rangle_2 \\
	  \end{array} \right]
\]

\[
  M^{-1} = \left[ \begin{array}{cccc}
		    \langle{\mathbf e}^1 |_1 &  \langle{\mathbf e}^1 |_2 & \dots & \langle{\mathbf e}^1 |_n \vspace{1mm}\\
		    \langle{\mathbf e}^2 |_1 &  \langle{\mathbf e}^2 |_2 & \dots & \langle{\mathbf e}^2 |_n \\
	    	\vdots                   &     \vdots                 & \ddots &   \vdots \\
		    \langle{\mathbf e}^n |_1 &  \langle{\mathbf e}^n |_2 & \dots & \langle{\mathbf e}^n |_n \vspace{1mm}\\
	  \end{array} \right]
\]


\[
  M M^{-1} = \left[ \begin{array}{cccc}
                 | {\mathbf e}_1 \rangle_1 \langle {\mathbf e}^1 | & \langle {\mathbf e}^1 | {\mathbf e}_2 \rangle & \dots & \langle {\mathbf e}^1 | {\mathbf e}_n \rangle \vspace{1mm}\\
                 \langle {\mathbf e}^2 | {\mathbf e}_1 \rangle & \langle {\mathbf e}^2 | {\mathbf e}_2 \rangle & \dots & \langle {\mathbf e}^2 | {\mathbf e}_n \rangle \\
	    	     \vdots                   &     \vdots                 & \ddots &   \vdots \\
                 \langle {\mathbf e}^n | {\mathbf e}_1 \rangle & \langle {\mathbf e}^n | {\mathbf e}_2 \rangle & \dots & \langle {\mathbf e}^n | {\mathbf e}_n \rangle \\
			   \end{array}
			   \right]
\]
\[
  {\mathcal L}_{\mathbf A} f = A^{\alpha} \partial_{q^{\alpha}} f
\]
\[
  {\mathcal L}_{\mathbf A} {\mathbf B} = [{\mathbf A},{\mathbf B}]
\]
\[
  {\mathcal L}_{\mathbf A} {\mathbf \omega} = ( \omega_{\sigma} {A^{\sigma}}_{,\gamma} + A^{\sigma} \omega_{\gamma,\sigma} ) d q^{\gamma}
\]

\newpage


\[
  \Delta_L = {\dot q}^{\alpha} \frac{\partial}{\partial q^{\alpha}} + {\ddot q}^{\alpha} \frac{\partial}{\partial {\dot q}^{\alpha}}
\]
\[
  \Theta_L = \frac{\partial L}{\partial {\dot q}^{\alpha}} d q^{\alpha}
\]
\[
  {\mathcal L}_{\Delta_L} \Theta_L = d L
\]
\[
  \Delta_H = {\dot \xi}^{j} \partial_j = {\dot q}^{\alpha} \frac{\partial}{\partial {q^{\alpha}}} + {\dot p}^{\alpha} \frac{\partial}{\partial {p^{\alpha}}}
\]
\[
  \Theta_H = p_{\alpha} d q^{\alpha}
\]
\[
  i_{\Delta_H} \omega = d H
\]
\[
 {\mathcal L}_{\Delta} f = \left\{ f , H \right\} + \partial_t f
\]
\[
  i_{\mathbf X} {\mathbf \alpha} = \langle {\mathbf \alpha} , {\mathbf X} \rangle
\]
\[
  i_{\mathbf X} {\mathbf \omega} = {\mathbf \omega} \left( \bullet , \bullet, \dots, \bullet, {\mathbf X} \right)
\]
\[
   {\mathcal L}_{\mathbf X} \left[ {\mathbf Y}, {\mathbf Z} \right] 
 = \left[ {\mathcal L}_{\mathbf X} {\mathbf Y}, {\mathbf Z} \right] + \left[ {\mathbf Y}, {\mathcal L}_{\mathbf X} {\mathbf Z} \right]
\]
\[
  {\mathcal L}_{\mathbf X} \left( \alpha \wedge \beta \right) 
 = \left( {\mathcal L}_{\mathbf X} \alpha \right) \wedge \beta
 = \alpha \wedge \left( {\mathcal L}_{\mathbf X} \beta \right) 
\]
\[
  \left[ {\mathcal L}_{\mathbf X}, {\mathcal L}_{\mathbf Y} \right] \alpha = {\mathcal L}_{\left[ {\mathbf X}, {\mathbf Y} \right] } \alpha
\]
\[
    \left[ {\mathcal L}_{\mathbf X} , i_{\mathbf Y} \right] \alpha 
  = \left[ i_{\mathbf Y}, {\mathcal L}_{\mathbf X} \right] \alpha 
  = i_{\left[ {\mathbf X}, {\mathbf Y} \right] } \alpha
  = i_{ \left( {\mathcal L}_{\mathbf X} {\mathbf Y} \right) } \alpha
\]
\[
 {\mathcal L}_{\mathbf X} \alpha = i_{\mathbf X} d \alpha + d i_{\mathbf X} \alpha
\]

\[
  \frac{d}{d t} {\mathbf S}_\phi = {\mathbf F}_\phi \left( {\mathbf S}_{\phi} ; t \right)
\]
\[
  \frac{d}{d t} {\mathbf S}_\eta = {\mathbf F}_\eta \left( {\mathbf S}_\eta ; t \right)
\]
\[
  \frac{d}{d t} {\mathbf T}_\phi = {\mathbf F}_\phi \left( {\mathbf T}_\phi ; t \right)
\]
\[
  \frac{d}{d t} {\mathbf T}_\eta = {\mathbf F}_\eta \left( {\mathbf T}_\eta ; t \right)
\]

\[
  \theta(x - q) = \left\{ \begin{array}{ll} 0 & x < q \\ 1 & x > q \end{array} \right.
\]
\[
 \int_{a}^{b} dx \, \theta(x-q) = \int_{a}^{q} d x
\]
\begin{eqnarray*}
 \int_{a}^{b} dx \, \theta(q - x)	& = & \int_{a}^{b} dx \left\{1 - \theta(x-q) \right\} \\
								& = & \int_{a}^{b} dx - \int_{a}^{q} dx \\
								& = & \int_{a}^{q} dx + \int_{q}^{b} dx - \int_{a}^{q} dx\\
								& = & \int_{q}^{b} dx
\end{eqnarray*}


\newpage
\null

Define an integrand
\[
  L(a,b)[f] = a f(x)^2 + b f'(x)^2
\]

\[
  I_v(a,b)[f] = \int_0^1 L(a,b)[f] dx = \int_0^1 \left( a f(x)^2 + b f'(x)^2 \right) dx
\]

\begin{tabular}{cc}
  $I_p(1,1)\left[x\right]$   & 1.33333 \\
  $I_p(1,1)\left[x^2\right]$ & 1.53333 \\
  $I_p(1,1)\left[x^3\right]$ & 1.94286
\end{tabular}
subject to the boundary conditions $f(0) = 0$ and $f(1) = 1$.
$I_p$ has as its domain any well-behaved function that is continuous and differentiable (this may be 
too restrictive).  Let the set of such functions be called $\mathcal D$.  Then let $\mathcal D'$ be
the subset of $\mathcal D$ that satisfies the boundary conditions.  Examples of $\mathcal D$ are:
$const, 0, \cos(x), x, x^2, (1-x^3), ln(x), \sin \left( \frac{\pi}{2} x \right), e^x$.  Examples
of $\mathcal D'$ are: $x, x^2, \sin \left( \frac{\pi}{2} \right)$.  


The goal is to find
$q \backepsilon {\mathcal D'}$ extremizes $I_p(a,b)[f]$ by taking the variation $\delta I_p$
and solve for $f$
\[
  \delta I_p = 2 \int_0^1 d\,x \left( a f \delta f + b f' \delta f' \right)
\]
\normalsize
\begin{figure}[htpb!]
\centering
\includegraphics[width=170mm,height=150mm]{var_1.eps}
\scriptsize\caption{Blah Blag}\label{fig:erptsqfit}
\end{figure}
\newpage
\null
dude





%%%%%%%%%%%%%%%%%%%%%%%%%%%%%%%%%%%%%%%%%%%%%%%%%%%%%%%%%%%%%%%%%%%%%%%%%%%%%%%%%%%%%%%%%%%%
\newpage
\null
\scriptsize

\textblockcolor{LightYellow}
\begin{textblock*}{65mm}(0mm,12.54mm)
One of the most basic 
\[
  I = \oint_{C} z^n dz 
\]
Assume on $C$ that $z = R e^{i \theta}$ where $\theta$ goes from $0$ to $2 \pi$.
With this parametrization $dz = R i e^{i \theta} d \theta$ and
\[
  I = i r^{n+1} \int_0^{2 \pi} e^{i \theta \left( n + 1 \right)} d \theta
\]
There are two cases to consider: 1) $n \neq -1$ and $n = -1$. 
In the first case
\begin{eqnarray*}
  I_{n \neq -1}  & = & i R^{n+1} \int_0^{2 \pi} e^{i \theta \left(n+1\right)} d \theta \\
                & = & \frac{R^{n+1}}{n+1} \left. e^{i \theta \left(n+1\right)} \right|^{2 \pi}_0 \\
                & = & 0
\end{eqnarray*}
and in the second
\[
  I_{n = -1}   = i \int_0^{2 \pi} d \theta = 2 \pi i
\]

\begin{figure}[htp]
\centering
\psscalebox{2}{
\begin{pspicture}(-1.5,-1.5)(1.5,1.5)
  \pscircle[fillstyle=solid,fillcolor=LightGreen,linestyle=none](0,0){1.005}
  \psarc[linewidth=0.5pt]{->}(0,0){1}{32.96709734}{212}
  \psarc[linewidth=0.5pt]{-}(0,0){1}{210}{27.03290266}
  \pscircle[fillstyle=solid,fillcolor=LightYellow,linestyle=none](0,0){0.21}  
  \psarc[linewidth=0.5pt]{-}(0,0){0.205}{45}{190}
  \psarc[linewidth=0.5pt]{<-}(0,0){0.205}{180}{15}
  \pspolygon[fillstyle=solid,fillcolor=LightYellow,linestyle=none,linewidth=0.5pt](0.19125331,0.05124617)(0.89067955,0.45446836)(0.83892092,0.54411694)(0.14140721,0.14140721)
  \psline[linewidth=0.5pt]{-}(0.19318517,0.05176381)(0.89076863,0.45451381)
  \psline[linewidth=0.5pt]{-}(0.14142136,0.14142136)(0.83900482,0.54417136)
  \psline[linewidth=0.5pt]{<-}(0.5419769,0.25313881)(0.89076863,0.45451381)
  \psline[linewidth=0.5pt]{->}(0.14142136,0.14142136)(0.49021309,0.34279636)
  \psline[linewidth=0.2pt]{->}(0,0)(1.29903811,0.75)
  \pscircle[fillstyle=solid,fillcolor=black,linestyle=none](0,0){0.05}
  \uput[45](0.6,0.6){${\scriptsize C}$}
  \uput[1](-0.8,-0.05){${z_0}$}
  \uput[1](-0.4,-0.4){${C_2}$}
  \uput[1](-0.2,0.45){${C_{in}}$}  
  \uput[1](0.2,0){${C_{out}}$}  
  %\psgrid(-1.5,-1.5)(1.5,1.5)
\end{pspicture}}
\caption{}\label{fig_simp_vec}
\end{figure}
\end{textblock*}

\newpage
\null
%%%%%%%%%%%%%%%%%%%%%%%%%%%%%%%%%%%%%%%%%%%%%%%%%%%%%%%%%%%%%%%
%          Stuff
%%%%%%%%%%%%%%%%%%%%%%%%%%%%%%%%%%%%%%%%%%%%%%%%%%%%%%%%%%%%%%%
\scriptsize
\textblockcolor{LightYellow}
\begin{textblock*}{74mm}(0mm,12.54mm)
\begin{tabular*}{74mm}{l @{\extracolsep{\fill}} l}
   & ~\\
\multicolumn{2}{c}{\bf Stuff} \\
\end{tabular*}
\end{textblock*}
\newpage
\null

\TPMargin{2mm}
\begin{textblock*}{69mm}(0mm,20mm)
\begin{tabular*}{69mm}{l @{\extracolsep{\fill}} l}
   & ~\\
\multicolumn{2}{c}{\bf Dual Vectors} \\
\end{tabular*}
Let a $n$-dimensional space be spanned by a
set of vectors $\{ | {\mathbf e}_i \rangle\}$.
Then the column-wise collection
  \[
    {\mathbf J} = \left[\, |{\mathbf e}_1 \rangle, \, | {\mathbf e}_2 \rangle, \,  \dots , \, | {\mathbf e}_n \rangle \, \vspace{10mm} \right]
  \]
is know as the \emph{Jacobian matrix}.  Its inverse
  \[
   {\mathbf J}^{-1} = \left[ \begin{array}{c} \langle {\mathbf e}^1 | \vspace{1mm}\\ \langle {\mathbf e}^2 | \\ \vdots \\ \langle {\mathbf e}^n | \end{array} \right]
\]
gives a vector space spanned by the \emph{dual vectors} that obey the contraction rule
\begin{eqnarray*}
  {\mathbf J}^{-1} {\mathbf J} & = &  \left[ \begin{array}{cccc}
                 \langle {\mathbf e}^1 | {\mathbf e}_1 \rangle & \langle {\mathbf e}^1 | {\mathbf e}_2 \rangle & \dots & \langle {\mathbf e}^1 | {\mathbf e}_n \rangle \vspace{1mm}\\
                 \langle {\mathbf e}^2 | {\mathbf e}_1 \rangle & \langle {\mathbf e}^2 | {\mathbf e}_2 \rangle & \dots & \langle {\mathbf e}^2 | {\mathbf e}_n \rangle \\
	    	     \vdots                   &     \vdots                 & \ddots &   \vdots \\
                 \langle {\mathbf e}^n | {\mathbf e}_1 \rangle & \langle {\mathbf e}^n | {\mathbf e}_2 \rangle & \dots & \langle {\mathbf e}^n | {\mathbf e}_n \rangle \\
			   \end{array} \right] \\
		   & = & \langle {\mathbf e}^i | {\mathbf e}_j \rangle = {\delta^i}_j
\end{eqnarray*}
and the completeness relation
\[
  {\mathbf J} {\mathbf J}^{-1} = \sum_i | {\mathbf e}_i \rangle \langle {\mathbf e}^i | = \mbox{Id}_{n \times n} .
\]
\end{textblock*}
\newpage
\null

\begin{textblock*}{50mm}(68.85mm,20mm)
  \begin{tabular*}{50mm}{l @{\extracolsep{\fill}} l}
    & \\
  \multicolumn{2}{c}{\bf Permutation Tensor}\\
  \end{tabular*}
  \[
    \epsilon_{ijk} = \det({\mathbf J}) [ijk] = J [ijk]
  \]
  \[
    \epsilon^{ijk} = \det \left( {\mathbf J}^{-1} \right) [ijk] = J^{-1} [ijk]
  \]
  \[
    \epsilon^{ijk} \epsilon_{pqr} = \left| 
	   \begin{array}{ccc}
	   {\delta^i}_p & {\delta^i}_q & {\delta^i}_p\\
	   {\delta^j}_p & {\delta^j}_q & {\delta^j}_r\\
	   {\delta^k}_p & {\delta^k}_q & {\delta^k}_r\\
	   \end{array}
	  \right|
  \]
  \[
    \epsilon^{ijk} \epsilon_{pqk} = {\delta^i}_p {\delta^j}_q - {\delta^i}_q {\delta^j}_p
  \]
  \[
    \epsilon^{ijk} \epsilon_{pjk} = 2 {\delta^i}_p
  \]
  \[
      \epsilon^{ijk} \epsilon_{ijk} = 6
  \]
\end{textblock*}
\newpage
\null

\begin{tabular}{cc}
  1: & $Q = \left[ \begin{array}{cc}
               1 & 2 \\
			   3 & 4
			 \end{array} \right]$
\end{tabular}

\[
  M = \left[ \begin{array}{cccc}
		|{\mathbf e}_1 \rangle_1 &  | {\mathbf e}_2 \rangle_1 & \dots & | {\mathbf e}_n \rangle_1 \vspace{1mm}\\
		|{\mathbf e}_1 \rangle_2 &  | {\mathbf e}_2 \rangle_2 & \dots & | {\mathbf e}_n \rangle_2 \\
		\vdots                   &     \vdots                 & \ddots &   \vdots \\
		|{\mathbf e}_1 \rangle_n &  | {\mathbf e}_2 \rangle_2 & \dots & | {\mathbf e}_n \rangle_2 \\
	  \end{array} \right]
\]

\[
  M^{-1} = \left[ \begin{array}{cccc}
		    \langle{\mathbf e}^1 |_1 &  \langle{\mathbf e}^1 |_2 & \dots & \langle{\mathbf e}^1 |_n \vspace{1mm}\\
		    \langle{\mathbf e}^2 |_1 &  \langle{\mathbf e}^2 |_2 & \dots & \langle{\mathbf e}^2 |_n \\
	    	\vdots                   &     \vdots                 & \ddots &   \vdots \\
		    \langle{\mathbf e}^n |_1 &  \langle{\mathbf e}^n |_2 & \dots & \langle{\mathbf e}^n |_n \vspace{1mm}\\
	  \end{array} \right]
\]


\[
  M M^{-1} = \left[ \begin{array}{cccc}
                 | {\mathbf e}_1 \rangle_1 \langle {\mathbf e}^1 | & \langle {\mathbf e}^1 | {\mathbf e}_2 \rangle & \dots & \langle {\mathbf e}^1 | {\mathbf e}_n \rangle \vspace{1mm}\\
                 \langle {\mathbf e}^2 | {\mathbf e}_1 \rangle & \langle {\mathbf e}^2 | {\mathbf e}_2 \rangle & \dots & \langle {\mathbf e}^2 | {\mathbf e}_n \rangle \\
	    	     \vdots                   &     \vdots                 & \ddots &   \vdots \\
                 \langle {\mathbf e}^n | {\mathbf e}_1 \rangle & \langle {\mathbf e}^n | {\mathbf e}_2 \rangle & \dots & \langle {\mathbf e}^n | {\mathbf e}_n \rangle \\
			   \end{array}
			   \right]
\]
\[
  {\mathcal L}_{\mathbf A} f = A^{\alpha} \partial_{q^{\alpha}} f
\]
\[
  {\mathcal L}_{\mathbf A} {\mathbf B} = [{\mathbf A},{\mathbf B}]
\]
\[
  {\mathcal L}_{\mathbf A} {\mathbf \omega} = ( \omega_{\sigma} {A^{\sigma}}_{,\gamma} + A^{\sigma} \omega_{\gamma,\sigma} ) d q^{\gamma}
\]

\null
\[
  \Delta_L = {\dot q}^{\alpha} \frac{\partial}{\partial q^{\alpha}} + {\ddot q}^{\alpha} \frac{\partial}{\partial {\dot q}^{\alpha}}
\]
\[
  \Theta_L = \frac{\partial L}{\partial {\dot q}^{\alpha}} d q^{\alpha}
\]
\[
  {\mathcal L}_{\Delta_L} \Theta_L = d L
\]
\[
  \Delta_H = {\dot \xi}^{j} \partial_j = {\dot q}^{\alpha} \frac{\partial}{\partial {q^{\alpha}}} + {\dot p}^{\alpha} \frac{\partial}{\partial {p^{\alpha}}}
\]
\[
  \Theta_H = p_{\alpha} d q^{\alpha}
\]
\[
  i_{\Delta_H} \omega = d H
\]
\[
 {\mathcal L}_{\Delta} f = \left\{ f , H \right\} + \partial_t f
\]
\[
  i_{\mathbf X} {\mathbf \alpha} = \langle {\mathbf \alpha} , {\mathbf X} \rangle
\]
\[
  i_{\mathbf X} {\mathbf \omega} = {\mathbf \omega} \left( \bullet , \bullet, \dots, \bullet, {\mathbf X} \right)
\]
\[
   {\mathcal L}_{\mathbf X} \left[ {\mathbf Y}, {\mathbf Z} \right] 
 = \left[ {\mathcal L}_{\mathbf X} {\mathbf Y}, {\mathbf Z} \right] + \left[ {\mathbf Y}, {\mathcal L}_{\mathbf X} {\mathbf Z} \right]
\]
\[
  {\mathcal L}_{\mathbf X} \left( \alpha \wedge \beta \right) 
 = \left( {\mathcal L}_{\mathbf X} \alpha \right) \wedge \beta
 = \alpha \wedge \left( {\mathcal L}_{\mathbf X} \beta \right) 
\]
\[
  \left[ {\mathcal L}_{\mathbf X}, {\mathcal L}_{\mathbf Y} \right] \alpha = {\mathcal L}_{\left[ {\mathbf X}, {\mathbf Y} \right] } \alpha
\]
\[
    \left[ {\mathcal L}_{\mathbf X} , i_{\mathbf Y} \right] \alpha 
  = \left[ i_{\mathbf Y}, {\mathcal L}_{\mathbf X} \right] \alpha 
  = i_{\left[ {\mathbf X}, {\mathbf Y} \right] } \alpha
  = i_{ \left( {\mathcal L}_{\mathbf X} {\mathbf Y} \right) } \alpha
\]
\[
 {\mathcal L}_{\mathbf X} \alpha = i_{\mathbf X} d \alpha + d i_{\mathbf X} \alpha
\]

\[
  \frac{d}{d t} {\mathbf S}_\phi = {\mathbf F}_\phi \left( {\mathbf S}_{\phi} ; t \right)
\]
\[
  \frac{d}{d t} {\mathbf S}_\eta = {\mathbf F}_\eta \left( {\mathbf S}_\eta ; t \right)
\]
\[
  \frac{d}{d t} {\mathbf T}_\phi = {\mathbf F}_\phi \left( {\mathbf T}_\phi ; t \right)
\]
\[
  \frac{d}{d t} {\mathbf T}_\eta = {\mathbf F}_\eta \left( {\mathbf T}_\eta ; t \right)
\]

\[
  \theta(x - q) = \left\{ \begin{array}{ll} 0 & x < q \\ 1 & x > q \end{array} \right.
\]
\[
 \int_{a}^{b} dx \, \theta(x-q) = \int_{a}^{q} d x
\]
\begin{eqnarray*}
 \int_{a}^{b} dx \, \theta(q - x)	& = & \int_{a}^{b} dx \left\{1 - \theta(x-q) \right\} \\
								& = & \int_{a}^{b} dx - \int_{a}^{q} dx \\
								& = & \int_{a}^{q} dx + \int_{q}^{b} dx - \int_{a}^{q} dx\\
								& = & \int_{q}^{b} dx
\end{eqnarray*}

$\bullet$  


\newpage
\null

%%%%%%%%%%%%%%%%%%%%%%%%%%%%%%%%%%%%%%%%%%%%%%%%%%%%%%%%%%%%%%%%%%%%%%%%%%%%%%%
%          Fourier Analysis
%%%%%%%%%%%%%%%%%%%%%%%%%%%%%%%%%%%%%%%%%%%%%%%%%%%%%%%%%%%%%%%%%%%%%%%%%%%%%%%
\textblockcolor{test}
\begin{textblock*}{7.5in}(0mm,0mm)
\begin{tabular*}{7.5in}{c @{\extracolsep{\fill}} c }
       \tiny ~ & ~\\
       \multicolumn{2}{c}{\normalsize \bf Periodic Functions, Fourier Analysis, and Sturm-Liouville Theory} \\
       \tiny~ & ~\\
\end{tabular*}
\end{textblock*}

\begin{textblock*}{90mm}(0mm,15mm)
Consider a periodic function $f(x-2\ell) = f(x) = f(x + 2\ell)$ defined on the interval $[-\ell,\ell]$.  
Because of the underlying periodicity of $f(x)$, both the derivative and the integral also inherit 
periodic properties.

In the case of the derivative, the observation is evident simply from 
the standard definition as follows:
\begin{eqnarray*}
  \frac{d f}{d x}(x) & = & \lim_{dx \rightarrow 0} \frac{ f(x+dx)-f(x)}{dx}\\
                     & = & \lim_{dx \rightarrow 0} \frac{ f(x + 2\ell + dx) - f(x + 2\ell) }{dx} \\
                     & = & \frac{d f}{d x}(x + 2\ell).
\end{eqnarray*}
In the case of the integral, a bit more thinking and manipulation is needed.
To begin, consider the integral of $f(x)$ over an interval $[a,b]$.  The value of this integral 
is invariant if the entire interval is shifted by $2 \ell$.  This is proven by direct substitution
as 
\begin{eqnarray}\label{eq_per_ab_int}
  \int_{a}^{b} f(x) dx & = & \int_{a}      ^{b}       f(x - 2\ell) dx            \\
                       & = & \int_{a+2\ell}^{b+2\ell} f(y)         dy  \nonumber \\
                       & = & \int_{a+2\ell}^{b+2\ell} f(x)         dx. \nonumber
\end{eqnarray}
With this result, it is easy (although somewhat subtle) to show that the integral
of $f(x)$ over the interval $[c-\ell,c+\ell]$ is equal to the integral over 
the interval $[-\ell,\ell]$.  To do so, start by letting $a = c - \ell$ and $b = -\ell$
in (\ref{eq_per_ab_int}), which yields the identity 
\begin{equation}\label{eq_per_int}
  \int_{c-\ell}^{-\ell} f(x) dx  =  \int_{c+\ell}^{\ell} f(x) dx,
\end{equation}
which is then used to manipulate the integral over $[c-\ell,c+\ell]$ as follows,
\begin{eqnarray} 
  \int_{c-\ell}^{c+\ell} f(x) dx  & = &  \int_{c-\ell}^{-\ell} f(x) dx + \int_{-\ell}^{c+\ell} f(x) dx \\
                                  & = &  \int_{c+\ell}^{\ell}  f(x) dx + \int_{-\ell}^{c+\ell} f(x) dx \nonumber \\
                                  & = &  \int_{-\ell}^{c+\ell} f(x) dx + \int_{c+\ell}^{\ell}  f(x) dx \nonumber \\
                                  & = &  \int_{-\ell}^{\ell}   f(x) dx                                 \nonumber
\end{eqnarray}
where (\ref{eq_per_int}) was used in going from the first to the second line.

\[
  -\frac{d}{d x}\left( p(x) \frac{d y}{d x} \right) + q(x) y = \lambda w(x) y
\]
\end{textblock*}

\begin{textblock*}{100mm}(89.85mm,15mm)
\[
  \int_{c}^{c+2L} \cos \left( \frac{m \pi x}{L} \right) dx = 0
 \]
 \[
  \int_{c}^{c+2L} \sin \left( \frac{m \pi x}{L} \right) dx = 0
 \]
 
\[
  \int_{c}^{c+2L} dx \, \cos \left( \frac{m \pi x}{L} \right) \cos \left( \frac{n \pi x}{L} \right) = 
      \left\{ \begin{array}{cc} 2 L \, \delta_{mn} & m = 0 \\ L \, \delta_{mn} & m \neq 0 \end{array} \right.
\]
\[
  \int_{c}^{c+2L} dx \, \sin \left( \frac{m \pi x}{L} \right) \sin \left( \frac{n \pi x}{L} \right) = 
      \left\{ \begin{array}{cc} 0 \, \delta_{mn} & m = 0 \\ L \, \delta_{mn} & m \neq 0 \end{array} \right.
\]
\[
 {\tilde f}(x) = \frac{a_0}{2} + 
    \sum_{n=1}^{\infty} \left( a_n \cos \left( \frac{n \pi x}{L} \right) + b_n \sin \left( \frac{n \pi x}{L} \right) \right)
\]
\[
  a_n = \frac{1}{L} \int_{c}^{c+2L} f(x) \cos \left( \frac{n \pi x}{L} \right) dx
\]
\[
  b_n = \frac{1}{L} \int_{c}^{c+2L} f(x) \sin \left( \frac{n \pi x}{L} \right) dx
\]

\[
  {\tilde f}(x) = \sum_{n = - \infty}^{\infty} c_n e^{i n \pi x/L}
\]
\[
  c_n = \frac{1}{2 L} \int_{c}^{c+2L} f(x) e^{-i n \pi x/L} dx
\]
\end{textblock*}
\newpage
\null
%%%%%%%%%%%%%%%%%%%%%%%%%%%%%%%%%%%%%%%%%%%%%%%%%%%%%%%%%%%%%%%
%          Harmonic Oscillators
%%%%%%%%%%%%%%%%%%%%%%%%%%%%%%%%%%%%%%%%%%%%%%%%%%%%%%%%%%%%%%%
\textblockcolor{test}
\begin{textblock*}{7.5in}(0mm,0mm)
\begin{tabular*}{7.5in}{c @{\extracolsep{\fill}} c }
       \tiny ~ & ~\\
       \multicolumn{2}{c}{\normalsize \bf Harmonic Oscillators} \\
       \tiny~ & ~\\
\end{tabular*}
\end{textblock*}

\begin{textblock*}{2.5in}(0mm,30mm)
The equation for a damped simple harmonic oscillator is
\[
  m {\ddot x} + \Gamma {\dot x} + k x = 0 .
\]
Assume a solution of the form $x = A e^{\imath \omega t}$ and then substitute into the equation of motion.
Factoring out $A e^{i \omega t}$, yields the characteristic equation
\[
  -m \omega^2 + i \Gamma \omega + k = 0 .
\]
Before solving the charateristic equation, divide by $m$ and then define the parameters ${\omega_0}^2 = k/m$ and 
$\gamma = \Gamma/2$.  The equation then becomes
\[
  \omega^2 - i \gamma \omega + {\omega_0}^2 = 0
\]
with corresponding roots
\[
  \omega = \frac{i \gamma}{2} \pm \sqrt{ {\omega_0}^2 - \frac{\gamma^2}{4} } 
         = \frac{i \gamma}{2} \pm \omega_D
\]
Thus the final solution looks like
\[
  x = e^{-i \gamma/2} \left[ x_0 \cos(\omega_D t) + \frac{v_0}{\omega_D} \sin(\omega_D t) \right]
\]
\end{textblock*}
\newpage
\null

%%%%%%%%%%%%%%%%%%%%%%%%%%%%%%%%%%%%%%%%%%%%%%%%%%%%%%%%%%%%%%%
%          Variational Example
%%%%%%%%%%%%%%%%%%%%%%%%%%%%%%%%%%%%%%%%%%%%%%%%%%%%%%%%%%%%%%%
Define an integrand
\[
  L(a,b)[f] = a f(x)^2 + b f'(x)^2
\]

\[
  I_v(a,b)[f] = \int_0^1 L(a,b)[f] dx = \int_0^1 \left( a f(x)^2 + b f'(x)^2 \right) dx
\]
\normalsize
%\begin{figure}[htp]
%\centering
%\scriptsize\caption{Blah Blag}\label{fig:erptsqfit}
%\includegraphics[width=170mm,height=150mm]{var_1.eps}
%\end{figure}

\newpage
\null
Suppose we have a vector specified against a set of coordinate
axes $\{ \hat \imath, \hat \jmath \}$ as shown in 
Fig. \ref{fig_simp_vec}.

\begin{figure}[htp]
\centering
%\begin{pspicture}(0,0)(4,4)
%  \psline[linecolor=black]{->}(4,0)
%  \psline[linecolor=black]{->}(0,4)
%  \psline[linecolor=black,linewidth=2pt]{->}(2,2)
%  \uput[45](4.2,0){$\hat \imath$}  
%  \uput[45](0,4.2){$\hat \jmath$}  
%  \uput[45](2.2,2.2){$\vec V$}
%\end{pspicture}
\caption{Vector $V$ against the $\{\hat \imath, \hat \jmath\}$ coordinate system}\label{fig_simp_vec}
\end{figure}

\noindent By definition, an expression for $\vec V$ in terms of this basis
is 
\begin{eqnarray}\label{vec_def_A}
  \vec V & =       &   \left( \vec V \cdot \hat \imath \right) \hat \imath 
                     + \left( \vec V \cdot \hat \jmath \right) \hat \jmath \\
         & \equiv  &   V_{\hat \imath} \hat \imath 
                     + V_{\hat \jmath} \hat \jmath
\end{eqnarray}
To see that Eq. (\ref{vec_def_A}) is a consistent definition, take the 
dot product of both sides first with $\hat \imath$ and then $\hat \jmath$.

On the surface of it, Eq. (\ref{vec_def_A}) may seem a tautology except
for a clever manipulation that immediately elevates its usefulness. To
see this utility, consider another coordinate system spanned by the set
$\{ \hat \imath ', \hat \jmath'\}$ as shown in Fig. 
\ref{fig_vec_trans_A}.

\begin{figure}[htp]
\centering
%\begin{pspicture}(0,0)(4.5,4.5)
%  \psline[linecolor=black]{->}(4,0)
%  \psline[linecolor=black]{->}(0,4)
%  \psline[linecolor=black]{->}(3.75877,1.36808)
%  \psline[linecolor=black]{->}(-1.36808,3.75877)  
%  \psline[linecolor=black,linewidth=2pt]{->}(2,2)
%  \uput[45](4.2,0){$\hat \imath$}  
%  \uput[45](0,4.2){$\hat \jmath$}  
%  \uput[45](3.8,1.4){$\hat \imath'$}  
%  \uput[45](-1.4,3.8){$\hat \jmath'$}  
%  \uput[45](2.2,2.2){$\vec V$}
%  \uput[45](1.2,0.02){$\phi$}
%  \psarc[linecolor=black]{->}(0,0){1}{0}{20}
%\end{pspicture}
\caption{}\label{fig_vec_trans_A}
\end{figure}

In terms of the new coordinates, $\vec V$ can be written as
\begin{eqnarray}\label{vec_def_B}
  \vec V & =       &   \left( \vec V \cdot \hat \imath' \right) \hat \imath 
                     + \left( \vec V \cdot \hat \jmath' \right) \hat \jmath \\
         & \equiv  &   V_{\hat \imath'} \hat \imath 
                     + V_{\hat \jmath'} \hat \jmath .
\end{eqnarray}
So far, there is nothing new.  But now take the dot product of both sides
of Eq. \ref{vec_def_B} with first $\hat \imath$ and then $\hat \jmath$.
The relations
\begin{eqnarray}\label{vec_rel_A}
\left(\hat \imath \cdot \vec V \right) \equiv V_{\hat \imath} 
   & = &
    \hat \imath \cdot \hat \imath' \, V_{\hat \imath'}
+  \hat \imath \cdot \hat \jmath' \, V_{\hat \jmath'} \\
\left(\hat \jmath \cdot \vec V \right) \equiv V_{\hat \jmath} 
   & = &
    \hat \jmath \cdot \hat \imath' \, V_{\hat \imath'}
+  \hat \jmath \cdot \hat \jmath' \, V_{\hat \jmath'}
\end{eqnarray}
result.  Placing these into matrix form gives
\begin{equation}\label{vec_rel_B}
\left[ \begin{array}{c} V_{\hat \imath} \\ V_{\hat \jmath} \end{array} \right]
   = 
\left[ \begin{array}{cc} \hat \imath \cdot \hat \imath' & \hat \imath \cdot \hat \jmath' \\
                          \hat \jmath \cdot \hat \imath' & \hat \jmath \cdot \hat \jmath' 
        \end{array} \right]
\left[ \begin{array}{c} V_{\hat \imath'} \\ V_{\hat \jmath'} \end{array} \right]        
\end{equation}
as the equation for relating the components of $\vec V$ against the set 
$\{ \hat \imath, \hat \jmath \}$ in terms of its components against the
set $\{ \hat \imath', \hat \jmath' \}$.

Using similar reasoning the dual relation of the components of $\vec V$ 
against the set $\{ \hat \imath', \hat \jmath' \}$ in terms of its
components against the set $\{\hat \imath, \hat \jmath\}$ must be
\begin{equation}\label{vec_rel_C}
\left[ \begin{array}{c} V_{\hat \imath'} \\ V_{\hat \jmath'} \end{array} \right]
   = 
\left[ \begin{array}{cc} \hat \imath' \cdot \hat \imath & \hat \imath' \cdot \hat \jmath \\
                          \hat \jmath' \cdot \hat \imath & \hat \jmath' \cdot \hat \jmath 
        \end{array} \right]
\left[ \begin{array}{c} V_{\hat \imath} \\ V_{\hat \jmath} \end{array} \right] .
\end{equation}

If, for convenience, we name the coordinate systems ${\mathcal A} \doteq \{\hat \imath, \hat \jmath\}$
and ${\mathcal B} \doteq \{\hat \imath', \hat \jmath'\}$ and we adopt the notation
that $\left[ \vec V \right]_{\mathcal A}$ are the components of $\vec V$ against $\mathcal A$
presented column-wise and likewise for $\left[ \vec V \right]_{\mathcal B}$.  Then we
can concisely write the transformation matrix from $\mathcal A$ to $\mathcal B$ as
\begin{equation}\label{trans_matrix_A}
  {\mathcal T}_{ {\mathcal A} -> {\mathcal B} } = 
     \left[ \begin{array}{c} {\left[ \hat \imath' \right]}^{T}_{\mathcal A} \\ 
                             {\left[ \hat \jmath' \right]}^{T}_{\mathcal A}
            \end{array} \right]
\end{equation}
which is to say that ${\mathcal T}_{ {\mathcal A} -> {\mathcal B} }$ can be thought of 
as the matrix whose rows are the components of the basis vectors of $\mathcal B$ in 
the $\mathcal A$ frame.

Finally, we can evaluate ${\mathcal T}_{ {\mathcal A} -> {\mathcal B} }$ in terms 
of the angle $\phi$ (see Fig. \ref{fig_vec_trans_A}).  Using standard trigonometric
relations
\begin{eqnarray}\label{vec_rel_D}
  \hat \imath' &= &  \cos \phi \hat \imath + \sin \phi \hat \jmath \\
  \hat \jmath' &= & -\sin \phi \hat \imath + \cos \phi \hat \jmath
\end{eqnarray}
so that ${\mathcal T}_{ {\mathcal A} -> {\mathcal B} }$  now specifically becomes
\begin{equation}\label{trans_matrix_B}
  {\mathcal T}_{ {\mathcal A} -> {\mathcal B} } = 
  \left[ \begin{array}{cc} \cos \phi & \sin \phi \\ -\sin \phi & \cos \phi \end{array} \right] .
\end{equation}

For those of a skeptical mind, a few test cases will help to build confindence.  First consider 
$\vec V$ parallel to $\hat \imath$ ($\vec V = V \hat \imath$), the using basic trigonometry,
$\vec V$ should have components in the $\mathcal B$ frame of
\begin{equation}\label{vec_test_A}
  \vec V = V \cos \phi \hat \imath' + V \sin \phi \hat \jmath'
\end{equation}
Multiplying out Eq. \ref{vec_rel_C} in this case we get
\begin{equation}
  \left[ \vec V \right]_{\mathcal B}
= \left[ \begin{array}{cc} \cos \phi & \sin \phi \\ -\sin \phi & \cos \phi \end{array} \right] 
  \left[ \begin{array}{c} V \\ 0 \end{array} \right]
= \left[ \begin{array}{c} V \cos \phi \\ -V \sin \phi \end{array} \right].
\end{equation}
Next consider $\vec V$ parallel to $\hat \jmath$ ($\vec V = V \hat \jmath$), then,
again from trigonometry, the components of $\vec V$ are
\begin{equation}\label{vec_test_B}
  \vec V = V \sin \phi \hat \imath' + V \cos \phi \hat \jmath' .
\end{equation}
Multiplying out Eq. \ref{vec_rel_C} in this case we get
\begin{equation}
  \left[ \vec V \right]_{\mathcal B}
= \left[ \begin{array}{cc} \cos \phi & \sin \phi \\ -\sin \phi & \cos \phi \end{array} \right] 
  \left[ \begin{array}{c} 0 \\ V \end{array} \right]
= \left[ \begin{array}{c} V \sin \phi \\ V \cos \phi \end{array} \right].
\end{equation}
For the final test suppose $\vec V$ is parallel to $\hat \imath'$ ($\vec V = V \hat \imath'$)
then $\vec V = V \cos \phi \hat \imath + V \sin \phi \hat \jmath$ and multiplying out Eq.
\ref{vec_rel_C} yields
\begin{eqnarray}
      \left[ \vec V \right]_{\mathcal B}
& = & \left[ \begin{array}{cc} \cos \phi & \sin \phi \\ -\sin \phi & \cos \phi \end{array} \right]
  \left[ \begin{array}{c} V \cos \phi \\ V \sin \phi \end{array} \right] \\
& = & \left[ \begin{array}{c} V \left( \cos^2 \phi + \sin^2 \phi \right) \\
                              V \left( \cos \phi \sin \phi - \cos \phi \sin \phi \right) 
             \end{array} \right] \\
& = & \left[ \begin{array}{c} V \\ 0 \end{array} \right] .
\end{eqnarray}
The result found in Eq. \ref{vec_rel_B} can be extended easily and straightforwardly 
to an arbitrary number of dimensions.  Specifically, for three dimensions 
\begin{equation}\label{vec_rel_3d}
 \left[ \begin{array}{c} V_{\hat \imath} \\ V_{\hat \jmath} \\ V_{\hat k} \end{array} \right]
 = \left[
      \begin{array}{ccc}
         \hat \imath \cdot \hat \imath' & \hat \imath \cdot \hat \jmath' & \hat \imath \cdot \hat k' \\
         \hat \jmath \cdot \hat \imath' & \hat \jmath \cdot \hat \jmath' & \hat \imath \cdot \hat k' \\
         \hat      k \cdot \hat \imath' &           k \cdot \hat \jmath' & \hat \imath \cdot \hat k'
	  \end{array}
	\right]
\left[ \begin{array}{c} V_{\hat \imath'} \\ V_{\hat \jmath'} \\ V_{\hat k'} \end{array} \right].
\end{equation}
Now how to actually construct the matrix found in Eq. \ref{vec_rel_3d}?  In two dimensions it was trivial
(see Eq. \ref{vec_rel_D}) since there was only one angle to deal with and the corresponding trigonometry
was easy to determine.  What if there are two separate angles that are not in the same plane?  To see how
tackle this consider Fig. \ref{2d_Euler}
\begin{figure}[htp]
\caption{Something goes here}\label{2d_Euler}
\end{figure}
First rotate about $z^{(0)}$ by $\theta$.  The unit vectors $\{x^{(1)},y^{(1)},z^{(1)}\}$ can be expressed
in terms of the unit vectors $\{x^{(0)},y^{(0)},z^{(0)}\}$ as
\begin{equation}
 \begin{array}{ccccccc}
 x^{(1)} & = &  \cos \theta \, x^{(0)} & + & \sin \theta \, y^{(0)} &   &   \\ 
 y^{(1)} & = & -\sin \theta \, x^{(0)} & + & \cos \theta \, y^{(0)} &   &   \\
 z^{(1)} & = &                         &   &                        &   &   z^{(0)}
 \end{array}.
\end{equation}
Next rotate about $y^{(1)}$ by $\phi$ and express the unit vectors $\{x^{(2)},y^{(2)},z^{(2)}\}$ in terms
of the unit vectors $\{x^{(1)},y^{(1)},z^{(1)}\}$ as
\begin{equation}
 \begin{array}{ccccccc}
 x^{(2)} & = &  \cos \phi \, x^{(1)} &   &                  & + & -\sin \phi z^{(1)} \\ 
 y^{(2)} & = &                       &   &          y^{(1)} &   &                   \\
 z^{(2)} & = &  \sin \phi \, x^{(1)} &   &                  & + & \cos \phi z^{(1)}
 \end{array}.
\end{equation}


\newpage
\null
\begin{figure}[htp]
\centering
\begin{pspicture}(-2.5,-1.5)(4,1.5)
  %create top anchor
  \psline[linecolor=black](-2,1)(2,1)
  \psline( 2.0,1)( 2.2,1.5)
  \psline( 1.5,1)( 1.7,1.5)
  \psline( 1.0,1)( 1.2,1.5)
  \psline( 0.5,1)( 0.7,1.5)
  \psline( 0.0,1)( 0.2,1.5)
  \psline(-0.5,1)(-0.3,1.5)
  \psline(-1.0,1)(-0.8,1.5)
  \psline(-1.5,1)(-1.3,1.5)
  \psline(-2.0,1)(-1.8,1.5)
  %create bottom anchor
  \psline[linecolor=black](-2,-1)(2,-1)
  \psline(-2.0,-1)(-2.2,-1.5)
  \psline(-1.5,-1)(-1.7,-1.5)
  \psline(-1.0,-1)(-1.2,-1.5)
  \psline(-0.5,-1)(-0.7,-1.5)
  \psline( 0.0,-1)(-0.2,-1.5)
  \psline( 0.5,-1)( 0.3,-1.5)
  \psline( 1.0,-1)( 0.8,-1.5)
  \psline( 1.5,-1)( 1.3,-1.5)
  \psline( 2.0,-1)( 1.8,-1.5)
  %create origin and coord axes
  \pscircle[fillstyle=solid,fillcolor=black](0,0){0.05}
  \psline[linecolor=black,linewidth=0.5pt,linestyle=dashed,dash=1pt 2pt](-2,0)(3,0)
  \psline[linecolor=black,linewidth=0.5pt,linestyle=dashed,dash=1pt 2pt](0,1)(0,-1)
  \psline[linecolor=black,linewidth=0.5pt]{->}(2.4,0)(2.9,0)
  \psline[linecolor=black,linewidth=0.5pt]{->}(2.4,0)(2.4,0.5)  
  %create slider and springs
  \pscoil[coilwidth=0.25cm](0,1) (1.7,0)
  \pscoil[coilwidth=0.25cm](0,-1)(1.7,0)  
  \pscircle[fillstyle=solid,fillcolor=white](1.7,0){0.22}
  \pscircle[fillstyle=solid,fillcolor=gray](0,1){0.08}
  \pscircle[fillstyle=solid,fillcolor=gray](0,-1){0.08}
  %create annotation
  \uput[45](1.44,-0.2){$m$}  
  \uput[45](2.8,0.0){$x$}  
  \uput[45](2.3,0.4){$y$}
  \uput[45](0.1,1) {$\mathcal P$}
  \uput[45](-0.1,-1.5){$\mathcal Q$}
  \uput[45](1.1,0.4){$k$}
  \uput[45](1.1,-0.9){$k$}
\end{pspicture}
\caption{}\label{fig_simp_vec}
\end{figure}

\begin{eqnarray}
   {\mathcal P} & \equiv & \left[0, h\right]^T\\
   {\mathcal Q} & \equiv & \left[0,-h\right]^T \nonumber \\
   {\mathcal R} & \equiv & \left[x,0\right]^T \nonumber
\end{eqnarray}


\begin{equation}
  \ell = || {\mathcal R} - {\mathcal P} || = \sqrt{x^2+h^2}
\end{equation}

\begin{equation}
  T = 1/2 m {\dot x}^2
\end{equation}

\begin{equation}
  U = k \left( \ell - \ell_0 \right)^2
\end{equation}

\begin{equation}
  L = T - U = \frac{1}{2} m \dot x ^2 - k \left( \ell - \ell_0 \right)^2
\end{equation}

\begin{equation}
  \frac{d}{dt} \left( \frac{\partial L}{\partial \dot x} \right) - \frac{\partial L}{\partial x} = 0 \rightarrow
  m \ddot x + 2 k x \left(1 - \frac{\ell_0}{\ell}\right) = 0
\end{equation}

\begin{equation}
  \omega_0^2 = k/m
\end{equation}

\begin{equation}
\frac{d}{dt} 
\underbrace{\left[ \begin{array}{c} x      \\ \dot x \end{array} \right]}_{\bar S}
  = 
\underbrace{\left[ \begin{array}{c} \dot x \\ - 2 \omega_0^2 x \left(1-\ell_0/\ell\right) \end{array} \right]}_{\bar f (\bar S)}
  =              
\end{equation}

\begin{equation}
  A   = \frac{\partial \bar f}{\partial \bar s}
      = \left[  \begin{array}{cc} 0 & 1 \\ 1 & \partial_x f_{\dot x} \end{array}\right]
\end{equation}
\begin{equation}
  \partial_x f_{\dot x} = -2 \omega_0^2 \left( 1-\ell_0/\ell +x^2 \ell_0/\ell^3 \right)
\end{equation}

\begin{equation}
  \frac{d}{dt} \Phi(t,t_0) = A(t) \Phi(t,t_0)
\end{equation}
with
\begin{equation}
  \Phi(t_0,t_0) = \left[ \begin{array}{cc} 1 & 0 \\ 0 & 1 \end{array} \right]
\end{equation}

\begin{equation}
  \frac{d}{dt} \left[ \begin{array}{c} \Phi_{11} \\ \Phi_{12} \\ \Phi_{21} \\ \Phi_{22} \end{array} \right]
    = 
  \left[ \begin{array}{c} \Phi_{21}                                   \\ \Phi_{22} \\
                          \Phi_{11} + \partial_x f_{\dot x} \Phi_{21} \\ \Phi_{12} + \partial_x f_{\dot x} \Phi_{22} 
         \end{array} 
  \right]
\end{equation}

\newpage
\null

\[
\sin^2 \theta + \cos^2 \theta  = 1
\]

\[
\tan \theta = \frac{\sin \theta}{\cos \theta}
\]

\[
1 + \tan^2 \theta = \sec^2 \theta
\]

\[
1 + \cot^2 \theta = \csc^2 \theta
\]

\[
\sin( u + v ) = \sin u \cos v + \cos u \sin v
\]

\[
\cos(u + v) = \cos u \cos v - \sin u \sin v
\]

\[
\tan(u + v) = \frac{ \tan u + \tan v}{1 - \tan u \tan v}
\]

\[
\sin \left( \frac{\pi}{2} - x \right) = \cos x
\]

\[
\cos \left( \frac{\pi}{2} - x \right) = \sin x
\]

\[
\tan \left( \frac{\pi}{2} - x \right) = \cot x
\]

\[
\cot \left( \frac{\pi}{2} - x \right) = \tan x
\]

\[
\sec \left( \frac{\pi}{2} - x \right) = \csc x
\]

\[
\csc \left( \frac{\pi}{2} - x \right) = \sec x
\]

\[
\sin(2 x) = 2 \sin x \cos x
\]

\[
\cos(2 x) = \cos^2 x - \sin^2 x 
          = 1 - 2 \sin^2 x
          = 2 \cos^2 x - 1
\]

\[
\tan(2 x) = \frac{2 tan x}{1 - \tan^2 x}
\]

\[
  \sin \left( \frac{x}{2} \right) = \pm \sqrt{ \frac{1-\cos x}{2} }
\]

\[
  \cos \left( \frac{x}{2} \right) = \pm \sqrt{ \frac{1+\cos x}{2} }
\]

\[
  \tan \left( \frac{x}{2} \right) = \frac{\sin x}{1 + \cos x}
                                  = \frac{1 - \cos x}{\sin x}
                                  = \pm \sqrt{ \frac{1-\cos x}{1+\cos x} }
\]

\[
A_{trapezoid} = 1/2 ( b_1 + b_2 ) h
\]
\[
A_{rhombus} = 1/2 d_1 d_2
\]
\[
A_{regular polygon} = 1/2 a p
\]
\[
LA_{cylinder} = C h
\]
\[
V_{pyramid or cone} = 1/3 B h
\]
\[
LA_{cone} = \pi r \ell \quad (\ell = slant height
\]
\[
V_{sphere} = 4/3 \pi r^3
\]
\[
TA_{sphere} = 4 \pi r^2
\]
\[
V_{hemisphere} = 2/3 \pi r^3
\]
\[
TA_{hemisphere} = 3 \pi r^2
\]
\newpage
\null
\scriptsize

An $n$-dimensional manifold is an $n$-dimensional space ${\mathcal M}$ such that any 
point $p \in {\mathcal M}$ has a neighborhood ${\mathcal U} \subset {\mathcal M}$ which
is homeomorphic (homeomorphism: bijective, continuous, with a continuous inverse; bijective: one-to-one and onto) 
to the interior of the $n$-dimensional unit ball. (Stephani 2003).

Homeomorphism: a continuous, bijective function with continuous inverse (confirmed)
Bijective: a function that is one-to-one and onto (confirmed)
Injective: a function that is one-to-one (confirmed)
Surjective: a function that is onto (confirmed)
Diffeomorphism: 

\begin{figure}[htp]

\psscalebox{1.6}{
\begin{pspicture}(-4,-1.2)(4,1.2)
%\psgrid(-4,-1.2)(4,1.2)
%define the bijective block
\psframe[linewidth=0.5pt, framearc=0.3](-4,-1.1)(-3,1.1)
\psframe[linewidth=0.5pt, framearc=0.3](-2.8,-1.1)(-1.8,1.1)
\dotnode*[dotstyle=*](-3.8,0.6){bi_A}
\dotnode*[dotstyle=*](-2.4,0.8){bi_A1}
\ncarc{->}{bi_A}{bi_A1}
\dotnode*[dotstyle=*](-3.4,0.4){bi_B}
\dotnode*[dotstyle=*](-2.4,0.2){bi_B1}
\ncarc{->}{bi_B}{bi_B1}
\dotnode*[dotstyle=*](-3.6,-0.6){bi_C}
\dotnode*[dotstyle=*](-1.9,-0.8){bi_C1}
\ncarc{->}{bi_C}{bi_C1}
\dotnode*[dotstyle=*](-3.2,-0.8){bi_D}
\dotnode*[dotstyle=*](-2.2,-0.4){bi_D1}
\ncarc{->}{bi_D}{bi_D1}
%
%define the injective block
\psframe[linewidth=0.5pt, framearc=0.3](-1.4,-1.1)(-0.4,1.1)
\psframe[linewidth=0.5pt, framearc=0.3](-0.2,-1.1)(0.8,1.1)
\psframe[linewidth=0.5pt, framearc=0.3](-0.2,-1.1)(0.8,0.0)
\dotnode*[dotstyle=*](-1.2,0.8){inj_A}
\dotnode*[dotstyle=*](0.2,0.7){inj_A1}
\ncarc{->}{inj_A}{inj_A1}
\dotnode*[dotstyle=*](-0.8,0.2){inj_B}
\dotnode*[dotstyle=*](0.4,0.4){inj_B1}
\ncarc{->}{inj_B}{inj_B1}
\dotnode*[dotstyle=*](-1.0,-0.4){inj_C}
\dotnode*[dotstyle=*](-0.1,0.1){inj_C1}
\ncarc{->}{inj_C}{inj_C1}
\dotnode*[dotstyle=*](0.2,-0.6){inj_D1}
\dotnode*[dotstyle=*](0.4,-0.4){inj_E1}
%
%define the surjective block
\psframe[linewidth=0.5pt, framearc=0.3](1.3,-1.1)(2.3,1.1)
\psframe[linewidth=0.5pt, framearc=0.3](2.5,-1.1)(3.5,1.1)
\dotnode*[dotstyle=*](1.4,0.9){sur_A}
\dotnode*[dotstyle=*](1.6,0.6){sur_B}
\dotnode*[dotstyle=*](2.0,0.4){sur_C}
\dotnode*[dotstyle=*](1.8,-0.2){sur_D}
\dotnode*[dotstyle=*](1.6,-0.6){sur_E}
\dotnode*[dotstyle=*](1.8,-0.7){sur_F}
%
\dotnode*[dotstyle=*](2.8,0.8){sur_A1}
\dotnode*[dotstyle=*](3.3,0.2){sur_B1}
\dotnode*[dotstyle=*](2.7,-0.4){sur_C1}
\dotnode*[dotstyle=*](3.15,-0.75){sur_D1}
\ncarc{->}{sur_A}{sur_A1}
\ncarc{->}{sur_B}{sur_A1}
\ncarc{->}{sur_C}{sur_B1}
\ncarc{->}{sur_E}{sur_C1}
\ncarc{->}{sur_F}{sur_D1}
%
\end{pspicture}}
\caption{}\label{fig_simp_duh}
\end{figure}

\end{document} 