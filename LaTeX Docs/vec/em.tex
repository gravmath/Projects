%Classical Electromagnetism- filename em.tex

\null
%%%%%%%%%%%%%%%%%%%%%%%%%%%%%%%%%%%%%%%%%%%%%%%%%%%%%%%%%%%%%%%%%%%%%%%%%%%%%%%
%          Classical Electromagnetism
%%%%%%%%%%%%%%%%%%%%%%%%%%%%%%%%%%%%%%%%%%%%%%%%%%%%%%%%%%%%%%%%%%%%%%%%%%%%%%%
\textblockcolor{test}
\begin{textblock*}{190.5mm}(0mm,0mm)
\begin{tabular*}{190.5mm}{c @{\extracolsep{\fill}} c }
       \tiny ~ & ~\\
       \multicolumn{2}{c}{\normalsize \bf Classical Electromagnetism} \\
       \tiny~ & ~\\
\end{tabular*}
\end{textblock*}

\begin{textblock*}{75mm}(0mm,15mm)
\tiny
Electromagnetic theory is built on::
\begin{itemize} \itemsep1pt \parskip0pt \parsep0pt
\item Maxwell's equations - 4 linear PDEs in 5 vector and 1 scalar functions (16 variables)
    \begin{itemize} \itemsep1pt \parskip0pt \parsep0pt
    \item $\vec E$ - electric field $[N/C]$ or $[V/m]$
    \item $\vec D$ - electric displacement $[C/m^2]$
    \item $\vec B$ - magnetic flux density $[Ns/Cm]$ or $[kg/Cs]$
    \item $\vec H$ - magnetic intensity $[A/m]$
    \item $\vec J$ - current density $[C/m^2/s]$
    \item $\rho$ - charge density $[C/m^3]$
    \end{itemize}
\item Lorentz Force Law $\vec F = q \left( \vec E + \vec v \times \vec B \right)$ relating the force to 
      the charge, electric field, velocity, and magnetic flux density.
\item Constitutive relations
    \begin{itemize} \itemsep1pt \parskip0pt \parsep0pt
    \item $\vec D = \vec D(\vec E)$; usually through $\vec D = \vec E + \vec P (\vec E) = \epsilon \vec E$ for linear polarization
    \item $\vec H = \vec H(\vec B)$; usually through $\vec H = \vec B + \vec M (\vec B) = \frac{1}{\mu} \vec B$ for linear magnetization
    \item $\vec J = \vec J(\vec E)$; usually through $\vec J = g \vec E$ for linear conductivity (Ohm's law)
    \item Note that $\epsilon$ (permittivity) $[C^2/Nm]$, $\mu$ (permeability) $[N/A^2]$, \& $g$ $[A/V]$ can either be scalars or rank 2 tensors for isotropic 
          or anisotropic linear materials, respectively
    \item Nonlinear behavior occurs for all three cases
    \end{itemize}
\end{itemize}
\end{textblock*}




\begin{textblock*}{50mm}(76mm,15mm)
\tiny
The Maxwell Equations exhibit the following properties:
\begin{itemize} \itemsep1pt \parskip0pt \parsep0pt
\item Charge conservation
\item Well-pose initial-value problem
\item Compact expression in terms of potentials
    \begin{itemize} \itemsep1pt \parskip0pt \parsep0pt
    \item Vector \& scalar potentials - connected to fields
    \item Connecting potentials \& sources - inhomogenous wave eqns.
    \item Super potential - all in one
    \item Gauge invariance
    \item Lorentz gauge
    \item Coulomb gauge
    \end{itemize}
\item Wave excitations
    \begin{itemize} \itemsep1pt \parskip0pt \parsep0pt
    \item Magnetic Intensity
    \item Electric Field
    \end{itemize}
\item Conservation of energy
    \begin{itemize} \itemsep1pt \parskip0pt \parsep0pt
    \item Poynting Theorem
    \item Energy \& momentum flow
    \end{itemize}
\item Well-defined boundary conditions
\end{itemize}
\end{textblock*}

\begin{textblock*}{190.5mm}(0mm,100mm)
\tiny
\begin{tabular}{lll}
   \centering{Point Form} & Integral Form & Label and Name \\
              &               &                \\
    $\vec \nabla \times \vec H = {\vec J}_c + \frac{\partial {\vec D}}{\partial t}$ & 
    $\oint {\vec H} \cdot d {\vec \ell} = \int_S \left( {\vec J}_c + \frac{\partial {\vec D}}{\partial t} \right) \cdot d {\vec S}$ &
    ME1 - Ampere's Law\\
              &               &                \\     
    $\vec \nabla \times \vec E = -\frac{\partial \vec B}{\partial t}$ &
    $\oint \vec E \cdot d \vec \ell = \int_S \left(-\frac{\partial \vec B}{\partial t} \right) \cdot d \vec S$ &
    ME2 - Faraday's Law ($S$ fixed) \\
              &               &                \\     
    $\vec \nabla \cdot \vec D = \rho$ &
    $\oint_S \vec D \cdot d \vec S = \int_V \rho d V$ &
    ME3 - Gauss' Law\\
              &               &                \\     
    $\vec \nabla \cdot \vec B = 0$ &
    $\oint_S \vec B \cdot d \vec S = 0 $ &
    ME4 - nonexistance of monopoles    \\
              &               &                \\
    ${\vec D} = \epsilon {\vec E}$ & & \\
    ${\vec B} = \mu      {\vec H}$ & & \\
    ${\vec F} = q \left( {\vec E} + {\vec v} \times {\vec B} \right)$    & &    \\
    $\epsilon = \epsilon_r \epsilon_0$ & & \\
    $\mu      = \mu_r      \mu_0$      & & \\
    $\vec H$   & magnetic field strength & \\
    $\vec E$   & electric field strength & \\
    $\vec B$   & magnetic flux density   & \\
    $\vec D$   & electric flux density   & \\
    $\vec J$   & current density         & \\
    $\rho  $   & charge density          & \\
    $\vec F$   & force                   & \\
    $q$        & charge                  & \\
    $\vec v$   & velocity                & \\
    $\mu$      &             & \\
    $\epsilon$ &             & \\
               &                         & \\
    ${\vec E} = \frac{\partial \vec A}{\partial t} - \nabla \phi$ & & \\
    ${\vec B} = \nabla \times {\vec A}$ & & \\
\end{tabular}
\vspace{10mm}
simplifying assumptions:  $EMA1:  \epsilon = const \& \mu = const$

\begin{tabular}{ll}
  \emph{Magnetic Fields}                                                                     &
  \emph{Electric Fields}                                                                     \\
                                                                                             &
                                                                                             \\
  $B_{n1} = B_{n2}$                                                                          & 
  $\left( {\vec D}_1 - {\vec D}_2 \right) \cdot {\vec a}_{n12} = - \rho_s$                   \\
                                                                                             &
                                                                                             \\
  $\left( {\vec H}_1 - {\vec H}_2 \right) \times {\vec a}_{n12}  = {\vec K}$                 &
  $E_{t1} = E_{t2}$                                                                          \\
                                                                                             &
                                                                                             \\
  $\frac{\tan \theta_1}{\tan \theta_2} = \frac{\mu_{r2}}     {\mu_{r1}}$      (current-free) &
  $\frac{\tan \theta_1}{\tan \theta_2} = \frac{\epsilon_{r2}}{\epsilon_{r1}}$ (charge-free)  \\
\end{tabular}
\end{textblock*}