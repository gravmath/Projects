\documentclass[10pt]{article}
\usepackage[absolute]{textpos}
\usepackage{graphicx}
\setlength{\parindent}{0pt}
\usepackage{color}
\usepackage{amssymb}
\textblockorigin{0.5in}{0.5in}
\definecolor{MintGreen}      {rgb}{0.2,0.85,0.5}
\definecolor{LightGreen}     {rgb}{0.9,1,0.9}
\definecolor{LightYellow}    {rgb}{1,1,0.6}
\definecolor{test}           {rgb}{0.8,0.95,0.95}


\usepackage[absolute]{textpos}
\TPshowboxestrue
\begin{document}
\null
%%%%%%%%%%%%%%%%%%%%%%%%%%%%%%%%%%%%%%%%%%%%%%%%%%%%%%%%%%%%%%%
%          Header
%%%%%%%%%%%%%%%%%%%%%%%%%%%%%%%%%%%%%%%%%%%%%%%%%%%%%%%%%%%%%%%
\textblockcolor{MintGreen}
\begin{textblock*}{7.5in}(0mm,0mm)
\begin{tabular*}{7.5in}{c @{\extracolsep{\fill}} c }
       \tiny ~ & ~\\
       \multicolumn{2}{c}{\normalsize \bf Notes on \emph{Introduction to Random Signals and Spplied Kalman Filtering}} \\
	   \multicolumn{2}{c}{\normalsize by Brown and Hwang}\\
       \multicolumn{2}{r}{\scriptsize \emph{C. Schiff - 1/22/08}} \\
\end{tabular*}
\end{textblock*}

%%%%%%%%%%%%%%%%%%%%%%%%%%%%%%%%%%%%%%%%%%%%%%%%%%%%%%%%%%%%%%%
%          Chapter 2 Notes
%%%%%%%%%%%%%%%%%%%%%%%%%%%%%%%%%%%%%%%%%%%%%%%%%%%%%%%%%%%%%%%
\scriptsize
\TPMargin{1mm}
\textblockcolor{LightYellow}
\begin{textblock*}{100mm}(0mm,16.75mm)
\begin{tabular*}{100mm}{l @{\extracolsep{\fill}} l}
                & \\
\multicolumn{2}{c}{\bf Chapter 2 Notes} \\
                & \\
\end{tabular*}
Sec. 2.1\\
$\bullet$ Signals considered represent some physical quantity (e.g. voltage)\\
$\bullet$ Time will usually be the independent variable\\
$\bullet$ A signal is said to \emph{deterministic} if it is exactly predictable for the time span of interest\\
\[
  x(t) = 10 \sin( 2 \pi t )
\]
\[
  x(t) = \left\{ \begin{array}{ll} 1,& t > 0 \\
                                   0,& t < 0
				 \end{array} \right.
\]
\[
 x(t) = \left\{ \begin{array}{ll} 1 - e^{-t},&  t > 0 \\
                                           0,&  t < 0
				\end{array} \right.
\]
$\bullet$ A \emph{random} signal always has some element of chance
\[
  X(t) = 10 \sin( 2 \pi t + \theta)
\]
\[
  X(t) = A sin( 2 \pi t + \theta)
\]
where $\theta$ is a random variable uniformly distributed between $0$ and $2 \pi$.

\end{textblock*}


\end{document}