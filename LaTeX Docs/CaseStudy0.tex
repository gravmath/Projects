%vec
\documentclass[10pt]{article}
\usepackage{pstricks}
\usepackage{pst-plot}
\usepackage{pst-node}
\usepackage{pst-tree}
\usepackage{pst-coil}
\usepackage{bbold}

\usepackage[absolute]{textpos}
\setlength{\parindent}{0pt}
\usepackage{color}
\usepackage{amssymb}
\usepackage{graphicx}
\textblockorigin{0.5in}{0.5in}
\definecolor{MintGreen}      {rgb}{0.2,0.85,0.5}
\definecolor{LightGreen}     {rgb}{0.9,1,0.9}
\definecolor{LightYellow}    {rgb}{1,1,0.6}
\definecolor{test}           {rgb}{0.8,0.95,0.95}

\TPshowboxestrue
\TPMargin{2mm}
\pagestyle{empty}

\begin{document}

\section{Introduction to Case Study 0}
We want to examine the solutions to the equation 
\begin{equation}\label{Eq:main_eq}
  \frac{d^2}{dt^2} z + p \frac{d}{dt} z + q z = t e^{2 t}
\end{equation}
with $p = -4$ and $q = 4$, 
by comparing and contrasting a variety of methods. First we will
solve the homogeneous equation equation through 2 separate methods.  The
first method will be through the standard second-order differential
equation methods.  The second method is based on the state space formalism. 

\section{Standard Second Order Method}
In this section, we tackle Eq.(\ref{Eq:main_eq}) by first calculating 
the characteristic equation for the homogeneous equation.  Ordinarily 
the characteristic equation yields two separate roots yielding two 
independent solutions.  However, in this case, the roots are repeated
and the reduction of order method must be used. Once two independent
solutions are obtained the particular solution is constructed by
using the Variation of Parameters method.

\subsection{Charateristic Equation}
To determine the characteristic equation assume that the solution 
has the form
\[
  z = \exp(r t) \; ,
\]
Substituting this form into Eq. (\ref{Eq:main_eq}) yields
\[
  r^2 - 4 r + 4 = 0 \;,
\]
which is solved immediately as $(r-2)^2 = 0$. The first independent
solution
\[
  y_1 = e^{2t}
\]
is then arrived at. However, this is as far as it goes due to the
double root.  

\subsection{Reduction of Order}

To obtain the second independent solution we turn to the \emph{reduction of order} technique.
Start by defining
\[
  y(t) = y_1(t) v(t) .
\]
Taking the first and second derivatives yields
\[
  {\dot y} = y_1 {\dot v} + {\dot y}_1 v
\]
and
\[
  {\ddot y} = {\ddot y}_1 v + 2 {\dot y}_1 {\dot v} + y_1 {\ddot v} ,
\]
respectively.
Substituting these forms for ${\dot y}$ and ${\ddot y}$ into the differential
equation ${\ddot y} + p {\dot y} + q y = 0$ yields
\[
    \left( {\ddot y}_1 + p {\dot y}_1 + q y_1 \right) v 
  + \left(2 {\dot y}_1 + p y_1 \right) {\dot v} 
  + y_1 {\ddot v} = 0 .
\]
The first term cancels since $y_1$ is a solution of the differential equation.
This leaves the following equation for $w = {\dot v}$
\[
  y_1 {\dot w} + (2 {\dot y}_1 + p y_1) w = 0
\]
which can be rearranged as
\[
 \frac{\dot w}{w} = -\frac{ (2 {\dot y}_1 + p y_1) {\dot v} }{y_1}
\]
which, since it is first-order, can be immediately integrated to
\begin{eqnarray*}
w & = & c \exp \left[ -\int dt \left(\frac{2 {\dot y}_1}{y_1} + p \right) \right] \\
  & = & c \exp \left[ -2 \int \frac{ {\dot y}_1' }{y_1} dt \right] \exp \left[ -\int dt p \right]  \\
  & = & c \frac{1}{{y_1}^2} \exp \left[ -\int dt p \right] 
\end{eqnarray*}
Now in the case of constant coeffcients, $p \neq p(t)$ and the integral can be 
performed trivially and $w$ becomes
\[
  w = \frac{c}{ {y_1}^2 } e^{-p t} .
\]
Since in this particular case we have equal roots $y_1 = e^{-p t /2}$ and $w$ can be 
written as
\[
  w = \frac{e^{-pt}}{\left( e^{-pt/2} \right)^2 } = 1 
\]
where the arbitrary constant of integration has been set to $1$ since we 
are interested only in the functional form.  This equation can be integrated 
immediately to give
\[
  v = \int dt \, w = \int dt = t
\]
so that the new independent solution is
\[
  y_2 = v y_1 = t y_1 = t e^{2 t}.
\]

\subsection{Variation of Parameters}
To respond to the inhomogeneous term $t e^{2t}$ start by assuming a solution
\[
  y(t) = u_1(t) y_1(t) + u_2(t) y_2(t)
\]
where $u_1(t)$ and $u_2(t)$ are the now time varying `constants' of the 
homogeneous solution.  Taking the first derivative gives
\[
{\dot y} = u_1 {\dot y}_1 + u_2 {\dot y}_2 + \left( {\dot u}_1 y_1 + {\dot u}_2 y_2 \right) \, .
\]
Taking the second derivative gives 
\[
  {\ddot y} =   {\dot u_1} {\dot y}_1 + u_1 {\ddot y_1}
              + {\dot u_2} {\dot y}_2 + u_2 {\ddot y_2}
              + \frac{d}{dt} \left( {\dot u}_1 y_1 + {\dot u}_2 y_2 \right)\, .
\]
Substituting these terms into the differential equation ${\ddot y} + p {\dot y} + q y = f$ and rearranging yields
\begin{eqnarray}
      u_1 \left( {\ddot y_1} + p {\dot y_1} + q y_1 \right) \,
   +  u_2 \left( {\ddot y_2} + p {\dot y_2} + q y_2 \right) \, & & \\
   +  {\dot u}_1 {\dot y}_1 + {\dot u}_2 {\dot y}_2                        
   +  \left( \frac{d}{dt} + p \right) \left( {\dot u}_1 y_1 
   + {\dot u}_2 y_2 \right) & = &  f \, .
\end{eqnarray}
The first two terms equal to zero since $y_1$ and $y_2$ are solutions to the homogeneous 
equation ${\ddot y} + p {\dot y} + q y = 0$. 
The equation now becomes
\[
  {\dot u}_1 {\dot y}_1 + {\dot u}_2 {\dot y}_2 
+ \left( \frac{d}{dt} + p \right) 
  \left( {\dot u}_1 y_1  + {\dot u}_2 y_2 \right) = f \, ,
\]
which can be simplfied by assuming that the term in parentheses is zero.  
Doing so yields the set of equations
\begin{eqnarray}
  {\dot u}_1 y_1        + {\dot u}_2 y_2        & = & 0 \\
  {\dot u}_1 {\dot y}_1 + {\dot u}_2 {\dot y}_2 & = & f \, .
\end{eqnarray}
These last two can be separated easily in two steps.  In the first, multiply 
the first equation by ${\dot y}_2$ and second by $y_2$ and then
subtract the second from the first.  In the second, multiply the first equation
by ${\dot y}_1$ and the second by $y_1$ and subtract the first from the second.
The following equations result:
\[
  {\dot u}_1 \left( y_1 {\dot y}_2 - {\dot y}_1 y_2 \right) = - f y_2
\]
and
\[
  {\dot u}_2 \left( y_1 {\dot y}_2 - {\dot y}_1 y_2 \right) =   f y_1 \, .
\]
Defining the Wronskian as $W(y_1,y_2) = y_1 {\dot y}_2 - {\dot y}_1 y_2$ gives
the final equations for $u_1$ 
\[
  {\dot u}_1 = \frac{-f \, y_2}{W(y_1,y_2)}
\]
and $u_2$
\[
  {\dot u}_2 = \frac{-f \, y_1}{W(y_1,y_2)} \, .
\]
These equations can be integrated directly to get the form for 
$u_1$ and $u_2$ to give
\[
  u_1(t) = - \int_{t_0}^t \frac{f(\tau) \, y_2(\tau)}{ W(y_1,y_2)(\tau) } d \tau
\]
and
\[
  u_2(t) =   \int_{t_0}^t \frac{f(\tau) \, y_1(\tau)}{ W(y_1,y_2)(\tau) } d \tau \, .
\]
Substituting these forms for $u_1(t)$ and $u_2(t)$ into the expression
$y_p(t) = u_1(t) y_1(t) + u_2(t) y_2(t)$ yields
\[
  y_p(t) = \int_{t_0}^{t} \frac{ y_1(\tau) y_2(t) - y_1(t) y_2(\tau) }{W(y_1,y_2)(\tau)} f(\tau) d \tau \, .
\]
Defining the one-sided Green's function
\[
  g_1(t,\tau) = \frac{ y_1(\tau) y_2(t) - y_1(t) y_2(\tau) }{W(y_1,y_2)(\tau)}
\]
gives the usual form
\[
  y_p(t) = \int_{t_0}^t g_1(t,\tau) f(\tau) d \tau \, .
\]
 In the current case of $f = t \exp{2 t}$, these 
integrations yield ($W = \exp{4 t}$), $u_1 = -t^3/3$ and $u_2 = t^2/2$.
So the particular solution is
\[
  y_p = \frac{t^3 e^{2t}}{6} \, ,
\]
to which can be added any linear multiples of the homogeneous solutions
$y_1 = \exp{2 t}$ and $y_2 = t \exp{2 t}$.

All things considered, the state space methods that 
\section{State Space Methods}

Since the 
\[
   \frac{d}{dt} \left[ \begin{array}{c}x \\ y \end{array} \right]
     = \left[ \begin{array}{cc}
                             0  & 1 \\
                             -4 & 4 
                       \end{array} \right]
      \left[ \begin{array}{c}x \\ y \end{array} \right]
\]
The eigenvalues of $\mathbb A$ are degenerate with the value of $2$ occuring twice.
The implication of this degeneracy is that there will only be one eigenvector that
cvan be obtained with the usual method.
\[
  \left[\begin{array}{cc} \lambda & -1 \\ 4 & \lambda - 4 \end{array}\right]
     \left[\begin{array}{c} a \\ b\end{array}\right]
   = \left[\begin{array}{c} 0 \\ 0 \end{array}\right]
\]
\[
  \left[\begin{array}{cc} 2 & -1 \\ 4 & -2 \end{array}\right]
     \left[\begin{array}{c} a \\ b\end{array}\right]
   = \left[\begin{array}{c} 0 \\ 0 \end{array}\right]
\]
yields $b = 2a$ with $a$ arbitrary.
To obtain another, independent eigenvalue first take a vector that is not a multiple of the
eigenvector determined above. Determine the action of 
the operator ${\mathbb A} - \lambda {\mathbb I}$ on this vector.  It will always be a multiple of 
the previous eigenvector.  Scaling the new vector so that exactly yields the eigenvector results in 
a valid independent vector for diagonalization.  For concreteness, consider $\left< e'_2 \right| = [1,0]$.
\[
\left( {\mathbb A} - \lambda {\mathbb I} \right) \left| e'_2 \right> 
    =    \left[ \begin{array}{cc} -2 & 1\\ -4 & 2\end{array}\right] \left[\begin{array}{c}1\\0\end{array} \right]
    =    \left[\begin{array}{c}-2\\-4\end{array}\right]
    = -2 \left[\begin{array}{c}1\\2\end{array}\right]
    = \left| e_1 \right>
\]
So a proper second vector is $\left<e_2\right| = \left[-1/2,0\right]$ with the resulting 
\[
  \mathbb{S}      = \left[ \begin{array}{cc}1 & -1/2 \\ 2 & 0 \end{array} \right]
\]
and corresponding inverse
\[
  \mathbb{S}^{-1} = \left[ \begin{array}{cc}0 & 1/2 \\ -2 & 1 \end{array} \right]
\]
\[
  {\mathbb S}^{-1} {\mathbb A} {\mathbb S} = {\mathbb M}
\]
where ${\mathbb M}$ takes the form
\[
  \mathbb{M} = \left[ \begin{array}{cc} 2 & 1 \\ 0 & 2 \end{array} \right]
\]
\[
  \mathbb{M}  =       
         \left[ \begin{array}{cc} 2 & 0 \\ 0 & 2 \end{array} \right] 
       + \left[ \begin{array}{cc} 0 & 1 \\ 0 & 0 \end{array} \right]
     \equiv   
         2 \mathbb{I} + \mathbb{V}   
\]
\[
  e^{\mathbb{M}t} = e^{2\mathbb{I}t} e^{\mathbb{V}t} 
\]
\[
  \Phi = \mathbb{S} \mathbb{M} \mathbb{S}^{-1} 
       = \left[ \begin{array}{cc} (1-2t) e^{2t} & t e^{2t} \\ -4 t e^{2 t} & (1+2t) e^{2t} \end{array} \right]
\]
%%%%%%%%%%%%%%%%%%%%%%%%%%%%%%%%%%%%%%%%%%%%%%%%%%%%%%%%%%%%%%%%%%%%
\[
   \frac{d}{dt} \left[ \begin{array}{c}x \\ y \end{array} \right]
     = \left[ \begin{array}{cc}
                             1  & 1 \\
                             -1 & 3 
                       \end{array} \right]
      \left[ \begin{array}{c}x \\ y \end{array} \right]
\]

\[
  \mathbb{S}      = \left[ \begin{array}{cc}1 & -1 \\ 1 & 0 \end{array} \right]
\]
\[
  \mathbb{S}^{-1} = \left[ \begin{array}{cc}0 & 1 \\ -1 & 1 \end{array} \right]
\]
\[
  \mathbb{M} = \left[ \begin{array}{cc} 2 & 1 \\ 0 & 2 \end{array} \right]
\]
\[
  \mathbb{M}  =       
         \left[ \begin{array}{cc} 2 & 0 \\ 0 & 2 \end{array} \right] 
       + \left[ \begin{array}{cc} 0 & 1 \\ 0 & 0 \end{array} \right]
     \equiv   
         \mathbb{I} + \mathbb{V}   
\]
\[
  e^{\mathbb{M}t} = e^{\mathbb{I}t} e^{\mathbb{V}t} 
\]
\[
  e^{\mathbb{M}t} =  
            \left[ \begin{array}{cc} e^{2t}   & 0          \\ 0 & e^{2 t} \end{array} \right]
            \left[ \begin{array}{cc}        1 & t          \\ 0 & 1       \end{array} \right]
          = \left[ \begin{array}{cc} e^{2t}   & t e^{2 t}  \\ 0 & e^{2 t} \end{array} \right]
\]
\[
  \Phi = \mathbb{S} \mathbb{M} \mathbb{S}^{-1} 
       = \left[ \begin{array}{cc} (1-t) e^{2t} & t e^{2t} \\ -t e^{2 t} & (1+t) e^{2t} \end{array} \right]
\]


\end{document} 