\documentclass{article}
\usepackage{graphicx}
\usepackage{pstricks}
\usepackage{pst-plot}
\usepackage{pst-node}
\usepackage{pst-tree}
\usepackage{pst-coil}

\usepackage[absolute]{textpos}
\setlength{\parindent}{0pt}
\usepackage{color}
\usepackage{amssymb}
\usepackage{graphicx}
\textblockorigin{0.5in}{0.5in}
\definecolor{MintGreen}      {rgb}{0.2,0.85,0.5}
\definecolor{LightGreen}     {rgb}{0.9,1,0.9}
\definecolor{LightYellow}    {rgb}{1,1,0.6}
\definecolor{test}           {rgb}{0.8,0.95,0.95}

\TPshowboxestrue
\TPMargin{2mm}
\pagestyle{empty}

\begin{document}

\title{Green's Functions, Causality, and the $i \epsilon$ Prescription}
\author{Conrad Schiff}

\maketitle

\begin{abstract}
The abstract text goes here.
\end{abstract}

\section{Introduction}
One of the most important strategies to follow in teaching a new topic or discipline
is avoiding using same words to describe different things. The use of the same 
words to describe closely related but nonetheless distinct concepts is particularly
confusing. An equally important teaching strategy relates the unknown to the known, 
along the way showing the connections between different phenomena described by the 
same the equations or between the different descriptions of the phenomena . 
Unfortunately, it seems that in the subject of Greens functions, 
most treatments fail to achieve one or both of the goals just mentioned.  

In this article, I first draw a very clear distinction between Greens functions 
used to solve initial value problems (IVPs) and those used for boundary value problems (BVPs). 
I adopt the terminology `one-sided Greens functions' (following \cite{}) to clearly label 
those Greens functions used to solve IVPs from the adjective-less Greens functions used for (BVPs).  
Secondly, I compare and contrast the simple Wronskian-based method presented in most
undergraduate texts (e.g. \cite{}) from the more involved methods based on contour integration
in the complex plane that are used in more advanced work. What emerges from this side-by-side
 It is my hope that the student,
encountering these techniques in the study of either classical or quantum field theory, will
be able This latter  

\[
  x(t) = x_h(t) + \int_{t_0}^{t} g_1(t,\tau) f(\tau)
\]

\[
  f(t) = v_0 \delta(t-\tau) + \theta(t-\tau) {\tilde f}(t)
\]
\[
  x_h(t) = x_0
\]
\[
  g_1(t,\tau) = \theta(t-\tau) (t-\tau)
\]

\begin{equation}
    \label{simple_equation}
    \alpha = \sqrt{ \beta }
\end{equation}

\subsection{Subsection Heading Here}
Write your subsection text here.

\begin{figure}[htp]
\centering
\psscalebox{2}{
\begin{pspicture}(-1.5,-1.5)(1.5,1.5)
  \pscircle[fillstyle=solid,fillcolor=LightGreen,linestyle=none](0,0){1.005}
  \psarc[linewidth=0.5pt]{->}(0,0){1}{32.96709734}{212}
  \psarc[linewidth=0.5pt]{-}(0,0){1}{210}{27.03290266}
  \pscircle[fillstyle=solid,fillcolor=LightYellow,linestyle=none](0,0){0.21}  
  \psarc[linewidth=0.5pt]{-}(0,0){0.205}{45}{190}
  \psarc[linewidth=0.5pt]{<-}(0,0){0.205}{180}{15}
  \pspolygon[fillstyle=solid,fillcolor=LightYellow,linestyle=none,linewidth=0.5pt](0.19125331,0.05124617)(0.89067955,0.45446836)(0.83892092,0.54411694)(0.14140721,0.14140721)
  \psline[linewidth=0.5pt]{-}(0.19318517,0.05176381)(0.89076863,0.45451381)
  \psline[linewidth=0.5pt]{-}(0.14142136,0.14142136)(0.83900482,0.54417136)
  \psline[linewidth=0.5pt]{<-}(0.5419769,0.25313881)(0.89076863,0.45451381)
  \psline[linewidth=0.5pt]{->}(0.14142136,0.14142136)(0.49021309,0.34279636)
  \psline[linewidth=0.2pt]{->}(0,0)(1.29903811,0.75)
  \pscircle[fillstyle=solid,fillcolor=black,linestyle=none](0,0){0.05}
  \uput[45](0.6,0.6){${\scriptsize {\mathcal C}}$}
  \uput[1](-0.8,-0.05){${z_0}$}
  \uput[1](-0.4,-0.4){${{\mathcal C}_2}$}
  \uput[1](-0.2,0.45){${{\mathcal C}_{in}}$}  
  \uput[1](0.2,0){${{\mathcal C}_{out}}$}  
  %\psgrid(-1.5,-1.5)(1.5,1.5)
\end{pspicture}}
\end{figure}

\section{Conclusion}
Write your conclusion here.

\end{document}