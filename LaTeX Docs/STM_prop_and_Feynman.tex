\documentclass[10pt]{article}
\usepackage{pstricks}
\usepackage{pst-plot}
\usepackage{pst-node}
\usepackage{pst-tree}
\usepackage{pst-coil}
\usepackage{bbold}

\usepackage[absolute]{textpos}
\setlength{\parindent}{0pt}
\usepackage{color}
\usepackage{amssymb}
\usepackage{graphicx}
\textblockorigin{0.5in}{0.5in}
\definecolor{MintGreen}      {rgb}{0.2,0.85,0.5}
\definecolor{LightGreen}     {rgb}{0.9,1,0.9}
\definecolor{LightYellow}    {rgb}{1,1,0.6}
\definecolor{test}           {rgb}{0.8,0.95,0.95}

\TPshowboxestrue
\TPMargin{2mm}
\pagestyle{empty}


\begin{document}
\section{Introduction}
In this note we go through the various approaches for formally solving  
\[
 \frac{d}{dt} \left| S \right> = {\mathbb A} \left| S \right> + \left| u \right> \, ,
\]
where $\left| S \right>$ is the state of the system, $A(t)$ called the process
matrix, and $\left| u(t) \right>$ is an inhomogeneous forcing function. 
This equation is the most general way of writing a linear nonhomogeneous coupled set 
of differential equations.  Despite its innocent form, this equation is 
general enough to encompass most linear equations (Maxwell's, Schrodingers, wave, 
heat, and selected Newtonian equations).  The steps taken in this note will be to first
solve the homogeneous equation formally in `time' space and show how the
state transition matrix (STM) serves as a kernel or Green's function
for solving the inhomogeneous equation.  We next explore the formal ways
for constructing the STM and closely related matrix, called the evolution
operator.  The evolution operator has the advantage of allowing the formal
solution to transformed into the `frequency' space via the Fourier transform.
 
\section{Homogeneous Equation}

Begin by solving the homogeneous equation by setting the 
$\left| u \right>$ term to zero.  Now multiply both sides
by $dt$ resulting in 
\[
  d \left| S \right> = {\mathbb A} \left| S \right> dt \, ,
\]
which integrates formally to 
\[
  \left| S \right> = \left| S_0 \right> + \int_{t_0}^{t} dt' {\mathbb A}(t') \left| S(t') \right>  \, .
\]
Since the equation is linear, there must be an operator that connnects
the state of the system at a latter time in terms of the initial 
conditions
\[
 \left| S \right> = \Phi(t,t_0) \left| S_0 \right> \, .
\]
This operator is known as the state transition matrix
(STM), denoted by $\Phi(t,t_0)$, and can be determined by an interative
procedure using the integral equation. Before starting the iteration
define the operator
\[
  {\mathbb B} \left[ \left| X \right> \right] = \left| X_0 \right> + \int_{t0}^t dt' {\mathbb A}(t') \left| X(t') \right>  \, ,
\]
where $\left| X_0 \right>$ is the argument evaluated at the initial
conditions.  The iteration to find $\Phi$ is affected by 
successive applications of $\mathbb B$ to produce the $n^{th}$ approximation
$\left| S \right>^{(n)}$ to $\left| S \right>$. Generalizing the 
obvious pattern to $\left| S \right>$ gives $\Phi$ as an
infinite nested set of integrals. The interation starts
with $\left| S \right>^{(0)} = \left| S_0 \right>$ and 
the first iteration gives
\begin{eqnarray*}
  \left| S \right>^{(1)} & = &  {\mathbb B}\left[ \left| S \right>^{(0)} \right] \\
                         & = & \left| S_0 \right> + \int_{t0}^t dt' {\mathbb A}(t') \left| S_0 \right> \, .
\end{eqnarray*}
Pluggin this result into the next iteration yields
\begin{eqnarray*}
  \left| S \right>^{(2)} & = &  {\mathbb B}\left[ \left| S \right>^{(1)}\right] \\
                         & = & \left| S_0 \right> + \int_{t_0}^t dt' {\mathbb A}(t') 
                                  \left[ \left| S_0 \right> + \int_{t_0}^{t'} dt'' {\mathbb A(t'')} \left| S_0 \right> \right] \\
                         & = & \left[ {\mathbb I} + \int_{t_0}^t dt' {\mathbb A}(t') + \int_{t_0}^t \int_{t_0}^{t'} dt' dt'' {\mathbb A}(t') {\mathbb A}(t'') \right] \left| S_0 \right> \, .
\end{eqnarray*}
Abstracting to $n \rightarrow \infty$ yields
\[
  \left| S \right> = \left[ {\mathbb I} + \int_{t_0}^t dt' {\mathbb A}(t') + \int_{t_0}^t \int_{t_0}^{t'} dt' dt'' {\mathbb A}(t') {\mathbb A}(t'') + ...\right] \left| S_0 \right>
\]
from which the STM can be read off as 
\[
  \Phi(t,t_0) = {\mathbb I} + \int_{t_0}^t dt' {\mathbb A}(t') + \int_{t_0}^t \int_{t_0}^{t'} dt' dt'' {\mathbb A}(t') {\mathbb A}(t'') + ... \, .
\]
Finally, we can derive the equation of motion for the STM by first noting that 
\[
  \frac{d}{dt} \left| S \right> = \frac{d}{dt} \Phi(t,t_0) \left| S_0 \right>
\]
and that 
\[
  \frac{d}{dt} \left| S \right> = A(t) \left| S \right> = A(t) \Phi(t,t_0) \left| S \right> \, .
\]
Equatating the two right-hand sides yields
\[
  \frac{d}{dt} \Phi(t,t_0) = {\mathbb A}(t) \Phi(t,t_0) \, .
\]

\section{Solving the Inhomogeneous Solution}

The STM offers a solution to the inhomogeneous equation by acting as a kernel
that propagates the influence of the forcing function $\left| u(t) \right>$
as follows
\[
 \left| S \right> = \left| S \right>_h + \int_{t_0}^t \Phi(t,t') \left| u(t') \right> dt' \, ,
\]
where $\left| S \right>_h$ is a solution to the homogeneous equation.
To see that the proposed solution satifies the inhomogeneous solution
start by taking the time derivative of each side
\begin{eqnarray*}
  \frac{d}{dt} \left| S \right> & = &  \frac{d}{dt} \left| S \right>_h + \frac{\partial}{\partial t} \int_{t_0}^t \Phi(t,t') \left|u(t')\right> dt' \\
                                & = &  {\mathbb A}(t) \left| S \right>_h + \int_{t_0}^t \frac{\partial \Phi(t,t')}{\partial t} \left| u(t')\right> dt'
                                       + \Phi(t,t) \left| u(t) \right> \\
                                & = & {\mathbb A}(t) \left| S \right>_h + \int_{t_0}^t {\mathbb A}(t) \Phi(t,t') \left| u(t') \right> dt' 
                                       + \left| u(t) \right>\\
                                & = & {\mathbb A}(t) \left[ \left| S\right>_h + \int_{t_0}^t \Phi(t,t') \left| u(t') \right> dt' \right]
                                       + \left| u(t) \right> \\
                                & = & {\mathbb A}(t) \left|S\right> + \left|u(t)\right> \, ,
\end{eqnarray*}
where the Leibnitz rule for differentiating under an integral is used in line 
two, the evolution equation for $\Phi$ is used in line three, and a regrouping
of terms in used in line four.

\section{Time Evolution Operator}

Generally the iterated integrals used in defining the STM are difficult
to implement and give a slow convergence to the final solution as
$n \rightarrow \infty$.  The convergence of the process can be 
increased by using a technique that involves the separation of the 
using the 
\[
  {\mathbb A} = {\mathbb H} + {\mathbb V}(t)
\]
where ${\mathbb H}$ is time-invariant.  Now define a new state 
$\left| T \right>$ by
\[
  \left| S \right> = e^{ {\mathbb H} (t-t_0) } \left| T \right> \,.
\]
Defining the free propagator 
${\mathcal P}(t-t_0) = e^{ {\mathbb H} (t-t_0) }$ yields a time
evolution for $\left| T \right>$ which depends only on $\mathbb V$. 
To determine this relationship first derive the evolution equation
for the free propagator.  Since the free propagator is the matrix 
exponential of $\mathbb H$ it obeys the particularly simple
evolution equation
\[
  \frac{d}{dt} {\mathcal P}(t-t_0) = {\mathbb H} \, {\mathcal P}(t-t_0) \, .
\]
\[
  \frac{d}{dt} \left| S \right> = \frac{d}{dt} \Pi(t-t_0) \left| T \right> 
                                = {\mathbb H} \, \Pi(t-t_0) \left| T \right> 
                                  + \Pi(t-t_0) \frac{d}{dt} \left| T \right>
\]
\[
  {\mathbb H} \, \Pi(t-t_0) \left| T \right> + \Phi(t-t_0) \frac{d}{dt} \left| T \right>
    = ( {\mathbb H} + {\mathbb V}(t) ) \Pi(t-t_0) \left| T \right> + \left| U(t) \right>
\]
\[
  \frac{d}{dt} \left| T \right> = \Pi(t_0-t) {\mathbb V}(t) \, \Pi(t-t_0) \left| T \right> + \Pi(t_0-t) \left| U(t) \right>
\]
\[
  {\mathcal A}(t) = \Pi(t_0-t) {\mathbb V}(t) \, \Pi(t-t_0)
\]
\[
  \left| {\mathcal U}(t) \right> = \Pi(t_0-t) \left| U(t) \right>
\]
\[
  \frac{d}{dt} \left| T \right> = {\mathcal A}(t) \left| T \right> + \left| {\mathcal U}(t) \right>
\]
\[
  \left| T \right> = {\mathbb U}(t,t_0) \left| T_0 \right>
\]
\[
  {\mathbb U}(t,t_0) =    {\mathbb I} + \int_{t_0}^t dt' {\mathcal A}(t') 
                        + \int_{t_0}^t \int_{t_0}^{t'} dt' dt'' {\mathcal A}(t') {\mathcal A}(t'') + ...
\]


Suppose that there exists a matrix $Q$ such that
\[
  Q \sum_{n=0}^{\infty} A^n = 1
\]
then by definition $Q^{-1}$ would be equal to $\sum_{n=0}^{\infty} A^n$. It 
turns out that such a matrix can be trivially constructed.  Assume 
$Q = 1 - A$ then:
\begin{eqnarray}
  (1-A) \sum_{n=0}^{\infty} A^n 
       & = & (1-A) \left(1 + A + A^2 + A^3 + ... \right) \\
       & = & (1-A) + (1-A)A + (1-A)A^2 + (1-A)A^3 + ...\\
       & = & 1 - A + A - A^2 + A^2 - A^3 + A^3 - A^4 + ... \\
       & = & 1 \, .
\end{eqnarray}
So therefore $(1-A)^{-1} = \sum_{n=0}^{\infty} A^n$
\end{document}