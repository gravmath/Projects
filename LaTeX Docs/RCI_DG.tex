%Derivation of the differential geometry of the RIC frame
\documentclass[10pt]{article}
\usepackage{epsfig}
\usepackage{pstricks}
\usepackage{pst-grad} % For gradients
\usepackage{pst-plot}
\usepackage{pst-node}
\usepackage{pst-tree}
\usepackage{pst-coil}

\usepackage[absolute]{textpos}
\setlength{\parindent}{0pt}
\usepackage{color}
\usepackage{amssymb}
\usepackage{graphicx}
\textblockorigin{0.5in}{0.5in}
\definecolor{MintGreen}      {rgb}{0.2,0.85,0.5}
\definecolor{LightGreen}     {rgb}{0.9,1,0.9}
\definecolor{LightYellow}    {rgb}{1,1,0.6}
\definecolor{Blue}           {rgb}{0,0,1}
\definecolor{RIC}            {rgb}{0,0,1}
\TPshowboxestrue


\begin{document}

\title{On the Differential Geometry of the $RIC$ Frame}
\author{Conrad Schiff}
\date{Oct. 18, 2013}
\maketitle
This brief note is meant to lay out the differential geometry associated with 
the $RIC$ frame.  The unit vectors that comprise the frame are defined at each 
step along the object's trajectory in terms of its position $\vec r$ and velocity 
$\vec v$ as
\[
  {\hat R} = \frac{\vec r}{| \vec r |} \, ,
\]
\[
  {\hat C} = \frac{\vec r \times \vec v}{| \vec r \times \vec v |} \, ,
\]
and 
\[
  {\hat I} = {\hat C} \times {\hat R} \, .
\]
\begin{figure}
\begin{center}
\scalebox{1.2}
{
\begin{pspicture}(0,-3)(6.2,2.4)
%\psgrid(0,0)(0,-3)(6.2,2.4)
\rput{45.0}(0.91000307,-2.1969419){\psellipse[linewidth=0.04,dimen=outer](3.1069448,0.0)(3.02,1.3738835)}
\pscircle[linewidth=0.04,dimen=outer,fillstyle=solid,fillcolor=black](1.8084271,-1.2814823){0.28}
\psline[linewidth=0.04cm,arrowsize=0.05291667cm 2.0,arrowlength=1.4,arrowinset=0.4]{->}(1.8084271,-1.2814823)(4.948427,0.19851768)
\psline[linewidth=0.04cm,arrowsize=0.05291667cm 2.0,arrowlength=1.4,arrowinset=0.4]{->}(5.008427,0.23851767)(5.768427,1.4585177)
\psline[linewidth=0.04cm,linecolor=RIC,arrowsize=0.05291667cm 2.0,arrowlength=1.4,arrowinset=0.4]{->}(4.968427,0.19851768)(6.028427,0.6785177)
\psline[linewidth=0.04cm,linecolor=RIC,arrowsize=0.05291667cm 2.0,arrowlength=1.4,arrowinset=0.4]{->}(4.988427,0.21851768)(4.448427,1.0185177)
\psline[linewidth=0.04cm,linecolor=RIC,arrowsize=0.05291667cm 2.0,arrowlength=1.4,arrowinset=0.4]{->}(5.008427,0.25851768)(5.228427,1.1185176)
\put(5.15,1.4){\textcolor{RIC}{$\vec I$}}
\put(5.8,0.8) {\textcolor{RIC}{$\vec R$}}
\put(4.4,1.2) {\textcolor{RIC}{$\vec C$}}
\put(4.2,0.2) {$\vec r$}
\put(5.8,1.55){$\vec v$}
\end{pspicture} 
}
\end{center}
\end{figure}
%         1         2         3         4         5         6         7         8 
%12345678901234567890123456789012345678901234567890123456789012345678901234567890
Since the objects position and velocity change as the object moves, the $RIC$
unit vectors change their orientation as a function of time. In order to fully 
understand how these vectors change, it is neccessary to analyze their
differential geometry.  That said, the general structure can be deduced simply 
based on physical implications as follows.

Generally define
\[
  \hat e_1 \equiv \hat R \, ,
\]
\[
  \hat e_2 \equiv \hat I \, ,
\]
and
\[
  \hat e_3 \equiv \hat C \, .
\]
With this convenient notation, the time derivatives of the dot-products between
these vectors yields
\[
  \frac{d}{dt} \left( \hat e_i \cdot \hat e_j \right) 
             = \frac{d}{dt} \delta_{ij} = 0 \, ,
\] 
which breaks into two relations
\[
  \hat e_i \cdot \frac{d}{dt} \hat e_i = 0
\]
and
\[
  \hat e_i \cdot \frac{d}{dt} \hat e_j 
       = - \hat e_j \cdot \frac{d}{dt} \hat e_i \, .
\]
Applying these two relations across the $RIC$ frame yields the coupled equations 
\[
  \frac{d}{dt} \left[ 
                       \begin{array}{c} \hat R \\ \hat I \\ \hat C \end{array} 
			   \right] = 
               \left[ \begin{array}{ccc} 0       & \alpha & 0      \\
			                             -\alpha & 0      & \beta  \\
										 0       & -\beta & 0      \\
					  \end{array}
			   \right]
			   \left[
			          \begin{array}{c} \hat R \\ \hat I \\ \hat C \end{array}
		       \right] \, ,
\]
where 
\[
  \alpha = \hat I \cdot \frac{d}{dt} \hat R \, 
\]
and
\[
  \beta = \hat C \cdot \frac{d}{dt} \hat I \, .
\]
\[
  \frac{d}{dt} \frac{\vec r}{|\vec r|} 
     = \frac{\vec v}{|\vec r|} - \frac{\vec r \vec v \cdot \vec r}{|\vec r|^3}
\]
\[
  \hat I \cdot \hat R = (\hat C \times 
\]
\end{document}