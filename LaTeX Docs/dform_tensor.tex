\documentclass[10pt]{article}
\usepackage[absolute]{textpos}
\usepackage{graphicx}
\setlength{\parindent}{0pt}
%\usepackage{geometry}
%\geometry{hmargin={1in,1in}}
%\geometry{vmargin={1in,1in}}
%\pagestyle{empty}
\usepackage{color}
\usepackage{amssymb}
\textblockorigin{0.5in}{0.5in}
\definecolor{MintGreen} {rgb}{0.2,0.85,0.5}
\definecolor{LightGreen}{rgb}{0.95,1,0.95}
\definecolor{LightYellow}    {rgb}{1,1,0.65}


\usepackage[absolute]{textpos}
\begin{document}
\null
%%%%%%%%%%%%%%%%%%%%%%%%%%%%%%%%%%%%%%%%%%%%%%%%%%%%%%%%%%%%%%%
%          Header
%%%%%%%%%%%%%%%%%%%%%%%%%%%%%%%%%%%%%%%%%%%%%%%%%%%%%%%%%%%%%%%
\textblockcolor{LightGreen}
\begin{textblock*}{7.5in}(0in,0in)
\begin{tabular*}{7.5in}{|c @{\extracolsep{\fill}} c |}
       \hline
       \small ~ & ~\\
       \multicolumn{2}{|c|}{\normalsize \bf Differential Forms and Tensors} \\
       \small~ & ~\\
       \hline
\end{tabular*}
\end{textblock*}

%%%%%%%%%%%%%%%%%%%%%%%%%%%%%%%%%%%%%%%%%%%%%%%%%%%%%%%%%%%%%%%
%          Definitions
%%%%%%%%%%%%%%%%%%%%%%%%%%%%%%%%%%%%%%%%%%%%%%%%%%%%%%%%%%%%%%%
\small
\textblockcolor{LightYellow}
\begin{textblock*}{93mm}(0mm,12.75mm)
\begin{tabular*}{93mm}{|l @{\extracolsep{\fill}} l|}\hline
\multicolumn{2}{|c|}{\bf Definitions} \\
\multicolumn{2}{|c|}{  }\\
\hline
1        & Differential forms exist to be integrate \cite{bress}\\
\hline
2        & A \emph{differential 1-form in one variable}, $g(x) dx$\\
         &is a mapping from the set of finite intervals to the\\
         &real numbers given by:\\
         & $g(x) dx:\left[a,b\right] \rightarrow \int_{a}^{b} g(x) dx \in {\mathcal R}$\\
\hline   
3        & In addition, this mapping is invariant under invertible\\
         & differential transformations as follow.  If $x=f(t)$,\\
         & then $g(x) dx = g \left( f(t) \right) f'(t) dt$, and\\
         &  $g \left( f(t) \right) f'(t) dt : 
            \left[ f^{-1}(a), f^{-1}(b) \right] 
            \rightarrow 
            \int_{f^{-1}(a)}^{f^{-1}(b)} g \left( f(t) \right) f'(t) dt$ \\
         &$   =           
            \int_{a}^{b} g(x) dx$\\
\hline           
\end{tabular*}
\end{textblock*}

\newpage
%%%%%%%%%%%%%%%%%%%%%%%%%%%%%%%%%%%%%%%%%%%%%%%%%%%%%%%%%%%%%%%%%%
%%%%%%%%%%%%%%%%%%%%%%%%%%%%%%%%%%%%%%%%%%%%%%%%%%%%%%%%%%%%%%%%%%
%     NEW PAGE
%%%%%%%%%%%%%%%%%%%%%%%%%%%%%%%%%%%%%%%%%%%%%%%%%%%%%%%%%%%%%%%%%%
%%%%%%%%%%%%%%%%%%%%%%%%%%%%%%%%%%%%%%%%%%%%%%%%%%%%%%%%%%%%%%%%%%
\textblockcolor{LightGreen}
\begin{textblock*}{7.5in}(0in,0in)
\begin{tabular*}{7.5in}{|c @{\extracolsep{\fill}} c |}
       \hline
       \small ~ & ~\\
       \multicolumn{2}{|c|}{\normalsize \bf Classical Mechanics} \\
       \small~ & ~\\
       \hline
\end{tabular*}
\end{textblock*}

Work Energy\\

\end{document}