The anger still stays with me, even after all these years.  True, I've learned to channel it, 
to put it to some use, but it's still there, sitting like an ache behind my eyes, surging forward at 
times, even when there seems to be little reason.  Over time, I've learned to trust it, to
recognize when it was giving me an early warning of the type of person who was talking to me.
It seems to be able to sniff out the liars, the manipulators, and the shallow fakers of the world, cluing me in
well before any my intellect has had time to accumulate the neccessary evidence. And once its signal has been sent, 
it rides herd on me mercilessly, driving me into action.

Colleages, tell me that the anger is my `vital push', that it spurs me to do the important
things that that need to be done and that it `transports me outside myself', turning me 
into an implaccable adversary.  Members of the adminstration applaud it for providing me the
backbone I need to be uncomprimising in my stance on moral principles.  The people I represent take joy in 
the way I show up fools and bring fitting punishment home to those who deserve it.  They're all right, of course, 
but it's small comfort.  I would give up all my successes as district attorney if I could just get rid of what feeds 
my anger - the image of Mr. McCloskey's face on that terrible day.  That image give my anger its power.  
I can never forget that image, it's burned onto my brain.


I suppose I should start by describing Mr. McCloskey - Timothy by first name for those people
who knew him well. Tim was the kind of guy that local newpapers call a pillar of the community 
and national ones call a greedy  
open-toed waddle.  What he lacked in physical graces, he more than made up for in business acumen.
Status Quo:  McCloskey's stores - made him money, entrepenaur, supported the community

Enemies:  Govenrment and activists (unions and intelligencia)


Action:  Son objects to the what they've done - tries for Freedom

Reaction:  They kill the son and burn down the business

I went to law school and haven't looked back